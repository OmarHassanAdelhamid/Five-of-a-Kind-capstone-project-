\documentclass{article}

\usepackage{tabularx}
\usepackage{booktabs}
\usepackage{amsmath}


\title{Problem Statement and Goals\\\progname}

\author{\authname}

\date{}

\input{../Comments}
%% Common Parts

\newcommand{\progname}{AutoVox} % PUT YOUR PROGRAM NAME HERE
\newcommand{\authname}{Team \#10, Five of a Kind
\\ Omar Abdelhamid
\\ Daniel Maurer
\\ Andrew Bovbel
\\ Olivia Reich
\\ Khalid Farag
} % AUTHOR NAMES                  

\usepackage{hyperref}
    \hypersetup{colorlinks=true, linkcolor=blue, citecolor=blue, filecolor=blue,
                urlcolor=blue, unicode=false}
    \urlstyle{same}
                                


\begin{document}

\maketitle

\begin{table}[hp]
\caption{Revision History} \label{TblRevisionHistory}
\begin{tabularx}{\textwidth}{llX}
\toprule
\textbf{Date} & \textbf{Developer(s)} & \textbf{Change}\\
\midrule
Sept. 16, 2025 & Khalid Farag & Extras\\
Sept. 18, 2025 & Omar Abdelhamid & Problem Statement, Goals, Stretch Goals\\
\bottomrule
\end{tabularx}
\end{table}

\section{Problem Statement}


\subsection{Problem}
There is currently no CAD workflow that lets users take a finished 3D design and
select individual voxels (note: a voxel is a cube building block in a 3D printer) by layers from bottom to top to assign
magnetization. Without voxel selection and per-layer editing, researchers are
forced to rebuild the object voxel-by-voxel (e.g., in COMSOL), manually choosing
a magnetization direction for each voxel. This consumes hours per part and
makes iteration fragile: if a mistake is made, there is no practical way to
reselect or change a prior voxel’s magnetization without redoing subsequent work.
Our system removes this bottleneck by enabling per-layer voxel viewing,
selection (single or multiple voxels), and re-assignment of magnetization
directions on top of an imported design.


\subsection{Inputs and Outputs}
\textbf{Inputs}
\begin{itemize}
  \item Mesh CAD file exported from common tools (e.g., AutoCAD) and ingested as a mesh for voxelization (e.g., STL/OBJ).
  \item Voxel grid initially fixed at $300 \times 300\,\mu\text{m}$ (XY) and $110\,\mu\text{m}$ (Z) to form cubes at the stated dimensions.
  \item User interaction via a UI that slices the object into layers (bottom$\rightarrow$top) and allows selecting one or multiple voxels within a layer to assign magnetization.
\end{itemize}

\noindent\textbf{Outputs }
\begin{itemize}
  \item A Java-readable export containing per-voxel data for the custom printer pipeline (voxel location/layer and magnetization direction metadata).
\end{itemize}


\subsection{Stakeholders}

\begin{tabularx}{\textwidth}{lX}
\toprule
\textbf{Stakeholder} & \textbf{Role / Interest}\\
\midrule
Researchers / Engineers & Primary users who design and print magnetized objects; need fast, reliable voxel-level magnetization after CAD.\\
Lab Operators / Printer Owners & Consume the exported file to run the custom printer workflow; care about correctness and repeatability.\\
Supervisors / PIs (Approvers) & Validate that a magnetization plan is correct for the experiment’s goals.\\
\bottomrule
\end{tabularx}
\subsection{Environment}
\begin{itemize}
  \item \textbf{Platforms:} Desktop/laptop; Windows with WSL or native Linux.
  \item \textbf{Runtime/Integration:} Reads mesh CAD files for voxelization; exports a Java-readable file to drive a custom printer program.
  \item \textbf{Assumptions:} Commodity hardware (laptop/desktop) sufficient for interactive layer viewing and voxel selection.
\end{itemize}



\section{Goals}
\begin{enumerate}
  \item \textbf{Import \& Voxelization:} Import a mesh CAD file and convert it to a voxel grid with fixed voxel size of $300 \times 300\,\mu\text{m}$ (XY) and $110\,\mu\text{m}$ (Z).
  \item \textbf{Layered UI \& Selection:} Provide a UI to browse layers from bottom to top and select voxels (single or multiple) within a layer.
  \item \textbf{Magnetization Assignment \& Editing:} Assign a magnetization direction to selected voxels and allow re-selection and changes without restarting the design (with undo/redo).
  \item \textbf{Export:} Generate a Java-readable file capturing per-voxel magnetization for the custom printer software.
\end{enumerate}

\section{Stretch Goals}
\begin{itemize}
  \item \textbf{Adjustable Voxel Size:} Make voxel dimensions configurable to match different printers (beyond the fixed initial values).
  \item \textbf{Solid CAD Ingest (Future):} Add support for solid CAD (e.g., STEP/IGES) with robust tessellation before voxelization.
  \item \textbf{Additional Export Formats:} Support exporting to formats beyond the Java-based output to integrate with other software/hardware.
\end{itemize}

\section{Extras}

The extras for this project are:

\begin{itemize}
    \item \textbf{User Manual}: This manual will provide a comprehensive guide and explanations on how to effectively use the software, emphasizing the key features that differentiate it from a normal CAD software. This is useful for helping users to operate the system independently, reducing the need for direct support, and ensuring they can utilize the system to its full potential with all of the features. To make it more user-friendly, the manual will be written in a way that is easy to understand and follow, with clear instructions and examples. It will also include any safety warnings that are relevant to the software.
    \item \textbf{Usability Report}: This report will document the findings from usability testing, including identified pain points, user feedback, and recommendations from stakeholders for improving the user experience. It will also include information about how the testing was held, including the number of participants, the tasks they were given, the questions they were asked, and its initial results. This is useful for ensuring the product is intuitive, efficient, and satisfying for its target audience, leading to higher user adoption and satisfaction.
\end{itemize}
\newpage{}

\section*{Appendix --- Reflection}

\wss{Not required for CAS 741}

\input{../Reflection.tex}

\begin{enumerate}
    \item What went well while writing this deliverable? 
    \item What pain points did you experience during this deliverable, and how
    did you resolve them?
    \item How did you and your team adjust the scope of your goals to ensure
    they are suitable for a Capstone project (not overly ambitious but also of
    appropriate complexity for a senior design project)?
\end{enumerate}  

\end{document}