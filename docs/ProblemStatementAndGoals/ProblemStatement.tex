\documentclass{article}

\usepackage{tabularx}
\usepackage{booktabs}
\usepackage{amsmath}


\title{Problem Statement and Goals\\\progname}

\author{\authname}

\date{}

%% Comments

\usepackage{color}

\newif\ifcomments\commentstrue %displays comments
%\newif\ifcomments\commentsfalse %so that comments do not display

\ifcomments
\newcommand{\authornote}[3]{\textcolor{#1}{[#3 ---#2]}}
\newcommand{\todo}[1]{\textcolor{red}{[TODO: #1]}}
\else
\newcommand{\authornote}[3]{}
\newcommand{\todo}[1]{}
\fi

\newcommand{\wss}[1]{\authornote{magenta}{SS}{#1}} 
\newcommand{\plt}[1]{\authornote{cyan}{TPLT}{#1}} %For explanation of the template
\newcommand{\an}[1]{\authornote{cyan}{Author}{#1}}

%% Common Parts

\newcommand{\progname}{Software Engineering} % PUT YOUR PROGRAM NAME HERE
\newcommand{\authname}{Team \#10, Five of a Kind
\\ Omar Abdelhamid
\\ Daniel Maurer
\\ Andrew Bovbel
\\ Olivia Reich
\\ Khalid Farag
} % AUTHOR NAMES                  

\usepackage{hyperref}
    \hypersetup{colorlinks=true, linkcolor=blue, citecolor=blue, filecolor=blue,
                urlcolor=blue, unicode=false}
    \urlstyle{same}
                                


\begin{document}

\maketitle

\begin{table}[hp]
\caption{Revision History} \label{TblRevisionHistory}
\begin{tabularx}{\textwidth}{llX}
\toprule
\textbf{Date} & \textbf{Developer(s)} & \textbf{Change}\\
\midrule
Sept. 16, 2025 & Khalid Farag & Extras\\
Sept. 18, 2025 & Omar Abdelhamid & Problem Statement, Goals, Stretch Goals\\
\bottomrule
\end{tabularx}
\end{table}

\section{Problem Statement}


\subsection{Problem}
There is currently no CAD workflow that lets users take a finished 3D design and
select individual voxels (note: a voxel is a cube building block in a 3D printer) by layers from bottom to top to assign
magnetization. Without voxel selection and per-layer editing, researchers are
forced to rebuild the object voxel-by-voxel (e.g., in COMSOL), manually choosing
a magnetization direction for each voxel. This consumes hours per part and
makes iteration fragile: if a mistake is made, there is no practical way to
reselect or change a prior voxel’s magnetization without redoing subsequent work.
Our system removes this bottleneck by enabling per-layer voxel viewing,
selection (single or multiple voxels), and re-assignment of magnetization
directions on top of an imported design.


\subsection{Inputs and Outputs}
\textbf{Inputs}
\begin{itemize}
  \item Mesh CAD file exported from common tools (e.g., AutoCAD) and ingested as a mesh for voxelization (e.g., STL/OBJ).
  \item Voxel grid initially fixed at $300 \times 300\,\mu\text{m}$ (XY) and $110\,\mu\text{m}$ (Z) to form cubes at the stated dimensions.
  \item User interaction via a UI that slices the object into layers (bottom$\rightarrow$top) and allows selecting one or multiple voxels within a layer to assign magnetization.
\end{itemize}

\noindent\textbf{Outputs }
\begin{itemize}
  \item A Java-readable export containing per-voxel data for the custom printer pipeline (voxel location/layer and magnetization direction metadata).
\end{itemize}


\subsection{Stakeholders}

\begin{tabularx}{\textwidth}{lX}
\toprule
\textbf{Stakeholder} & \textbf{Role / Interest}\\
\midrule
Researchers / Engineers & Primary users who design and print magnetized objects; need fast, reliable voxel-level magnetization after CAD.\\
Lab Operators / Printer Owners & Consume the exported file to run the custom printer workflow; care about correctness and repeatability.\\
Supervisors / PIs (Approvers) & Validate that a magnetization plan is correct for the experiment’s goals.\\
\bottomrule
\end{tabularx}
\subsection{Environment}
\begin{itemize}
  \item \textbf{Platforms:} Desktop/laptop; Windows with WSL or native Linux.
  \item \textbf{Runtime/Integration:} Reads mesh CAD files for voxelization; exports a Java-readable file to drive a custom printer program.
  \item \textbf{Assumptions:} Commodity hardware (laptop/desktop) sufficient for interactive layer viewing and voxel selection.
\end{itemize}



\section{Goals}
\begin{enumerate}
  \item \textbf{Import \& Voxelization:} Import a mesh CAD file and convert it to a voxel grid with fixed voxel size of $300 \times 300\,\mu\text{m}$ (XY) and $110\,\mu\text{m}$ (Z).
  \item \textbf{Layered UI \& Selection:} Provide a UI to browse layers from bottom to top and select voxels (single or multiple) within a layer.
  \item \textbf{Magnetization Assignment \& Editing:} Assign a magnetization direction to selected voxels and allow re-selection and changes without restarting the design (with undo/redo).
  \item \textbf{Export:} Generate a Java-readable file capturing per-voxel magnetization for the custom printer software.
\end{enumerate}

\section{Stretch Goals}
\begin{itemize}
  \item \textbf{Adjustable Voxel Size:} Make voxel dimensions configurable to match different printers (beyond the fixed initial values).
  \item \textbf{Solid CAD Ingest (Future):} Add support for solid CAD (e.g., STEP/IGES) with robust tessellation before voxelization.
  \item \textbf{Additional Export Formats:} Support exporting to formats beyond the Java-based output to integrate with other software/hardware.
\end{itemize}

\section{Extras}

The extras for this project are:

\begin{itemize}
    \item \textbf{User Manual}: This manual will provide a comprehensive guide and explanations on how to effectively use the software, emphasizing the key features that differentiate it from a normal CAD software. This is useful for helping users to operate the system independently, reducing the need for direct support, and ensuring they can utilize the system to its full potential with all of the features. To make it more user-friendly, the manual will be written in a way that is easy to understand and follow, with clear instructions and examples. It will also include any safety warnings that are relevant to the software.
    \item \textbf{Usability Report}: This report will document the findings from usability testing, including identified pain points, user feedback, and recommendations from stakeholders for improving the user experience. It will also include information about how the testing was held, including the number of participants, the tasks they were given, the questions they were asked, and its initial results. This is useful for ensuring the product is intuitive, efficient, and satisfying for its target audience, leading to higher user adoption and satisfaction.
\end{itemize}
\newpage{}

\section*{Appendix --- Reflection}

\iffalse
\wss{Not required for CAS 741}
\fi

The purpose of reflection questions is to give you a chance to assess your own
learning and that of your group as a whole, and to find ways to improve in the
future. Reflection is an important part of the learning process.  Reflection is
also an essential component of a successful software development process.  

Reflections are most interesting and useful when they're honest, even if the
stories they tell are imperfect. You will be marked based on your depth of
thought and analysis, and not based on the content of the reflections
themselves. Thus, for full marks we encourage you to answer openly and honestly
and to avoid simply writing ``what you think the evaluator wants to hear.''

Please answer the following questions.  Some questions can be answered on the
team level, but where appropriate, each team member should write their own
response:


\subsection*{1. What went well while writing this deliverable?}
We were able to write the problem statement and goals in timely and efficient manner due to our ability to effectively gather and apply guidance from the supervisor. The meeting held with the supervisor helped us to clarify the project's scope and goals, and to ensure that we were on the right track. This ensured that the both our team and the supervisor were on the same page, in terms of the project's requirements and expectations. Having this clear understanding early on reduced any ambiguity or confusion, allowing us to focus on the actual implementation of the project.

\subsection*{2. What pain points did you experience during this deliverable, and how did you resolve them?}
\bigskip
\textbf{Khalid:} The pain point I experienced was the time it took to select a project for the capstone. It was difficult to find a project that was both interesting and challenging, and that would be a good fit for our team. Therefore it reduced the amount of time we had to work on this document. To overcome this, the meeting with the supervisor helped us to clarify the project's scope and goals, and I was able to prioritize the work on this document.
\newline
\newline
\textbf{Daniel:} One pain point I experienced was in identifying possible risks for the project's future. It required a lot of forward thinking, estimating where things could go wrong based on my current understanding of the work our group was taking on. Ultimately, by looking into exactly what we would want to demonstrate as a proof of concept, I was able to generate possible routes of failure as well as steps we would take in case any of these possibilities becomes reality.
\newline
\newline
\textbf{Andrew:} The pain point I experienced during this deliverable was some merge conflicts for latex compiled pdfs. It was very frustrating having to do git magic to fix it, and I even gave up at one point and deleted my branch to fix it. A way to fix this, instead of just adding it to gitignore would be ideal. 
\newline
\newline
\textbf{Olivia:} Overall, one of the major challenges was navigating the scope of this project. The size and undertaking of this project are very daunting when I compare it to the project I’ve done in the past. Therefore, I found it difficult during this milestone to determine where to start in defining this project and how to make a plan that encapsulates all that will be required, both in technical skills as well as group collaboration. To resolve this challenge, our group took the necessary steps to understand where we stood as a group in both skills and desired team dynamics. We also took the time to properly explore different project options that we believed would be a good fit. This led to lots of debate and individual introspection before we were able to come together as a group to decide on the project. Even then, once we decided on a project, we had to have a meeting with our supervisor to gain a better understanding of the exact scope we were looking to take on with this project.
\newline
\newline
\textbf{Omar:}The pain point I experienced was the lack of initial understanding of the project requirements and scope. This made it difficult to define clear and achievable goals for the project. To resolve this, we scheduled a meeting with our supervisor to discuss the project in detail. This meeting helped us clarify the project's objectives, constraints, and expectations, allowing us to refine our goals and ensure they were aligned with the project's requirements. My other pain point was dealing with merge conflicts in LaTeX files. To resolve this, I coordinated with my team members to ensure that we were working on separate sections of the document and regularly communicated about changes to avoid overlapping edits. Also we maintained a clear version control strategy to manage contributions effectively while having a pull request review process to catch any potential conflicts early and we were in constant communication to ensure everyone was on the same page and able to help each other when needed.


\subsection*{3. How did you and your team adjust the scope of your goals to ensure they are suitable for a Capstone project (not overly ambitious but also of appropriate complexity for a senior design project)?}
The team did not need to adjust the scope of our goals as we were able to meet the requirements of the project. The meeting with the supervisor further clarified the project's scope and goals, and we were able to ensure that our goals were suitable for a Capstone project.

\end{document}