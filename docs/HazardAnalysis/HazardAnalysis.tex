\documentclass{article}

\usepackage{booktabs}
\usepackage{tabularx}
\usepackage{hyperref}
\usepackage{float}
\usepackage{pdflscape} % For landscape pages
\usepackage{geometry}

\hypersetup{
    colorlinks=true,       % false: boxed links; true: colored links
    linkcolor=red,          % color of internal links (change box color with linkbordercolor)
    citecolor=green,        % color of links to bibliography
    filecolor=magenta,      % color of file links
    urlcolor=cyan           % color of external links
}

\title{Hazard Analysis\\\progname}

\author{\authname}

\date{}

\input{../Comments}
%% Common Parts

\newcommand{\progname}{AutoVox} % PUT YOUR PROGRAM NAME HERE
\newcommand{\authname}{Team \#10, Five of a Kind
\\ Omar Abdelhamid
\\ Daniel Maurer
\\ Andrew Bovbel
\\ Olivia Reich
\\ Khalid Farag
} % AUTHOR NAMES                  

\usepackage{hyperref}
    \hypersetup{colorlinks=true, linkcolor=blue, citecolor=blue, filecolor=blue,
                urlcolor=blue, unicode=false}
    \urlstyle{same}
                                


\begin{document}

\maketitle
\thispagestyle{empty}

~\newpage

\pagenumbering{roman}

\begin{table}[hp]
\caption{Revision History} \label{TblRevisionHistory}
\begin{tabularx}{\textwidth}{llX}
\toprule
\textbf{Date} & \textbf{Developer(s)} & \textbf{Change}\\
\midrule
Date1 & Name(s) & Description of changes\\
Date2 & Name(s) & Description of changes\\
... & ... & ...\\
\bottomrule
\end{tabularx}
\end{table}

~\newpage

\tableofcontents

~\newpage

\pagenumbering{arabic}

\wss{You are free to modify this template.}

\section{Introduction}

\wss{You can include your definition of what a hazard is here.}

\section{Scope and Purpose of Hazard Analysis}

\wss{You should say what \textbf{loss} could be incurred because of the
hazards.}

\section{System Boundaries and Components}

\wss{Dividing the system into components will help you brainstorm the hazards.
You shouldn't do a full design of the components, just get a feel for the major
ones.  For projects that involve hardware, the components will typically include
each individual piece of hardware.  If your software will have a database, or an
important library, these are also potential components.}

\section{Critical Assumptions}

\wss{These assumptions that are made about the software or system.  You should
minimize the number of assumptions that remove potential hazards.  For instance,
you could assume a part will never fail, but it is generally better to include
this potential failure mode.}


\iffalse
\wss{Include your FMEA table here. This is the most important part of this document.}
\wss{The safety requirements in the table do not have to have the prefix SR.
The most important thing is to show traceability to your SRS. You might trace to
requirements you have already written, or you might need to add new
requirements.}
\wss{If no safety requirement can be devised, other mitigation strategies can be
entered in the table, including strategies involving providing additional
documentation, and/or test cases.}
\fi

\newgeometry{top=1.5cm, bottom=2cm, left=1.5cm, right=3cm} 
\begin{landscape}
\section{Failure Mode and Effect Analysis}
\begin{table}[H]
\centering
\caption{Failure Mode and Effect Analysis (FMEA)}
\renewcommand{\arraystretch}{1.3} 
\footnotesize 
\begin{tabular}{|p{3cm}|p{3cm}|p{3.8cm}|p{3.8cm}|p{5.5cm}|p{2cm}|p{1cm}|}
\hline
\textbf{Component} & \textbf{Failure Mode} & \textbf{Causes of Failure(s)} & \textbf{Effects of Failure(s)} & \textbf{Recommended Action(s)} & \textbf{SR} & \textbf{Ref.} \\
\hline
Slicing Manager & The system does not process the CAD file correctly as the user is expecting. & \textbullet{} The inputed STL file is corrupted therefore it can't be read by the system.\newline \textbullet{} The input file is not in the correct format so the system can't process it.& \textbullet{} The system can't process the file therefore it can't be sliced. \newline \textbullet{} Due to the confusion in system not reading the file, this will result in user frustration. \newline \textbullet{} It will lead to user spending more time to get the system to process the file. & When the file is first inputted into the system, the system will validate that the file is first in the correct format, and that it does not contain any corrupt information. This will then lead to the system prompting the user to re-enter the file or input a new one in. & F211, NF211 & H1 \\
\hline
Imaging Manager & The system fails to render the 3D image of the model & \textbullet{} The device that is being used to run the system does not have enough memory to render the image. \newline \textbullet{} The device's GPU driver is not working correctly or is not up to date as required for the system to render the image. & \textbullet{} Model is not displayed leading to poor user experience. \newline \textbullet{} Designers can't proceed with the design review, and fall back to using their old design software to add magnetization and material properties to the design. & The system should check if the device has enough memory to render the image, and if the GPU driver is working correctly. If not, the system should prompt the user to use a different device or update the GPU driver. & F221, NF221 & H2 \\
\hline
Exportation Manager & Exported file is missing magnetization and material properties& \textbullet{} The system does not export the magnetization and material properties to the CSV file correctly. \newline \textbullet{} The system fails to save the magnetization and material properties for certain voxels & \textbullet{} The Java program that reads the CSV file receives incomplete data causing print failure or defects. \newline \textbullet{} The Java program reads the file correctly, however the printer crashes due to missing magnetization and material properties. \newline \textbullet{} Material will be wasted if printer crashes after printing a few layers. & The system should validate that the magnetization and material properties are saved for all voxels. If not, the system should prompt the user to re-export the file. & F242, NF241 & H3 \\
\hline
Exportation Manager & File format is exported in the incorrect format & \textbullet{} The system does not export the CSV file in the correct structure and format. & The Java program can't read the file correctly because of the wrong format therefore causing the program to crash. & The system should check that the structure of the CSV file is correct and matches the required format. & F241 & H4 \\
\hline
\end{tabular}
\end{table}
\end{landscape}
\restoregeometry

\section{Safety and Security Requirements}

\wss{Newly discovered requirements.  These should also be added to the SRS.  (A
rationale design process how and why to fake it.)}

\section{Roadmap}

\wss{Which safety requirements will be implemented as part of the capstone timeline?
Which requirements will be implemented in the future?}

\newpage{}

\section*{Appendix --- Reflection}

\wss{Not required for CAS 741}

\input{../Reflection.tex}

\begin{enumerate}
    \item What went well while writing this deliverable? 
    \item What pain points did you experience during this deliverable, and how
    did you resolve them?
    \item Which of your listed risks had your team thought of before this
    deliverable, and which did you think of while doing this deliverable? For
    the latter ones (ones you thought of while doing the Hazard Analysis), how
    did they come about?
    \item Other than the risk of physical harm (some projects may not have any
    appreciable risks of this form), list at least 2 other types of risk in
    software products. Why are they important to consider?
\end{enumerate}

\end{document}