\documentclass{article}

\usepackage{booktabs}
\usepackage{tabularx}
\usepackage{hyperref}
\usepackage{float}
\usepackage{pdflscape} % For landscape pages
\usepackage{geometry}
\usepackage{enumerate}

\hypersetup{
    colorlinks=true,       % false: boxed links; true: colored links
    linkcolor=red,          % color of internal links (change box color with linkbordercolor)
    citecolor=green,        % color of links to bibliography
    filecolor=magenta,      % color of file links
    urlcolor=cyan           % color of external links
}

\title{Hazard Analysis\\\progname}

\author{\authname}

\date{}

%% Comments

\usepackage{color}

\newif\ifcomments\commentstrue %displays comments
%\newif\ifcomments\commentsfalse %so that comments do not display

\ifcomments
\newcommand{\authornote}[3]{\textcolor{#1}{[#3 ---#2]}}
\newcommand{\todo}[1]{\textcolor{red}{[TODO: #1]}}
\else
\newcommand{\authornote}[3]{}
\newcommand{\todo}[1]{}
\fi

\newcommand{\wss}[1]{\authornote{magenta}{SS}{#1}} 
\newcommand{\plt}[1]{\authornote{cyan}{TPLT}{#1}} %For explanation of the template
\newcommand{\an}[1]{\authornote{cyan}{Author}{#1}}

%% Common Parts

\newcommand{\progname}{Software Engineering} % PUT YOUR PROGRAM NAME HERE
\newcommand{\authname}{Team \#10, Five of a Kind
\\ Omar Abdelhamid
\\ Daniel Maurer
\\ Andrew Bovbel
\\ Olivia Reich
\\ Khalid Farag
} % AUTHOR NAMES                  

\usepackage{hyperref}
    \hypersetup{colorlinks=true, linkcolor=blue, citecolor=blue, filecolor=blue,
                urlcolor=blue, unicode=false}
    \urlstyle{same}
                                


\begin{document}

\maketitle
\thispagestyle{empty}

~\newpage

\pagenumbering{roman}

\begin{table}[hp]
\caption{Revision History} \label{TblRevisionHistory}
\begin{tabularx}{\textwidth}{llX}
\toprule
\textbf{Date} & \textbf{Developer(s)} & \textbf{Change}\\
\midrule
October 10, 2025 & All & Created initial revision of Hazard Analysis\\
\bottomrule
\end{tabularx}
\end{table}

~\newpage

\tableofcontents

~\newpage

\pagenumbering{arabic}

\section{Introduction}

The following sections aim to provide insight into our project and the role that hazard analysis
plays throughout its development. It introduces the concept of a hazard and how it may manifest
in software development. It then connects those ideas back to the scope of our current project.

\subsection{Problem Statement}

Recently, master's students at McMaster University have been developing cutting-edge 3D
printer technology that has the capabilities of incorporating complex elements of
magnetization and multi-material components. However, with this innovative new design
comes the challenge of limited software capabilities that are unable to support the
unique needs of property assignment while simultaneously allowing an efficient and
intuitive design. Currently, developers of the 3D printer are forced to design their
model outside of COMSOL, which is the current software that permits magnetization assignment.
To build their model for printing, they must build their entire model voxel-by-voxel in
COMSOL, individually assigning each voxel a specific magnetization. This leads to an
unintuitive, time-consuming process that is unideal as they look to further test and develop
their printer. They require new software that can take a model built through AutoCAD and
provide a replica that has been sliced into voxels with the ability to easily assign
magnetization and material values to each voxel.

\subsection{Hazard Analysis Introduction}

As defined by the Canadian Centre for Occupational Health and Safety, a hazard can be
defined as any source of potential for harm, damage, or adverse effect to people, property,
or the environment [1]. Many of the typical factors that can contribute to a hazard,
including human error, unsafe working conditions, equipment malfunction, environmental
influences, and inadequate safety procedures. However, from the perspective of a software
developer, hazards can extend outside the factors listed above and may take form in data
corruption, unstable system components, poor development processes, security risks and
technological limitations. \par \vspace{1em}

As described above, our project aims to create software that supports slicing a
pre-existing model into voxels and permitting easy property assignment for future
printing processes.This project has the potential to greatly reduce the time, energy
and workload exerted on those using the multi-material 3D printer. However, to accomplish
this, there are many potential hazards that must be mitigated during system development.
These hazards may come from impractical technical implementations, incompatible input data
structures with pre-existing software or faulty output generation that could result in
unsafe or unintended behaviour in the physical system. Therefore, the hazard analysis
outlined below aims to identify and assess all hazards to ensure that our team is able to
successfully deploy a safe and reliable software system.

\section{Scope and Purpose of Hazard Analysis}

\textbf{Scope:} 
This hazard analysis examines the potential risks and failures that may arise during the development and operation of the voxel-based design platform used for automating multi-material 3D printing. The analysis covers all major stages of the system’s lifecycle, including CAD model import, voxelization, visualization, editing, and export. Each of these stages introduces possible hazards such as file corruption, rendering issues, data loss, or miscommunication between the user interface and internal processes. The scope also considers hazards resulting from external factors such as hardware limitations, printer software compatibility, and computational resource constraints. The analysis is limited to hazards directly related to the software system and its interactions, and excludes physical hardware malfunctions or the user misuse outside normal operating conditions.

\textbf{Purpose:} 
The purpose of this hazard analysis is to identify, evaluate, and mitigate risks that could compromise the reliability, safety, or effectiveness of the voxel design platform. Conducting this analysis helps the development team anticipate failures early in the project lifecycle, reduce potential losses of time, data, and materials, and improve the system’s overall dependability. The findings from this process will inform safety requirements, guide testing and validation efforts, and contribute to the creation of a robust, efficient, and trustworthy tool for research and development in multi-material 3D printing.

\section{System Boundaries and Components}

\iffalse
\wss{Dividing the system into components will help you brainstorm the hazards.
You shouldn't do a full design of the components, just get a feel for the major
ones.  For projects that involve hardware, the components will typically include
each individual piece of hardware.  If your software will have a database, or an
important library, these are also potential components.}
\fi

As discussed in the SRS, the system consists of four primary managers, each responsible
for a specific part of the transformation from the initial CAD file to the completed 
voxel file. All components operate locally, with dependencies limited to
the library files needed to interact with necessary files.

\subsection{Import Manager}
\begin{itemize}
    \item \textbf{Purpose:} Converts input external files into internal native type and interprets existing native files.
    \item \textbf{Key Functions:} 
    \begin{itemize}
        \item Parse CAD files (as per F211)
        \item Generate output voxel files for program use (as per F213, F214, F215, NF211)
    \end{itemize}
    \item \textbf{Integration:} Indirectly receives CAD files from external CAD software.
\end{itemize}

\subsection{Visualization Manager}
\begin{itemize}
    \item \textbf{Purpose:} Renders interactible 3D images for model visualization.
    \item \textbf{Key Functions:} 
    \begin{itemize}
        \item Generate an interactible 3D visualization of the voxel model (as per F221, F223, NF221, NF222)
        \item Allow users to view layers of the voxel model in isolation (as per F224)
        \item Visualize voxels selected by the user (as per F225)
    \end{itemize}
\end{itemize}

\subsection{Editing Manager}
\begin{itemize}
    \item \textbf{Purpose:} Tracks and stores user edits to voxel properties.
    \item \textbf{Key Functions:} 
    \begin{itemize}
        \item Set properties for sets of selected voxels (as per F231, F233, F235)
        \item Save per-voxel properties within project files (as per F236, NF232)
    \end{itemize}
\end{itemize}

\subsection{Export Manager}
\begin{itemize}
    \item \textbf{Purpose:} Handles saving and exporting of user files.
    \item \textbf{Key Functions:} 
    \begin{itemize}
        \item Export voxel models as a native project file (as per F243)
        \item Export project file as printer-ready format (as per F242, NF241)
        \item Validate that each voxel has properties assigned correctly (as per F241)
    \end{itemize}
    \item \textbf{Integration:} Indirectly integrates with printer software through export files.
\end{itemize}


\section{Critical Assumptions}

The following critical assumptions define the scope of the hazard analysis by identifying conditions that are assumed to hold true. These assumptions eliminate certain classes of hazards from consideration in the FMEA.

\begin{itemize}
    \item \textbf{No Direct Hardware Control}: The system will never directly control physical 3D printer hardware. Physical safety hazards related to direct hardware manipulation (printer collision, overheating, mechanical failures) are out of scope for the current version.

    \item \textbf{Local Operation Only}: The system operates in a completely local environment without network dependencies. Network-related hazards (data breaches, denial of service attacks, connectivity issues) are out of scope for the current version.

    \item \textbf{Non-Malicious User Intent}: Users act in good faith and use the system as intended. Potential hazards from malicious misuse (intentional data corruption, system exploitation) are out of scope for the current version but may be addressed later.

    \item \textbf{Controlled Laboratory Environment}: The system operates within a controlled laboratory or research environment with appropriate workspace and environmental conditions. Hazards related to extreme operating conditions (temperature extremes, electromagnetic interference) are out of scope for the current version.

    \item \textbf{Single-User Operation}: The system assumes single-user operation at any given time. Concurrency hazards, access control conflicts, and collaborative editing synchronization issues are out of scope for the current version.

    \item \textbf{Experienced User Base}: Users are experienced researchers and lab operators with knowledge of CAD software and voxel magnetization workflows. Basic usability and learning curve hazards are out of scope for the current version.

    \item \textbf{Standard Hardware Requirements}: The system operates on commodity hardware meeting minimum specifications for 3D visualization. Hardware-specific performance degradation and compatibility issues are out of scope for the current version.
\end{itemize}


\iffalse
\wss{Include your FMEA table here. This is the most important part of this document.}
\wss{The safety requirements in the table do not have to have the prefix SR.
The most important thing is to show traceability to your SRS. You might trace to
requirements you have already written, or you might need to add new
requirements.}
\wss{If no safety requirement can be devised, other mitigation strategies can be
entered in the table, including strategies involving providing additional
documentation, and/or test cases.}
\fi

\newgeometry{top=1.5cm, bottom=2cm, left=1.5cm, right=3cm} 
\begin{landscape}
\section{Failure Mode and Effect Analysis}
In this section, we will be analyzing the failure modes and effects of the system. The following table will break down the potential failure modes, their causes, effects, recommended actions, and the safety requirements that are associated with them.
\begin{table}[H]
\centering
\caption{Failure Mode and Effect Analysis (FMEA) - Part 1}
\renewcommand{\arraystretch}{1.1} 
\footnotesize 
\begin{tabular}{|p{3cm}|p{3cm}|p{3.8cm}|p{3.8cm}|p{5.5cm}|p{2cm}|p{1cm}|}
\hline
\textbf{Component} & \textbf{Failure Mode} & \textbf{Causes of Failure(s)} & \textbf{Effects of Failure(s)} & \textbf{Recommended Action(s)} & \textbf{SR} & \textbf{Ref.} \\
\hline
Import Manager & The system does not process the CAD file correctly as the user is expecting. & \textbullet{} The inputed STL file is corrupted therefore it can't be read by the system.\newline \textbullet{} The input file is not in the correct format so the system can't process it.& \textbullet{} The system can't process the file therefore it can't be sliced. \newline \textbullet{} Due to the confusion in system not reading the file, this will result in user frustration. \newline \textbullet{} It will lead to user spending more time to get the system to process the file. & When the file is first inputted into the system, the system will validate that the file is first in the correct format, and that it does not contain any corrupt information. This will then lead to the system prompting the user to re-enter the file or input a new one in. & F211, NF211, SCR1, SCR9 & H1 \\
\hline
Visualization Manager & The system fails to render the 3D image of the model & \textbullet{} The device that is being used to run the system does not have enough memory to render the image. \newline \textbullet{} The device's GPU driver is not working correctly or is not up to date as required for the system to render the image. & \textbullet{} Model is not displayed leading to poor user experience. \newline \textbullet{} Designers can't proceed with the design review, and fall back to using their old design software to add magnetization and material properties to the design. & The system should check if the device has enough memory to render the image, and if the GPU driver is working correctly. If not, the system should prompt the user to use a different device or update the GPU driver. & F221, NF221, SCR2, SCR9 & H2 \\
\hline
Visualization Manager & The voxels displayed as selected/currently editing do not match the voxels actually being edited. & \textbullet{} Logical error in the system's selection/editing mechanism. \newline \textbullet{} UI rendering bug not accurately reflecting the backend state. \newline \textbullet{} Inconsistent data synchronization between front-end display and back-end data model. & \textbullet{} Poor user experience (UX) as the user thinks they're doing one thing, but the system is actually editing something completely different. \newline \textbullet{} Incorrect modifications to the model. \newline \textbullet{} Loss of user trust in the system's accuracy. \newline \textbullet{} Time wasted correcting errors. & The system shall ensure real-time synchronization between the UI display of selected/edited voxels and the backend data model.  & F225, F227, F228, NF221, SCR3, SCR8 & H3 \\
\hline
\end{tabular}
\end{table}

\begin{table}[H]
\centering
\caption{Failure Mode and Effect Analysis (FMEA) - Part 2}
\renewcommand{\arraystretch}{1.1} 
\footnotesize 
\begin{tabular}{|p{3cm}|p{3cm}|p{3.8cm}|p{3.8cm}|p{5.5cm}|p{2cm}|p{1cm}|}
\hline
\textbf{Component} & \textbf{Failure Mode} & \textbf{Causes of Failure(s)} & \textbf{Effects of Failure(s)} & \textbf{Recommended Action(s)} & \textbf{SR} & \textbf{Ref.} \\
\hline
Export Manager and Editing Manager & Exported file is missing magnetization and material properties& \textbullet{} The system does not export the magnetization and material properties to the metadata file correctly. \newline \textbullet{} The system fails to save the magnetization and material properties for certain voxels & \textbullet{} The custom 3D printer software that reads the metadata file receives incomplete data causing print failure or defects. \newline \textbullet{} The custom 3D printer software reads the file correctly, however the printer crashes due to missing magnetization and material properties. \newline \textbullet{} Material will be wasted if printer crashes after printing a few layers. & The system should validate that the magnetization and material properties are saved for all voxels. If not, the system should prompt the user to re-export the file. & F231, F233, NF232, F241, F242, SCR4 & H4 \\
\hline
Export Manager & File format is exported in the incorrect format & \textbullet{} The system does not export the metadata file in the correct structure and format. & The custom 3D printer software can't read the file correctly because of the wrong format therefore causing the program to crash. & The system should check that the structure of the metadata file is correct and matches the required format. & F241, F242, NF241, SCR5 & H5 \\
\hline
Full System & User progress is lost & \textbullet{} Unexpected software bug that causes the system to crash, which causes the system to lose its state. \newline \textbullet{} The user experiences a hardware malfunction (e.g., power loss, memory failure), which causes the system to lose its state. & \textbullet{} User loses signifcant work and would need to re-do the work. \newline \textbullet{} User will start getting frustrated and lose trust in the system. \newline \textbullet{} User will lose signicant time in the total workflow of their design due to the need to re-do the work. & The system shall implement a auto-saving mechanism to periodically save the state of the user's process to allow for recovery in case of a system shutdown or failure. & F236, NF241, SCR6, SCR7 & H6 \\
\hline
\end{tabular}
\end{table}
\end{landscape}
\restoregeometry

\section{Safety and Security Requirements}

In this section, we will be analyzing the safety requirements of the system. The following table will break down the potential safety requirements, their associated hazards, and their priority.

\begin{table}[H]
    \centering
    \renewcommand{\arraystretch}{1.2}
    \begin{tabular}{|l|c|}
        \toprule \textbf{Symbol} & \textbf{Value} \\
        \midrule
        OPTIMAL\_VOXEL\_SIZE & 5.5nm \\
        MAX\_VOXELS & 13,996,800,000 voxels \\
        MAX\_LAYER\_DISPLAY & 518,400 voxels \\
        MAX\_DISPLAY & 103,680,000 voxels \\
        MAX\_VOXEL\_SELECTION & 1000 voxels \\
        MIN\_RATE & 500,000 voxels per second \\
        MAX\_EDIT\_LATENCY & 1 second \\
        MAX\_VIEW\_LATENCY & 500ms \\
        MAX\_INTERACTIONS & 5\\
        INPUT\_FILE\_VALIDATION\_THRESH & 100\% \\
        MIN\_RENDER\_MEMORY & 16 GB \\
        UI\_SYNC\_ACCURACY & 100\% \\
        EXPORT\_COMPLETENESS\_THRESH & 100\% \\
        METADATA\_FORMAT\_VALIDATION & 100\% \\
        AUTO\_SAVE\_INTERVAL & 5 minutes \\
        UNDO\_HISTORY\_SIZE & 10 actions \\
        FILE\_IMPORT\_EXPORT\_TIMEOUT &  2 minutes \\
        FEEDBACK\_UPDATE\_TIME & 30 seconds \\
        \bottomrule
    \end{tabular}
    \caption{Expanded Table of Symbolic Constants. This includes constants from SRS S.2, with the addition of new constants derived during the hazard analysis process.}
    \label{tab:sym}
\end{table}
\begin{enumerate}
\renewcommand{\labelenumi}{SCR \arabic{enumi}.}

\item \emph{The system shall validate the imported CAD files for their format correctness before processing the model.}\\
    \quad {\bf Rationale:} The corrupted files can cause the system to fail and prevent successful voxelization to the workflow disruption.\\
    \quad {\bf Fit Criterion:} \texttt{INPUT\_FILE\_VALIDATION\_THRESH} of the imported files must pass the validation checks, and the system should reject any files that fail to do so.\\
    \quad {\bf Associated Hazards:} H1\\
    \quad {\bf Priority:} High

\item \emph{The system shall check if the device has enough memory and GPU capabilities before attempting to render 3D models and voxel grids.}\\
    \quad {\bf Rationale:} Insufficient memory or outdated GPU drivers can cause rendering failures with the system, preventing users from visualizing their models and proceeding with property assignments.\\
    \quad {\bf Fit Criterion:} The hardware where the system is running on must have at least \texttt{MIN\_RENDER\_MEMORY} of memory, and the GPU driver must be compatible with the system.\\
    \quad {\bf Associated Hazards:} H2\\
    \quad {\bf Priority:} Medium

\item \emph{The system shall maintain real-time synchronization between the UI component of selected/edited voxels and the backend data model.}\\
    \quad {\bf Rationale:} Any failures in this synchronization can lead to the user editing unintended voxels, causing incorrect property assignment and loss of user trust in system accuracy.\\
    \quad {\bf Fit Criterion:} \texttt{UI\_SYNC\_ACCURACY} of voxel selections and edits must be accurately reflected in both UI display and backend data model.\\
    \quad {\bf Associated Hazards:} H3\\
    \quad {\bf Priority:} High

\item \emph{The system shall validate all magnetization and material properties before exporting the file containing the metadata for all voxels.}\\
    \quad {\bf Rationale:} Missing these important information can cause excessive time and material waste, and in some cases can cause printer failures.\\
    \quad {\bf Fit Criterion:} \texttt{EXPORT\_COMPLETENESS\_THRESH} of exported files must contain complete magnetization and material property data for all voxels.\\
    \quad {\bf Associated Hazards:} H4\\
    \quad {\bf Priority:} High

\item \emph{The system shall validate the metadata file format and structure before exporting the file, ensuring that it is compatible with the next system that will read the file.}\\
    \quad {\bf Rationale:} Incorrect metadata format can cause the custom 3D printer program to crash, preventing the user from continuing with the workflow, leading to user frustration.\\
    \quad {\bf Fit Criterion:} \texttt{METADATA\_FORMAT\_VALIDATION} of exported metadata files must match the required structure and format expected by the next system that will read the file.\\
    \quad {\bf Associated Hazards:} H5\\
    \quad {\bf Priority:} High

\item \emph{The system shall implement auto-saving functionality to preserve user progress and enable recovery from any system crashes or hardware failures.}\\
    \quad {\bf Rationale:} Any system crashes or hardware failures can lead to signicant amount of work loss, which would require the user to restart the workflow from scratch, which would be very time consuming and frustrating.\\
    \quad {\bf Fit Criterion:} The system must automatically save user progress every \texttt{AUTO\_SAVE\_INTERVAL}.\\
    \quad {\bf Associated Hazards:} H6\\
    \quad {\bf Priority:} High

\item \emph{The system shall provide undo/redo functionality to allow users to correct their mistakes without losing signifcant work.}\\
    \quad {\bf Rationale:} Human errors in design are common and users need the ability to easily correct these mistakes without restarting the process.\\
    \quad {\bf Fit Criterion:} The system must maintain an undo history of at least \texttt{UNDO\_HISTORY\_SIZE} actions and provide clear undo/redo controls.\\
    \quad {\bf Associated Hazards:} H6\\
    \quad {\bf Priority:} Medium

\item \emph{The system shall limit the number of voxels that can be selected simultaneously to prevent any performance issues.}\\
    \quad {\bf Rationale:} Selecting too many voxels at a time can cause system slowdowns, UI freezing and an overall poor user experience.\\
    \quad {\bf Fit Criterion:} The system must limit voxel selection to a maximum of \texttt{MAX\_VOXEL\_SELECTION} voxels at any time and provide clear feedback when limits are reached.\\
    \quad {\bf Associated Hazards:} H3\\
    \quad {\bf Priority:} Medium

\item \emph{The system shall provide progress updates and timeout handling for long running operations such as importing/exporting the file, and voxelization.}\\
    \quad {\bf Rationale:} Large CAD files can take significant time to process, and users need feedback on progress to avoid waiting indefinitely.\\
    \quad {\bf Fit Criterion:} All import and export operations must timeout after \texttt{FILE\_IMPORT\_EXPORT\_TIMEOUT}, and all operations exceeding \texttt{FEEDBACK\_UPDATE\_TIME} must provide progress updates.\\
    \quad {\bf Associated Hazards:} H1, H2\\
    \quad {\bf Priority:} Medium

\end{enumerate}

\iffalse
\wss{Newly discovered requirements.  These should also be added to the SRS.  (A
rationale design process how and why to fake it.)}
\fi

\section{Roadmap}

In this section, we will state which safety requirements will be implemented as part of the capstone timeline, and which requirements will be implemented in the future.

Requirements that will be implemented as part of the capstone timeline, as they are considered to be high priority, and will be implemented first:
\begin{itemize}
    \item SCR 1
    \item SCR 2
    \item SCR 3
    \item SCR 4
    \item SCR 5
\end{itemize}

Requirements that will be implemented in the future, this also contains requirements that are considered to be high priority, and will only be implemented if time permits:
\begin{itemize}
    \item SCR 6
    \item SCR 7
    \item SCR 8
\end{itemize}

\iffalse
\wss{Which safety requirements will be implemented as part of the capstone timeline?
Which requirements will be implemented in the future?}
\fi


\newpage{}

\section*{Appendix --- Reflection}

\iffalse
\wss{Not required for CAS 741}
\fi

The purpose of reflection questions is to give you a chance to assess your own
learning and that of your group as a whole, and to find ways to improve in the
future. Reflection is an important part of the learning process.  Reflection is
also an essential component of a successful software development process.  

Reflections are most interesting and useful when they're honest, even if the
stories they tell are imperfect. You will be marked based on your depth of
thought and analysis, and not based on the content of the reflections
themselves. Thus, for full marks we encourage you to answer openly and honestly
and to avoid simply writing ``what you think the evaluator wants to hear.''

Please answer the following questions.  Some questions can be answered on the
team level, but where appropriate, each team member should write their own
response:


\subsection*{1. What went well while writing this deliverable?} 
Collaboration was the key word with this deliverable. With how this document was intrinsically tied 
to pieces of the SRS (e.g. components, functional requirements), we each had to ensure our parts not only 
remained consistent across both documents, but who finished what when was also a concern. Thankfully, 
this was able to be sorted out. To ensure all members were on the same page about what was to be discussed, 
initial brainstorms from one member were received and implemented by another, whose work was finally 
judged by the first member. Suggestions of possible hazards were also taken not just from the group, but 
from discussions with the client, who verified our concerns. Despite our project not exhibiting traditional 
safety concerns (beyond the remote possibility of printer damage), possible hazards were nevertheless 
identified as they relate to user productivity, an essential facet of project success.

\subsection*{2. What pain points did you experience during this deliverable, and how
did you resolve them?}
\bigskip
\textbf{Omar: }Responsible for writing the \textit{Scope and Purpose} section, my main difficulty was distinguishing between what should be included in the hazard analysis versus what belonged in the SRS. It was challenging to define the boundaries of the system’s intent without repeating previously established details. Another challenge was ensuring that the scope was broad enough to cover all potential hazards, yet still focused on the system’s core functions. To resolve this, I revisited the SRS and refined my statements to ensure consistency while emphasizing how the defined purpose directly supported the hazard identification proce\newline
\newline
\textbf{Daniel: } Responsible for writing the System Boundaries, my main concern was finding a succinct way of explaining each component without simply restating what was already discussed in the SRS. Providing the necessary detail for the HA whilst also summarizing was an interesting balance to strike. Ultimately I settled on leaving out items not discussed within the analysis itself. In addition, when brainstorming possible hazards, tracing back possible causes when the system obviously isn't developed was a challenge. In the end, we realised that discussing causes more generally was sufficient in indicating a possible cause.
\newline
\newline
\textbf{Andrew: } I was responsible for writing the Critical Assumptions section. The main challenge I faced was ensuring that the assumptions didn't simply dismiss potential hazards by assuming failures would never occur. The template guidance emphasized that assumptions should not be used to remove hazards from consideration, so I needed to frame each assumption alongside its corresponding mitigation strategy through safety requirements. 
\newline
\newline
\textbf{Olivia: } I took on a larger section of the SRS document, so within the hazard analysis I only did the introduction. Consequently, my pain points weren’t as significant within this part of the deliverable. I would say that a very minor pain point would be having to introduce hazards in a way that reframes the concept beyond the typical, inherently physical perspective. Even as a software engineer, my first instinctual thought when I hear the word 'hazard' is to think of things that can cause observable, physical harm. I think it's human nature to have greater fear of harm when a hazard is more tangible and apparent in nature because it feels more immediate and concrete. To address this issue, I made sure that the hazard introduction effectively introduced software hazards by including examples that shift the tone and direction towards recognizing damaging, non-physical hazards that are inherent to software systems.
\newline
\newline
\textbf{Khalid: } I was primarly responsible on the Failure Mode and Effects Analysis (FMEA) for this deliverable. This involved identify any potentail failure modes with the system that we have not thought about during the requirements gathering process. This included analyzing their causes and effects, and assessing their severity and likelihood. The challenge that I had during this process was identifying any safety requirements or hazards. This is because our system is a software tool that doesn't deal with any physical harm to a user, so we had to think more abstractly about the hazards. The hazards that I came up where more directly towards user productivity, data corruption, loss of work, or connectivity issues with the external systems.

\subsection*{3. Which of your listed risks had your team thought of before this
deliverable, and which did you think of while doing this deliverable? For
the latter ones (ones you thought of while doing the Hazard Analysis), how
did they come about?}
The major risks already identified were (which were refined in this analysis):
\begin{enumerate}
    \item System not processing the file correctly. This was considered during work on the development plan, as it is the starting point of the workflow.
    \item Performance degradation from selecting too many voxels simultaneously. This risk was identified during the development plan as a potential risk, and was refined in this analysis.
    \item System not providing an undo/redo functionality to combat the risk of unintended edits being made to the model. This is a risk that we identified after meeting with the supervisor and discussing the system's requirements.
    \item System not saving user progress. This is a risk that was thought of during the development plan, as we discussed how the system should handle user progress by autosaving.
    \item System validating the material and magnetization properties before exporting the file. This is a risk that was thought of during the development plan, when we discussed the requirements for the system with the supervisor.
\end{enumerate}
The risks that were thought of while doing this deliverable:
\begin{enumerate}
    \item The device does not have enough memory or GPU capabilities to render 3D models and voxel grids. We came up with this risk after researching how much 3D imaging requires memory and how the GPU driver can affect the rendering process.
    \item The system is not maintaining real-time synchronization between the UI and backend data model. We came up with this risk after noticing how this could lead to unintended edits being made to the model and cause the backend data to potentially become corrupted.
    \item Long-running operations causing user frustration due to lack of progress feedback. We came up with risk after learning about feedback in the SFRWENG 4HC3 course, where we learned how the Norman Principle of Feedback is important for user experience.
    \item System not exporting the file in the correct format. We came up with this risk after exploring how the Java program that reads the metadata file can crash if the file is not in the correct format.
\end{enumerate}

\subsection*{4. Other than the risk of physical harm (some projects may not have any
appreciable risks of this form), list at least 2 other types of risk in
software products. Why are they important to consider?}
The terms listed here are general suggestions as areas of risk in software products.
\begin{enumerate}
    \item "Usability harm." This includes any hazard that might harm the user's experience, satisfaction, or productivity when using the product, and can arise from faulty user interfaces or common errors that the user must find workarounds for. This is important to consider as if the product's usability is a substantial decrease versus another solution, users will be encouraged to find alternate products that accomplish their goals without these pain points. In the context of our project, many of the risks identified relate to this concept (e.g. progress loss).
    \item "Virtual harm." Whereas physical harm relates to humans being harmed, virtual harm involves harm to files, programs, and posssibly hardware. This is important to consider as files can be just as precious as human safety, especially in the case where no backups exist. This can also factor into usability harm; if these errors are common, users can completely lose trust in the product. In our project, this manifests as both incorrect edits being made to user files, as well as complete progress loss.
\end{enumerate}


\section*{References}

[1]  C. C. for O. H. and S. Government of Canada, “CCOHS: Hazard and Risk - Hazard Identification.”
Accessed: Oct. 08, 2025. [Online]. Available: https://www.ccohs.ca/oshanswers/hsprograms/hazard
/hazards\_identification.html

\end{document}
