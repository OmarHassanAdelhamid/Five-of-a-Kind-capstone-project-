\documentclass{article}

\usepackage{float}
\restylefloat{table}

\usepackage{booktabs}

\title{Team Contributions: POC\\\progname}

\author{\authname}

\date{}

\input{../Comments}
%% Common Parts

\newcommand{\progname}{AutoVox} % PUT YOUR PROGRAM NAME HERE
\newcommand{\authname}{Team \#10, Five of a Kind
\\ Omar Abdelhamid
\\ Daniel Maurer
\\ Andrew Bovbel
\\ Olivia Reich
\\ Khalid Farag
} % AUTHOR NAMES                  

\usepackage{hyperref}
    \hypersetup{colorlinks=true, linkcolor=blue, citecolor=blue, filecolor=blue,
                urlcolor=blue, unicode=false}
    \urlstyle{same}
                                


\begin{document}

\maketitle

This document summarizes the contributions of each team member up to the POC
Demo.  The time period of interest is the time between the beginning of the term
and the POC demo.

\section{Demo Plans}

For our proof of concept demonstration, we will showcase the core functionality of our 
voxel magnetization system for 3D printing. The demonstration will focus on the following key aspects:

\begin{enumerate}
    \item \textbf{Import and Voxelization:} We will demonstrate how our system imports CAD 
    files (STL/OBJ format) and converts them into a voxel grid with fixed dimensions, showing the file parsing and voxelization of a simple model.
    
    \item \textbf{3D Rendering:} We will demonstrate the 3D rendering of the voxelized model using Three.js, showing how users can view the model from different angles.
    
    \item \textbf{Export Functionality:} We will demosntrate how our system exports the voxel data (magnetization, material ID, location) to a CSV file format, demonstrating the file export capability for a simple model. Dummy data will be used for the export.
    
\end{enumerate}

\section{Team Meeting Attendance}
\iffalse
\wss{For each team member how many team meetings have they attended over the
time period of interest.  This number should be determined from the meeting
issues in the team's repo.  The first entry in the table should be the total
number of team meetings held by the team.}
\fi
\begin{table}[H]
\centering
\begin{tabular}{ll}
\toprule
\textbf{Student} & \textbf{Meetings}\\
\midrule
Total & 8\\
Omar Hassan & 7\\
Daniel Maurer & 8\\
Andrew Bovbel & 7\\
Olivia Reich & 8\\
Khalid Farag & 7\\
\bottomrule
\end{tabular}
\end{table}

%update these numbers after tomorrow!
We often try to schedule our meetings such that everyone can attend; if people missed, it is due to them not
able to attend for a previously discussed reason.

\section{Supervisor/Stakeholder Meeting Attendance}
\iffalse
\wss{For each team member how many supervisor/stakeholder team meetings have
they attended over the time period of interest.  This number should be determined
from the supervisor meeting issues in the team's repo.  The first entry in the
table should be the total number of supervisor and team meetings held by the
team.  If there is no supervisor, there will usually be meetings with
stakeholders (potential users) that can serve a similar purpose.}
\fi
\noindent \textbf{Supervisor's Name: } [Dr.Onaizah]

\begin{table}[H]
\centering
\begin{tabular}{ll}
\toprule
\textbf{Student} & \textbf{Meetings}\\
\midrule
Total & 3\\
Omar Hassan & 3\\
Daniel Maurer & 3\\
Andrew Bovbel & 3\\
Olivia Reich & 3\\
Khalid Farag & 3\\
\bottomrule
\end{tabular}
\end{table}

Scheduling meetings with the supervisor has been difficult given our schedules and Dr. Onaizah's schedule as well, 
however we do our best to remain in contact when possible.

\section{Lecture Attendance}
\iffalse
\wss{For each team member how many lectures have they attended over the time
period of interest.  This number should be determined from the lecture issues in
the team's repo. You can find the number of lectures in the time period of
interest by looking at the
\href{https://calendar.google.com/calendar/u/0/embed?src=rnboqiaki1k2la7rpu3bn0um58@group.calendar.google.com&ctz=America/Toronto}
{Google calendar} for the capstone course.}
\fi
\begin{table}[H]
\centering
\begin{tabular}{ll}
\toprule
\textbf{Student} & \textbf{Lectures}\\
\midrule
Total & 13\\
Omar Hassan & 9\\
Daniel Maurer & 13\\
Andrew Bovbel & 7\\
Olivia Reich & 9\\
Khalid Farag & 8\\
\bottomrule
\end{tabular}
\end{table}

We aim to have at least one person attending every lecture in order to be most up to date (ideally whoever can attend, attends).

\section{TA Document Discussion Attendance}
\iffalse
\wss{For each team member how many of the informal document discussion meetings
with the TA were attended over the time period of interest.}
\fi
\noindent \textbf{TA's Name: } [Chris Schankula]

\begin{table}[H]
\centering
\begin{tabular}{ll}
\toprule
\textbf{Student} & \textbf{Lectures}\\
\midrule
Total & 2\\
Omar Hassan & 2\\
Daniel Maurer & 2\\
Andrew Bovbel & 2\\
Olivia Reich & 2\\
Khalid Farag & 2\\
\bottomrule
\end{tabular}
\end{table}

Our team had fewer meetings with our TA due to his unavailability at the beginning of the term.

\section{Commits}
\iffalse
\wss{For each team member how many commits to the main branch have been made
over the time period of interest.  The total is the total number of commits for
the entire team since the beginning of the term.  The percentage is the
percentage of the total commits made by each team member.}
\fi
\begin{table}[H]
\centering
\begin{tabular}{lll}
\toprule
\textbf{Student} & \textbf{Commits} & \textbf{Percent}\\
\midrule
Total & 240 & 100\% \\
Omar Hassan & 16 & 6.7\%\\
Daniel Maurer & 81 & 33.7\%\\
Andrew Bovbel & 49 & 20.4\%\\
Olivia Reich & 35 & 14.6\%\\
Khalid Farag & 59 & 24.6\%\\
\bottomrule
\end{tabular}
\end{table}

\iffalse
\wss{If needed, an explanation for the counts can be provided here.  For
instance, if a team member has more commits to unmerged branches, these numbers
can be provided here.  If multiple people contribute to a commit, git allows for
multi-author commits.}
\fi

Some members have lower commits due to commit sizes being larger and less frequent (e.g. only commiting once an entire assigned
section is complete), and some have more due to commits being smaller and more frequent (this is especially true for hot-fix commits).

\section{Issue Tracker}
\iffalse
\wss{For each team member how many issues have they authored (including open and
closed issues (O+C)) and how many have they been assigned (only counting closed
issues (C only)) over the time period of interest.}
\fi
\begin{table}[H]
\centering
\begin{tabular}{lll}
\toprule
\textbf{Student} & \textbf{Authored (O+C)} & \textbf{Assigned (C only)}\\
\midrule
Omar Hassan & 0 & 12\\
Daniel Maurer & 52 & 12\\
Andrew Bovbel & 0 & 11\\
Olivia Reich & 0 & 12\\
Khalid Farag & 3 & 11\\
\bottomrule
\end{tabular}
\end{table}

The large discrepancy in issue authorship is due to the assumed roles within our group; assigned much clearly reflects
the work divisions.

\section{CICD}

This section explains the Continuous Integration and Continuous Deployment (CI/CD) plan for our project. The plan will help ensure consistent code quality, testing, and deployment processes.

\subsection{Source Control and Branching Strategy}

\begin{itemize}
    \item \textbf{Repository Setup}: Our team will use GitHub for version control and collaboration.
    
    \item \textbf{Branching Strategy}:
    \begin{itemize}
        \item \texttt{main}: The main branch will be used to store the production-ready code.
        \item \texttt{<name of team member>-<issue name>}: The temporary branches will be created for each team member to work on their assigned issues. These branches will be merged into the main branch after the task is completed.
    \end{itemize}
    
    \item \textbf{Merging Policy}: All pull requests should have at least two reviews before merging, as outlined in the Development Plan. This will help ensure that the code is of high quality and meets the requirements of the project.
\end{itemize}

\subsection{Testing and Code Analysis Pipeline}

Our team will use GitHub Actions for CI/CD to automate testing and code analysis on pull requests. They will include the following:

\begin{itemize}
    
    \item \textbf{Static Code Analysis \& Linting}
    \begin{itemize}
        \item \textbf{Frontend}: \texttt{ESLint} will be used to enforce React coding standards and identify potential issues.
        \item \textbf{Backend}: \texttt{PyLint} will be used to handle both code smells for static analysis and enforce the coding standards for the backend.
    \end{itemize}
    
    \item \textbf{Testing}
    \begin{itemize}
        \item \textbf{Frontend Unit Tests}: Unit tests will be written using \texttt{Jest} for React components and utility functions, the GitHub Actions will run the tests using the command `npm test`.
        \item \textbf{Backend Unit Tests}: Unit tests will be written using \texttt{pytest} for API endpoints and core functionality, the GitHub Actions will run the tests using the command `pytest`.
    \end{itemize}
    
    \item \textbf{Documentation}: LaTeX documentation files will be automatically compiled to PDF on commits to the \texttt{main} branch. The GitHub Actions will run the compilation using the command `make all`.
\end{itemize}

\section{Team Charter Trigger Items}

The triggers for team performance and accountability are as follows:

\begin{itemize}
    \item \textbf{Meeting Frequency:} Team meetings must be held at least once per week, with all members expected to attend unless otherwise specified.
    \item \textbf{Supervisor Updates:} The supervisor must be updated at least once every two weeks via email, unless questions arise that require more frequent communication.
    \item \textbf{Code Review Requirements:} Each pull request must be reviewed by at least two other team members before being merged into the main branch.
    \item \textbf{Performance Metrics:} Team member contributions are tracked using commits, meetings attended, and issues completed to evaluate performance.
    \item \textbf{Underperformance Criteria:} A team member is considered underperforming if they are not contributing to the project or are not meeting deadlines, which triggers a team meeting to address the issue.
\end{itemize}

Our team has experienced no violations of these triggers in the team charter.

%SAME HERE

\section{Additional Productivity Metrics}

Our team has no other productivity metrics in use.

\end{document}