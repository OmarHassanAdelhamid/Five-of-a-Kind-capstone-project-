\documentclass{article}

\usepackage{float}
\restylefloat{table}

\usepackage{booktabs}

\title{Team Contributions: Rev 0\\\progname}

\author{\authname}

\date{}

\input{../Comments}
%% Common Parts

\newcommand{\progname}{AutoVox} % PUT YOUR PROGRAM NAME HERE
\newcommand{\authname}{Team \#10, Five of a Kind
\\ Omar Abdelhamid
\\ Daniel Maurer
\\ Andrew Bovbel
\\ Olivia Reich
\\ Khalid Farag
} % AUTHOR NAMES                  

\usepackage{hyperref}
    \hypersetup{colorlinks=true, linkcolor=blue, citecolor=blue, filecolor=blue,
                urlcolor=blue, unicode=false}
    \urlstyle{same}
                                


\begin{document}

\maketitle

This document summarizes the contributions of each team member for the Rev 0
Demo.  The time period of interest is the time between the PoC demo and the Rev
0 demo; the contributions prior to the PoC are NOT included.

\section{Demo Plans}
\iffalse
\wss{What will you be demonstrating}
\fi

For our Rev 0 Demo, we aim to demonstrate the full functionality of our product, AutoVox.
In addition to the polished functionality shown in the POC demo, the demo will focus on:

\begin{itemize}
    \item \textbf{Refreshed UI:} We will showcase an improved UI that follows traditional principles of human-centred design, is organized logically and efficiently, and aligns with the non-functional requirements listed in the SRS.
    \item \textbf{Layer Editing Interface:} We will demonstrate the workflow of how voxels are edited within a layer of the voxel model, including layer selection, voxel selection, and property assignment.
    \item \textbf{Partitioning of models:} We will show how larger models are broken down into segments that are rendered separately, and swapping between them.
\end{itemize}

\section{Team Meeting Attendance}
\iffalse
\wss{For each team member how many team meetings have they attended over the
time period of interest.  This number should be determined from the meeting
issues in the team's repo.  The first entry in the table should be the total
number of team meetings held by the team.}
\fi

\begin{table}[H]
\centering
\begin{tabular}{ll}
\toprule
\textbf{Student} & \textbf{Meetings}\\
\midrule
Total & 10\\
Omar & 9\\
Daniel & 9\\
Andrew & 8\\
Olivia & 10\\
Khalid & 10\\
\bottomrule
\end{tabular}
\end{table}
\iffalse
\wss{If needed, an explanation for the counts can be provided here.}
\fi
We try to schedule meetings so that all can attend; if one has missed meetings, it is either discussed prior the meeting, or the meeting was not relevant to all members.

\section{Supervisor/Stakeholder Meeting Attendance}
\iffalse
\wss{For each team member how many supervisor/stakeholder team meetings have
they attended over the time period of interest.  This number should be determined
from the supervisor meeting issues in the team's repo.  The first entry in the
table should be the total number of supervisor and team meetings held by the
team.  If there is no supervisor, there will usually be meetings with
stakeholders (potential users) that can serve a similar purpose.}
\fi

\noindent \textbf{Supervisor's Name: } [Dr. Onaizah]

\begin{table}[H]
\centering
\begin{tabular}{ll}
\toprule
\textbf{Student} & \textbf{Meetings}\\
\midrule
Total & 1\\
Omar & 1\\
Daniel & 1\\
Andrew & 0\\
Olivia & 0\\
Khalid & 1\\
\bottomrule
\end{tabular}
\end{table}
\iffalse
\wss{If needed, an explanation for the counts can be provided here.}
\fi
Due to scheduling issues, our team has not been able to meet with Dr. Onaizah as frequently as we would like. During the winter term, she has become quite busy. We have sent requests to meet throughout the month of January and have spoken to her in passing, but as of the submission of this report we have not received a response to confirm a meeting. We aim to address this issue as soon as possible, including potentially reaching out to one of her graduate students to speak to as a proxy.

\section{Lecture Attendance}
\iffalse
\wss{For each team member how many lectures have they attended over the time
period of interest.  This number should be determined from the lecture issues in
the team's repo. You can find the number of lectures in the time period of
interest by looking at the
\href{https://calendar.google.com/calendar/u/0/embed?src=rnboqiaki1k2la7rpu3bn0um58@group.calendar.google.com&ctz=America/Toronto}
{Google calendar} for the capstone course.}
\fi

\begin{table}[H]
\centering
\begin{tabular}{ll}
\toprule
\textbf{Student} & \textbf{Lectures}\\
\midrule
Total & 2\\
Omar & 0\\
Daniel & 2\\
Andrew & 0\\
Olivia & 0\\
Khalid & 0\\
\bottomrule
\end{tabular}
\end{table}
\iffalse
\wss{If needed, an explanation for the lecture attendance can be provided here.}
\fi
We aim to always have one group member attend lectures in order to take notes, who is then responsible for updating the rest of the team.

\section{TA Document Discussion Attendance}
\iffalse
\wss{For each team member how many of the informal document discussion meetings
with the TA were attended over the time period of interest.}
\fi

\noindent \textbf{TA's Name: } [Chris Schankula]

\begin{table}[H]
\centering
\begin{tabular}{ll}
\toprule
\textbf{Student} & \textbf{Lectures}\\
\midrule
Total & 2\\
Omar & 2\\
Daniel & 2\\
Andrew & 2\\
Olivia & 2\\
Khalid & 2\\
\bottomrule
\end{tabular}
\end{table}
\iffalse
\wss{If needed, an explanation for the attendance can be provided here.}
\fi

\section{Commits}
\iffalse
\wss{For each team member how many commits to the main branch have been made
over the time period of interest.  The total is the total number of commits for
the entire team since the beginning of the term.  The percentage is the
percentage of the total commits made by each team member.}
\fi

\begin{table}[H]
\centering
\begin{tabular}{lll}
\toprule
\textbf{Student} & \textbf{Commits} & \textbf{Percent}\\
\midrule
Total & 118 & 100\% \\
Omar & 7 & 5.9\% \\
Daniel & 45 & 38.1\% \\
Andrew & 10 & 8.5\% \\
Olivia & 27 & 22.9\% \\
Khalid & 29 & 24.6\% \\
\bottomrule
\end{tabular}
\end{table}
\iffalse
\wss{If needed, an explanation for the counts can be provided here.  For
instance, if a team member has more commits to unmerged branches, these numbers
can be provided here.  If multiple people contribute to a commit, git allows for
multi-author commits.}
\fi
Some members have lower commits due to commit sizes being larger and less
frequent (e.g. only commiting once an entire assigned section is complete),
and some have more due to commits being smaller and more frequent (this is
especially true for hot-fix commits).

\section{Issue Tracker}
\iffalse
\wss{For each team member how many issues have they authored (including open and
closed issues (O+C)) and how many have they been assigned (only counting closed
issues (C only)) over the time period of interest.}
\fi

\begin{table}[H]
\centering
\begin{tabular}{lll}
\toprule
\textbf{Student} & \textbf{Authored (O+C)} & \textbf{Assigned (C only)}\\
\midrule
Omar & 2 & 18 \\
Daniel & 74 & 32 \\
Andrew & 0 & 2 \\
Olivia & 16 & 16 \\
Khalid & 4 & 3 \\
\bottomrule
\end{tabular}
\end{table}
\iffalse
\wss{If needed, an explanation for the counts can be provided here.}
\fi
The large discrepancy in issue authorship is due to the assumed roles within
our group; discrepancy in issue assignment/closed is due to group members forgetting
to link issues until after the fact. Additionally, a set of issues closed dealt with small fixes
from either peer review or TA feedback.

\section{CICD}
\iffalse
\wss{Say how CICD is used in your project}

\section{Team Charter Trigger Items}

\wss{Provide a summary of the quantified triggers identified in the team's
charter.}

\wss{Provide a list of any violations of the triggers.  If the team wishes, the
violations can be summarized on aggregate, instead of naming specific team
members.}

\wss{Provide a plan to address the violations.  This could include revising the
triggers, if they are found to be too weak, strong or ambiguous.}
\fi

This section explains the Continuous Integration and Continuous Deployment (CI/CD) plan that has been utilized so far in our project.

\subsection{Source Control and Branching Strategy}

\begin{itemize}
    \item \textbf{Repository Setup}: Our team has used GitHub for version control and collaboration.
    
    \item \textbf{Branching Strategy}:
    \begin{itemize}
        \item \texttt{main}: The main branch used to store the production-ready code.
        \item \texttt{<name of team member>-<issue name>}: The temporary branches created for each team member to work on their assigned issues. These branches are merged into the main branch after the task is completed.
    \end{itemize}
    
    \item \textbf{Merging Policy}: All pull requests have at least two reviews before merging, as outlined in the Development Plan. This helps ensure that the code is of high quality and meets the requirements of the project.
\end{itemize}

\subsection{Testing and Code Analysis Pipeline}

Our team uses GitHub Actions for CI/CD to automate testing and code analysis on pull requests. They include the following:

\begin{itemize}
    
    \item \textbf{Static Code Analysis \& Linting}
    \begin{itemize}
        \item \textbf{Frontend}: \texttt{ESLint} is used to enforce React coding standards and identify potential issues.
        \item \textbf{Backend}: \texttt{PyLint} is used to handle both code smells for static analysis and enforce the coding standards for the backend.
    \end{itemize}
    
    \item \textbf{Testing}
    \begin{itemize}
        \item \textbf{Frontend Unit Tests}: Unit tests are written using \texttt{Jest} for React components and utility functions, the GitHub Actions run the tests using the command `npm test`.
        \item \textbf{Backend Unit Tests}: Unit tests are written using \texttt{pytest} for API endpoints and core functionality, the GitHub Actions run the tests using the command `pytest`.
    \end{itemize}
    
    \item \textbf{Documentation}: LaTeX documentation files are automatically compiled to PDF on commits to the \texttt{main} branch. The GitHub Actions run the compilation using the command `make all`.
\end{itemize}

\section{Team Charter Trigger Items}

The triggers for team performance and accountability are as follows:

\begin{itemize}
    \item \textbf{Meeting Frequency:} Team meetings must be held at least once per week, with all members expected to attend unless otherwise specified.
    \item \textbf{Supervisor Updates:} The supervisor must be updated at least once every two weeks via email, unless questions arise that require more frequent communication.
    \item \textbf{Code Review Requirements:} Each pull request must be reviewed by at least two other team members before being merged into the main branch.
    \item \textbf{Performance Metrics:} Team member contributions are tracked using commits, meetings attended, and issues completed to evaluate performance.
    \item \textbf{Underperformance Criteria:} A team member is considered underperforming if they are not contributing to the project or are not meeting deadlines, which triggers a team meeting to address the issue.
\end{itemize}

Our team has experienced no violations of these triggers in the team charter.

\section{Additional Productivity Metrics}
\iffalse
\wss{If your team has additional metrics of productivity, please feel free to
add them to this report.}
\fi

Our team has no other productivity metrics in use.

\end{document}