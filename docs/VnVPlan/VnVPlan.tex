\documentclass[12pt, titlepage]{article}

\usepackage{booktabs}
\usepackage{tabularx}
\usepackage{hyperref}
\hypersetup{
    colorlinks,
    citecolor=blue,
    filecolor=black,
    linkcolor=red,
    urlcolor=blue
}
\usepackage[round]{natbib}

%% Comments

\usepackage{color}

\newif\ifcomments\commentstrue %displays comments
%\newif\ifcomments\commentsfalse %so that comments do not display

\ifcomments
\newcommand{\authornote}[3]{\textcolor{#1}{[#3 ---#2]}}
\newcommand{\todo}[1]{\textcolor{red}{[TODO: #1]}}
\else
\newcommand{\authornote}[3]{}
\newcommand{\todo}[1]{}
\fi

\newcommand{\wss}[1]{\authornote{magenta}{SS}{#1}} 
\newcommand{\plt}[1]{\authornote{cyan}{TPLT}{#1}} %For explanation of the template
\newcommand{\an}[1]{\authornote{cyan}{Author}{#1}}

%% Common Parts

\newcommand{\progname}{Software Engineering} % PUT YOUR PROGRAM NAME HERE
\newcommand{\authname}{Team \#10, Five of a Kind
\\ Omar Abdelhamid
\\ Daniel Maurer
\\ Andrew Bovbel
\\ Olivia Reich
\\ Khalid Farag
} % AUTHOR NAMES                  

\usepackage{hyperref}
    \hypersetup{colorlinks=true, linkcolor=blue, citecolor=blue, filecolor=blue,
                urlcolor=blue, unicode=false}
    \urlstyle{same}
                                


\begin{document}

\title{System Verification and Validation Plan for \progname{}} 
\author{\authname}
\date{\today}
	
\maketitle

\pagenumbering{roman}

\section*{Revision History}

\begin{tabularx}{\textwidth}{p{3cm}p{2cm}X}
\toprule {\bf Date} & {\bf Version} & {\bf Notes}\\
\midrule
Date 1 & 1.0 & Notes\\
Date 2 & 1.1 & Notes\\
\bottomrule
\end{tabularx}

~\\
\wss{The intention of the VnV plan is to increase confidence in the software.
However, this does not mean listing every verification and validation technique
that has ever been devised.  The VnV plan should also be a \textbf{feasible}
plan. Execution of the plan should be possible with the time and team available.
If the full plan cannot be completed during the time available, it can either be
modified to ``fake it'', or a better solution is to add a section describing
what work has been completed and what work is still planned for the future.}

\wss{The VnV plan is typically started after the requirements stage, but before
the design stage.  This means that the sections related to unit testing cannot
initially be completed.  The sections will be filled in after the design stage
is complete.  the final version of the VnV plan should have all sections filled
in.}

\newpage

\tableofcontents

\listoftables
\wss{Remove this section if it isn't needed}

\listoffigures
\wss{Remove this section if it isn't needed}

\newpage

\section{Symbols, Abbreviations, and Acronyms}

\renewcommand{\arraystretch}{1.2}
\begin{tabular}{l l} 
  \toprule		
  \textbf{symbol} & \textbf{description}\\
  \midrule 
  T & Test\\
  \bottomrule
\end{tabular}\\

\wss{symbols, abbreviations, or acronyms --- you can simply reference the SRS
  \citep{SRS} tables, if appropriate}

\wss{Remove this section if it isn't needed}

\newpage

\pagenumbering{arabic}

This document ... \wss{provide an introductory blurb and roadmap of the
  Verification and Validation plan}

\section{General Information}

\subsection{Summary}

\wss{Say what software is being tested.  Give its name and a brief overview of
  its general functions.}

\subsection{Objectives}

\wss{State what is intended to be accomplished.  The objective will be around
  the qualities that are most important for your project.  You might have
  something like: ``build confidence in the software correctness,''
  ``demonstrate adequate usability.'' etc.  You won't list all of the qualities,
  just those that are most important.}

\wss{You should also list the objectives that are out of scope.  You don't have 
the resources to do everything, so what will you be leaving out.  For instance, 
if you are not going to verify the quality of usability, state this.  It is also 
worthwhile to justify why the objectives are left out.}

\wss{The objectives are important because they highlight that you are aware of 
limitations in your resources for verification and validation.  You can't do everything, 
so what are you going to prioritize?  As an example, if your system depends on an 
external library, you can explicitly state that you will assume that external library 
has already been verified by its implementation team.}

\subsection{Challenge Level and Extras}

\wss{State the challenge level (advanced, general, basic) for your project.
Your challenge level should exactly match what is included in your problem
statement.  This should be the challenge level agreed on between you and the
course instructor.  You can use a pull request to update your challenge level
(in TeamComposition.csv or Repos.csv) if your plan changes as a result of the
VnV planning exercise.}

\wss{Summarize the extras (if any) that were tackled by this project.  Extras
can include usability testing, code walkthroughs, user documentation, formal
proof, GenderMag personas, Design Thinking, etc.  Extras should have already
been approved by the course instructor as included in your problem statement.
You can use a pull request to update your extras (in TeamComposition.csv or
Repos.csv) if your plan changes as a result of the VnV planning exercise.}

\subsection{Relevant Documentation}

\wss{Reference relevant documentation.  This will definitely include your SRS
  and your other project documents (design documents, like MG, MIS, etc).  You
  can include these even before they are written, since by the time the project
  is done, they will be written.  You can create BibTeX entries for your
  documents and within those entries include a hyperlink to the documents.}

\citet{SRS}

\wss{Don't just list the other documents.  You should explain why they are relevant and 
how they relate to your VnV efforts.}

\section{Plan}

\wss{Introduce this section.  You can provide a roadmap of the sections to
  come.}

\subsection{Verification and Validation Team}

\wss{Your teammates.  Maybe your supervisor.
  You should do more than list names.  You should say what each person's role is
  for the project's verification.  A table is a good way to summarize this information.}

\subsection{SRS Verification}

\wss{List any approaches you intend to use for SRS verification.  This may
  include ad hoc feedback from reviewers, like your classmates (like your
  primary reviewer), or you may plan for something more rigorous/systematic.}

\wss{If you have a supervisor for the project, you shouldn't just say they will
read over the SRS.  You should explain your structured approach to the review.
Will you have a meeting?  What will you present?  What questions will you ask?
Will you give them instructions for a task-based inspection?  Will you use your
issue tracker?}

\wss{Maybe create an SRS checklist?}

\subsection{Design Verification}

\wss{Plans for design verification}

\wss{The review will include reviews by your classmates}

\wss{Create a checklists?}

\subsection{Verification and Validation Plan Verification}

\wss{The verification and validation plan is an artifact that should also be
verified.  Techniques for this include review and mutation testing.}

\wss{The review will include reviews by your classmates}

\wss{Create a checklists?}

\subsection{Implementation Verification}

\wss{You should at least point to the tests listed in this document and the unit
  testing plan.}

\wss{In this section you would also give any details of any plans for static
  verification of the implementation.  Potential techniques include code
  walkthroughs, code inspection, static analyzers, etc.}

\wss{The final class presentation in CAS 741 could be used as a code
walkthrough.  There is also a possibility of using the final presentation (in
CAS741) for a partial usability survey.}

\subsection{Automated Testing and Verification Tools}

\wss{What tools are you using for automated testing.  Likely a unit testing
  framework and maybe a profiling tool, like ValGrind.  Other possible tools
  include a static analyzer, make, continuous integration tools, test coverage
  tools, etc.  Explain your plans for summarizing code coverage metrics.
  Linters are another important class of tools.  For the programming language
  you select, you should look at the available linters.  There may also be tools
  that verify that coding standards have been respected, like flake9 for
  Python.}

\wss{If you have already done this in the development plan, you can point to
that document.}

\wss{The details of this section will likely evolve as you get closer to the
  implementation.}

\subsection{Software Validation}

\wss{If there is any external data that can be used for validation, you should
  point to it here.  If there are no plans for validation, you should state that
  here.}

\wss{You might want to use review sessions with the stakeholder to check that
the requirements document captures the right requirements.  Maybe task based
inspection?}

\wss{For those capstone teams with an external supervisor, the Rev 0 demo should 
be used as an opportunity to validate the requirements.  You should plan on 
demonstrating your project to your supervisor shortly after the scheduled Rev 0 demo.  
The feedback from your supervisor will be very useful for improving your project.}

\wss{For teams without an external supervisor, user testing can serve the same purpose 
as a Rev 0 demo for the supervisor.}

\wss{This section might reference back to the SRS verification section.}

\section{System Tests}

This section outlines the tests for verifying functional requirements
of the software, ensuring it meets user expectations and performs reliably.
This includes tests covering essential aspects of the software's operation
and compliance.

\subsection{Tests for Functional Requirements}

The subsections below outline tests corresponding to functional
requirements in the SRS. Each test is associated with a
unique functional area, helping to confirm that the tool meets the
specified requirements. Each functional area has its own subsection for clarity.

\noindent
\rule{\textwidth}{0.5pt}

\subsubsection{Editing Manager Tests}
\rule{\textwidth}{0.5pt}

\medskip

\noindent
This section covers the tests for ensuring the Editing Manager correctly
handles magnetization and material assignment operations, property management,
and user interface interactions as specified in requirements \textbf{F231} through \textbf{F2310}.

\begin{enumerate}
  \item \textbf{test-FR-EM-1 Individual Voxel Magnetization Assignment} \\[2mm]
    \textbf{Control:} Manual \\ 
    \textbf{Initial State:} 3D model loaded with voxels visible, Editing Manager active. \\ 
    \textbf{Input:} User selects a single voxel and assigns a magnetization vector (x=1.0, y=0.0, z=0.0). \\ 
    \textbf{Output:} The selected voxel displays the assigned magnetization vector and updates its visual representation. \\[2mm]
    \textbf{Test Case Derivation:} Confirming that the system correctly processes individual voxel magnetization assignment, as specified in \textbf{F231}. \\[2mm]
    \textbf{How test will be performed:} Load a 3D model, select a voxel using the interface, input magnetization values, and verify the assignment is recorded and visually displayed.

  \item \textbf{test-FR-EM-2 Group Voxel Magnetization Assignment} \\[2mm]
    \textbf{Control:} Manual \\ 
    \textbf{Initial State:} 3D model loaded with multiple voxels visible. \\ 
    \textbf{Input:} User selects multiple voxels and assigns the same magnetization vector to all selected voxels. \\ 
    \textbf{Output:} All selected voxels display the assigned magnetization vector and update their visual representation. \\[2mm]
    \textbf{Test Case Derivation:} Ensures the system can handle bulk magnetization assignment operations, satisfying \textbf{F231}. \\[2mm]
    \textbf{How test will be performed:} Select multiple voxels using the interface, assign magnetization values, and verify all selected voxels receive the assignment.

  \item \textbf{test-FR-EM-3 Favourite Magnetization Vector Management} \\[2mm]
    \textbf{Control:} Manual \\ 
    \textbf{Initial State:} Editing Manager with favourite bar accessible. \\ 
    \textbf{Input:} User creates a favourite magnetization vector (x=0.5, y=0.5, z=0.0) and saves it to the favourite bar. \\ 
    \textbf{Output:} The magnetization vector is added to the favourite bar and can be selected for future use. \\[2mm]
    \textbf{Test Case Derivation:} Validates that users can maintain and reuse frequently used magnetization vectors, per \textbf{F232}. \\[2mm]
    \textbf{How test will be performed:} Create a favourite magnetization vector, save it to the favourite bar, and verify it can be selected and applied to voxels.

  \item \textbf{test-FR-EM-4 Individual Voxel Material Assignment} \\[2mm]
    \textbf{Control:} Manual \\ 
    \textbf{Initial State:} 3D model loaded with voxels visible. \\ 
    \textbf{Input:} User selects a single voxel and assigns material ID 2. \\ 
    \textbf{Output:} The selected voxel displays the assigned material and updates its visual representation. \\[2mm]
    \textbf{Test Case Derivation:} Confirming that the system correctly processes individual voxel material assignment, as specified in \textbf{F233}. \\[2mm]
    \textbf{How test will be performed:} Select a voxel, assign a material ID, and verify the assignment is recorded and visually displayed.

  \item \textbf{test-FR-EM-5 Group Voxel Material Assignment} \\[2mm]
    \textbf{Control:} Manual \\ 
    \textbf{Initial State:} 3D model loaded with multiple voxels visible. \\ 
    \textbf{Input:} User selects multiple voxels and assigns material ID 3 to all selected voxels. \\ 
    \textbf{Output:} All selected voxels display the assigned material and update their visual representation. \\[2mm]
    \textbf{Test Case Derivation:} Ensures the system can handle bulk material assignment operations, satisfying \textbf{F233}. \\[2mm]
    \textbf{How test will be performed:} Select multiple voxels, assign a material ID, and verify all selected voxels receive the assignment.

  \item \textbf{test-FR-EM-6 Material Label Assignment} \\[2mm]
    \textbf{Control:} Manual \\ 
    \textbf{Initial State:} Material assignment interface accessible. \\ 
    \textbf{Input:} User assigns label "Steel" to material ID 1. \\ 
    \textbf{Output:} Material ID 1 is now associated with the label "Steel" in the interface. \\[2mm]
    \textbf{Test Case Derivation:} Validates that users can assign meaningful labels to material IDs for better organization, per \textbf{F234}. \\[2mm]
    \textbf{How test will be performed:} Access material management interface, assign a label to a material ID, and verify the label is displayed and associated correctly.

  \item \textbf{test-FR-EM-7 Property Replication Across Layers} \\[2mm]
    \textbf{Control:} Manual \\ 
    \textbf{Initial State:} 3D model with multiple layers loaded. \\ 
    \textbf{Input:} User selects voxels in one layer with assigned properties and replicates them to another layer. \\ 
    \textbf{Output:} The target layer voxels receive the same material and magnetization properties as the source voxels. \\[2mm]
    \textbf{Test Case Derivation:} Ensures efficient workflow by allowing property replication across layers, as specified in \textbf{F235}. \\[2mm]
    \textbf{How test will be performed:} Select voxels with assigned properties in one layer, use the replication function to copy properties to another layer, and verify the properties are correctly applied.

  \item \textbf{test-FR-EM-8 Auto Save Functionality} \\[2mm]
    \textbf{Control:} Automated \\ 
    \textbf{Initial State:} Model with unsaved changes. \\ 
    \textbf{Input:} System automatically saves changes after a property assignment. \\ 
    \textbf{Output:} Changes are saved without user intervention and confirmation is displayed. \\[2mm]
    \textbf{Test Case Derivation:} Validates that the system automatically preserves user work without manual intervention, per \textbf{F236}. \\[2mm]
    \textbf{How test will be performed:} Make property assignments, verify auto-save triggers, and confirm changes are preserved in the project file.

  \item \textbf{test-FR-EM-9 Edit History Access and Revert} \\[2mm]
    \textbf{Control:} Manual \\ 
    \textbf{Initial State:} Model with multiple edit operations completed. \\ 
    \textbf{Input:} User accesses edit history and reverts to a previous version. \\ 
    \textbf{Output:} Model returns to the selected previous state with all subsequent changes undone. \\[2mm]
    \textbf{Test Case Derivation:} Ensures users can recover from mistakes by accessing edit history, as specified in \textbf{F237}. \\[2mm]
    \textbf{How test will be performed:} Perform multiple edits, access edit history, select a previous version, and verify the model reverts correctly.

  \item \textbf{test-FR-EM-10 Entire Layer Selection and Assignment} \\[2mm]
    \textbf{Control:} Manual \\ 
    \textbf{Initial State:} 3D model with multiple layers loaded. \\ 
    \textbf{Input:} User selects an entire layer and assigns common material and magnetization properties. \\ 
    \textbf{Output:} All voxels in the selected layer receive the assigned properties. \\[2mm]
    \textbf{Test Case Derivation:} Validates efficient bulk operations by allowing entire layer selection, per \textbf{F238}. \\[2mm]
    \textbf{How test will be performed:} Select an entire layer, assign material and magnetization properties, and verify all voxels in the layer receive the assignments.

  \item \textbf{test-FR-EM-11 Manual Voxel Addition} \\[2mm]
    \textbf{Control:} Manual \\ 
    \textbf{Initial State:} 3D model loaded with defined voxel dimensions. \\ 
    \textbf{Input:} User adds a new voxel with the same dimensions as existing voxels. \\ 
    \textbf{Output:} New voxel is added to the model with unassigned properties. \\[2mm]
    \textbf{Test Case Derivation:} Ensures users can manually add voxels for model adjustments, as specified in \textbf{F239}. \\[2mm]
    \textbf{How test will be performed:} Use the voxel addition tool to add a new voxel, verify it has correct dimensions, and confirm it starts with unassigned properties.

  \item \textbf{test-FR-EM-12 Manual Voxel Deletion} \\[2mm]
    \textbf{Control:} Manual \\ 
    \textbf{Initial State:} 3D model with multiple voxels. \\ 
    \textbf{Input:} User selects and deletes a voxel. \\ 
    \textbf{Output:} Selected voxel is removed from the model. \\[2mm]
    \textbf{Test Case Derivation:} Ensures users can manually remove voxels for model adjustments, as specified in \textbf{F239}. \\[2mm]
    \textbf{How test will be performed:} Select a voxel, use the deletion tool, and verify the voxel is removed from the model.

  \item \textbf{test-FR-EM-13 Individual Voxel Property Reset} \\[2mm]
    \textbf{Control:} Manual \\ 
    \textbf{Initial State:} Voxel with assigned material and magnetization properties. \\ 
    \textbf{Input:} User selects the voxel and resets its properties. \\ 
    \textbf{Output:} Voxel returns to unassigned state for both material and magnetization. \\[2mm]
    \textbf{Test Case Derivation:} Validates that users can reset individual voxel properties, per \textbf{F2310}. \\[2mm]
    \textbf{How test will be performed:} Select a voxel with assigned properties, use the reset function, and verify the voxel returns to unassigned state.

  \item \textbf{test-FR-EM-14 Group Voxel Property Reset} \\[2mm]
    \textbf{Control:} Manual \\ 
    \textbf{Initial State:} Multiple voxels with assigned properties. \\ 
    \textbf{Input:} User selects multiple voxels and resets their properties. \\ 
    \textbf{Output:} All selected voxels return to unassigned state. \\[2mm]
    \textbf{Test Case Derivation:} Ensures users can reset multiple voxel properties simultaneously, per \textbf{F2310}. \\[2mm]
    \textbf{How test will be performed:} Select multiple voxels with assigned properties, use the reset function, and verify all selected voxels return to unassigned state.


\end{enumerate}

\noindent\rule{\textwidth}{0.5pt}

\subsubsection{Export Manager Tests}
\rule{\textwidth}{0.5pt}

\medskip

\noindent
This section covers the tests for ensuring the Export Manager correctly
handles file export operations, property validation, and data integrity
as specified in requirements \textbf{F241} through \textbf{F245}.

\begin{enumerate}
  \item \textbf{test-FR-XM-1 Property Validation for Complete Magnetization} \\[2mm]
    \textbf{Control:} Automated \\ 
    \textbf{Initial State:} Model with all voxels having assigned magnetization values. \\ 
    \textbf{Input:} User requests export for printing. \\ 
    \textbf{Output:} System validates all voxels have magnetization assignments and proceeds with export. \\[2mm]
    \textbf{Test Case Derivation:} Confirming that the system validates complete magnetization before export, as specified in \textbf{F241}. \\[2mm]
    \textbf{How test will be performed:} Load a model with complete magnetization assignments, request export, and verify validation passes and export proceeds.

  \item \textbf{test-FR-XM-2 Property Validation for Incomplete Magnetization} \\[2mm]
    \textbf{Control:} Automated \\ 
    \textbf{Initial State:} Model with some voxels missing magnetization assignments. \\ 
    \textbf{Input:} User requests export for printing. \\ 
    \textbf{Output:} System identifies unassigned voxels and prompts user to complete assignments or assign null values. \\[2mm]
    \textbf{Test Case Derivation:} Ensures the system prevents export of incomplete models, per \textbf{F241}. \\[2mm]
    \textbf{How test will be performed:} Load a model with incomplete magnetization, request export, and verify the system identifies missing assignments and provides appropriate feedback.

  \item \textbf{test-FR-XM-3 Standalone File Export} \\[2mm]
    \textbf{Control:} Manual \\ 
    \textbf{Initial State:} Model with complete property assignments ready for export. \\ 
    \textbf{Input:} User exports model with custom filename "test\_model.vox". \\ 
    \textbf{Output:} Standalone file containing all voxel metadata is created and saved locally. \\[2mm]
    \textbf{Test Case Derivation:} Validates that the system can produce complete export files, as specified in \textbf{F242}. \\[2mm]
    \textbf{How test will be performed:} Export a model with a custom filename, verify the file is created with all metadata, and confirm it can be accessed outside the software.

  \item \textbf{test-FR-XM-4 Project Export in Native Format} \\[2mm]
    \textbf{Control:} Manual \\ 
    \textbf{Initial State:} Project with model and all associated data. \\ 
    \textbf{Input:} User exports project with custom filename "backup\_project.fok". \\ 
    \textbf{Output:} Native format project file is created and saved locally with all project data preserved. \\[2mm]
    \textbf{Test Case Derivation:} Ensures users can create project backups and iterations, per \textbf{F243}. \\[2mm]
    \textbf{How test will be performed:} Export a project in native format, verify the file contains all project data, and confirm it can be imported back into the software.

  \item \textbf{test-FR-XM-5 Export Progress Tracking} \\[2mm]
    \textbf{Control:} Manual \\ 
    \textbf{Initial State:} Large model ready for export. \\ 
    \textbf{Input:} User initiates export operation. \\ 
    \textbf{Output:} Progress bar displays export progress with visual indicators. \\[2mm]
    \textbf{Test Case Derivation:} Validates that users receive feedback during export operations, as specified in \textbf{F244}. \\[2mm]
    \textbf{How test will be performed:} Export a large model, observe the progress bar updates, and verify the progress indicators accurately reflect export status.

  \item \textbf{test-FR-XM-6 Model Summary Export} \\[2mm]
    \textbf{Control:} Manual \\ 
    \textbf{Initial State:} Model with complete property assignments. \\ 
    \textbf{Input:} User exports model with summary option enabled. \\ 
    \textbf{Output:} Model file and summary file are created with statistics about the model. \\[2mm]
    \textbf{Test Case Derivation:} Ensures users can access model statistics for analysis, per \textbf{F245}. \\[2mm]
    \textbf{How test will be performed:} Export a model with summary enabled, verify both files are created, and confirm the summary contains relevant model statistics.

  \item \textbf{test-FR-XM-7 Export Failure Recovery} \\[2mm]
    \textbf{Control:} Automated \\ 
    \textbf{Initial State:} Model ready for export. \\ 
    \textbf{Input:} Export operation is interrupted (simulated system failure). \\ 
    \textbf{Output:} Original project data remains intact and unaltered. \\[2mm]
    \textbf{Test Case Derivation:} Validates that export failures do not corrupt project data, per \textbf{F241}. \\[2mm]
    \textbf{How test will be performed:} Initiate export, simulate failure during the process, and verify the original project file remains unchanged and accessible.

  \item \textbf{test-FR-XM-8 Export Performance Validation} \\[2mm]
    \textbf{Control:} Automated \\ 
    \textbf{Initial State:} Model with MAX\_VOXELS voxels ready for export. \\ 
    \textbf{Input:} User initiates export operation. \\ 
    \textbf{Output:} Export completes at minimum rate of MIN\_RATE (500,000 voxels per second). \\[2mm]
    \textbf{Test Case Derivation:} Ensures export performance meets specified thresholds, per \textbf{F241}. \\[2mm]
    \textbf{How test will be performed:} Export a large model, measure the export rate, and verify it meets or exceeds the MIN\_RATE threshold.

\end{enumerate}

\noindent\rule{\textwidth}{0.5pt}


\subsection{Traceability Between Test Cases and Requirements}

The following tables show the traceability between test cases and their corresponding functional requirements.

\begin{table}[h!]
  \centering
  \caption{Editing Manager Tests and Corresponding Requirements}
  \begin{tabular}{c c}
    \hline
    \textbf{Test ID (test-)} & \textbf{Requirement and/or Relevant Documentation} \\ \hline
    FR-EM-1 & F231\\
    FR-EM-2 & F231\\
    FR-EM-3 & F232\\
    FR-EM-4 & F233\\
    FR-EM-5 & F233\\
    FR-EM-6 & F234\\
    FR-EM-7 & F235\\
    FR-EM-8 & F236\\
    FR-EM-9 & F237\\
    FR-EM-10 & F238\\
    FR-EM-11 & F239\\
    FR-EM-12 & F239\\
    FR-EM-13 & F2310\\
    FR-EM-14 & F2310\\
  \end{tabular}
\end{table}

\begin{table}[h!]
  \centering
  \caption{Export Manager Tests and Corresponding Requirements}
  \begin{tabular}{c c}
    \hline
    \textbf{Test ID (test-)} & \textbf{Requirement and/or Relevant Documentation} \\ \hline
    FR-XM-1 & F241\\
    FR-XM-2 & F241\\
    FR-XM-3 & F242\\
    FR-XM-4 & F243\\
    FR-XM-5 & F244\\
    FR-XM-6 & F245\\
    FR-XM-7 & F241\\
    FR-XM-8 & F241\\
  \end{tabular}
\end{table}

\section{Unit Test Description}

\wss{This section should not be filled in until after the MIS (detailed design
  document) has been completed.}

\wss{Reference your MIS (detailed design document) and explain your overall
philosophy for test case selection.}  

\wss{To save space and time, it may be an option to provide less detail in this section.  
For the unit tests you can potentially layout your testing strategy here.  That is, you 
can explain how tests will be selected for each module.  For instance, your test building 
approach could be test cases for each access program, including one test for normal behaviour 
and as many tests as needed for edge cases.  Rather than create the details of the input 
and output here, you could point to the unit testing code.  For this to work, you code 
needs to be well-documented, with meaningful names for all of the tests.}

\subsection{Unit Testing Scope}

\wss{What modules are outside of the scope.  If there are modules that are
  developed by someone else, then you would say here if you aren't planning on
  verifying them.  There may also be modules that are part of your software, but
  have a lower priority for verification than others.  If this is the case,
  explain your rationale for the ranking of module importance.}

\subsection{Tests for Functional Requirements}

\wss{Most of the verification will be through automated unit testing.  If
  appropriate specific modules can be verified by a non-testing based
  technique.  That can also be documented in this section.}

\subsubsection{Module 1}

\wss{Include a blurb here to explain why the subsections below cover the module.
  References to the MIS would be good.  You will want tests from a black box
  perspective and from a white box perspective.  Explain to the reader how the
  tests were selected.}

\begin{enumerate}

\item{test-id1\\}

Type: \wss{Functional, Dynamic, Manual, Automatic, Static etc. Most will
  be automatic}
					
Initial State: 
					
Input: 
					
Output: \wss{The expected result for the given inputs}

Test Case Derivation: \wss{Justify the expected value given in the Output field}

How test will be performed: 
					
\item{test-id2\\}

Type: \wss{Functional, Dynamic, Manual, Automatic, Static etc. Most will
  be automatic}
					
Initial State: 
					
Input: 
					
Output: \wss{The expected result for the given inputs}

Test Case Derivation: \wss{Justify the expected value given in the Output field}

How test will be performed: 

\item{...\\}
    
\end{enumerate}

\subsubsection{Module 2}

...


\subsection{Traceability Between Test Cases and Modules}

\wss{Provide evidence that all of the modules have been considered.}
				
\bibliographystyle{plainnat}

\bibliography{../../refs/References}

\newpage

\section{Appendix}

This is where you can place additional information.

\subsection{Symbolic Parameters}

The definition of the test cases will call for SYMBOLIC\_CONSTANTS.
Their values are defined in this section for easy maintenance.

\subsection{Usability Survey Questions?}

\wss{This is a section that would be appropriate for some projects.}

\newpage{}
\section*{Appendix --- Reflection}

\wss{This section is not required for CAS 741}

The information in this section will be used to evaluate the team members on the
graduate attribute of Lifelong Learning.

The purpose of reflection questions is to give you a chance to assess your own
learning and that of your group as a whole, and to find ways to improve in the
future. Reflection is an important part of the learning process.  Reflection is
also an essential component of a successful software development process.  

Reflections are most interesting and useful when they're honest, even if the
stories they tell are imperfect. You will be marked based on your depth of
thought and analysis, and not based on the content of the reflections
themselves. Thus, for full marks we encourage you to answer openly and honestly
and to avoid simply writing ``what you think the evaluator wants to hear.''

Please answer the following questions.  Some questions can be answered on the
team level, but where appropriate, each team member should write their own
response:


\begin{enumerate}
  \item What went well while writing this deliverable? 
  \item What pain points did you experience during this deliverable, and how
    did you resolve them?
  \item What knowledge and skills will the team collectively need to acquire to
  successfully complete the verification and validation of your project?
  Examples of possible knowledge and skills include dynamic testing knowledge,
  static testing knowledge, specific tool usage, Valgrind etc.  You should look to
  identify at least one item for each team member.
  \item For each of the knowledge areas and skills identified in the previous
  question, what are at least two approaches to acquiring the knowledge or
  mastering the skill?  Of the identified approaches, which will each team
  member pursue, and why did they make this choice?
\end{enumerate}


\end{document}