\documentclass[12pt, titlepage]{article}
\usepackage{float}

\usepackage{booktabs}
\usepackage{tabularx}
\usepackage{hyperref}
\usepackage{float}

\hypersetup{
    colorlinks,
    citecolor=blue,
    filecolor=black,
    linkcolor=red,
    urlcolor=blue
}
\usepackage[round]{natbib}

%% Comments

\usepackage{color}

\newif\ifcomments\commentstrue %displays comments
%\newif\ifcomments\commentsfalse %so that comments do not display

\ifcomments
\newcommand{\authornote}[3]{\textcolor{#1}{[#3 ---#2]}}
\newcommand{\todo}[1]{\textcolor{red}{[TODO: #1]}}
\else
\newcommand{\authornote}[3]{}
\newcommand{\todo}[1]{}
\fi

\newcommand{\wss}[1]{\authornote{magenta}{SS}{#1}} 
\newcommand{\plt}[1]{\authornote{cyan}{TPLT}{#1}} %For explanation of the template
\newcommand{\an}[1]{\authornote{cyan}{Author}{#1}}

%% Common Parts

\newcommand{\progname}{Software Engineering} % PUT YOUR PROGRAM NAME HERE
\newcommand{\authname}{Team \#10, Five of a Kind
\\ Omar Abdelhamid
\\ Daniel Maurer
\\ Andrew Bovbel
\\ Olivia Reich
\\ Khalid Farag
} % AUTHOR NAMES                  

\usepackage{hyperref}
    \hypersetup{colorlinks=true, linkcolor=blue, citecolor=blue, filecolor=blue,
                urlcolor=blue, unicode=false}
    \urlstyle{same}
                                


\begin{document}

\title{System Verification and Validation Plan for \progname{}} 
\author{\authname}
\date{\today}
	
\maketitle

\pagenumbering{roman}

\section*{Revision History}

\begin{tabularx}{\textwidth}{p{3cm}p{2cm}X}
\toprule {\bf Date} & {\bf Version} & {\bf Notes}\\
\midrule
October 27, 2025 & 1.0 & Initial draft by All\\
\bottomrule
\end{tabularx}

~\\
\iffalse
\wss{The intention of the VnV plan is to increase confidence in the software.
However, this does not mean listing every verification and validation technique
that has ever been devised.  The VnV plan should also be a \textbf{feasible}
plan. Execution of the plan should be possible with the time and team available.
If the full plan cannot be completed during the time available, it can either be
modified to ``fake it'', or a better solution is to add a section describing
what work has been completed and what work is still planned for the future.}

\wss{The VnV plan is typically started after the requirements stage, but before
the design stage.  This means that the sections related to unit testing cannot
initially be completed.  The sections will be filled in after the design stage
is complete.  the final version of the VnV plan should have all sections filled
in.}
\fi

\newpage

\tableofcontents

\newpage

\listoftables

\newpage

\pagenumbering{arabic}

This document outlines the Verification and Validation (V\&V) Plan for \textbf{AutoVox}, developed by Team 10 --- \textit{Five of a Kind}. It describes how the system will be tested to ensure that all functional, non-functional, and safety requirements defined in the Software Requirements Specification (SRS) are satisfied. The plan establishes the strategies, team roles, and procedures that will be used to verify correctness, performance, and usability of the software.

The document is organized into two main sections. The first section, \textbf{General Information}, provides background context including a summary of the system, testing objectives, project challenge level, and relevant supporting documents. The second section, \textbf{Plan}, outlines the structure of the V\&V process, identifies team responsibilities, and describes the approach for verifying the SRS through structured reviews, stakeholder feedback, and continuous validation throughout development.

\section{General Information}

\subsection{Summary}

The software system being tested is called \textbf{AutoVox}, developed by Team 10 --- \textit{Five of a Kind}. AutoVox is a desktop-based CAD enhancement tool that allows researchers and engineers to assign magnetic properties to individual voxels within a 3D model. It is designed to streamline the process of magnetization planning for multi-material 3D printing workflows within a research laboratory environment.

Currently, magnetized objects are modeled voxel-by-voxel in COMSOL, requiring researchers to manually assign magnetization directions for each voxel and rebuild the model from scratch if an error occurs. AutoVox removes this bottleneck by enabling users to import pre-built CAD models (e.g., from AutoCAD), automatically slice them into voxel layers, and interactively select single or multiple voxels to assign or modify magnetization directions. The software also provides per-layer visualization, undo/redo functionality, and validation mechanisms to ensure complete and accurate voxel data before export.

The final output is a \textbf{Java-readable metadata file} that contains per-voxel magnetization information, which integrates directly with the custom 3D printer pipeline used in the lab. By automating voxel processing and enabling efficient magnetization editing, AutoVox dramatically reduces the time and effort required to prepare print-ready models, ensuring a faster, more reliable, and user-friendly workflow for advanced magnetized 3D printing research.

\subsection{Objectives}

The primary objective of the verification and validation (V\&V) process for \textbf{AutoVox} is to ensure that the system functions correctly, reliably, and efficiently according to the requirements defined in the Software Requirements Specification (SRS). The V\&V activities will confirm that each system component performs as intended, meets stakeholder expectations, and aligns with the project’s functional and non-functional goals.

Specifically, the objectives of V\&V are to:

\begin{itemize}
  \item \textbf{Correctness:} Verify that all functional requirements are implemented as specified, including accurate import, voxelization, visualization, editing, and export of CAD models.
  \item \textbf{Reliability:} Ensure the system maintains consistent behaviour under normal and high-load conditions, preventing crashes, data corruption, or voxel property loss.
  \item \textbf{Usability:} Validate that the interface allows users to intuitively view, select, and edit voxels across layers, and that feedback mechanisms (such as undo/redo, autosave, and validation prompts) function as expected.
  \item \textbf{Performance:} Confirm that voxel rendering, editing, and export operations meet timing thresholds and remain responsive on standard laboratory hardware.
  \item \textbf{Safety and Security:} Validate that safeguards identified in the Hazard Analysis (e.g., input file validation, export completeness, autosave recovery) are correctly implemented and meet the defined safety requirements.
  \item \textbf{Traceability:} Demonstrate that all SRS requirements are testable, measurable, and traceable through verification procedures and testing artifacts.
\end{itemize}

Overall, the V\&V process aims to provide stakeholders with confidence that \textbf{AutoVox} achieves its intended purpose as a robust, user-friendly, and trustworthy software solution for voxel-based magnetization design.

\subsection{Challenge Level and Extras}

The challenge level for \textbf{AutoVox} is classified as \textbf{advanced}. The system integrates multiple complex subsystems, including 3D model import and voxelization, real-time voxel visualization and editing, magnetization assignment, and structured export to a custom printer pipeline. Each subsystem involves substantial technical and architectural considerations, such as handling large voxel datasets, maintaining UI synchronization, and ensuring data integrity across modules. These factors collectively establish the project as an advanced-level software system.

The extras included in this project are:

\begin{itemize}
  \item \textbf{User Manual:} A comprehensive guide explaining how to operate the software, including detailed instructions, annotated screenshots, and workflow examples. The manual emphasizes ease of use and clarity to enable independent operation by research personnel.
  \item \textbf{Usability Report:} A formal documentation of the usability testing process, summarizing user feedback, identified issues, and recommendations for improving interface design and user experience. The report also outlines the methodology used during testing, including participant demographics and task-based evaluations.
\end{itemize}

These extras complement the technical objectives of the project by ensuring that the final product is not only functionally correct but also accessible, intuitive, and well-documented for its target users.

\subsection{Relevant Documentation}



\citet{SRS}

The documentation that supports the verification and validation (V\&V) process for \textbf{ AutoVox} includes the Software Requirements Specification (SRS), Hazard Analysis (HA), Development Plan (DP), and Problem Statement and Goals. Each document serves a distinct purpose in guiding and supporting the testing strategy.

The \textbf{Software Requirements Specification (SRS)} defines all functional and non-functional requirements for the system. It provides the foundation for test case design, ensuring that every requirement can be verified through measurable criteria and traced throughout the V\&V process.

The \textbf{Hazard Analysis (HA)} identifies potential system hazards, their causes, and mitigation strategies. It establishes the safety requirements (SCRs) that must be verified to ensure the system operates safely and consistently under expected laboratory conditions.

The \textbf{Development Plan (DP)} outlines the project’s workflow, coding standards, continuous integration setup, and proof-of-concept demonstration plan. It informs the structure and scheduling of testing activities by defining the methods and tools used during development.

The \textbf{Problem Statement and Goals} document provides the original motivation, scope, and intended outcomes of the project. It ensures that the V\&V process remains aligned with the project’s high-level objectives and stakeholder expectations.

Finally, the project’s \textbf{GitHub Repository} serves as the central hub for all implementation artifacts, testing scripts, and documentation updates.\\
\url{https://github.com/OmarHassanAdelhamid/Five-of-a-Kind-capstone-project}

Together, these documents ensure that all verification and validation activities are traceable, comprehensive, and consistent with the system’s intended functionality and safety goals.
\section{Plan}

The \textbf{Plan} section outlines the structure and approach of the verification and validation (V\&V) process for \textbf{AutoVox}. It describes the composition of the V\&V team, their assigned roles, and responsibilities across different testing domains, ensuring that all system components are thoroughly verified and validated. This section also details the planned approach for verifying the Software Requirements Specification (SRS), including the review strategy, feedback mechanisms, and formal verification methods that will be employed to confirm that all requirements are correct, testable, and traceable throughout development.

The verification and validation (V\&V) efforts for \textbf{AutoVox} will be conducted by \textbf{Team 10 --- \textit{Five of a Kind}}. Each member plays a specific role to ensure that all functional, non-functional, and safety requirements are properly verified and validated. The distribution of responsibilities ensures comprehensive coverage of testing, documentation, and review activities across all project domains.

\begin{table}[H]
\centering
\begin{tabular}{|p{3.5cm}|p{4cm}|p{7cm}|}
\hline
\textbf{Name} & \textbf{Role} & \textbf{Responsibilities} \\
\hline
\textbf{Omar Abdelhamid} & Lead Tester / Integration Coordinator & Oversees the overall testing strategy and ensures all verification activities align with the SRS. Manages issue tracking for validation tasks, coordinates integration testing, and co-leads structured review meetings with the supervisor. \\
\hline
\textbf{Daniel Maurer} & Functional Requirements Tester / Supervisor Liaison & Focuses on verifying functional aspects such as import, voxelization, and export. Coordinates communication with the supervisor, organizes feedback sessions, and ensures review comments are integrated into revisions. \\
\hline
\textbf{Andrew Bovbel} & Non-Functional Requirements Tester & Responsible for evaluating system performance, responsiveness, and reliability. Conducts stress and performance tests to ensure compliance with non-functional requirements (e.g., rendering latency, autosave frequency). \\
\hline
\textbf{Olivia Reich} & Usability and Documentation Tester & Leads usability testing, gathers user feedback, and evaluates interface intuitiveness. Reviews and refines the User Manual to ensure clarity and alignment with system behavior. \\
\hline
\textbf{Khalid Farag} & Safety and Validation Tester & Validates safety and security requirements derived from the Hazard Analysis. Maintains traceability between identified hazards, safety requirements (SCRs), and corresponding test cases. \\
\hline
\end{tabular}
\caption{Team roles and responsibilities for verification and validation.}
\end{table}

This structure ensures that all verification and validation efforts—ranging from requirement coverage and performance testing to usability evaluation and safety assurance—are systematically planned, executed, and reviewed in collaboration with the supervisor.
\subsection{SRS Verification}

Verification of the Software Requirements Specification (SRS) for \textbf{AutoVox} will follow a structured, multi-stage process to ensure that all requirements are correct, complete, testable, and traceable. This process includes internal reviews, supervisor meetings, peer review feedback, and continuous verification. Each activity is designed to improve requirement quality and maintain traceability between the SRS and subsequent project deliverables.

\subsubsection{Internal Review and Checklist Verification}

The first stage of SRS verification will involve a detailed internal review conducted by all team members. Each member will focus on a particular category of requirements:

\begin{itemize}
  \item \textbf{Omar Abdelhamid} and \textbf{Daniel Maurer} will verify that all functional requirements are clear, consistent, and feasible to implement.
  \item \textbf{Andrew Bovbel} will confirm that non-functional requirements, such as performance and reliability metrics, are measurable and verifiable.
  \item \textbf{Khalid Farag} will validate that safety and security requirements (SCRs) properly trace to hazards identified in the Hazard Analysis.
  \item \textbf{Olivia Reich} will review documentation-related and usability-focused requirements for clarity and completeness.
\end{itemize}

A collaborative \textbf{SRS Verification Checklist} will guide the review. The checklist will include the following key items:

\begin{enumerate}
  \item Each requirement is uniquely identified and unambiguous.
  \item Functional requirements are measurable and testable.
  \item Non-functional requirements specify clear performance criteria.
  \item Safety and security requirements trace back to identified hazards.
  \item Each requirement includes a defined verification or validation method.
  \item Requirements are free of contradictions or redundancies.
\end{enumerate}

All findings and issues identified during the review will be documented as GitHub issues and assigned to the appropriate team member for correction prior to the next SRS revision.

\subsubsection{Supervisor Review}

Following internal review, the team will conduct a structured SRS review meeting with the project supervisor. \textbf{Omar Abdelhamid} and \textbf{Daniel Maurer} will organize and co-lead the meeting, preparing a summary document highlighting high-risk and complex requirements that need confirmation (e.g., voxel visualization performance, data export structure, and UI synchronization accuracy).

The review meeting will include:

\begin{itemize}
  \item A walkthrough of the SRS to verify alignment between requirements and project scope.
  \item Discussion of any ambiguous, conflicting, or technically challenging requirements.
  \item Presentation of requirements identified as high-risk and their proposed validation methods.
  \item A feedback session for the supervisor to provide clarifications or suggest modifications.
\end{itemize}

Supervisor feedback will be recorded within the GitHub issue tracker. Each comment will be linked to the corresponding requirement ID and tracked until resolution. Follow-up actions will be reviewed in subsequent supervisor meetings.

\subsubsection{Peer Review and Class Feedback}

In addition to the supervisor review, the SRS will undergo peer evaluation during the class-wide review process. The primary reviewer team will assess the document’s clarity, consistency, and completeness. Their feedback will be formally logged and categorized according to:

\begin{itemize}
  \item Readability and clarity of requirement descriptions.
  \item Consistency of terminology and format across sections.
  \item Logical organization of requirements by subsystem or feature.
  \item Appropriateness of fit criteria and testability.
\end{itemize}

All peer feedback will be integrated systematically into the SRS through a traceability matrix, ensuring that each comment maps to a specific requirement and its revision status.

\subsubsection{Continuous Verification and Traceability Maintenance}

Throughout development, the team will maintain continuous verification of the SRS using GitHub issues, milestones, and project boards. Each requirement will be linked to corresponding design elements, test cases, and validation results. Regular bi-weekly internal reviews will verify that any new or modified requirements remain traceable and verifiable.

This iterative approach ensures that the SRS remains accurate, actionable, and aligned with the evolving implementation. Any changes introduced after supervisor or peer feedback will trigger updates to both the traceability matrix and the V\&V Plan to maintain consistency across documents.

\textbf{Summary:} This structured, multi-level verification process guarantees that the SRS for \textbf{AutoVox} is validated both internally and externally. Through continuous review, stakeholder feedback, and traceability enforcement, the team ensures that all requirements are precise, measurable, and fully verifiable before system testing begins.

\subsection{Design Verification}

The design verification will primarily focus on the system framework and integrated architecture that details the system implementation. Verification will take place through structured classmate and supervisor reviews. Reviews conducted by classmates will be designed to emphasize the interface design that permits usage of external libraries, internal refactoring and the primary database structure that encapsulates all voxel metadata. Feedback will be created into GitHub issues and discussed amongst the group in order to determine the best course of action to effectively integrate the feedback in subsequent development cycles.\\

\noindent Reviews conducted with our supervisor, Dr. Onaizah, will primarily verify tolerance parameters within the output framework during integration with specialized 3D printing software. Diagrams may be developed to provide visual aids to help facilitate the technical conversations and enhance understanding. These diagrams will default to standard UML diagrams, unless a more simplified diagram is required due to disparity in technical proficiency.\\

\noindent A checklist evaluation has been provided below for reference:\\

\begin{table}[H]
\centering
\caption{Design Verification Checklist}
\renewcommand{\arraystretch}{1.3}
\setlength{\tabcolsep}{8pt}
\begin{tabular}{|p{0.25\textwidth}|p{0.7\textwidth}|}
\hline
\textbf{Focus Area} & \textbf{Verification Tasks} \\ \hline
Core Architecture &
\begin{itemize}
\item Defined system modularity.
\item Compliance with standard coding conventions and language-specific best practices.
\item Software architecture patterns verified.
\end{itemize} \\ \hline
IDE Integration &
\begin{itemize}
\item External library integration for voxel slicing validated.
\item UI/UX patterns verified.
\end{itemize} \\ \hline
Database Integrity &
\begin{itemize}
\item Database structure confirmed.
\item Voxel data management operations verified for accuracy.
\end{itemize} \\ \hline
External Data Integrity &
\begin{itemize}
\item Structure of data files confirmed.
\end{itemize} \\ \hline
\end{tabular}
\end{table}


\subsection{Verification and Validation Plan Verification}

The verification \& validation testing plan will also incorporate peer reviews and certain testing methodologies. It will be evaluated on the following 4 key metric categories: coverage, test effectiveness, testing methodology, and traceability. Peer reviews will be centered around structured checklists that guide meaningful assessment on the scope and completeness of the verification \& validation plan. Completeness within the scope of the plan should effectively address system scalability in regard to the software’s ability to manage growth in project size, complexity, and anticipated functional demands. There will be emphasis placed on ensuring effective coverage of user interactions with 3D renditions and all voxel data management operations. Feedback will be created into GitHub issues and discussed amongst the group in order to determine the best course of action to effectively integrate the feedback in subsequent development cycles.\\

\noindent Feedback will be integrated with an interactive development approach. Established functionality within the system will not be considered complete until complete testing in accordance with the verification \& validation plan has been completed. During weekly group meetings, updates related to a specific feature will require a corresponding update regarding related test cases that will be established. Tests must also be documented as GitHub issues. Upon closing a test case, the results must be documented within the issue. Results will be evaluated against a pre-defined baseline success metrics. \\

\noindent Within test validation, mutation testing will be integrated to assess the effectiveness of the system. This will include the assertion of defective input files, faulty voxel metadata updates, and unsupported display commands. Our goal with mutation testing is to generate quantifiable observations regarding our system’s capabilities of detecting undesirable anomalies.\\

\noindent A checklist evaluation has been provided below for reference:\\

\begin{table}[H]
\centering
\caption{V\&V Plan Verification Checklist}
\renewcommand{\arraystretch}{1.3}
\setlength{\tabcolsep}{8pt}
\begin{tabular}{|p{0.25\textwidth}|p{0.7\textwidth}|}
\hline
\textbf{Criteria} & \textbf{Verification Tasks} \\ \hline
Coverage &
\begin{itemize}
\item Quantitative evaluation metrics defined.
\item Established configuration of testing domains.
\item Established coverage cases.
\item Quantitative coverage summaries.
\end{itemize} \\ \hline
Testing Methodology &
\begin{itemize}
\item Established baseline success metrics.
\item Testing classification completed.
\end{itemize} \\ \hline
Traceability &
\begin{itemize}
\item Test tracking integrated within GitHub issues.
\item Formalized feedback process.
\end{itemize} \\ \hline
\end{tabular}
\end{table}


\subsection{Implementation Verification}

Implementation verification will ensure that all segments of code strictly adhere to the requirements and constraints outlined in the SRS document as well as ensure adherence to the mandatory programming language’s conventions. Unit tests will be used in order to validate the core functionalities with the system. Pytest will likely be the primary testing framework for unit tests relating to the system components' development in Python. In a similar manner, Jest will be used to facilitate unit testing with the UI/UX of our system. Test cases will primarily focus on testing the functionality related to voxel modification requirements identified in F1-F4.\\

\noindent Static analysis will be conducted to ensure accordance with PEP 8, which is the official style guide for writing Python code. There will also be code inspections to verify secure and consistent file handling across all interconnected systems prior to running the code.  This is in accordance with SCR1, SCR4, and SCR5.\\

\noindent To ensure high implementation quality throughout all development cycles, code reviews will be on a minimum weekly basis (granted the code was modified throughout that period). These code reviews will incorporate the following:
\begin{itemize}
\item effective usage of external Python libraries in accordance with their usage guidelines
\item optimized voxel modification processes (F231, F233, F235, F239)
\item precise 3D rendition (F221, F222)
\item verification of future scalability (NF211)
\end{itemize}

\noindent Testing will also be done to evaluate the performance of implementation. In accordance with thresholds as defined in associated requirements, it will be evaluated as follows:
\begin{itemize}
\item data processing across file input and output workflows (NF212, NF242)
\item optimized interface responsiveness (F223)
\item large database interaction (NF232, SCR3)
\end{itemize}

\begin{table}[H]
\centering
\caption{Implementation Verification Checklist}
\renewcommand{\arraystretch}{1.3}
\setlength{\tabcolsep}{8pt}
\begin{tabular}{|p{0.25\textwidth}|p{0.7\textwidth}|}
\hline
\textbf{Category} & \textbf{Verification Tasks} \\ \hline
Functionality &
\begin{itemize}
\item Defined test case sets for component requirements (SRS, S.2).
\item Visual rendering validation.
\item Voxel modification validation.
\end{itemize} \\ \hline
Quality &
\begin{itemize}
\item Structured code walkthrough completed.
\item Static analysis in accordance with the development environment.
\item Weekly code reviews conducted.
\end{itemize} \\ \hline
Performance &
\begin{itemize}
\item File handling benchmarks.
\item User interaction processing time benchmarks.
\item Database interaction benchmarks.
\end{itemize} \\ \hline
Traceability &
\begin{itemize}
\item Test tracking integrated within GitHub issues.
\end{itemize} \\ \hline
\end{tabular}
\end{table}


\subsection{Automated Testing and Verification Tools}

\textbf{Unit Testing Framework (backend):} Since the backend of our system will be coded in Python, the primary testing framework used will be PyTest. This framework was chosen due to its flexible testing framework that can be scaled to support complex function testing. \\

\noindent\textbf{Code coverage tools  in PyTest:} PyTest offers a variety of tools, including the usage of the coverage.py library (pytest-cov), which allows extensive code coverage. It is able to monitor the program during execution and report what parts of the code have been executed. The plugin pytest-mock also provides a convenient way to mock objects and functions, including those in external libraries. This will be useful for testing an external voxel slicing library.\\

\noindent\textbf{Unit Testing Framework (frontend):} Currently, the primary testing framework that will be used for the front end is Jest, as we will be using the JavaScript library, Three.js, to generate the 3D image.\\

\noindent\textbf{Code coverage tools in Jest:} Jest offers several specialized libraries and techniques that can create image analysis tools for 3D image data evaluation. This includes tools such as snapshot test (jest-image-snapshot), which allows visual comparison. It also offers a mocking tool, which allows testing of specific property image components.\\

\noindent Ideally, the goal is to achieve 50\% coverage across both the frontend and backend of the codebases by the end of Rev 0. It is expected the PoC demo product will have significantly reduced coverage. However, coverage will be integrated through an interactive approach. Weekly reports will be generated from all coverage tools to track testing through the project. This will allow us to gain insight into overall test progress, understand general trends within code coverage, and provide clarity on what areas still require additional validation.\\

\subsection{Software Validation}

\textit{User Test Group}\\
\noindent To validate the usability of the system, a specialized test group will be selected based on the current typical users of the 3D printer. Testing will take the form of exploratory and acceptance testing. Users will be provided a list of tasks that they will attempt to navigate through with no prior instruction. The goal is to assess the system on the following aspects: usability, accessibility, discoverability, natural flow of interaction and system behaviour.\\

\noindent The list of tasks will closely relate to essential requirements. The list of tasks will depend on the current status of the project. However, tasks may include:
\begin{itemize}
\item create a new project with a given CAS file (F221)
\item generate a 3D image with a given voxel size (F213)
\item access the bottom face of the object (F223)
\item given desired voxel properties of 3 layers, replicate structure on software (F224 - F227, F231, F2313)
\begin{itemize}
\item all 3 layers will have the same voxel properties to see if the user experiments with property replication (F235)
\end{itemize}
\item erase all properties from the third layer
\item assign properties to the remaining voxels and export the project
\end{itemize}

\noindent \textit{Rev 0}\\
\noindent This will include a meeting with Dr. Onaizah to validate the following:
\begin{itemize}
\item implementation of SRS requirements
\item desired modifications to requirements or changes in user needs
\item how the system behaviour compares to specifications
\end{itemize}


\section{System Tests}

This section outlines the tests for verifying both functional and nonfunctional requirements of the system. It helps ensure that the system meets stakeholders' expectations. This section includes
tests covering the essential aspects of the systems requirements.

\subsection{Tests for Functional Requirements}

The subsections below outline tests corresponding to functional requirements in the SRS.
Each test is associated with a unique functional area, helping to confirm that the system meets
the specified requirements. 

\subsubsection{Import Manager Tests}

\paragraph{CAD File Import and Project Management Tests}

\begin{enumerate}

\item{test-FR-IM-1 Valid CAD File Import for New Project\\}

\textbf{Type:} Functional, Dynamic, Manual

\textbf{Covers:} F211
					
\textbf{Initial State:} No active project is loaded in the system, and the system is ready to create a new project.
					
\textbf{Input:} A CAD file (STL file) is provided to the system through the import interface by the user.
					
\textbf{Output:} The system creates the new project with the imported CAD file, initiates the voxel slicing process, and the imported model is ready for visualization.

\textbf{Test Case Derivation:} The test case is derived from the functional requirement F211, which states that the Import Manager should allow a user to start a new project with the option to import an initial CAD file from their personal device that will be sliced.
					
\textbf{How test will be performed:} Create a new project on the system and import a CAD file into the new project to verify that the system will correctly process the file and create a new project. The system should be able to handle the CAD file and initiate the voxel slicing process. This will be tested manually by the developer of the system.

\item{test-FR-IM-2 Past Project File Import\\}

\textbf{Type:} Functional, Dynamic, Manual

\textbf{Covers:} F212
					
\textbf{Initial State:} No active project is loaded in the system, and the system is ready to open a past project.
					
\textbf{Input:} A previously saved project file containing magnetization and material properties is provided to the system through the import interface by the user.
					
\textbf{Output:} The system loads the project containing all of the magnetization and material properties preserved. The model is ready for further editing and visualization.

\textbf{Test Case Derivation:} The test case is derived from the functional requirement F212, which states that the Import Manager should allow a user to import a past project file in order to reopen a project with all magnetization and material properties preserved.
					
\textbf{How test will be performed:} Open a previously saved project containing magnetization and material properties on the system. Verify that the system correctly loads the project and all of the magnetization and material properties are preserved. The model should be ready for further editing and visualization. This will be tested manually by the developer of the system.

\item{test-FR-IM-3 Invalid File Format Handling\\}

\textbf{Type:} Functional, Dynamic, Manual

\textbf{Covers:} F211
					
\textbf{Initial State:} No active project is loaded in the system, and the system is ready to import a file.
					
\textbf{Input:} A file with unsupported format (e.g. .txt, .jpg, .pdf) is provided to the system through the import interface by the user.
					
\textbf{Output:} The system rejects the file and displays an appropriate error message indicating the file format is unsupported, without creating a project or crashing.

\textbf{Test Case Derivation:} The test case is derived from the functional requirement F211, which states that the Import Manager should allow a user to start a new project with the option to import an initial CAD file from their personal device that will be sliced. 
					
\textbf{How test will be performed:} Try to import a file with an unsupported format (e.g. .txt, .jpg, .pdf) into the system. Verify that the system correctly rejects the file and displays an appropriate error message indicating the file format is unsupported. This will be tested manually by the developer of the system.

\end{enumerate}

\paragraph{Model Configuration and Slicing Tests}

\begin{enumerate}

\item{test-FR-IM-4 Voxel Size Configuration\\}

\textbf{Type:} Functional, Dynamic, Automated

\textbf{Covers:} F213
					
\textbf{Initial State:} A valid CAD model has been imported and the system is ready to configure the voxel dimensions.
					
\textbf{Input:} User specifies custom voxel dimensions (e.g. 0.1mm x 0.1mm x 0.1mm) by entering the dimensions into the system through the import interface by the user.
					
\textbf{Output:} The system applies the specified voxel dimensions and successfully slices the model into voxels of the requested dimensions.

\textbf{Test Case Derivation:} The test case is derived from the functional requirement F213, which states that the Import Manager should allow a user to configure the voxel dimensions into which the model will be sliced.
					
\textbf{How test will be performed:} Configure the voxel dimensions for the imported model by entering the dimensions into the system through the import interface by the user. Verify that the system correctly applies the specified voxel dimensions and successfully slices the model into voxels of the requested dimensions. This will be tested automatically by the developer of the system by running a script that configures the voxel dimensions and verifies that the system correctly slices the model into voxels of the requested dimensions.

\item{test-FR-IM-5 Model Scaling Functionality\\}

\textbf{Type:} Functional, Dynamic, Manual

\textbf{Covers:} F214
					
\textbf{Initial State:} A valid CAD model has been imported and the system is ready to scale the model.
					
\textbf{Input:} User modifies the model dimensions to the desired printing size by entering the new dimensions into the system through the import interface by the user.
					
\textbf{Output:} The model is scaled to the desired printing size, maintaining its geometric integrity and proportions.

\textbf{Test Case Derivation:} The test case is derived from the functional requirement F214, which states that the Import Manager should allow a user to scale the model to the desired printing size before being sliced into voxels.
					
\textbf{How test will be performed:} Modify the model dimensions to the desired printing size by entering the new dimensions into the system through the import interface by the user. The developer will then manually inspect the rendered 3D model within the application. Verification will involve comparing the visual representation of the sliced model against the original CAD model and the expected scaling of the model to confirm that model integrity and accuracy are preserved as specified in F214

\item{test-FR-IM-6 Partial Voxel Resolution\\}

\textbf{Type:} Functional, Dynamic, Manual

\textbf{Covers:} F215
					
\textbf{Initial State:} A CAD model with complex geometry has been imported and the system is ready to resolve the partial voxels.
					
\textbf{Input:} A model containing surfaces that intersect voxel boundaries is imported by the user through the import interface.
					
\textbf{Output:} The system correctly resolves partial voxels and preserves model integrity and accuracy.

\textbf{Test Case Derivation:} The test case is derived from the functional requirement F215, which states that the Import Manager shall resolve partial voxels and interpret them in the best way that preserves the integrity and accuracy of the provided model.
					
\textbf{How test will be performed:} Import a model containing surfaces that intersect voxel boundaries. The developer will then manually inspect the rendered 3D model within the application. Verification will involve comparing the visual representation of the sliced model against the original CAD model and the expected resolution of partial voxels to confirm that model integrity and accuracy are preserved as specified in F215

\item{test-FR-IM-7 Model Division for Display\\}

\textbf{Type:} Functional, Dynamic, Manual

\textbf{Covers:} F216
					
\textbf{Initial State:} A large CAD model has been imported and the system is ready to divide the model into manageable display sections.
					
\textbf{Input:} A model containing more than MAX\_DISPLAY voxels is imported by the user through the import interface.
					
\textbf{Output:} The system automatically partitions the model into manageable display sections that do not exceed the MAX\_DISPLAY threshold, enabling smooth visualization.

\textbf{Test Case Derivation:} The test case is derived from the functional requirement F216, which states that the Import Manager shall partition the model into user-manageable display sections that do not surpass the MAX\_DISPLAY threshold.
					
\textbf{How test will be performed:} Import a model containing more than MAX\_DISPLAY voxels. The developer will then manually inspect the rendered 3D model within the application. Verification will involve comparing the visual representation of the sliced model against the original CAD model and the expected partitioning of the model to confirm that model integrity and accuracy are preserved as specified in F216

\end{enumerate}

\subsubsection{Visualization Manager Tests}

\paragraph{3D Model Rendering and Display Tests}

\begin{enumerate}

\item{test-FR-VM-1 3D Model Rendition with Clear Voxel Visualization\\}

\textbf{Type:} Functional, Dynamic, Manual

\textbf{Covers:} F221
					
\textbf{Initial State:} A CAD model has been imported and sliced into voxels by the Import Manager and the system is ready to render the model.
					
\textbf{Input:} Sliced voxel data from Import Manager is provided to the Visualization Manager.
					
\textbf{Output:} The system renders a 3D model with clear visualization of individual voxels, maintaining geometric accuracy and providing distinct voxel boundaries.

\textbf{Test Case Derivation:} The test case is derived from the functional requirement F221, which states that the Visualization Manager shall recreate a 3D model with clear visualization of sliced voxels.
					
\textbf{How test will be performed:} Render a 3D model with clear visualization of sliced voxels by providing the Visualization Manager with sliced voxel data from the Import Manager. Verify that the system correctly renders the model and displays the individual voxels with proper boundaries and geometric accuracy. This will be tested manually by the developer of the system by providing the system with a CAD model and verifying that the system correctly renders the model and displays the individual voxels with proper boundaries and geometric accuracy. 

\item{test-FR-VM-2 Partition Navigation Functionality\\}

Type: Functional, Dynamic, Manual

\textbf{Covers:} F222
					
\textbf{Initial State:} A large model has been divided into multiple display partitions and the system is ready to navigate between the partitions.
					
\textbf{Input:} User navigates between model partitions by clicking on the different display sections of the model.
					
\textbf{Output:} The system provides smooth navigation between partitions, allowing users to access all model sections without performance degradation.

\textbf{Test Case Derivation:} The test case is derived from the functional requirement F222, which states that the Visualization Manager shall provide users with simplistic navigation across user-manageable display sections to ensure access to all model partitions.
					
\textbf{How test will be performed:} Navigate between model partitions by clicking on the different display sections of the model. Verify that the system correctly provides smooth navigation between partitions, allowing users to access all model sections without performance degradation. This will be tested manually by the developer of the system.

\item{test-FR-VM-3 Multi-Perspective 3D Model Interaction\\}

\textbf{Type:} Functional, Dynamic, Manual

\textbf{Covers:} F223
					
\textbf{Initial State:} A 3D model is rendered and displayed and the system is ready to interact with the model from different perspectives.
					
\textbf{Input:} User rotates and zooms the model to view it from different perspectives using intuitive interface controls.
					
\textbf{Output:} The system correctly provides intuitive interface controls that allow seamless navigation across multiple perspectives of the 3D model with responsive interaction.

\textbf{Test Case Derivation:} The test case is derived from the functional requirement F223, which states that the Visualization Manager shall provide users with an intuitive interface that permits seamless navigation across multiple perspectives of the 3D model.
					
\textbf{How test will be performed:} Rotate and zoom the model to view it from different perspectives using intuitive interface controls. Verify that the system correctly provides intuitive interface controls that allow seamless navigation across multiple perspectives of the 3D model with responsive interaction. This will be tested manually by the developer of the system.

\end{enumerate}

\paragraph{Layer Focus and Voxel Selection Tests}

\begin{enumerate}

\item{test-FR-VM-4 Layer Isolation Functionality\\}

Type: Functional, Dynamic, Manual

\textbf{Covers:} F224
					
\textbf{Initial State:} A 3D model with multiple layers is displayed and the system is ready to isolate a specific layer.
					
\textbf{Input:} User selects a specific layer to focus on by clicking on the layer in the display.
					
\textbf{Output:} The system isolates the selected layer, rendering all other voxels irrelevant or transparent, facilitating property assignment on the focused layer.

\textbf{Test Case Derivation:} The test case is derived from the functional requirement F224, which states that the Visualization Manager shall allow users to isolate a specific layer to facilitate property assignments, rendering all other voxels irrelevant while that layer is in focus.
					
\textbf{How test will be performed:} Select a specific layer to focus on by clicking on the layer in the display. Verify that the system correctly isolates the selected layer and renders all other voxels irrelevant while that layer is in focus. This will be tested manually by the developer of the system.

\item{test-FR-VM-5 Voxel Selection Highlighting\\}

\textbf{Type:} Functional, Dynamic, Manual

\textbf{Covers:} F225
					
\textbf{Initial State:} A specific layer is in focus and displayed and the system is ready to highlight the selected voxels.
					
\textbf{Input:} User selects individual voxels or groups of voxels within the focused layer by clicking on the voxels in the display.
					
\textbf{Output:} The system correctly provides clear visual feedback showing which voxels are currently selected using distinct highlighting.

\textbf{Test Case Derivation:} The test case is derived from the functional requirement F225, which states that the Visualization Manager shall provide users with visualization that showcases which voxels are currently selected within a specific layer.
					
\textbf{How test will be performed:} Select individual voxels or groups of voxels within the focused layer by clicking on the voxels in the display. Verify that the system correctly provides clear visual feedback showing which voxels are currently selected using distinct highlighting. This will be tested manually by the developer of the system.

\end{enumerate}

\paragraph{Material and Magnetization Tracking Tests}

\begin{enumerate}

\item{test-FR-VM-6 Material Assignment Visual Tracking\\}

Type: Functional, Dynamic, Manual

\textbf{Covers:} F226
					
\textbf{Initial State:} A model with unassigned materials is displayed and the system is ready to track the material assignment.
					
\textbf{Input:} User assigns material IDs to various voxels by clicking on the voxels in the display and assigning the material IDs.
					
\textbf{Output:} The system updates the voxel colors to indicate material assignment completeness, providing clear visual feedback about the assignment status.

\textbf{Test Case Derivation:} The test case is derived from the functional requirement F226, which states that the Visualization Manager shall integrate easy tracking of voxels that have been assigned material IDs by adjusting the colour of the voxel to indicate assignment completeness.
					
\textbf{How test will be performed:} Assign material IDs to various voxels by clicking on the voxels in the display and assigning the material IDs. Verify that the system correctly updates the voxel colors to indicate material assignment completeness, providing clear visual feedback about assignment status. This will be tested manually by the developer of the system.

\item{test-FR-VM-7 Magnetization Assignment Visual Tracking\\}

\textbf{Type:} Functional, Dynamic, Manual

\textbf{Covers:} F227
					
\textbf{Initial State:} A model with unassigned magnetization vectors is displayed and the system is ready to track the magnetization assignment.
					
\textbf{Input:} User assigns magnetization vectors to various voxels by clicking on the voxels in the display and assigning the magnetization vectors.
					
\textbf{Output:} The system updates the voxel colors to indicate magnetization assignment completeness, providing clear visual feedback about the assignment status.

\textbf{Test Case Derivation:} The test case is derived from the functional requirement F227, which states that the Visualization Manager shall integrate easy tracking of voxels that have been assigned magnetization vectors by adjusting the colour of the voxel to indicate assignment completeness.
					
\textbf{How test will be performed:} Assign magnetization vectors to various voxels by clicking on the voxels in the display and assigning the magnetization vectors. Verify that the system correctly updates the voxel colors to indicate magnetization assignment completeness, providing clear visual feedback about the assignment status. This will be tested manually by the developer of the system.

\end{enumerate}

\subsubsection{Integration Tests Between Import and Visualization Managers}


\paragraph{Data Flow and Communication Tests}

\begin{enumerate}

\item{test-FR-INT-1 Import to Visualization Data Transfer\\}

\textbf{Type:} Functional, Dynamic, Automated

\textbf{Covers:} F211, F221
					
\textbf{Initial State:} No active project is loaded in the system, and the system is ready to import a file.
					
\textbf{Input:} A valid CAD file is imported by the user through the import interface.
					
\textbf{Output:} The Visualization Manager receives complete voxel data and successfully renders the 3D model without data loss or corruption.

\textbf{Test Case Derivation:} The test case is derived from the functional requirement F211 and F221, which states that the Import Manager should allow a user to import a CAD file from their personal device that will be sliced and the Visualization Manager should successfully render the 3D model without data loss or corruption.
					
\textbf{How test will be performed:} Import a valid CAD file. Verify that the system correctly imports the CAD file and the Visualization Manager successfully renders the 3D model without data loss or corruption. This will be tested automatically by the developer of the system by running a script that imports a CAD file and verifies that the system correctly imports the CAD file and the Visualization Manager successfully renders the 3D model without data loss or corruption.


\end{enumerate}

\subsubsection{Editing Manager Tests}
This section covers the tests for ensuring the Editing Manager correctly
handles magnetization and material assignment operations, property management,
and user interface interactions as specified in requirements F231 through F2310.

\begin{enumerate}

\item{test-FR-EM-1 Individual Voxel Magnetization Assignment\\}

\textbf{Type:} Functional, Dynamic, Manual

\textbf{Covers:} F231
					
\textbf{Initial State:} 3D model loaded with voxels visible, Editing Manager active.
					
\textbf{Input:} User selects a single voxel and assigns a magnetization vector (x=1.0, y=0.0, z=0.0).
					
\textbf{Output:} The selected voxel displays the assigned magnetization vector and updates its visual representation.

\textbf{Test Case Derivation:} Confirming that the system correctly processes individual voxel magnetization assignment, as specified in F231.
					
\textbf{How test will be performed:} Load a 3D model, select a voxel using the interface, input magnetization values, and verify the assignment is recorded and visually displayed.

\item{test-FR-EM-2 Group Voxel Magnetization Assignment\\}

\textbf{Type:} Functional, Dynamic, Manual

\textbf{Covers:} F231
					
\textbf{Initial State:} 3D model loaded with multiple voxels visible.
					
\textbf{Input:} User selects multiple voxels and assigns the same magnetization vector to all selected voxels.
					
\textbf{Output:} All selected voxels display the assigned magnetization vector and update their visual representation.

\textbf{Test Case Derivation:} Ensures the system can handle bulk magnetization assignment operations, satisfying F231.
					
\textbf{How test will be performed:} Select multiple voxels using the interface, assign magnetization values, and verify all selected voxels receive the assignment.

\item{test-FR-EM-3 Favourite Magnetization Vector Management\\}

\textbf{Type:} Functional, Dynamic, Manual

\textbf{Covers:} F232
					
\textbf{Initial State:} Editing Manager with favourite bar accessible.
					
\textbf{Input:} User creates a favourite magnetization vector (x=0.5, y=0.5, z=0.0) and saves it to the favourite bar.
					
\textbf{Output:} The magnetization vector is added to the favourite bar and can be selected for future use.

\textbf{Test Case Derivation:} Validates that users can maintain and reuse frequently used magnetization vectors, per F232.
					
\textbf{How test will be performed:} Create a favourite magnetization vector, save it to the favourite bar, and verify it can be selected and applied to voxels.

\item{test-FR-EM-4 Individual Voxel Material Assignment\\}

\textbf{Type:} Functional, Dynamic, Manual

\textbf{Covers:} F233
					
\textbf{Initial State:} 3D model loaded with voxels visible.
					
\textbf{Input:} User selects a single voxel and assigns material ID 2.
					
\textbf{Output:} The selected voxel displays the assigned material and updates its visual representation.

\textbf{Test Case Derivation:} Confirming that the system correctly processes individual voxel material assignment, as specified in F233.
					
\textbf{How test will be performed:} Select a voxel, assign a material ID, and verify the assignment is recorded and visually displayed.

\item{test-FR-EM-5 Group Voxel Material Assignment\\}

\textbf{Type:} Functional, Dynamic, Manual

\textbf{Covers:} F233
					
\textbf{Initial State:} 3D model loaded with multiple voxels visible.
					
\textbf{Input:} User selects multiple voxels and assigns material ID 3 to all selected voxels.
					
\textbf{Output:} All selected voxels display the assigned material and update their visual representation.

\textbf{Test Case Derivation:} Ensures the system can handle bulk material assignment operations, satisfying F233.
					
\textbf{How test will be performed:} Select multiple voxels, assign a material ID, and verify all selected voxels receive the assignment.

\item{test-FR-EM-6 Material Label Assignment\\}

\textbf{Type:} Functional, Dynamic, Manual

\textbf{Covers:} F234
					
\textbf{Initial State:} Material assignment interface accessible.
					
\textbf{Input:} User assigns label "Steel" to material ID 1.
					
\textbf{Output:} Material ID 1 is now associated with the label "Steel" in the interface.

\textbf{Test Case Derivation:} Validates that users can assign meaningful labels to material IDs for better organization, per F234.
					
\textbf{How test will be performed:} Access material management interface, assign a label to a material ID, and verify the label is displayed and associated correctly.

\item{test-FR-EM-7 Property Replication Across Layers\\}

\textbf{Type:} Functional, Dynamic, Manual

\textbf{Covers:} F235
					
\textbf{Initial State:} 3D model with multiple layers loaded.
					
\textbf{Input:} User selects voxels in one layer with assigned properties and replicates them to another layer.
					
\textbf{Output:} The target layer voxels receive the same material and magnetization properties as the source voxels.

\textbf{Test Case Derivation:} Ensures efficient workflow by allowing property replication across layers, as specified in F235.
					
\textbf{How test will be performed:} Select voxels with assigned properties in one layer, use the replication function to copy properties to another layer, and verify the properties are correctly applied.

\item{test-FR-EM-8 Auto Save Functionality\\}

\textbf{Type:} Functional, Dynamic, Automated

\textbf{Covers:} F236
					
\textbf{Initial State:} Model with unsaved changes.
					
\textbf{Input:} System automatically saves changes after a property assignment.
					
\textbf{Output:} Changes are saved without user intervention and confirmation is displayed.

\textbf{Test Case Derivation:} Validates that the system automatically preserves user work without manual intervention, per F236.
					
\textbf{How test will be performed:} Make property assignments, verify auto-save triggers, and confirm changes are preserved in the project file.

\item{test-FR-EM-9 Edit History Access and Revert\\}

\textbf{Type:} Functional, Dynamic, Manual

\textbf{Covers:} F237
					
\textbf{Initial State:} Model with multiple edit operations completed.
					
\textbf{Input:} User accesses edit history and reverts to a previous version.
					
\textbf{Output:} Model returns to the selected previous state with all subsequent changes undone.

\textbf{Test Case Derivation:} Ensures users can recover from mistakes by accessing edit history, as specified in F237.
					
\textbf{How test will be performed:} Perform multiple edits, access edit history, select a previous version, and verify the model reverts correctly.

\item{test-FR-EM-10 Entire Layer Selection and Assignment\\}

\textbf{Type:} Functional, Dynamic, Manual

\textbf{Covers:} F238
					
\textbf{Initial State:} 3D model with multiple layers loaded.
					
\textbf{Input:} User selects an entire layer and assigns common material and magnetization properties.
					
\textbf{Output:} All voxels in the selected layer receive the assigned properties.

\textbf{Test Case Derivation:} Validates efficient bulk operations by allowing entire layer selection, per F238.
					
\textbf{How test will be performed:} Select an entire layer, assign material and magnetization properties, and verify all voxels in the layer receive the assignments.

\item{test-FR-EM-11 Manual Voxel Addition\\}

\textbf{Type:} Functional, Dynamic, Manual

\textbf{Covers:} F239
					
\textbf{Initial State:} 3D model loaded with defined voxel dimensions.
					
\textbf{Input:} User adds a new voxel with the same dimensions as existing voxels.
					
\textbf{Output:} New voxel is added to the model with unassigned properties.

\textbf{Test Case Derivation:} Ensures users can manually add voxels for model adjustments, as specified in F239.
					
\textbf{How test will be performed:} Use the voxel addition tool to add a new voxel, verify it has correct dimensions, and confirm it starts with unassigned properties.

\item{test-FR-EM-12 Manual Voxel Deletion\\}

\textbf{Type:} Functional, Dynamic, Manual

\textbf{Covers:} F239
					
\textbf{Initial State:} 3D model with multiple voxels.
					
\textbf{Input:} User selects and deletes a voxel.
					
\textbf{Output:} Selected voxel is removed from the model.

\textbf{Test Case Derivation:} Ensures users can manually remove voxels for model adjustments, as specified in F239.
					
\textbf{How test will be performed:} Select a voxel, use the deletion tool, and verify the voxel is removed from the model.

\item{test-FR-EM-13 Individual Voxel Property Reset\\}

Type: Functional, Dynamic, Manual

\textbf{Covers:} F2310
					
\textbf{Initial State:} Voxel with assigned material and magnetization properties.
					
\textbf{Input:} User selects the voxel and resets its properties.
					
\textbf{Output:} Voxel returns to unassigned state for both material and magnetization.

\textbf{Test Case Derivation:} Validates that users can reset individual voxel properties, per F2310.
					
\textbf{How test will be performed:} Select a voxel with assigned properties, use the reset function, and verify the voxel returns to unassigned state.

\item{test-FR-EM-14 Group Voxel Property Reset\\}

\textbf{Type:} Functional, Dynamic, Manual

\textbf{Covers:} F2310
					
\textbf{Initial State:} Multiple voxels with assigned properties.
					
\textbf{Input:} User selects multiple voxels and resets their properties.
					
\textbf{Output:} All selected voxels return to unassigned state.

\textbf{Test Case Derivation:} Ensures users can reset multiple voxel properties simultaneously, per F2310.
					
\textbf{How test will be performed:} Select multiple voxels with assigned properties, use the reset function, and verify all selected voxels return to unassigned state.


\end{enumerate}


\subsubsection{Export Manager Tests}

This section covers the tests for ensuring the Export Manager correctly
handles file export operations, property validation, and data integrity
as specified in requirements F241 through F245.

\begin{enumerate}

\item{test-FR-XM-1 Property Validation for Complete Magnetization\\}

\textbf{Type:} Functional, Dynamic, Automated

\textbf{Covers:} F241
					
\textbf{Initial State:} Model with all voxels having assigned magnetization values.
					
\textbf{Input:} User requests export for printing.
					
\textbf{Output:} System validates all voxels have magnetization assignments and proceeds with export.

\textbf{Test Case Derivation:} Confirming that the system validates complete magnetization before export, as specified in F241.
					
\textbf{How test will be performed:} Load a model with complete magnetization assignments, request export, and verify validation passes and export proceeds.

\item{test-FR-XM-2 Property Validation for Incomplete Magnetization\\}

\textbf{Type:} Functional, Dynamic, Automated

\textbf{Covers:} F241
					
\textbf{Initial State:} Model with some voxels missing magnetization assignments.
					
\textbf{Input:} User requests export for printing.
					
\textbf{Output:} System identifies unassigned voxels and prompts user to complete assignments or assign null values.

\textbf{Test Case Derivation:} Ensures the system prevents export of incomplete models, per F241.
					
\textbf{How test will be performed:} Load a model with incomplete magnetization, request export, and verify the system identifies missing assignments and provides appropriate feedback.

\item{test-FR-XM-3 Standalone File Export\\}

\textbf{Type:} Functional, Dynamic, Manual

\textbf{Covers:} F242
					
\textbf{Initial State:} Model with complete property assignments ready for export.
					
\textbf{Input:} User exports model with custom filename "test\_model.vox".
					
\textbf{Output:} Standalone file containing all voxel metadata is created and saved locally.

\textbf{Test Case Derivation:} Validates that the system can produce complete export files, as specified in F242.
					
\textbf{How test will be performed:} Export a model with a custom filename, verify the file is created with all metadata, and confirm it can be accessed outside the software.

\item{test-FR-XM-4 Project Export in Native Format\\}

\textbf{Type:} Functional, Dynamic, Manual

\textbf{Covers:} F243
					
\textbf{Initial State:} Project with model and all associated data.
					
\textbf{Input:} User exports project with custom filename "backup\_project.fok".
					
\textbf{Output:} Native format project file is created and saved locally with all project data preserved.

\textbf{Test Case Derivation:} Ensures users can create project backups and iterations, per F243.
					
\textbf{How test will be performed:} Export a project in native format, verify the file contains all project data, and confirm it can be imported back into the software.

\item{test-FR-XM-5 Export Progress Tracking\\}

\textbf{Type:} Functional, Dynamic, Manual

\textbf{Covers:} F244
					
\textbf{Initial State:} Large model ready for export.
					
\textbf{Input:} User initiates export operation.
					
\textbf{Output:} Progress bar displays export progress with visual indicators.

\textbf{Test Case Derivation:} Validates that users receive feedback during export operations, as specified in F244.
					
\textbf{How test will be performed:} Export a large model, observe the progress bar updates, and verify the progress indicators accurately reflect export status.

\item{test-FR-XM-6 Model Summary Export\\}

Type: Functional, Dynamic, Manual

\textbf{Covers:} F245
					
\textbf{Initial State:} Model with complete property assignments.
					
\textbf{Input:} User exports model with summary option enabled.
					
\textbf{Output:} Model file and summary file are created with statistics about the model.

\textbf{Test Case Derivation:} Ensures users can access model statistics for analysis, per F245.
					
\textbf{How test will be performed:} Export a model with summary enabled, verify both files are created, and confirm the summary contains relevant model statistics.

\item{test-FR-XM-7 Export Failure Recovery\\}

\textbf{Type:} Functional, Dynamic, Automated

\textbf{Covers:} F241
					
\textbf{Initial State:} Model ready for export.
					
\textbf{Input:} Export operation is interrupted (simulated system failure).
					
\textbf{Output:} Original project data remains intact and unaltered.

\textbf{Test Case Derivation:} Validates that export failures do not corrupt project data, per F241.
					
\textbf{How test will be performed:} Initiate export, simulate failure during the process, and verify the original project file remains unchanged and accessible.

\item{test-FR-XM-8 Export Performance Validation\\}

\textbf{Type:} Functional, Dynamic, Automated

\textbf{Covers:} F241
					
\textbf{Initial State:} Model with MAX\_VOXELS voxels ready for export.
					
\textbf{Input:} User initiates export operation.
					
\textbf{Output:} Export completes at minimum rate of MIN\_RATE (500,000 voxels per second).

\textbf{Test Case Derivation:} Ensures export performance meets specified thresholds, per F241.
					
\textbf{How test will be performed:} Export a large model, measure the export rate, and verify it meets or exceeds the MIN\_RATE threshold.

\end{enumerate}

\subsection{Tests for Nonfunctional Requirements}
\iffalse
\wss{The nonfunctional requirements for accuracy will likely just reference the
  appropriate functional tests from above.  The test cases should mention
  reporting the relative error for these tests.  Not all projects will
  necessarily have nonfunctional requirements related to accuracy.}

\wss{For some nonfunctional tests, you won't be setting a target threshold for
passing the test, but rather describing the experiment you will do to measure
the quality for different inputs.  For instance, you could measure speed versus
the problem size.  The output of the test isn't pass/fail, but rather a summary
table or graph.}

\wss{Tests related to usability could include conducting a usability test and
  survey.  The survey will be in the Appendix.}

\wss{If you introduce static tests in your plan, you need to provide details.
How will they be done?  In cases like code (or document) walkthroughs, who will
be involved? Be specific.}
\fi
\subsubsection{Usability and Look \& Feel}

\begin{enumerate}

\item{test-US-1 Design Acceptance \& Intuitivity Survey\\}

\textbf{Type:} Non-Functional, Dynamic, Manual, Survey-based

\textbf{Covers:} G4-HNF-1
					
\textbf{Initial State:} Application open
					
Input/Condition: User interacts with the application
					
Output/Result: Recorded observations, feedback, and survey response that indicates whether the design works for the user.
					
\textbf{How test will be performed:} Users will be exposed to the application with minimal direction and asked to perform 
some basic tasks. Observations such as where users get confused, when they need to consult the manual, and any comments
made during use will be recorded. Afterwards, users will fill out a survey relating to their satisfaction with the interface, 
as well as how they would rank how intuitive functionalities are on a 5-point scale. Observations and survey responses will then be analysed, with
changes implemented where necessary to improve UI design in the case that expectations are not reached. A draft of the survey
is included in the appendix of this document.

Quantifiable Metric: 90\% of survey responses must rank 4 or above for intuitiveness of each feature.

\item{test-US-2 Accessible Colours Check\\}

\textbf{Type:} Static, Manual

\textbf{Covers:} NF223, GEN1 %Need to add a general requirement to SRS about entire system colours.
					
\textbf{Initial State:} Default colours for materials and application chosen
					
Input/Condition: Constrast analyser run on colours
					
Output/Result: Compliance level meets WCAG standards
					
\textbf{How test will be performed:} A web-based contrast checking tool such as WebAIM will be utilised with both the set of colours chosen 
for material IDs, as well as colours used in the UI (e.g. text, buttons, sidebars).

Acceptance Criteria: All ratios must meet 3:1 contrast ratio for WCAG 2.0 compliance.

\item{test-US-3 Design Ease of Use Test\\}

\textbf{Type:} Non-Functional, Dynamic, Manual, Survey-based

\textbf{Covers:} NF231, GEN2 %need reqs for rest of app about max-interactions
					
\textbf{Initial State:} Application open
					
\textbf{Input:} User completes core tasks (import file, material/magnetization assignment, export file)
					
\textbf{Output:} 90\% of users tested complete tasks in less than MAX\_INTERACTIONS.
					
\textbf{How test will be performed:} After a short explanation of core tasks, users will be asked to perform core tasks. Clicks per task
are recorded and later analysed to verify that expectations are satisfied.

\item{test-US-4 View Model and Layer Focus within application\\}

\textbf{Type:} Non-Functional, Dynamic, Manual

\textbf{Covers:} S3 %Need a better way of citing this?
					
\textbf{Initial State:} Application open with an example project file pre-loaded
					
Input/Condition: Developer attempts to view an individual layer of the model
					
Output/Result: Application displays layer and full model separately, alongside interface to edit the layer.
					
\textbf{How test will be performed:} Tester will select a layer and attempt to view the layer separately, as well as the model as a whole, 
as if they intend to edit voxels within the layer. This confirms that the primary design notion of being able to focus on a specific
layer when editing voxels is satisfied.

\item{test-US-5 Clear and Interpretable Warnings and Errors\\}

\textbf{Type:} Survey-based

\textbf{Covers:} S4.1-7 %better way of citing this as well?
					
\textbf{Initial State:} Application open
					
Input/Condition: User performs interactions that are known to result in error
					
Output/Result: 90\% of respondents agree that errors are informative
					
\textbf{How test will be performed:} The user will be given direction in how to trigger different errors, and be exposed to
all warning and error messages the system can generate. Users will explain whether they find it easy to understand
or not (e.g. pass/fail). Investigation in how to reword prompts will take place for messages that under 90\% of users
do not understand.

Quantifiable Metric: For all messages, 90\% of users 'pass'.

\item{test-US-6 Concise and Interpretable User Manual\\}

\textbf{Type:} Survey-based

\textbf{Covers:} GEN3 %need to add a general requirement about availability/readability of user manual.
					
\textbf{Initial State:} Manual is completed
					
Input/Condition: User reads the full user manual
					
Output/Result: User agrees the manual is helpful with installation, general usage, and interpretation of common errors.
					
\textbf{How test will be performed:} User accesses the manual and reviews it. Afterwards, they give feedback on each section of 
the manual, pointing out areas of confusion or notes that should be added. Feedback is then incorporated upon revision
of the manual.

Quantifiable Metric: Considered an initial success if areas of confusion for each section is less than one.

\end{enumerate}

\subsubsection{Performance \& Scalability}

\begin{enumerate}

\item{test-PS-1 Scalability Validation for Import and Visualization\\}

\textbf{Type:} Non-Functional, Dynamic, Automated

\textbf{Covers:} NF211, NF222, implicitly tests NF212
					
\textbf{Initial State:} Application is open with a set of CAD files of varying sizes prepared.
					
Input/Condition: Sequentially, each CAD file is imported and the resulting voxel model is viewed within the application. 
					
Output/Result: Application completed import within expected times given MIN\_RATE and the estimated number of voxels in the resulting project
file from each input. Resulting voxel models render correctly within the application
					
\textbf{How test will be performed:} Import will be initiated on each CAD file (five files will be prepared, ranging from small (estimated 
100 voxels) to very large (1.5 * MAX\_VOXELS)). Timing will be started upon import, and stopped when the program reports the file was 
successfully imported and opens the project file. It will also be recorded whether the resulting project file renders without error.
If recorded times are not what is expected given MIN\_RATE and the size of the
resulting model, the import module will be investigated for possible sources of optimization. If the resulting file is not openable or does
not render, error messages will be investigated for possible sources of refactoring. %check this with HA times
					
\item{test-PS-2 Performance Validation for Import, Visualization, Editing, and Export\\}

\textbf{Type:} Non-Functional, Dynamic, Automated

\textbf{Covers:} NF212, NF221, NF232, NF242
					
\textbf{Initial State:} Application open with a set of CAD files of average sizes prepared.
					
Input/Condition: Sequentially, each file is imported, core operations completed, and exported
					
Output/Result: Latencies for import/view change/edit/export matches those defined in the above non-functional requirements.
					
\textbf{How test will be performed:} Import will be initiated on each CAD file (three files will be prepared, ranging from small (estimated 100 voxels)
to semi-large (100,000 voxels)). Timing and latency will be recorded for importing the CAD file, changing the orientation of the 
project file, opening layer focus, editing properties of voxel selections of varying sizes (1 voxel, 100, and a full layer), and exporting
the completed project file. These times will then be compared against the defined minimum latencies and expected times given minimum processing
rates. Should performance fall short of these expectations, specific sections that fail will have their code reviewed to identify areas 
that can be optimized. %check this with HA times (in/export)

\item{test-PS-3 Resource Usage Validation Test\\}

\textbf{Type:} Non-Functional, Dynamic, Automated

\textbf{Covers:} E3.4 %check citing this form.
					
\textbf{Initial State:} Application open, with a CAD file of average size prepared.
					
Input/Condition: File is imported, core operations completed, and exported
					
Output/Result: Recorded CPU and GPU usage over the course of the test does not exceed thresholds. %define these somewhere.
					
\textbf{How test will be performed:} Import will be initiated on the CAD file, and core operations will be performed (see test-PS-2's description). 
Throughout the test, CPU and GPU usage of the application on the host computer will be recorded. Should usage exceed defined thresholds, the operation
that lead to increased usage will be investigated for precisely what part of the operation triggered it, with refactoring of the unsatisfactory module.

\end{enumerate}

\subsubsection{Compliance}

\begin{enumerate}

\item{test-CO-1 Windows and Linux Compatibility Test\\}

\textbf{Type:} Non-Functional, Dynamic, Manual

\textbf{Covers:} E3.1
					
\textbf{Initial State:} Logged in to Windows and Linux and application is not installed
					
Input/Condition: User installs the application on both Windows and Linux
					
Output/Result: Application is successfully installed and functions as intended.
					
\textbf{How test will be performed:} The user will download the installer on each operating system, run it, and attempt to open the
application. If the application opens successfully on each OS and can perform basic functionalities, the test is ruled a success.
					
\item{test-CO-2 Import and Export Compatibility Test\\}

\textbf{Type:} Non-Functional, Dynamic, Manual

\textbf{Covers:} E3.2, E3.3
					
\textbf{Initial State:} Application open, with a CAD file of average size prepared.
					
Input/Condition: File is imported, core operations completed, and exported
					
Output/Result: Application converts input file correctly, exported file is correctly formatted
					
\textbf{How test will be performed:} The tester selects the CAD file to be imported, and verifies that the resulting project file 
is correctly generated (e.g. resulting voxel model is an accurate approximation of the original model). The tester then assigns 
dummy data to each layer and exports the project file to the 3D-printer ready format. The resulting file is then checked for
completeness and correct formatting. 

\item{test-CO-3 Project File Fail-safe\\} 

\textbf{Type:} Non-Functional, Dynamic, Manual

\textbf{Covers:} NF241
					
\textbf{Initial State:} Application open, with a completed project file prepared.
					
Input/Condition: File is open, export is initiated and interrupted
					
Output/Result: Project file remains unaltered
					
\textbf{How test will be performed:} The tester begins the export process on the loaded project file, before intentionally ending the program
in the middle of the export. The project file is then checked against a copy of itself made before the test to see if there are any differences;
if there are no differences, the test is considered a success. 

\item{test-CO-4 PEP8 Standard Compliance\\}

\textbf{Type:} Static Analysis

\textbf{Covers:} DP.10
					
\textbf{Initial State:} Application is fully implemented
					
Input/Condition: Codebase is scanned with a linter (e.g. Pylint) %check this against dev plan
					
Output/Result: Code complies with PEP8 Standard
					
\textbf{How test will be performed:} Developer inputs the entirety of the application's backend code into a linting program and takes note of all
warnings. All areas of note are then addressed, with possible refactorings or small fixes resulting.

\end{enumerate}

\subsubsection{Maintainability}

\begin{enumerate}

\item{test-MA-1 Maintainable Codebase\\}

\textbf{Type:} Static Analysis

\textbf{Covers:} DP.10, GEN4 % this req needs to be added to the SRS
					
\textbf{Initial State:} Application is fully implemented, and related documentation is complete
					
Input/Condition: Developers review all code and documentation
					
Output/Result: Code is sufficiently modular, and fully commented; associated documentation is easy to understand
					
\textbf{How test will be performed:} Reviewers inspect all code, evaluating based on modularity (functionality is sufficiently separated,
archicture supports editing one section not mandating updates throughout entire codebase), comments (both function/class headers and notes 
within functions, explaining complex or unclear code), and naming conventions (names of functions, classes, variables are consistent
and clear). Associated documentation is reviewed separately, evaluating based on completeness (all modules are detailed) and readability
(jargon is clearly defined; non-team members can understand the content).

\end{enumerate}

\subsubsection{Security}

Regarding the requirements listed in the Hazard Analysis document, they will not be explicitly tested for; examining each, we see that they
are implicitly tested for by the following system tests and/or unit tests.

\begin{itemize}
  \item SCR1: Implicitly tested for by test-FR-IM-3.
  \item SCR2: Infeasible to test at runtime as this would require access to hardware we have assumed we will not be working with within the SRS.
  Will be verified by unit tests that simulate a system with less memory.
  \item SCR3: Will be verified by unit tests related to the Editing manager that verify latency.
  \item SCR4: Implicitly tested for by test-FR-XM-1 and test-FR-XM-2.
  \item SCR5: Implicitly tested for by test FR-XM-3.
  \item SCR6: Implicitly tested for by test-FR-EM-8.
  \item SCR7: Implicitly tested for by test-FR-EM-9.
  \item SCR8: Will be verified by unit tests related to both the Visualization and Editing managers that verify limits of voxel selection.
  \item SCR9: Implicitly tested for by test-FR-XM-5.
\end{itemize}

\newpage

\subsection{Traceability Between Test Cases and Requirements}

The following tables show the traceability between test cases and their corresponding functional requirements.

\iffalse
\wss{Provide a table that shows which test cases are supporting which
  requirements.}
\fi

\begin{table}[h!]
  \centering
  \caption{Import Manager Tests and Corresponding Requirements}
  \begin{tabular}{c c}
    \hline
    \textbf{Test ID (test-)} & \textbf{Requirement and/or Relevant Documentation} \\ \hline
    FR-IM-1 & F211\\
    FR-IM-2 & F212\\
    FR-IM-3 & F211\\
    FR-IM-4 & F213\\
    FR-IM-5 & F214\\
    FR-IM-6 & F215\\
    FR-IM-7 & F216\\
  \end{tabular}
\end{table}

\begin{table}[h!]
  \centering
  \caption{Visualization Manager Tests and Corresponding Requirements}
  \begin{tabular}{c c}
    \hline
    \textbf{Test ID (test-)} & \textbf{Requirement and/or Relevant Documentation} \\ \hline
    FR-VM-1 & F221\\
    FR-VM-2 & F222\\
    FR-VM-3 & F223\\
    FR-VM-4 & F224\\
    FR-VM-5 & F225\\
    FR-VM-6 & F226\\
    FR-VM-7 & F227\\
    FR-INT-1 & F211, F221\\
  \end{tabular}
\end{table}

\begin{table}[h!]
  \centering
  \caption{Editing Manager Tests and Corresponding Requirements}
  \begin{tabular}{c c}
    \hline
    \textbf{Test ID (test-)} & \textbf{Requirement and/or Relevant Documentation} \\ \hline
    FR-EM-1 & F231\\
    FR-EM-2 & F231\\
    FR-EM-3 & F232\\
    FR-EM-4 & F233\\
    FR-EM-5 & F233\\
    FR-EM-6 & F234\\
    FR-EM-7 & F235\\
    FR-EM-8 & F236\\
    FR-EM-9 & F237\\
    FR-EM-10 & F238\\
    FR-EM-11 & F239\\
    FR-EM-12 & F239\\
    FR-EM-13 & F2310\\
    FR-EM-14 & F2310\\
  \end{tabular}
\end{table}

\begin{table}[h!]
  \centering
  \caption{Export Manager Tests and Corresponding Requirements}
  \begin{tabular}{c c}
    \hline
    \textbf{Test ID (test-)} & \textbf{Requirement and/or Relevant Documentation} \\ \hline
    FR-XM-1 & F241\\
    FR-XM-2 & F241\\
    FR-XM-3 & F242\\
    FR-XM-4 & F243\\
    FR-XM-5 & F244\\
    FR-XM-6 & F245\\
    FR-XM-7 & F241\\
    FR-XM-8 & F241\\
  \end{tabular}
\end{table}

\begin{table}[h!]
  \centering
  \caption{Non-Functional Tests and Corresponding Requirements}
  \begin{tabular}{c c}
    \hline
    \textbf{Test ID (test-)} & \textbf{Requirement and/or Relevant Documentation} \\ \hline
    US-1 & G4-HNF-1\\
    US-2 & NF223, GEN1\\
    US-3 & NF231, GEN2\\
    US-4 & S3\\
    US-5 & S4.1-7\\
    US-6 & GEN3\\
    PS-1 & NF211, NF222, NF212\\
    PS-2 & NF212, NF221, NF232, NF242\\
    PS-3 & E3.4\\
    CO-1 & E3.1\\
    CO-2 & E3.2, E3.3\\
    CO-3 & NF241\\
    CO-4 & DP.10\\
    MA-1 & DP.10, GEN4\\
  \end{tabular}
\end{table}

\newpage

\clearpage

				
\bibliographystyle{plainnat}

\bibliography{../../refs/References}

\clearpage

\section{Appendix}

This is where you can place additional information.

\subsection{Symbolic Parameters}

The definition of the test cases will call for SYMBOLIC\_CONSTANTS.
Their values are defined in this section for easy maintenance.


\begin{itemize}
  \item \textbf{OPTIMAL\_VOXEL\_SIZE} = 5.5nm
  
  This was derived using the optimal voxel size specified by the stakeholder.
  
  \item \textbf{MAX\_VOXELS} = 13,996,800,000 voxels
  
  This considers the maximum number of voxels contained within the entire model.
  
  \item \textbf{MAX\_LAYER\_DISPLAY} = 518,400 voxels
  
  This was derived assuming 960 voxels x 540 voxels per layer. It considers the maximum number of voxels contained within a single layer for a single display window.
  
  \item \textbf{MAX\_DISPLAY} = 103,680,000 voxels
  
  This was derived assuming 960 voxels x 540 voxels per layer. It considers the maximum number of voxels that will be displayed at a given time.
  
  \item \textbf{MIN\_RATE} = 500,000 voxels per second
  
  This considers how fast voxel data will be generated during file import and export.
  
  \item \textbf{MAX\_EDIT\_LATENCY} = 1 second
  
  This considers how quickly the system should respond to model modifications performed by the user.
  
  \item \textbf{MAX\_VIEW\_LATENCY} = 500 ms
  
  This considers how quickly the system should respond to changes in model view orientation by the user.
  
  \item \textbf{MAX\_INTERACTIONS} = 5
  
  This considers the maximum number of steps (e.g. clicks, enter details into a text box) a user should take to complete a part of a task.
\end{itemize}

\subsection{Usability Survey Questions?}

\iffalse
\wss{This is a section that would be appropriate for some projects.}
\fi

Each of the following questions is to be answered via a 5-point scale, with 1 being the least and 5 being 
the most except where otherwise listed.
This list may change or be updated as functionalities are added or removed.

\begin{enumerate}
  \item How difficult was it for you to navigate the overall interface? (5 - very easily, 1 - with difficulty)
  \item Do you consider the current method of moving the model to inspect it intuitive?
  \item How natural do you find the layer select functionality?
  \item How intuitive did you find the method of selecting voxels?
  \item How easy was it for you to assign a magnetization vector to a set of voxels?
  \item How easy was it for you to assign a material to a set of voxels?
  \item Do you find the current signifiers for completed voxels helpful? (5 - very helpful, 1 - not very helpful)
  \item How intuitive did you find the file importing process?
  \item How intuitive did you find the file exporting process?
\end{enumerate}

\newpage{}
\section*{Appendix --- Reflection}

\iffalse
\wss{This section is not required for CAS 741}
\fi

The information in this section will be used to evaluate the team members on the
graduate attribute of Lifelong Learning.

The purpose of reflection questions is to give you a chance to assess your own
learning and that of your group as a whole, and to find ways to improve in the
future. Reflection is an important part of the learning process.  Reflection is
also an essential component of a successful software development process.  

Reflections are most interesting and useful when they're honest, even if the
stories they tell are imperfect. You will be marked based on your depth of
thought and analysis, and not based on the content of the reflections
themselves. Thus, for full marks we encourage you to answer openly and honestly
and to avoid simply writing ``what you think the evaluator wants to hear.''

Please answer the following questions.  Some questions can be answered on the
team level, but where appropriate, each team member should write their own
response:


\begin{enumerate}
  \item What went well while writing this deliverable? 

  Overall this deliverable went well. We were able to write up all the parts of minimal issues. The peer feedback we received during the TA review session was also helpful in ensuring that we were on the right track. One problem that we had was the time constraints, since all the members were busy with other assignments and midterms, so we had to work on this deliverable in a timely manner. Another thing that went well was our ability to communicate with each other, and ensuring that everyone was on the same page.

  \item What pain points did you experience during this deliverable, and how
    did you resolve them?
  

\textbf{Omar: } The pain points that I experienced was trying to free up time to work on this deliverable, as all the members were busy with other assignments and midterms. I worked on the general information section of this deliverable, and the challenges I faced was keeping the information concise and to the point, while still being able to cover all the functional requirements that were listed for both managers on the SRS.
\newline
\newline
\textbf{Daniel: }Responsible for the non-functional system tests, one pain point I dealt with was finding general system requirements that were missing from the SRS as I thought of possible tests for the areas of non-functional testing. Our group chose the Meyer template for the SRS, which I still believe to be the right choice for the project; however, with how the template divides up requirements between components, it becomes possible to miss requirements that the system should have as a whole. For example, not just one feature should have accessible colours, the entire application should. In addition, the SRS makes no mention of the manual, which of course needs to be accessible and user-friendly. As I developed tests that our system needed to satisfy, there were a handful of times I found the SRS did not have these types of general requirements, which will need to be added in a future iteration. 
\newline
\newline
\textbf{Andrew: } One of the main pain points I experienced was ensuring consistency in test numbering and formatting across all the system tests. Initially, there was confusion about the proper test naming convention (test-FR-EM-X vs test-FR-XM-X format) and I had to go back and standardize all the test IDs to follow Khalid's established pattern. Another challenge was removing all non-functional requirements while keeping only the functional tests for Editing Manager and Export Manager, which required careful review to ensure no important content was lost. I resolved these issues by systematically going through the document, updating all test references, and maintaining clear communication with the team about the changes.
\newline
\newline
\textbf{Olivia: } One of the challenges that I came across was trying to decide on the best code coverage tool for the frontend of the system. The process of 3D rendering is fairly complex, which means it will likely be fairly difficult to test. Without any prior experience in the development of 3D rendering, I found it hard to understand which tools would be able to provide enough coverage given the needs of our code. It helped a little once we decided what language we were going to use to develop our frontend. In the end, it still required a fair amount of research, and I decided to just choose one that I think right now would be a good choice. I’ve come to realize that as the project progresses, it’s likely that things will change. This might mean that the chosen coverage tool might change. It doesn’t have to be perfect because adapting to changing needs is part of the development process.
\newline
\newline
\textbf{Khalid: }I was primarily responsible for writing the system tests for the Import and Visualization Managers. The main challenges that I had when writing up this section was ensuring that the tests were comprehensive and covered all the functional requirements that were listed for both manaers on the SRS. Additionally, I had to ensure that the tests were easy to understand and follow, and that they were able to catch all the potential errors and edge cases.

  \item What knowledge and skills will the team collectively need to acquire to
  successfully complete the verification and validation of your project?
  Examples of possible knowledge and skills include dynamic testing knowledge,
  static testing knowledge, specific tool usage, Valgrind etc.  You should look to
  identify at least one item for each team member.

  There are five major knowledges/skills our team will need to acquire:

  \begin{itemize}
    \item \textbf{Knowledge of PyTest Unit testing framework.} With PyTest being the main testing framework for the application's backend, 
    all team members will need to know its conventions.
    \item \textbf{Knowledge of Jest Unit testing framework.} Similar to above reasoning, Jest is the identified framework for testing
    the application's frontend.
    \item \textbf{Organizing Usability tests.} Beyond developing the test with the users themselves, knowing the right observations to take 
    note of is important to integrating feedback correctly.
    \item \textbf{Mutation testing skills.} Mutation testing is only as good as the variety of mutants produced.
    \item \textbf{How to conduct a code review.} Given the potential scale of the codebase and the limited time of team members, effective,
    time-efficient code reviews will be essential.
  \end{itemize}

  \item For each of the knowledge areas and skills identified in the previous
  question, what are at least two approaches to acquiring the knowledge or
  mastering the skill?  Of the identified approaches, which will each team
  member pursue, and why did they make this choice?

  \begin{itemize}
    \item \textbf{Knowledge of PyTest:} Approaches include reviewing existing online tutorials and documentation with the goal of achieving a
    broad understanding of the framework, with an alternative option of experimenting with the tool with a toy example program to gain practical experience.
    \item \textbf{Knowledge of Jest:} Similar to the above, a first option is to review documentation as needed, and a second to experiment creating tests
    in the framework to gain experience.
    \item \textbf{Usability Tests:} A first approach is to review course content from Human-Computer Interfaces, a course that covers how to 
    observe users and derive feedback from not just what they say, but how they behave with the interface, as well as non-obvious prompting methods.
    Another option is to review online material about how to conduct user testing that focuses on usability, and to possibly conduct a trial run 
    with peers.
    \item \textbf{Mutation testing:} Approaches include a reviewing content from Software Testing, a course all team members have taken that not only 
    covers how to come up with test cases, but mutation testing as well, and how to develop varying mutants. Another option is to review 
    academic papers detailing mutatation testing strategies, some of which can be found in the course materials of Software Testing.
    \item \textbf{Code Reviews:} Software Testing also included content detailing how code reviews are conducted in formal contexts, which could be 
    a helpful resource; ideas could be adapted to this project and provide a standard to consult. Another option involves running code reviews in other 
    projects team members are involved in throughout term to gain experience, which would benefit both this project and those involved.
  \end{itemize}

  \textbf{Omar: }
  \textbf{Knowledge of PyTest:} Since I already have some experience with PyTest, I will just focus on reviewing the documentation and the framework to ensure I have a solid understanding of it. Then I will practice with the tool by creating some test cases for the project.
  \textbf{Knowledge of Jest:} As this is a new framework for me, I will use the online tutorials and documentation to help me understand how the framework works. Then I will try testing it out some of the components of this project to gain practical experience.
  \textbf{Usability tests:} For this skill, I think the best way to learn is it to go through the 4HC3 course material, which covers many of the usability testing concepts and how to conduct them. This way I can better understand the process and align it with the project.
  \textbf{Mutation testing:} For this skill, I can go through my last year's notes from the Software Testing course that covers the many type of testing including mutatation testing. This way I can recall the concept and apply it to the project.
  \textbf{Code reviews:} For this skill, I can practice by doing code reviews on the other projects I am involved in. Currently I am working on a project for 4HC3 and 4AL3. This will help me gain experience and also help me understand the process of code reviews.

  \textbf{Daniel: }
  \textit{Knowledge of PyTest:} To become familiar, I will plan to rely on documentation as I feel I am familiar with unit testing frameworks and feel a majority of experience will likely carry over, with differences mainly centered around syntax.

  \textit{Knowledge of Jest:} Similar to PyTest, I plan to rely on documentation, likely moreso due to unfamiliarity with Javascript.

  \textit{Usability tests:} I plan to employ a mix of the two approaches, consulting online materials when I feel the course does not go into enough detail. I also plan to leverage the group project within the course to try out different user observation methods.

  \textit{Mutation testing:} I plan to review the academic papers discussed within Software Testing, as I personally feel the topic was glossed over within the course and is a subject I found intriguing. 

  \textit{Code reviews:} Similar to usability tests, I plan to leverage the other group projects I'm involved in this term, not only to benefit my experience with code reviews but to benefit those projects as well; with a skill that is as collaboration-centric as code reviews, the best way to learn is through experience.
  
  \textbf{Andrew: }
  \textit{Knowledge of PyTest:} I will start by reviewing the official PyTest documentation and online tutorials to understand the framework's core concepts and best practices. Then I'll create a small test project to experiment with different testing patterns and features before applying it to our main codebase.
  
  \textit{Knowledge of Jest:} I have experience with Jest from my co-op work, so I'll leverage that existing knowledge and review the documentation to refresh my understanding of advanced testing patterns and best practices for our frontend components.
  
  \textit{Usability Tests:} I'll review my notes from the Human-Computer Interfaces course to refresh my understanding of user testing methodologies. I'll also look into online resources about conducting effective usability tests and practice with mock scenarios before running actual user tests for our application.
  
  \textit{Mutation testing:} I'll revisit my Software Testing course materials on mutation testing strategies and techniques. I'll also experiment with mutation testing tools on sample code to understand how to create effective mutants and interpret results.
  
  \textit{Code Reviews:} I'll review the formal code review processes covered in Software Testing and adapt them for our project. 

  \textbf{Olivia: }\\
  \textit{Knowledge of PyTest:} Given that I have minimal experience working with PyTest, I will likely integrate a mixture of the two approaches. To ensure I have a solid foundation of knowledge regarding the framework, I’ll start with reviewing online documents. However, once I have that foundational knowledge, I feel that I will learn much more through engaging with the tool.\\

  \textit{Knowledge of Jest:} Similar to my answer above, I will probably combine the approaches. Since I expect the testing of 3D rendition to be much more complex, I feel like I will have to spend more time reviewing the documentation to adequately understand the testing information. However, once I have a good idea of the basics of Jest testing, I feel that active engagement with the tool is better for me, as I tend to have a higher retention rate for what I learn.\\

  \textit{Usability tests:} Due to the fact that this system will only be used by a fairly small user group on a regular basis, I feel that it is more beneficial to focus my learning on how to conduct real user testing trial runs. While generalized fundamentals are important in creating a usable design, it might better suit the scope of the project to focus on usability in specific regards to the select group of current users.\\

  \textit{Mutation testing:} I believe that I will use a combination of the two approaches. When looking to recall general mutation testing information at the beginning stages, reviewing course material will be helpful due to the familiarity with the content structure. However, when deriving the specific mutations for these test cases, I think that engaging with papers and online resources will allow me to better tailor mutation cases to the specific scope of this project.\\

  \textit{Code reviews:} I personally would like to focus on doing code reviews with other team members, as I find the fresh perspective can be helpful. When I spend several hours on a project, I find that it is easy for me to grow complacent during review due to loss of motivation. Therefore, I believe code reviews would be beneficial to actively re-engage myself with the project and take the opportunity to learn from others.

  \textbf{Khalid:}\\
  \textit{Knowledge of PyTest:} I will use the online tutorials and documentation to help me gain a broading understanding of the testing framework. I think this will be a good starting point, as it will help me understand the framework theoretically, and then I can apply it to the project to gain practical experience.
  
  \textit{Knowledge of Jest:} Similar to the above, I will also use the online tutorials and documentation to help me gain a broading understanding of the testing framework. I will also experiment with the tool with a toy example program to gain practical experience.
  
  \textit{Usability tests:} For this skill, I will look over my notes from the 4HC3 (Human-Computer Interfaces) course, which covers how to observe users and derive feedback from both their responses. This was I can recall the content and apply it to the project.
  
  \textit{Mutation testing:} Similar to the above, I will look over my notes from the Software Testing course that covers not just test case development but also mutation testing strategies and how to develop varying mutants. I will also experiement with a simple program to gain practical experience.
  
  \textit{Code reviews:} I will look over my notes from the Software Testing course that covers how to conduct a code review. I will also experiment using some old code from other projects just to get a feel for the process.
  
\end{enumerate}


\end{document}