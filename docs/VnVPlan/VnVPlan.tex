\documentclass[12pt, titlepage]{article}

\usepackage{booktabs}
\usepackage{tabularx}
\usepackage{hyperref}
\hypersetup{
    colorlinks,
    citecolor=blue,
    filecolor=black,
    linkcolor=red,
    urlcolor=blue
}
\usepackage[round]{natbib}

%% Comments

\usepackage{color}

\newif\ifcomments\commentstrue %displays comments
%\newif\ifcomments\commentsfalse %so that comments do not display

\ifcomments
\newcommand{\authornote}[3]{\textcolor{#1}{[#3 ---#2]}}
\newcommand{\todo}[1]{\textcolor{red}{[TODO: #1]}}
\else
\newcommand{\authornote}[3]{}
\newcommand{\todo}[1]{}
\fi

\newcommand{\wss}[1]{\authornote{magenta}{SS}{#1}} 
\newcommand{\plt}[1]{\authornote{cyan}{TPLT}{#1}} %For explanation of the template
\newcommand{\an}[1]{\authornote{cyan}{Author}{#1}}

%% Common Parts

\newcommand{\progname}{Software Engineering} % PUT YOUR PROGRAM NAME HERE
\newcommand{\authname}{Team \#10, Five of a Kind
\\ Omar Abdelhamid
\\ Daniel Maurer
\\ Andrew Bovbel
\\ Olivia Reich
\\ Khalid Farag
} % AUTHOR NAMES                  

\usepackage{hyperref}
    \hypersetup{colorlinks=true, linkcolor=blue, citecolor=blue, filecolor=blue,
                urlcolor=blue, unicode=false}
    \urlstyle{same}
                                


\begin{document}

\title{System Verification and Validation Plan for \progname{}} 
\author{\authname}
\date{\today}
	
\maketitle

\pagenumbering{roman}

\section*{Revision History}

\begin{tabularx}{\textwidth}{p{3cm}p{2cm}X}
\toprule {\bf Date} & {\bf Version} & {\bf Notes}\\
\midrule
Date 1 & 1.0 & Notes\\
Date 2 & 1.1 & Notes\\
\bottomrule
\end{tabularx}

~\\
\wss{The intention of the VnV plan is to increase confidence in the software.
However, this does not mean listing every verification and validation technique
that has ever been devised.  The VnV plan should also be a \textbf{feasible}
plan. Execution of the plan should be possible with the time and team available.
If the full plan cannot be completed during the time available, it can either be
modified to ``fake it'', or a better solution is to add a section describing
what work has been completed and what work is still planned for the future.}

\wss{The VnV plan is typically started after the requirements stage, but before
the design stage.  This means that the sections related to unit testing cannot
initially be completed.  The sections will be filled in after the design stage
is complete.  the final version of the VnV plan should have all sections filled
in.}

\newpage

\tableofcontents

\listoftables
\wss{Remove this section if it isn't needed}

\listoffigures
\wss{Remove this section if it isn't needed}

\newpage

\section{Symbols, Abbreviations, and Acronyms}

\renewcommand{\arraystretch}{1.2}
\begin{tabular}{l l} 
  \toprule		
  \textbf{symbol} & \textbf{description}\\
  \midrule 
  T & Test\\
  \bottomrule
\end{tabular}\\

\wss{symbols, abbreviations, or acronyms --- you can simply reference the SRS
  \citep{SRS} tables, if appropriate}

\wss{Remove this section if it isn't needed}

\newpage

\pagenumbering{arabic}

This document ... \wss{provide an introductory blurb and roadmap of the
  Verification and Validation plan}

\section{General Information}

\subsection{Summary}

\wss{Say what software is being tested.  Give its name and a brief overview of
  its general functions.}

\subsection{Objectives}

\wss{State what is intended to be accomplished.  The objective will be around
  the qualities that are most important for your project.  You might have
  something like: ``build confidence in the software correctness,''
  ``demonstrate adequate usability.'' etc.  You won't list all of the qualities,
  just those that are most important.}

\wss{You should also list the objectives that are out of scope.  You don't have 
the resources to do everything, so what will you be leaving out.  For instance, 
if you are not going to verify the quality of usability, state this.  It is also 
worthwhile to justify why the objectives are left out.}

\wss{The objectives are important because they highlight that you are aware of 
limitations in your resources for verification and validation.  You can't do everything, 
so what are you going to prioritize?  As an example, if your system depends on an 
external library, you can explicitly state that you will assume that external library 
has already been verified by its implementation team.}

\subsection{Challenge Level and Extras}

\wss{State the challenge level (advanced, general, basic) for your project.
Your challenge level should exactly match what is included in your problem
statement.  This should be the challenge level agreed on between you and the
course instructor.  You can use a pull request to update your challenge level
(in TeamComposition.csv or Repos.csv) if your plan changes as a result of the
VnV planning exercise.}

\wss{Summarize the extras (if any) that were tackled by this project.  Extras
can include usability testing, code walkthroughs, user documentation, formal
proof, GenderMag personas, Design Thinking, etc.  Extras should have already
been approved by the course instructor as included in your problem statement.
You can use a pull request to update your extras (in TeamComposition.csv or
Repos.csv) if your plan changes as a result of the VnV planning exercise.}

\subsection{Relevant Documentation}

\wss{Reference relevant documentation.  This will definitely include your SRS
  and your other project documents (design documents, like MG, MIS, etc).  You
  can include these even before they are written, since by the time the project
  is done, they will be written.  You can create BibTeX entries for your
  documents and within those entries include a hyperlink to the documents.}

\citet{SRS}

\wss{Don't just list the other documents.  You should explain why they are relevant and 
how they relate to your VnV efforts.}

\section{Plan}

\wss{Introduce this section.  You can provide a roadmap of the sections to
  come.}

\subsection{Verification and Validation Team}

\wss{Your teammates.  Maybe your supervisor.
  You should do more than list names.  You should say what each person's role is
  for the project's verification.  A table is a good way to summarize this information.}

\subsection{SRS Verification}

\wss{List any approaches you intend to use for SRS verification.  This may
  include ad hoc feedback from reviewers, like your classmates (like your
  primary reviewer), or you may plan for something more rigorous/systematic.}

\wss{If you have a supervisor for the project, you shouldn't just say they will
read over the SRS.  You should explain your structured approach to the review.
Will you have a meeting?  What will you present?  What questions will you ask?
Will you give them instructions for a task-based inspection?  Will you use your
issue tracker?}

\wss{Maybe create an SRS checklist?}

\subsection{Design Verification}

\wss{Plans for design verification}

\wss{The review will include reviews by your classmates}

\wss{Create a checklists?}

\subsection{Verification and Validation Plan Verification}

\wss{The verification and validation plan is an artifact that should also be
verified.  Techniques for this include review and mutation testing.}

\wss{The review will include reviews by your classmates}

\wss{Create a checklists?}

\subsection{Implementation Verification}

\wss{You should at least point to the tests listed in this document and the unit
  testing plan.}

\wss{In this section you would also give any details of any plans for static
  verification of the implementation.  Potential techniques include code
  walkthroughs, code inspection, static analyzers, etc.}

\wss{The final class presentation in CAS 741 could be used as a code
walkthrough.  There is also a possibility of using the final presentation (in
CAS741) for a partial usability survey.}

\subsection{Automated Testing and Verification Tools}

\wss{What tools are you using for automated testing.  Likely a unit testing
  framework and maybe a profiling tool, like ValGrind.  Other possible tools
  include a static analyzer, make, continuous integration tools, test coverage
  tools, etc.  Explain your plans for summarizing code coverage metrics.
  Linters are another important class of tools.  For the programming language
  you select, you should look at the available linters.  There may also be tools
  that verify that coding standards have been respected, like flake9 for
  Python.}

\wss{If you have already done this in the development plan, you can point to
that document.}

\wss{The details of this section will likely evolve as you get closer to the
  implementation.}

\subsection{Software Validation}

\wss{If there is any external data that can be used for validation, you should
  point to it here.  If there are no plans for validation, you should state that
  here.}

\wss{You might want to use review sessions with the stakeholder to check that
the requirements document captures the right requirements.  Maybe task based
inspection?}

\wss{For those capstone teams with an external supervisor, the Rev 0 demo should 
be used as an opportunity to validate the requirements.  You should plan on 
demonstrating your project to your supervisor shortly after the scheduled Rev 0 demo.  
The feedback from your supervisor will be very useful for improving your project.}

\wss{For teams without an external supervisor, user testing can serve the same purpose 
as a Rev 0 demo for the supervisor.}

\wss{This section might reference back to the SRS verification section.}

\section{System Tests}

This section outlines the tests for verifying both functional and nonfunctional requirements of the system. It helps ensure that the system meets stakeholders' expectations. This section includes
tests covering the essential aspects of the systems requirements.

\subsection{Tests for Functional Requirements}

The subsections below outline tests corresponding to functional requirements in the SRS.
Each test is associated with a unique functional area, helping to confirm that the system meets
the specified requirements. 

\subsubsection{Import Manager Tests}

\paragraph{CAD File Import and Project Management Tests}

\begin{enumerate}

\item{test-FR-IM-1 Valid CAD File Import for New Project\\}

Type: Functional, Dynamic, Manual

Covers: F211
					
Initial State: No active project is loaded in the system, and the system is ready to create a new project.
					
Input: A CAD file (STL file) is provided to the system through the import interface by the user.
					
Output: The system creates the new project with the imported CAD file, initiates the voxel slicing process, and the imported model is ready for visualization.

Test Case Derivation: The test case is derived from the functional requirement F211, which states that the Import Manager should allow a user to start a new project with the option to import an initial CAD file from their personal device that will be sliced.
					
How test will be performed: Create a new project on the system and import a CAD file into the new project to verify that the system will correctly process the file and create a new project. The system should be able to handle the CAD file and initiate the voxel slicing process. This will be tested manually by the developer of the system.

\item{test-FR-IM-2 Past Project File Import\\}

Type: Functional, Dynamic, Manual

Covers: F212
					
Initial State: No active project is loaded in the system, and the system is ready to open a past project.
					
Input: A previously saved project file containing magnetization and material properties is provided to the system through the import interface by the user.
					
Output: The system loads the project containing all of the magnetization and material properties preserved. The model is ready for further editing and visualization.

Test Case Derivation: The test case is derived from the functional requirement F212, which states that the Import Manager should allow a user to import a past project file in order to reopen a project with all magnetization and material properties preserved.
					
How test will be performed: Open a previously saved project containing magnetization and material properties on the system. Verify that the system correctly loads the project and all of the magnetization and material properties are preserved. The model should be ready for further editing and visualization. This will be tested manually by the developer of the system.

\item{test-FR-IM-3 Invalid File Format Handling\\}

Type: Functional, Dynamic, Manual

Covers: F211
					
Initial State: No active project is loaded in the system, and the system is ready to import a file.
					
Input: A file with unsupported format (e.g. .txt, .jpg, .pdf) is provided to the system through the import interface by the user.
					
Output: The system rejects the file and displays an appropriate error message indicating the file format is unsupported, without creating a project or crashing.

Test Case Derivation: The test case is derived from the functional requirement F211, which states that the Import Manager should allow a user to start a new project with the option to import an initial CAD file from their personal device that will be sliced. 
					
How test will be performed: Try to import a file with an unsupported format (e.g. .txt, .jpg, .pdf) into the system. Verify that the system correctly rejects the file and displays an appropriate error message indicating the file format is unsupported. This will be tested manually by the developer of the system.

\end{enumerate}

\paragraph{Model Configuration and Slicing Tests}

\begin{enumerate}

\item{test-FR-IM-4 Voxel Size Configuration\\}

Type: Functional, Dynamic, Automated

Covers: F213
					
Initial State: A valid CAD model has been imported and the system is ready to configure the voxel dimensions.
					
Input: User specifies custom voxel dimensions (e.g. 0.1mm x 0.1mm x 0.1mm) by entering the dimensions into the system through the import interface by the user.
					
Output: The system applies the specified voxel dimensions and successfully slices the model into voxels of the requested dimensions.

Test Case Derivation: The test case is derived from the functional requirement F213, which states that the Import Manager should allow a user to configure the voxel dimensions into which the model will be sliced.
					
How test will be performed: Configure the voxel dimensions for the imported model by entering the dimensions into the system through the import interface by the user. Verify that the system correctly applies the specified voxel dimensions and successfully slices the model into voxels of the requested dimensions. This will be tested automatically by the developer of the system by running a script that configures the voxel dimensions and verifies that the system correctly slices the model into voxels of the requested dimensions.

\item{test-FR-IM-5 Model Scaling Functionality\\}

Type: Functional, Dynamic, Manual

Covers: F214
					
Initial State: A valid CAD model has been imported and the system is ready to scale the model.
					
Input: User modifies the model dimensions to the desired printing size by entering the new dimensions into the system through the import interface by the user.
					
Output: The model is scaled to the desired printing size, maintaining its geometric integrity and proportions.

Test Case Derivation: The test case is derived from the functional requirement F214, which states that the Import Manager should allow a user to scale the model to the desired printing size before being sliced into voxels.
					
How test will be performed: Modify the model dimensions to the desired printing size by entering the new dimensions into the system through the import interface by the user. The developer will then manually inspect the rendered 3D model within the application. Verification will involve comparing the visual representation of the sliced model against the original CAD model and the expected scaling of the model to confirm that model integrity and accuracy are preserved as specified in F214

\item{test-FR-IM-6 Partial Voxel Resolution\\}

Type: Functional, Dynamic, Manual

Covers: F215
					
Initial State: A CAD model with complex geometry has been imported and the system is ready to resolve the partial voxels.
					
Input: A model containing surfaces that intersect voxel boundaries is imported by the user through the import interface.
					
Output: The system correctly resolves partial voxels and preserves model integrity and accuracy.

Test Case Derivation: The test case is derived from the functional requirement F215, which states that the Import Manager shall resolve partial voxels and interpret them in the best way that preserves the integrity and accuracy of the provided model.
					
How test will be performed: Import a model containing surfaces that intersect voxel boundaries. The developer will then manually inspect the rendered 3D model within the application. Verification will involve comparing the visual representation of the sliced model against the original CAD model and the expected resolution of partial voxels to confirm that model integrity and accuracy are preserved as specified in F215

\item{test-FR-IM-7 Model Division for Display\\}

Type: Functional, Dynamic, Manual

Covers: F216
					
Initial State: A large CAD model has been imported and the system is ready to divide the model into manageable display sections.
					
Input: A model containing more than MAX\_DISPLAY voxels is imported by the user through the import interface.
					
Output: The system automatically partitions the model into manageable display sections that do not exceed the MAX\_DISPLAY threshold, enabling smooth visualization.

Test Case Derivation: The test case is derived from the functional requirement F216, which states that the Import Manager shall partition the model into user-manageable display sections that do not surpass the MAX\_DISPLAY threshold.
					
How test will be performed: Import a model containing more than MAX\_DISPLAY voxels. The developer will then manually inspect the rendered 3D model within the application. Verification will involve comparing the visual representation of the sliced model against the original CAD model and the expected partitioning of the model to confirm that model integrity and accuracy are preserved as specified in F216

\end{enumerate}

\subsubsection{Visualization Manager Tests}

\paragraph{3D Model Rendering and Display Tests}

\begin{enumerate}

\item{test-FR-VM-1 3D Model Rendition with Clear Voxel Visualization\\}

Type: Functional, Dynamic, Manual

Covers: F221
					
Initial State: A CAD model has been imported and sliced into voxels by the Import Manager and the system is ready to render the model.
					
Input: Sliced voxel data from Import Manager is provided to the Visualization Manager.
					
Output: The system renders a 3D model with clear visualization of individual voxels, maintaining geometric accuracy and providing distinct voxel boundaries.

Test Case Derivation: The test case is derived from the functional requirement F221, which states that the Visualization Manager shall recreate a 3D model with clear visualization of sliced voxels.
					
How test will be performed: Render a 3D model with clear visualization of sliced voxels by providing the Visualization Manager with sliced voxel data from the Import Manager. Verify that the system correctly renders the model and displays the individual voxels with proper boundaries and geometric accuracy. This will be tested manually by the developer of the system by providing the system with a CAD model and verifying that the system correctly renders the model and displays the individual voxels with proper boundaries and geometric accuracy. 

\item{test-FR-VM-2 Partition Navigation Functionality\\}

Type: Functional, Dynamic, Manual

Covers: F222
					
Initial State: A large model has been divided into multiple display partitions and the system is ready to navigate between the partitions.
					
Input: User navigates between model partitions by clicking on the different display sections of the model.
					
Output: The system provides smooth navigation between partitions, allowing users to access all model sections without performance degradation.

Test Case Derivation: The test case is derived from the functional requirement F222, which states that the Visualization Manager shall provide users with simplistic navigation across user-manageable display sections to ensure access to all model partitions.
					
How test will be performed: Navigate between model partitions by clicking on the different display sections of the model. Verify that the system correctly provides smooth navigation between partitions, allowing users to access all model sections without performance degradation. This will be tested manually by the developer of the system.

\item{test-FR-VM-3 Multi-Perspective 3D Model Interaction\\}

Type: Functional, Dynamic, Manual

Covers: F223
					
Initial State: A 3D model is rendered and displayed and the system is ready to interact with the model from different perspectives.
					
Input: User rotates and zooms the model to view it from different perspectives using intuitive interface controls.
					
Output: The system correctly provides intuitive interface controls that allow seamless navigation across multiple perspectives of the 3D model with responsive interaction.

Test Case Derivation: The test case is derived from the functional requirement F223, which states that the Visualization Manager shall provide users with an intuitive interface that permits seamless navigation across multiple perspectives of the 3D model.
					
How test will be performed: Rotate and zoom the model to view it from different perspectives using intuitive interface controls. Verify that the system correctly provides intuitive interface controls that allow seamless navigation across multiple perspectives of the 3D model with responsive interaction. This will be tested manually by the developer of the system.

\end{enumerate}

\paragraph{Layer Focus and Voxel Selection Tests}

\begin{enumerate}

\item{test-FR-VM-4 Layer Isolation Functionality\\}

Type: Functional, Dynamic, Manual

Covers: F224
					
Initial State: A 3D model with multiple layers is displayed and the system is ready to isolate a specific layer.
					
Input: User selects a specific layer to focus on by clicking on the layer in the display.
					
Output: The system isolates the selected layer, rendering all other voxels irrelevant or transparent, facilitating property assignment on the focused layer.

Test Case Derivation: The test case is derived from the functional requirement F224, which states that the Visualization Manager shall allow users to isolate a specific layer to facilitate property assignments, rendering all other voxels irrelevant while that layer is in focus.
					
How test will be performed: Select a specific layer to focus on by clicking on the layer in the display. Verify that the system correctly isolates the selected layer and renders all other voxels irrelevant while that layer is in focus. This will be tested manually by the developer of the system.

\item{test-FR-VM-5 Voxel Selection Highlighting\\}

Type: Functional, Dynamic, Manual

Covers: F225
					
Initial State: A specific layer is in focus and displayed and the system is ready to highlight the selected voxels.
					
Input: User selects individual voxels or groups of voxels within the focused layer by clicking on the voxels in the display.
					
Output: The system correctly provides clear visual feedback showing which voxels are currently selected using distinct highlighting.

Test Case Derivation: The test case is derived from the functional requirement F225, which states that the Visualization Manager shall provide users with visualization that showcases which voxels are currently selected within a specific layer.
					
How test will be performed: Select individual voxels or groups of voxels within the focused layer by clicking on the voxels in the display. Verify that the system correctly provides clear visual feedback showing which voxels are currently selected using distinct highlighting. This will be tested manually by the developer of the system.

\end{enumerate}

\paragraph{Material and Magnetization Tracking Tests}

\begin{enumerate}

\item{test-FR-VM-6 Material Assignment Visual Tracking\\}

Type: Functional, Dynamic, Manual

Covers: F226
					
Initial State: A model with unassigned materials is displayed and the system is ready to track the material assignment.
					
Input: User assigns material IDs to various voxels by clicking on the voxels in the display and assigning the material IDs.
					
Output: The system updates the voxel colors to indicate material assignment completeness, providing clear visual feedback about the assignment status.

Test Case Derivation: The test case is derived from the functional requirement F226, which states that the Visualization Manager shall integrate easy tracking of voxels that have been assigned material IDs by adjusting the colour of the voxel to indicate assignment completeness.
					
How test will be performed: Assign material IDs to various voxels by clicking on the voxels in the display and assigning the material IDs. Verify that the system correctly updates the voxel colors to indicate material assignment completeness, providing clear visual feedback about assignment status. This will be tested manually by the developer of the system.

\item{test-FR-VM-7 Magnetization Assignment Visual Tracking\\}

Type: Functional, Dynamic, Manual

Covers: F227
					
Initial State: A model with unassigned magnetization vectors is displayed and the system is ready to track the magnetization assignment.
					
Input: User assigns magnetization vectors to various voxels by clicking on the voxels in the display and assigning the magnetization vectors.
					
Output: The system updates the voxel colors to indicate magnetization assignment completeness, providing clear visual feedback about the assignment status.

Test Case Derivation: The test case is derived from the functional requirement F227, which states that the Visualization Manager shall integrate easy tracking of voxels that have been assigned magnetization vectors by adjusting the colour of the voxel to indicate assignment completeness.
					
How test will be performed: Assign magnetization vectors to various voxels by clicking on the voxels in the display and assigning the magnetization vectors. Verify that the system correctly updates the voxel colors to indicate magnetization assignment completeness, providing clear visual feedback about the assignment status. This will be tested manually by the developer of the system.

\end{enumerate}

\subsubsection{Integration Tests Between Import and Visualization Managers}


\paragraph{Data Flow and Communication Tests}

\begin{enumerate}

\item{test-FR-INT-1 Import to Visualization Data Transfer\\}

Type: Functional, Dynamic, Automated

Covers: F211, F221
					
Initial State: No active project is loaded in the system, and the system is ready to import a file.
					
Input: A valid CAD file is imported by the user through the import interface.
					
Output: The Visualization Manager receives complete voxel data and successfully renders the 3D model without data loss or corruption.

Test Case Derivation: The test case is derived from the functional requirement F211 and F221, which states that the Import Manager should allow a user to import a CAD file from their personal device that will be sliced and the Visualization Manager should successfully render the 3D model without data loss or corruption.
					
How test will be performed: Import a valid CAD file. Verify that the system correctly imports the CAD file and the Visualization Manager successfully renders the 3D model without data loss or corruption. This will be tested automatically by the developer of the system by running a script that imports a CAD file and verifies that the system correctly imports the CAD file and the Visualization Manager successfully renders the 3D model without data loss or corruption.


\end{enumerate}

\subsection{Tests for Nonfunctional Requirements}
\iffalse
\wss{The nonfunctional requirements for accuracy will likely just reference the
  appropriate functional tests from above.  The test cases should mention
  reporting the relative error for these tests.  Not all projects will
  necessarily have nonfunctional requirements related to accuracy.}

\wss{For some nonfunctional tests, you won't be setting a target threshold for
passing the test, but rather describing the experiment you will do to measure
the quality for different inputs.  For instance, you could measure speed versus
the problem size.  The output of the test isn't pass/fail, but rather a summary
table or graph.}

\wss{Tests related to usability could include conducting a usability test and
  survey.  The survey will be in the Appendix.}

\wss{Static tests, review, inspections, and walkthroughs, will not follow the
format for the tests given below.}

\wss{If you introduce static tests in your plan, you need to provide details.
How will they be done?  In cases like code (or document) walkthroughs, who will
be involved? Be specific.}
\fi
\subsubsection{Usability and Look \& Feel}

\begin{enumerate}

\item{test-US-1 Design Acceptance \& Intuitivity Survey\\}

Type: Non-Functional, Dynamic, Manual, Survey-based

Covers: G4-HNF-1
					
Initial State: Application open
					
Input/Condition: User interacts with the application
					
Output/Result: Recorded observations, feedback, and survey response that indicates whether the design works for the user.
					
How test will be performed: Users will be exposed to the application with minimal direction and asked to perform 
some basic tasks. Observations such as where users get confused, when they need to consult the manual, and any comments
made during use will be recorded. Afterwards, users will fill out a survey relating to their satisfaction with the interface, 
as well as how they would rank how intuitive functionalities are on a 5-point scale. Observations and survey responses will then be analysed, with
changes implemented where necessary to improve UI design in the case that expectations are not reached. A draft of the survey
is included in the appendix of this document.

Quantifiable Metric: 90\% of survey responses must rank 4 or above for intuitiveness of each feature.

\item{test-US-2 Accessible Colours Check\\}

Type: Static, Manual

Covers: NF223, GEN1 %Need to add a general requirement to SRS about entire system colours.
					
Initial State: Default colours for materials and application chosen
					
Input/Condition: Constrast analyser run on colours
					
Output/Result: Compliance level meets WCAG standards
					
How test will be performed: A web-based contrast checking tool such as WebAIM will be utilised with both the set of colours chosen 
for material IDs, as well as colours used in the UI (e.g. text, buttons, sidebars).

Acceptance Criteria: All ratios must meet 3:1 contrast ratio for WCAG 2.0 compliance.

\item{test-US-3 Design Ease of Use Test\\}

Type: Non-Functional, Dynamic, Manual, Survey-based

Covers: NF231, GEN2 %need reqs for rest of app about max-interactions
					
Initial State: Application open
					
Input: User completes core tasks (import file, material/magnetization assignment, export file)
					
Output: 90\% of users tested complete tasks in less than MAX\_INTERACTIONS.
					
How test will be performed: After a short explanation of core tasks, users will be asked to perform core tasks. Clicks per task
are recorded and later analysed to verify that expectations are satisfied.

\item{test-US-4 View Model and Layer Focus within application\\}

Type: Non-Functional, Dynamic, Manual

Covers: S3 %Need a better way of citing this?
					
Initial State: Application open with an example project file pre-loaded
					
Input/Condition: Developer attempts to view an individual layer of the model
					
Output/Result: Application displays layer and full model separately, alongside interface to edit the layer.
					
How test will be performed: Tester will select a layer and attempt to view the layer separately, as well as the model as a whole, 
as if they intend to edit voxels within the layer. This confirms that the primary design notion of being able to focus on a specific
layer when editing voxels is satisfied.

\item{test-US-5 Clear and Interpretable Warnings and Errors\\}

Type: Survey-based

Covers: S4.1-7 %better way of citing this as well?
					
Initial State: Application open
					
Input/Condition: User performs interactions that are known to result in error
					
Output/Result: 90\% of respondents agree that errors are informative
					
How test will be performed: The user will be given direction in how to trigger different errors, and be exposed to
all warning and error messages the system can generate. Users will explain whether they find it easy to understand
or not (e.g. pass/fail). Investigation in how to reword prompts will take place for messages that under 90\% of users
do not understand.

Quantifiable Metric: For all messages, 90\% of users 'pass'.

\item{test-US-6 Concise and Interpretable User Manual\\}

Type: Survey-based

Covers: GEN3 %need to add a general requirement about availability/readability of user manual.
					
Initial State: Manual is completed
					
Input/Condition: User reads the full user manual
					
Output/Result: User agrees the manual is helpful with installation, general usage, and interpretation of common errors.
					
How test will be performed: User accesses the manual and reviews it. Afterwards, they give feedback on each section of 
the manual, pointing out areas of confusion or notes that should be added. Feedback is then incorporated upon revision
of the manual.

Quantifiable Metric: Considered an initial success if areas of confusion for each section is less than one.

\end{enumerate}

\subsubsection{Performance \& Scalability}

\begin{enumerate}

\item{test-PS-1 Scalability Validation for Import and Visualization\\}

Type: Non-Functional, Dynamic, Automated

Covers: NF211, NF222, implicitly tests NF212
					
Initial State: Application is open with a set of CAD files of varying sizes prepared.
					
Input/Condition: Sequentially, each CAD file is imported and the resulting voxel model is viewed within the application. 
					
Output/Result: Application completed import within expected times given MIN\_RATE and the estimated number of voxels in the resulting project
file from each input. Resulting voxel models render correctly within the application
					
How test will be performed: Import will be initiated on each CAD file (five files will be prepared, ranging from small (estimated 
100 voxels) to very large (1.5 * MAX\_VOXELS)). Timing will be started upon import, and stopped when the program reports the file was 
successfully imported and opens the project file. It will also be recorded whether the resulting project file renders without error.
If recorded times are not what is expected given MIN\_RATE and the size of the
resulting model, the import module will be investigated for possible sources of optimization. If the resulting file is not openable or does
not render, error messages will be investigated for possible sources of refactoring. %check this with HA times
					
\item{test-PS-2 Performance Validation for Import, Visualization, Editing, and Export\\}

Type: Non-Functional, Dynamic, Automated

Covers: NF212, NF221, NF232, NF242
					
Initial State: Application open with a set of CAD files of average sizes prepared.
					
Input/Condition: Sequentially, each file is imported, core operations completed, and exported
					
Output/Result: Latencies for import/view change/edit/export matches those defined in the above non-functional requirements.
					
How test will be performed: Import will be initiated on each CAD file (three files will be prepared, ranging from small (estimated 100 voxels)
to semi-large (100,000 voxels)). Timing and latency will be recorded for importing the CAD file, changing the orientation of the 
project file, opening layer focus, editing properties of voxel selections of varying sizes (1 voxel, 100, and a full layer), and exporting
the completed project file. These times will then be compared against the defined minimum latencies and expected times given minimum processing
rates. Should performance fall short of these expectations, specific sections that fail will have their code reviewed to identify areas 
that can be optimized. %check this with HA times (in/export)

\item{test-PS-3 Resource Usage Validation Test\\}

Type: Non-Functional, Dynamic, Automated

Covers: E3.4 %check citing this form.
					
Initial State: Application open, with a CAD file of average size prepared.
					
Input/Condition: File is imported, core operations completed, and exported
					
Output/Result: Recorded CPU and GPU usage over the course of the test does not exceed thresholds. %define these somewhere.
					
How test will be performed: Import will be initiated on the CAD file, and core operations will be performed (see test-PS-2's description). 
Throughout the test, CPU and GPU usage of the application on the host computer will be recorded. Should usage exceed defined thresholds, the operation
that lead to increased usage will be investigated for precisely what part of the operation triggered it, with refactoring of the unsatisfactory module.

\end{enumerate}

\subsubsection{Compliance}

\begin{enumerate}

\item{test-CO-1 Windows and Linux Compatibility Test\\}

Type: Non-Functional, Dynamic, Manual

Covers: E3.1
					
Initial State: Logged in to Windows and Linux and application is not installed
					
Input/Condition: User installs the application on both Windows and Linux
					
Output/Result: Application is successfully installed and functions as intended.
					
How test will be performed: The user will download the installer on each operating system, run it, and attempt to open the
application. If the application opens successfully on each OS and can perform basic functionalities, the test is ruled a success.
					
\item{test-CO-2 Import and Export Compatibility Test\\}

Type: Non-Functional, Dynamic, Manual

Covers: E3.2, E3.3
					
Initial State: Application open, with a CAD file of average size prepared.
					
Input/Condition: File is imported, core operations completed, and exported
					
Output/Result: Application converts input file correctly, exported file is correctly formatted
					
How test will be performed: The tester selects the CAD file to be imported, and verifies that the resulting project file 
is correctly generated (e.g. resulting voxel model is an accurate approximation of the original model). The tester then assigns 
dummy data to each layer and exports the project file to the 3D-printer ready format. The resulting file is then checked for
completeness and correct formatting. 

\item{test-CO-3 Project File Fail-safe\\} 

Type: Non-Functional, Dynamic, Manual

Covers: NF241
					
Initial State: Application open, with a completed project file prepared.
					
Input/Condition: File is open, export is initiated and interrupted
					
Output/Result: Project file remains unaltered
					
How test will be performed: The tester begins the export process on the loaded project file, before intentionally ending the program
in the middle of the export. The project file is then checked against a copy of itself made before the test to see if there are any differences;
if there are no differences, the test is considered a success. 

\item{test-CO-4 PEP8 Standard Compliance\\}

Type: Static Analysis

Covers: DP.10
					
Initial State: Application is fully implemented
					
Input/Condition: Codebase is scanned with a linter (e.g. Pylint) %check this against dev plan
					
Output/Result: Code complies with PEP8 Standard
					
How test will be performed: Developer inputs the entirety of the application's backend code into a linting program and takes note of all
warnings. All areas of note are then addressed, with possible refactorings or small fixes resulting.

\end{enumerate}

\subsubsection{Maintainability}

\begin{enumerate}

\item{test-MA-1 Maintainable Codebase\\}

Type: Static Analysis

Covers: DP.10, GEN4 % this req needs to be added to the SRS
					
Initial State: Application is fully implemented, and related documentation is complete
					
Input/Condition: Developers review all code and documentation
					
Output/Result: Code is sufficiently modular, and fully commented; associated documentation is easy to understand
					
How test will be performed: Reviewers inspect all code, evaluating based on modularity (functionality is sufficiently separated,
archicture supports editing one section not mandating updates throughout entire codebase), comments (both function/class headers and notes 
within functions, explaining complex or unclear code), and naming conventions (names of functions, classes, variables are consistent
and clear). Associated documentation is reviewed separately, evaluating based on completeness (all modules are detailed) and readability
(jargon is clearly defined; non-team members can understand the content).

\end{enumerate}

\subsubsection{Security}

Regarding the requirements listed in the Hazard Analysis document, they will not be explicitly tested for; examining each, we see that they
are implicitly tested for by the following system tests and/or unit tests.

\begin{itemize}
  \item SCR1: Implicitly tested for by test-FR-IM-3.
  \item SCR2: Infeasible to test at runtime as this would require access to hardware we have assumed we will not be working with within the SRS.
  Will be verified by unit tests that simulate a system with less memory.
  \item SCR3: Will be verified by unit tests related to the Editing manager that verify latency.
  \item SCR4: Implicitly tested for by test-FR-XM-1 and test-FR-XM-2.
  \item SCR5: Implicitly tested for by test FR-XM-3.
  \item SCR6: Implicitly tested for by test-FR-EM-6.
  \item SCR7: Implicitly tested for by test-FR-EM-7.
  \item SCR8: Will be verified by unit tests related to both the Visualization and Editing managers that verify limits of voxel selection.
  \item SCR9: Implicitly tested for by test-FR-XM-5.
\end{itemize}


\subsection{Traceability Between Test Cases and Requirements}

\iffalse
\wss{Provide a table that shows which test cases are supporting which
  requirements.}
\fi

\begin{table}[h!]
  \centering
  \caption{Import Manager Tests and Corresponding Requirements}
  \begin{tabular}{c c}
    \hline
    \textbf{Test ID (test-)} & \textbf{Requirement and/or Relevant Documentation} \\ \hline
    FR-IM-1 & F211\\
    FR-IM-2 & F212\\
    FR-IM-3 & F211\\
    FR-IM-4 & F213\\
    FR-IM-5 & F214\\
    FR-IM-6 & F215\\
    FR-IM-7 & F216\\
  \end{tabular}
\end{table}

\begin{table}[h!]
  \centering
  \caption{Visualization Manager Tests and Corresponding Requirements}
  \begin{tabular}{c c}
    \hline
    \textbf{Test ID (test-)} & \textbf{Requirement and/or Relevant Documentation} \\ \hline
    FR-VM-1 & F221\\
    FR-VM-2 & F222\\
    FR-VM-3 & F223\\
    FR-VM-4 & F224\\
    FR-VM-5 & F225\\
    FR-VM-6 & F226\\
    FR-VM-7 & F227\\
    FR-INT-1 & F211, F221\\
  \end{tabular}
\end{table}

\begin{table}[h!]
  \centering
  \caption{Non-Functional Tests and Corresponding Requirements}
  \begin{tabular}{c c}
    \hline
    \textbf{Test ID (test-)} & \textbf{Requirement and/or Relevant Documentation} \\ \hline
    US-1 & G4-HNF-1\\
    US-2 & NF223, GEN1\\
    US-3 & NF231, GEN2\\
    US-4 & S3\\
    US-5 & S4.1-7\\
    US-6 & GEN3\\
    PS-1 & NF211, NF222, NF212\\
    PS-2 & NF212, NF221, NF232, NF242\\
    PS-3 & E3.4\\
    CO-1 & E3.1\\
    CO-2 & E3.2, E3.3\\
    CO-3 & NF241\\
    CO-4 & DP.10\\
    MA-1 & DP.10, GEN4\\
  \end{tabular}
\end{table}

\iffalse
\section{Unit Test Description}

\wss{This section should not be filled in until after the MIS (detailed design
  document) has been completed.}

\wss{Reference your MIS (detailed design document) and explain your overall
philosophy for test case selection.}  

\wss{To save space and time, it may be an option to provide less detail in this section.  
For the unit tests you can potentially layout your testing strategy here.  That is, you 
can explain how tests will be selected for each module.  For instance, your test building 
approach could be test cases for each access program, including one test for normal behaviour 
and as many tests as needed for edge cases.  Rather than create the details of the input 
and output here, you could point to the unit testing code.  For this to work, you code 
needs to be well-documented, with meaningful names for all of the tests.}


\subsection{Unit Testing Scope}

\wss{What modules are outside of the scope.  If there are modules that are
  developed by someone else, then you would say here if you aren't planning on
  verifying them.  There may also be modules that are part of your software, but
  have a lower priority for verification than others.  If this is the case,
  explain your rationale for the ranking of module importance.}

\subsection{Tests for Functional Requirements}

\wss{Most of the verification will be through automated unit testing.  If
  appropriate specific modules can be verified by a non-testing based
  technique.  That can also be documented in this section.}

\subsubsection{Module 1}

\wss{Include a blurb here to explain why the subsections below cover the module.
  References to the MIS would be good.  You will want tests from a black box
  perspective and from a white box perspective.  Explain to the reader how the
  tests were selected.}

\fi

\iffalse
\begin{enumerate}

\item{test-id1\\}

Type: \wss{Functional, Dynamic, Manual, Automatic, Static etc. Most will
  be automatic}
					
Initial State: 
					
Input: 
					
Output: \wss{The expected result for the given inputs}

Test Case Derivation: \wss{Justify the expected value given in the Output field}

How test will be performed: 
					
\item{test-id2\\}

Type: \wss{Functional, Dynamic, Manual, Automatic, Static etc. Most will
  be automatic}
					
Initial State: 
					
Input: 
					
Output: \wss{The expected result for the given inputs}

Test Case Derivation: \wss{Justify the expected value given in the Output field}

How test will be performed: 

\item{...\\}
    
\end{enumerate}

\subsubsection{Module 2}
\

...

\subsection{Tests for Nonfunctional Requirements}

\wss{If there is a module that needs to be independently assessed for
  performance, those test cases can go here.  In some projects, planning for
  nonfunctional tests of units will not be that relevant.}

\wss{These tests may involve collecting performance data from previously
  mentioned functional tests.}

\subsubsection{Module ?}
		
\begin{enumerate}

\item{test-id1\\}

Type: \wss{Functional, Dynamic, Manual, Automatic, Static etc. Most will
  be automatic}
					
Initial State: 
					
Input/Condition: 
					
Output/Result: 
					
How test will be performed: 
					
\item{test-id2\\}

Type: Functional, Dynamic, Manual, Static etc.
					
Initial State: 
					
Input: 
					
Output: 
					
How test will be performed: 

\end{enumerate}

\subsubsection{Module ?}

...

\subsection{Traceability Between Test Cases and Modules}

\wss{Provide evidence that all of the modules have been considered.}

\fi
				
\bibliographystyle{plainnat}

\bibliography{../../refs/References}

\newpage

\section{Appendix}

This is where you can place additional information.

\subsection{Symbolic Parameters}

The definition of the test cases will call for SYMBOLIC\_CONSTANTS.
Their values are defined in this section for easy maintenance.

\begin{itemize}
  \item \textbf{OPTIMAL\_VOXEL\_SIZE} = 5.5nm
  
  This was derived using the optimal voxel size specified by the stakeholder.
  
  \item \textbf{MAX\_VOXELS} = 13,996,800,000 voxels
  
  This considers the maximum number of voxels contained within the entire model.
  
  \item \textbf{MAX\_LAYER\_DISPLAY} = 518,400 voxels
  
  This was derived assuming 960 voxels x 540 voxels per layer. It considers the maximum number of voxels contained within a single layer for a single display window.
  
  \item \textbf{MAX\_DISPLAY} = 103,680,000 voxels
  
  This was derived assuming 960 voxels x 540 voxels per layer. It considers the maximum number of voxels that will be displayed at a given time.
  
  \item \textbf{MIN\_RATE} = 500,000 voxels per second
  
  This considers how fast voxel data will be generated during file import and export.
  
  \item \textbf{MAX\_EDIT\_LATENCY} = 1 second
  
  This considers how quickly the system should respond to model modifications performed by the user.
  
  \item \textbf{MAX\_VIEW\_LATENCY} = 500 ms
  
  This considers how quickly the system should respond to changes in model view orientation by the user.
  
  \item \textbf{MAX\_INTERACTIONS} = 5
  
  This considers the maximum number of steps (e.g. clicks, enter details into a text box) a user should take to complete a part of a task.
\end{itemize}

\subsection{Usability Survey Questions?}

\iffalse
\wss{This is a section that would be appropriate for some projects.}
\fi

Each of the following questions is to be answered via a 5-point scale, with 1 being the least and 5 being 
the most except where otherwise listed.
This list may change or be updated as functionalities are added or removed.

\begin{enumerate}
  \item How difficult was it for you to navigate the overall interface? (5 - very easily, 1 - with difficulty)
  \item Do you consider the current method of moving the model to inspect it intuitive?
  \item How natural do you find the layer select functionality?
  \item How intuitive did you find the method of selecting voxels?
  \item How easy was it for you to assign a magnetization vector to a set of voxels?
  \item How easy was it for you to assign a material to a set of voxels?
  \item Do you find the current signifiers for completed voxels helpful? (5 - very helpful, 1 - not very helpful)
  \item How intuitive did you find the file importing process?
  \item How intuitive did you find the file exporting process?
\end{enumerate}

\newpage{}
\section*{Appendix --- Reflection}

\iffalse
\wss{This section is not required for CAS 741}
\fi

The information in this section will be used to evaluate the team members on the
graduate attribute of Lifelong Learning.

The purpose of reflection questions is to give you a chance to assess your own
learning and that of your group as a whole, and to find ways to improve in the
future. Reflection is an important part of the learning process.  Reflection is
also an essential component of a successful software development process.  

Reflections are most interesting and useful when they're honest, even if the
stories they tell are imperfect. You will be marked based on your depth of
thought and analysis, and not based on the content of the reflections
themselves. Thus, for full marks we encourage you to answer openly and honestly
and to avoid simply writing ``what you think the evaluator wants to hear.''

Please answer the following questions.  Some questions can be answered on the
team level, but where appropriate, each team member should write their own
response:


\begin{enumerate}
  \item What went well while writing this deliverable? 

  Overall this deliverable went well. We were able to write up all the parts of minimal issues. The peer feedback we received during the TA review session was also helpful in ensuring that we were on the right track. One problem that we had was the time constraints, since all the members were busy with other assignments and midterms, so we had to work on this deliverable in a timely manner. Another thing that went well was our ability to communicate with each other, and ensuring that everyone was on the same page.

  \item What pain points did you experience during this deliverable, and how
    did you resolve them?
  

\textbf{Omar: }
\newline
\textbf{Daniel: } 
\newline
\newline
\textbf{Andrew: } 
\newline
\newline
\textbf{Olivia: } 
\newline
\newline
\textbf{Khalid: }I was primarily responsible for writing the system tests for the Import and Visualization Managers. The main challenges that I had when writing up this section was ensuring that the tests were comprehensive and covered all the functional requirements that were listed for both manaers on the SRS. Additionally, I had to ensure that the tests were easy to understand and follow, and that they were able to catch all the potential errors and edge cases.

  \item What knowledge and skills will the team collectively need to acquire to
  successfully complete the verification and validation of your project?
  Examples of possible knowledge and skills include dynamic testing knowledge,
  static testing knowledge, specific tool usage, Valgrind etc.  You should look to
  identify at least one item for each team member.

  There are five major knowledges/skills our team will need to acquire:

  \begin{itemize}
    \item \textbf{Knowledge of PyTest Unit testing framework.} With PyTest being the main testing framework for the application's backend, 
    all team members will need to know its conventions.
    \item \textbf{Knowledge of Jest Unit testing framework.} Similar to above reasoning, Jest is the identified framework for testing
    the application's frontend.
    \item \textbf{Organizing Usability tests.} Beyond developing the test with the users themselves, knowing the right observations to take 
    note of is important to integrating feedback correctly.
    \item \textbf{Mutation testing skills.} Mutation testing is only as good as the variety of mutants produced.
    \item \textbf{How to conduct a code review.} Given the potential scale of the codebase and the limited time of team members, effective,
    time-efficient code reviews will be essential.
  \end{itemize}

  \item For each of the knowledge areas and skills identified in the previous
  question, what are at least two approaches to acquiring the knowledge or
  mastering the skill?  Of the identified approaches, which will each team
  member pursue, and why did they make this choice?

  \begin{itemize}
    \item \textbf{Knowledge of PyTest:} Approaches include reviewing existing online tutorials and documentation with the goal of achieving a
    broad understanding of the framework, with an alternative option of experimenting with the tool with a toy example program to gain practical experience.
    \item \textbf{Knowledge of Jest:} Similar to the above, a first option is to review documentation as needed, and a second to experiment creating tests
    in the framework to gain experience.
    \item \textbf{Usability Tests:} A first approach is to review course content from Human-Computer Interfaces, a course that covers how to 
    observe users and derive feedback from not just what they say, but how they behave with the interface, as well as non-obvious prompting methods.
    Another option is to review online material about how to conduct user testing that focuses on usability, and to possibly conduct a trial run 
    with peers.
    \item \textbf{Mutation testing:} Approaches include a reviewing content from Software Testing, a course all team members have taken that not only 
    covers how to come up with test cases, but mutation testing as well, and how to develop varying mutants. Another option is to review 
    academic papers detailing mutatation testing strategies, some of which can be found in the course materials of Software Testing.
    \item \textbf{Code Reviews:} Software Testing also included content detailing how code reviews are conducted in formal contexts, which could be 
    a helpful resource; ideas could be adapted to this project and provide a standard to consult. Another option involves running code reviews in other 
    projects team members are involved in throughout term to gain experience, which would benefit both this project and those involved.
  \end{itemize}

  \textbf{Omar: }
  \newline
  \textbf{Daniel: }
  \newline
  \textbf{Andrew: }
  \newline
  \textbf{Olivia: }
  \newline
  \textbf{Khalid:}
  \textit{Knowledge of PyTest:} I will use the online tutorials and documentation to help me gain a broading understanding of the testing framework. I think this will be a good starting point, as it will help me understand the framework theoretically, and then I can apply it to the project to gain practical experience.
  
  \textit{Knowledge of Jest:} Similar to the above, I will also use the online tutorials and documentation to help me gain a broading understanding of the testing framework. I will also experiment with the tool with a toy example program to gain practical experience.
  
  \textit{Usability tests:} For this skill, I will look over my notes from the 4HC3 (Human-Computer Interfaces) course, which covers how to observe users and derive feedback from both their responses. This was I can recall the content and apply it to the project.
  
  \textit{Mutation testing:} Similar to the above, I will look over my notes from the Software Testing course that covers not just test case development but also mutation testing strategies and how to develop varying mutants. I will also experiement with a simple program to gain practical experience.
  
  \textit{Code reviews:} I will look over my notes from the Software Testing course that covers how to conduct a code review. I will also experiment using some old code from other projects just to get a feel for the process.
  
\end{enumerate}

\end{document}