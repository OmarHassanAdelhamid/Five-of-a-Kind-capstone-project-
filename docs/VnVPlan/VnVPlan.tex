\documentclass[12pt, titlepage]{article}

\usepackage{booktabs}
\usepackage{tabularx}
\usepackage{hyperref}
\hypersetup{
    colorlinks,
    citecolor=blue,
    filecolor=black,
    linkcolor=red,
    urlcolor=blue
}
\usepackage[round]{natbib}

\input{../Comments}
%% Common Parts

\newcommand{\progname}{AutoVox} % PUT YOUR PROGRAM NAME HERE
\newcommand{\authname}{Team \#10, Five of a Kind
\\ Omar Abdelhamid
\\ Daniel Maurer
\\ Andrew Bovbel
\\ Olivia Reich
\\ Khalid Farag
} % AUTHOR NAMES                  

\usepackage{hyperref}
    \hypersetup{colorlinks=true, linkcolor=blue, citecolor=blue, filecolor=blue,
                urlcolor=blue, unicode=false}
    \urlstyle{same}
                                


\begin{document}

\title{System Verification and Validation Plan for \progname{}} 
\author{\authname}
\date{\today}
	
\maketitle

\pagenumbering{roman}

\section*{Revision History}

\begin{tabularx}{\textwidth}{p{3cm}p{2cm}X}
\toprule {\bf Date} & {\bf Version} & {\bf Notes}\\
\midrule
Date 1 & 1.0 & Notes\\
Date 2 & 1.1 & Notes\\
\bottomrule
\end{tabularx}

~\\
\wss{The intention of the VnV plan is to increase confidence in the software.
However, this does not mean listing every verification and validation technique
that has ever been devised.  The VnV plan should also be a \textbf{feasible}
plan. Execution of the plan should be possible with the time and team available.
If the full plan cannot be completed during the time available, it can either be
modified to ``fake it'', or a better solution is to add a section describing
what work has been completed and what work is still planned for the future.}

\wss{The VnV plan is typically started after the requirements stage, but before
the design stage.  This means that the sections related to unit testing cannot
initially be completed.  The sections will be filled in after the design stage
is complete.  the final version of the VnV plan should have all sections filled
in.}

\newpage

\tableofcontents

\listoftables
\wss{Remove this section if it isn't needed}

\listoffigures
\wss{Remove this section if it isn't needed}

\newpage

\section{Symbols, Abbreviations, and Acronyms}

\renewcommand{\arraystretch}{1.2}
\begin{tabular}{l l} 
  \toprule		
  \textbf{symbol} & \textbf{description}\\
  \midrule 
  T & Test\\
  \bottomrule
\end{tabular}\\

\wss{symbols, abbreviations, or acronyms --- you can simply reference the SRS
  \citep{SRS} tables, if appropriate}

\wss{Remove this section if it isn't needed}

\newpage

\pagenumbering{arabic}

This document ... \wss{provide an introductory blurb and roadmap of the
  Verification and Validation plan}

\section{General Information}

\subsection{Summary}

\wss{Say what software is being tested.  Give its name and a brief overview of
  its general functions.}

\subsection{Objectives}

\wss{State what is intended to be accomplished.  The objective will be around
  the qualities that are most important for your project.  You might have
  something like: ``build confidence in the software correctness,''
  ``demonstrate adequate usability.'' etc.  You won't list all of the qualities,
  just those that are most important.}

\wss{You should also list the objectives that are out of scope.  You don't have 
the resources to do everything, so what will you be leaving out.  For instance, 
if you are not going to verify the quality of usability, state this.  It is also 
worthwhile to justify why the objectives are left out.}

\wss{The objectives are important because they highlight that you are aware of 
limitations in your resources for verification and validation.  You can't do everything, 
so what are you going to prioritize?  As an example, if your system depends on an 
external library, you can explicitly state that you will assume that external library 
has already been verified by its implementation team.}

\subsection{Challenge Level and Extras}

\wss{State the challenge level (advanced, general, basic) for your project.
Your challenge level should exactly match what is included in your problem
statement.  This should be the challenge level agreed on between you and the
course instructor.  You can use a pull request to update your challenge level
(in TeamComposition.csv or Repos.csv) if your plan changes as a result of the
VnV planning exercise.}

\wss{Summarize the extras (if any) that were tackled by this project.  Extras
can include usability testing, code walkthroughs, user documentation, formal
proof, GenderMag personas, Design Thinking, etc.  Extras should have already
been approved by the course instructor as included in your problem statement.
You can use a pull request to update your extras (in TeamComposition.csv or
Repos.csv) if your plan changes as a result of the VnV planning exercise.}

\subsection{Relevant Documentation}

\wss{Reference relevant documentation.  This will definitely include your SRS
  and your other project documents (design documents, like MG, MIS, etc).  You
  can include these even before they are written, since by the time the project
  is done, they will be written.  You can create BibTeX entries for your
  documents and within those entries include a hyperlink to the documents.}

\citet{SRS}

\wss{Don't just list the other documents.  You should explain why they are relevant and 
how they relate to your VnV efforts.}

\section{Plan}

\wss{Introduce this section.  You can provide a roadmap of the sections to
  come.}

\subsection{Verification and Validation Team}

\wss{Your teammates.  Maybe your supervisor.
  You should do more than list names.  You should say what each person's role is
  for the project's verification.  A table is a good way to summarize this information.}

\subsection{SRS Verification}

\wss{List any approaches you intend to use for SRS verification.  This may
  include ad hoc feedback from reviewers, like your classmates (like your
  primary reviewer), or you may plan for something more rigorous/systematic.}

\wss{If you have a supervisor for the project, you shouldn't just say they will
read over the SRS.  You should explain your structured approach to the review.
Will you have a meeting?  What will you present?  What questions will you ask?
Will you give them instructions for a task-based inspection?  Will you use your
issue tracker?}

\wss{Maybe create an SRS checklist?}

\subsection{Design Verification}

\wss{Plans for design verification}

\wss{The review will include reviews by your classmates}

\wss{Create a checklists?}

\subsection{Verification and Validation Plan Verification}

\wss{The verification and validation plan is an artifact that should also be
verified.  Techniques for this include review and mutation testing.}

\wss{The review will include reviews by your classmates}

\wss{Create a checklists?}

\subsection{Implementation Verification}

\wss{You should at least point to the tests listed in this document and the unit
  testing plan.}

\wss{In this section you would also give any details of any plans for static
  verification of the implementation.  Potential techniques include code
  walkthroughs, code inspection, static analyzers, etc.}

\wss{The final class presentation in CAS 741 could be used as a code
walkthrough.  There is also a possibility of using the final presentation (in
CAS741) for a partial usability survey.}

\subsection{Automated Testing and Verification Tools}

\wss{What tools are you using for automated testing.  Likely a unit testing
  framework and maybe a profiling tool, like ValGrind.  Other possible tools
  include a static analyzer, make, continuous integration tools, test coverage
  tools, etc.  Explain your plans for summarizing code coverage metrics.
  Linters are another important class of tools.  For the programming language
  you select, you should look at the available linters.  There may also be tools
  that verify that coding standards have been respected, like flake9 for
  Python.}

\wss{If you have already done this in the development plan, you can point to
that document.}

\wss{The details of this section will likely evolve as you get closer to the
  implementation.}

\subsection{Software Validation}

\wss{If there is any external data that can be used for validation, you should
  point to it here.  If there are no plans for validation, you should state that
  here.}

\wss{You might want to use review sessions with the stakeholder to check that
the requirements document captures the right requirements.  Maybe task based
inspection?}

\wss{For those capstone teams with an external supervisor, the Rev 0 demo should 
be used as an opportunity to validate the requirements.  You should plan on 
demonstrating your project to your supervisor shortly after the scheduled Rev 0 demo.  
The feedback from your supervisor will be very useful for improving your project.}

\wss{For teams without an external supervisor, user testing can serve the same purpose 
as a Rev 0 demo for the supervisor.}

\wss{This section might reference back to the SRS verification section.}

\section{System Tests}

This section outlines the tests for verifying both functional and nonfunctional requirements
of the software, ensuring it meets user expectations and performs reliably. This includes
tests for code quality, usability, performance, security, and traceability, covering essential
aspects of the software's operation and compliance.

\subsection{Tests for Functional Requirements}

The subsections below outline tests corresponding to functional requirements in the SRS \citep{SRS}.
Each test is associated with a unique functional area, helping to confirm that the tool meets
the specified requirements. Each functional area has its own subsection for clarity.

The system tests are organized by the major components identified in the SRS: Import Manager
and Visualization Manager. These components handle the core functionality of CAD file processing,
voxel slicing, 3D model visualization, and user interaction. The tests ensure that all functional
requirements (F211-F216 for Import Manager and F221-F227 for Visualization Manager) are properly
validated through both automated and manual testing approaches.

\subsubsection{Import Manager Tests}

This section covers tests for the Import Manager component, which handles CAD file interpretation,
voxel slicing, and model manipulation. These tests verify the functional requirements F211-F216
from the SRS, ensuring proper file handling, model processing, and user configuration capabilities.
The tests validate both automated processing and user interaction scenarios.

\paragraph{CAD File Import and Project Management Tests}

\begin{enumerate}

\item{test-FR-IM-1 Valid CAD File Import for New Project\\}

Control: Automated
					
Initial State: System is idle, no active project loaded
					
Input: A valid CAD file (e.g., .stl, .obj, .ply format) containing a 3D mesh model
					
Output: The system successfully creates a new project, imports the CAD file, and initiates
the voxel slicing process. The imported model is ready for visualization.

Test Case Derivation: Confirms that the system correctly processes valid CAD files and creates
new projects as specified in F211. This validates the core functionality of starting new projects
with imported CAD files.
					
How test will be performed: Provide a valid CAD file through the import interface and verify
that a new project is created, the file is processed without errors, and the model data is
available for subsequent operations.

\item{test-FR-IM-2 Past Project File Import\\}

Control: Automated
					
Initial State: System is idle, no active project loaded
					
Input: A previously saved project file containing magnetization and material properties
					
Output: The system successfully loads the project with all magnetization and material
properties preserved, and the model is ready for further editing.

Test Case Derivation: Ensures that past projects can be reopened with complete data integrity,
satisfying F212. This prevents users from losing work and enables iterative design processes.
					
How test will be performed: Load a previously saved project file and verify that all voxel
properties, magnetization vectors, and material assignments are correctly restored.

\item{test-FR-IM-3 Invalid File Format Handling\\}

Control: Automated
					
Initial State: System is idle, no active project loaded
					
Input: A file with unsupported format (e.g., .txt, .jpg, .pdf)
					
Output: The system rejects the file and displays an appropriate error message indicating
unsupported file format, without creating a project or crashing.

Test Case Derivation: Validates robust error handling for invalid inputs, ensuring system
stability when users attempt to import incompatible files.
					
How test will be performed: Attempt to import various unsupported file formats and verify
that the system gracefully handles the error without system failure.

\end{enumerate}

\paragraph{Model Configuration and Slicing Tests}

\begin{enumerate}

\item{test-FR-IM-4 Voxel Size Configuration\\}

Control: Manual
					
Initial State: A valid CAD model has been imported and is ready for slicing
					
Input: User specifies custom voxel dimensions (e.g., 0.1mm x 0.1mm x 0.1mm)
					
Output: The system applies the specified voxel dimensions and successfully slices the
model into voxels of the requested size.

Test Case Derivation: Confirms that users can configure voxel dimensions as required by F213,
enabling customization of model resolution for different printing requirements.
					
How test will be performed: Import a CAD model, configure custom voxel dimensions through
the user interface, and verify that the resulting voxel grid matches the specified dimensions.

\item{test-FR-IM-5 Model Scaling Functionality\\}

Control: Manual
					
Initial State: A valid CAD model has been imported
					
Input: User modifies model dimensions (e.g., scale to 50\% of original size)
					
Output: The model is successfully scaled to the specified dimensions while maintaining
geometric integrity and proportions.

Test Case Derivation: Validates F214 by ensuring users can modify model dimensions before
slicing, enabling proper scaling for different printing requirements.
					
How test will be performed: Import a CAD model, use the scaling interface to modify dimensions,
and verify that the scaled model maintains proper proportions and geometric accuracy.

\item{test-FR-IM-6 Partial Voxel Resolution\\}

Control: Automated
					
Initial State: A CAD model with complex geometry has been imported
					
Input: A model containing surfaces that intersect voxel boundaries (creating partial voxels)
					
Output: The system correctly resolves partial voxels using appropriate algorithms (e.g.,
majority rule, surface area weighting) and preserves model integrity and accuracy.

Test Case Derivation: Ensures F215 is satisfied by validating that partial voxels are handled
correctly, maintaining model accuracy during the slicing process.
					
How test will be performed: Import a CAD model with complex geometry that creates partial
voxels, and verify that the resolution algorithm produces accurate results that preserve
the original model's shape and volume.

\item{test-FR-IM-7 Model Division for Display\\}

Control: Automated
					
Initial State: A large CAD model has been imported and sliced into voxels
					
Input: A model containing more than MAX\_DISPLAY voxels
					
Output: The system automatically partitions the model into manageable display sections
that do not exceed the MAX\_DISPLAY threshold, enabling smooth visualization.

Test Case Derivation: Confirms F216 by ensuring large models are properly divided into
manageable sections for optimal display performance and user interaction.
					
How test will be performed: Import a large CAD model that exceeds MAX\_DISPLAY voxels,
and verify that the system creates appropriate partitions that maintain model coherence
while staying within display limits.

\end{enumerate}

\subsubsection{Visualization Manager Tests}

This section covers tests for the Visualization Manager component, which handles 3D model
rendering, user interaction, and visual feedback. These tests verify the functional requirements
F221-F227 from the SRS, ensuring proper 3D visualization, navigation, and user interface
functionality. The tests validate both rendering capabilities and interactive features.

\paragraph{3D Model Rendering and Display Tests}

\begin{enumerate}

\item{test-FR-VM-1 3D Model Rendition with Clear Voxel Visualization\\}

Control: Manual
					
Initial State: A CAD model has been imported and sliced into voxels by the Import Manager
					
Input: Sliced voxel data from Import Manager
					
Output: The system successfully renders a 3D model with clear visualization of individual
voxels, maintaining geometric accuracy and providing distinct voxel boundaries.

Test Case Derivation: Confirms F221 by ensuring the Visualization Manager can recreate
3D models with clear voxel visualization, which is essential for property assignment
and user understanding of the model structure.
					
How test will be performed: Import and slice a CAD model, then verify that the 3D
rendition clearly displays individual voxels with proper boundaries and geometric accuracy.

\item{test-FR-VM-2 Partition Navigation Functionality\\}

Control: Manual
					
Initial State: A large model has been divided into multiple display partitions
					
Input: User navigation commands to switch between model partitions
					
Output: The system provides smooth navigation between partitions, allowing users to
access all model sections without performance degradation.

Test Case Derivation: Validates F222 by ensuring users can navigate across all model
partitions, supporting the Import Manager's model division functionality.
					
How test will be performed: Load a large model with multiple partitions and verify
that navigation controls allow seamless movement between all sections of the model.

\item{test-FR-VM-3 Multi-Perspective 3D Model Interaction\\}

Control: Manual
					
Initial State: A 3D model is rendered and displayed
					
Input: User interaction commands (rotate, zoom, pan) to view the model from different perspectives
					
Output: The system provides intuitive interface controls that allow seamless navigation
across multiple perspectives of the 3D model with responsive interaction.

Test Case Derivation: Confirms F223 by ensuring users can interact with the 3D model
from all perspectives, providing comprehensive visualization capabilities.
					
How test will be performed: Use the 3D interaction controls to rotate, zoom, and pan
the model, verifying that all perspectives are accessible and the interface remains
responsive and intuitive.

\end{enumerate}

\paragraph{Layer Focus and Voxel Selection Tests}

\begin{enumerate}

\item{test-FR-VM-4 Layer Isolation Functionality\\}

Control: Manual
					
Initial State: A 3D model with multiple layers is displayed
					
Input: User selects a specific layer to focus on
					
Output: The system isolates the selected layer, rendering all other voxels irrelevant
or transparent, facilitating property assignment on the focused layer.

Test Case Derivation: Validates F224 by ensuring users can isolate specific layers
for focused interaction, which is essential for efficient property assignment.
					
How test will be performed: Select different layers of a multi-layer model and verify
that only the selected layer remains visible and interactive while other layers are
appropriately dimmed or hidden.

\item{test-FR-VM-5 Voxel Selection Highlighting\\}

Control: Manual
					
Initial State: A specific layer is in focus and displayed
					
Input: User selects individual voxels or groups of voxels within the focused layer
					
Output: The system provides clear visual feedback showing which voxels are currently
selected, using distinct highlighting or color changes.

Test Case Derivation: Confirms F225 by ensuring users receive clear visual feedback
about their current selections, improving usability and reducing errors.
					
How test will be performed: Select various voxels within a focused layer and verify
that the selection is clearly highlighted with distinct visual indicators.

\end{enumerate}

\paragraph{Material and Magnetization Tracking Tests}

\begin{enumerate}

\item{test-FR-VM-6 Material Assignment Visual Tracking\\}

Control: Manual
					
Initial State: A model with unassigned voxels is displayed
					
Input: User assigns material IDs to various voxels
					
Output: The system updates voxel colors to indicate material assignment completeness,
providing clear visual feedback about assignment status.

Test Case Derivation: Validates F226 by ensuring users can easily track which voxels
have been assigned materials through visual color changes.
					
How test will be performed: Assign materials to various voxels and verify that the
color changes provide clear indication of assignment status and remaining work.

\item{test-FR-VM-7 Magnetization Assignment Visual Tracking\\}

Control: Manual
					
Initial State: A model with some magnetized and some unmagnetized voxels
					
Input: User toggles the magnetization visualization display
					
Output: The system provides the option to toggle color display of all magnetized
voxels, allowing users to easily identify magnetization status.

Test Case Derivation: Confirms F227 by ensuring users can track magnetization
assignments through visual toggles, improving workflow efficiency.
					
How test will be performed: Toggle the magnetization visualization and verify that
magnetized voxels are clearly distinguished from unmagnetized ones through appropriate
color coding.

\end{enumerate}

\subsubsection{Integration Tests Between Import and Visualization Managers}

This section covers tests that validate the interaction and data flow between the Import Manager
and Visualization Manager components. These tests ensure that the two managers work together
seamlessly to provide a complete user experience from CAD file import to 3D visualization.

\paragraph{Data Flow and Communication Tests}

\begin{enumerate}

\item{test-FR-INT-1 Import to Visualization Data Transfer\\}

Control: Automated
					
Initial State: System is idle, no active project
					
Input: A valid CAD file is imported and processed by the Import Manager
					
Output: The Visualization Manager receives complete voxel data and successfully renders
the 3D model without data loss or corruption.

Test Case Derivation: Ensures proper data flow between managers, validating that
voxel slicing results are correctly transmitted to the visualization system.
					
How test will be performed: Import a CAD file and verify that the resulting 3D
visualization accurately represents the sliced voxel data from the Import Manager.

\item{test-FR-INT-2 Model Division Integration\\}

Control: Automated
					
Initial State: A large CAD model has been imported
					
Input: Import Manager divides the model into display partitions
					
Output: Visualization Manager correctly displays and allows navigation between all
partitions created by the Import Manager.

Test Case Derivation: Validates that model division from Import Manager is properly
handled by Visualization Manager, ensuring seamless user experience with large models.
					
How test will be performed: Import a large model that requires division and verify
that all partitions are accessible through the visualization interface.

\item{test-FR-INT-3 Configuration Parameter Synchronization\\}

Control: Manual
					
Initial State: A model is ready for slicing
					
Input: User configures voxel size and model scaling parameters
					
Output: Both Import Manager (for slicing) and Visualization Manager (for display)
correctly apply the configuration parameters and maintain consistency.

Test Case Derivation: Ensures that user configuration changes are properly synchronized
between managers, maintaining data consistency throughout the system.
					
How test will be performed: Modify configuration parameters and verify that both
slicing results and visualization accurately reflect the changes.

\end{enumerate}

\subsection{Tests for Nonfunctional Requirements}

\wss{The nonfunctional requirements for accuracy will likely just reference the
  appropriate functional tests from above.  The test cases should mention
  reporting the relative error for these tests.  Not all projects will
  necessarily have nonfunctional requirements related to accuracy.}

\wss{For some nonfunctional tests, you won't be setting a target threshold for
passing the test, but rather describing the experiment you will do to measure
the quality for different inputs.  For instance, you could measure speed versus
the problem size.  The output of the test isn't pass/fail, but rather a summary
table or graph.}

\wss{Tests related to usability could include conducting a usability test and
  survey.  The survey will be in the Appendix.}

\wss{Static tests, review, inspections, and walkthroughs, will not follow the
format for the tests given below.}

\wss{If you introduce static tests in your plan, you need to provide details.
How will they be done?  In cases like code (or document) walkthroughs, who will
be involved? Be specific.}

\subsubsection{Area of Testing1}
		
\paragraph{Title for Test}

\begin{enumerate}

\item{test-id1\\}

Type: Functional, Dynamic, Manual, Static etc.
					
Initial State: 
					
Input/Condition: 
					
Output/Result: 
					
How test will be performed: 
					
\item{test-id2\\}

Type: Functional, Dynamic, Manual, Static etc.
					
Initial State: 
					
Input: 
					
Output: 
					
How test will be performed: 

\end{enumerate}

\subsubsection{Area of Testing2}

...

\subsection{Traceability Between Test Cases and Requirements}

\wss{Provide a table that shows which test cases are supporting which
  requirements.}

\section{Unit Test Description}

\wss{This section should not be filled in until after the MIS (detailed design
  document) has been completed.}

\wss{Reference your MIS (detailed design document) and explain your overall
philosophy for test case selection.}  

\wss{To save space and time, it may be an option to provide less detail in this section.  
For the unit tests you can potentially layout your testing strategy here.  That is, you 
can explain how tests will be selected for each module.  For instance, your test building 
approach could be test cases for each access program, including one test for normal behaviour 
and as many tests as needed for edge cases.  Rather than create the details of the input 
and output here, you could point to the unit testing code.  For this to work, you code 
needs to be well-documented, with meaningful names for all of the tests.}

\subsection{Unit Testing Scope}

\wss{What modules are outside of the scope.  If there are modules that are
  developed by someone else, then you would say here if you aren't planning on
  verifying them.  There may also be modules that are part of your software, but
  have a lower priority for verification than others.  If this is the case,
  explain your rationale for the ranking of module importance.}

\subsection{Tests for Functional Requirements}

\wss{Most of the verification will be through automated unit testing.  If
  appropriate specific modules can be verified by a non-testing based
  technique.  That can also be documented in this section.}

\subsubsection{Module 1}

\wss{Include a blurb here to explain why the subsections below cover the module.
  References to the MIS would be good.  You will want tests from a black box
  perspective and from a white box perspective.  Explain to the reader how the
  tests were selected.}

\begin{enumerate}

\item{test-id1\\}

Type: \wss{Functional, Dynamic, Manual, Automatic, Static etc. Most will
  be automatic}
					
Initial State: 
					
Input: 
					
Output: \wss{The expected result for the given inputs}

Test Case Derivation: \wss{Justify the expected value given in the Output field}

How test will be performed: 
					
\item{test-id2\\}

Type: \wss{Functional, Dynamic, Manual, Automatic, Static etc. Most will
  be automatic}
					
Initial State: 
					
Input: 
					
Output: \wss{The expected result for the given inputs}

Test Case Derivation: \wss{Justify the expected value given in the Output field}

How test will be performed: 

\item{...\\}
    
\end{enumerate}

\subsubsection{Module 2}

...

\subsection{Tests for Nonfunctional Requirements}

\wss{If there is a module that needs to be independently assessed for
  performance, those test cases can go here.  In some projects, planning for
  nonfunctional tests of units will not be that relevant.}

\wss{These tests may involve collecting performance data from previously
  mentioned functional tests.}

\subsubsection{Module ?}
		
\begin{enumerate}

\item{test-id1\\}

Type: \wss{Functional, Dynamic, Manual, Automatic, Static etc. Most will
  be automatic}
					
Initial State: 
					
Input/Condition: 
					
Output/Result: 
					
How test will be performed: 
					
\item{test-id2\\}

Type: Functional, Dynamic, Manual, Static etc.
					
Initial State: 
					
Input: 
					
Output: 
					
How test will be performed: 

\end{enumerate}

\subsubsection{Module ?}

...

\subsection{Traceability Between Test Cases and Modules}

\wss{Provide evidence that all of the modules have been considered.}
				
\bibliographystyle{plainnat}

\bibliography{../../refs/References}

\newpage

\section{Appendix}

This is where you can place additional information.

\subsection{Symbolic Parameters}

The definition of the test cases will call for SYMBOLIC\_CONSTANTS.
Their values are defined in this section for easy maintenance.

\subsection{Usability Survey Questions?}

\wss{This is a section that would be appropriate for some projects.}

\newpage{}
\section*{Appendix --- Reflection}

\wss{This section is not required for CAS 741}

The information in this section will be used to evaluate the team members on the
graduate attribute of Lifelong Learning.

\input{../Reflection.tex}

\begin{enumerate}
  \item What went well while writing this deliverable? 
  \item What pain points did you experience during this deliverable, and how
    did you resolve them?
  \item What knowledge and skills will the team collectively need to acquire to
  successfully complete the verification and validation of your project?
  Examples of possible knowledge and skills include dynamic testing knowledge,
  static testing knowledge, specific tool usage, Valgrind etc.  You should look to
  identify at least one item for each team member.
  \item For each of the knowledge areas and skills identified in the previous
  question, what are at least two approaches to acquiring the knowledge or
  mastering the skill?  Of the identified approaches, which will each team
  member pursue, and why did they make this choice?
\end{enumerate}

\end{document}