\documentclass{article}

\usepackage{booktabs}
\usepackage{tabularx}

\title{Development Plan\\\progname}

\author{\authname}

\date{}

%% Comments

\usepackage{color}

\newif\ifcomments\commentstrue %displays comments
%\newif\ifcomments\commentsfalse %so that comments do not display

\ifcomments
\newcommand{\authornote}[3]{\textcolor{#1}{[#3 ---#2]}}
\newcommand{\todo}[1]{\textcolor{red}{[TODO: #1]}}
\else
\newcommand{\authornote}[3]{}
\newcommand{\todo}[1]{}
\fi

\newcommand{\wss}[1]{\authornote{magenta}{SS}{#1}} 
\newcommand{\plt}[1]{\authornote{cyan}{TPLT}{#1}} %For explanation of the template
\newcommand{\an}[1]{\authornote{cyan}{Author}{#1}}

%% Common Parts

\newcommand{\progname}{Software Engineering} % PUT YOUR PROGRAM NAME HERE
\newcommand{\authname}{Team \#10, Five of a Kind
\\ Omar Abdelhamid
\\ Daniel Maurer
\\ Andrew Bovbel
\\ Olivia Reich
\\ Khalid Farag
} % AUTHOR NAMES                  

\usepackage{hyperref}
    \hypersetup{colorlinks=true, linkcolor=blue, citecolor=blue, filecolor=blue,
                urlcolor=blue, unicode=false}
    \urlstyle{same}
                                


\begin{document}

\maketitle

\begin{table}[hp]
\caption{Revision History} \label{TblRevisionHistory}
\begin{tabularx}{\textwidth}{llX}
\toprule
\textbf{Date} & \textbf{Developer(s)} & \textbf{Change}\\
\midrule
Date1 & Name(s) & Description of changes\\
Date2 & Name(s) & Description of changes\\
... & ... & ...\\
\bottomrule
\end{tabularx}
\end{table}

\newpage{}

\iffalse
\wss{Put your introductory blurb here.  Often the blurb is a brief roadmap of
what is contained in the report.}

\wss{Additional information on the development plan can be found in the
\href{https://gitlab.cas.mcmaster.ca/courses/capstone/-/blob/main/Lectures/L02b_POCAndDevPlan/POCAndDevPlan.pdf?ref_type=heads}
{lecture slides}.}
\fi

The following document will provide an overview of the team plans for the project development process. It will
establish plans to promote effective collaboration throughout the entire project as well as define roles and
responsibilities to help guide successful team dynamics. It will provide insight into the plans for project
execution through a workflow plan that is supported by an outline of the methods that will successfully demonstrate
proof of concept as the project progresses. It also provides a high-level understanding of the technological
concepts that will be explored and adopted throughout the project.

\section{Confidential Information?}

\iffalse
\wss{State whether your project has confidential information from industry, or
not.  If there is confidential information, point to the agreement you have in
place.}\\
\fi

At this time, there is no confidential information that is relevant to note within the current project scope.

\iffalse
\wss{For most teams this section will just state that there is no confidential
information to protect.}
\fi
\section{IP to Protect}

\iffalse
\wss{State whether there is IP to protect.  If there is, point to the agreement.
All students who are working on a project that requires an IP agreement are also
required to sign the ``Intellectual Property Guide Acknowledgement.''}
\fi

At this time, an IP is not relevant to the current project scope.

\section{Copyright License}

\iffalse
\wss{What copyright license is your team adopting.  Point to the license in your
repo.}
\fi

This project is licensed under the MIT License. See the \href{./LICENSE}{LICENSE} file for details.

\section{Team Meeting Plan}

\iffalse
\wss{How often will you meet? where?}

\wss{If the meeting is a physical location (not virtual), out of an abundance of
caution for safety reasons you shouldn't put the location online}

\wss{How often will you meet with your industry advisor?  when?  where?}

\wss{Will meetings be virtual?  At least some meetings should likely be
in-person.}

\wss{How will the meetings be structured?  There should be a chair for all meetings.  There should be an agenda for all meetings.}
\fi

A basic group standard will be the organization of a group meeting that occurs a minimum of at least once a week to
allow for discussion on current project progress and future planning. Given weekly variations in personal schedules, there
will not be a specific date and time that the group abides by each week. Instead, the individual who takes on the
role of the Meeting Chair will be in charge of organizing a date and time each week that best accommodates the
schedule of each person. The nature of the meeting, regarding whether it will be in-person or virtual, will be
established upon determining the date and time of the meeting. As a general principle, in-person meetings will be
favoured as they typically help facilitate deeper collaboration and build better relationships amongst team
members. All in-person meetings should be conducted on campus, unless a different location is agreed upon by the
group. If a group member is unable to attend a certain meeting, they should notify the group through the WhatsApp
group chat as soon as possible.\\

The individual who takes on the role of the Notetaker will document all parts of the meeting. Certain points of
discussion may be established prior to the meeting, either during a previous meeting or during the period leading
up to the meeting. The Notetaker will work with the Team Manager to establish an agenda of all discussion points
and deadline reminders prior to the meeting. This agenda will be documented in the meeting report. During the
meeting, the Meeting Chair will be responsible for ensuring all points of discussion are adequately covered. At the
end of the meeting, it is the responsbility of the Notetaker to ensure the meeting has been properly documented.
The meeting documentation expectations have been specified under the Notetaker description in Section 6.\\

Meetings with our supervisor, Dr. Onaizah, will be organized based on the need to provide more in-depth explanations
on the progress of the project or to facilitate a discussion if there are many questions in which Dr. Onaizah can
provide insight. There will be no set schedule for when these meetings will occur. Once the team establishes that a
meeting with Dr. Onaizah is required, the team will create a list of potential dates and times that they are able
to present as options for Dr. Onaizah. Unless the group decides to specifically request either an in-person or
online meeting to accommodate specific group needs, the nature of the meeting will be determined based on the
preference of Dr. Onaizah.

\section{Team Communication Plan}

\iffalse
\wss{Issues on GitHub should be part of your communication plan.}
\fi

All informal communication should be initiated through a WhatsApp group chat that has already been established. It
is the responsibility of all group members to ensure they have access to the group chat and are actively monitoring
all messages that occur through this channel of communication. The primary focus of this group chat is to facilitate
easy and quick communication regarding small issues and meeting organization.\\

Meetings should be a regular occurrence to establish consistent and clear communication that can’t be achieved
solely through text communication. As established within the Team Meeting Plan, there will be a standard weekly scheduled
team meeting. If a discussion that starts through the group chat seems to require additional time, effort, or
broader input to consider different perspectives, the discussion may be postponed until it is able to be discussed
during a meeting. This means that it can either be noted as an official point of discussion within the weekly
scheduled team meeting or a separate meeting may be scheduled depending on the urgency of the discussion.\\

The creation of issues in the GitHub team repository will be the primary way to formally document all plans,
discussions and deadlines that have been discussed. All notable information that should be known by all group
members must be documented as a GitHub issue. Even if information has been stated in a different communication channel,
such as a group chat or team meeting, it must still be documented in GitHub to provide consistency and ensure there
are no oversights.\\

In the event that any communication occurs outside the WhatsApp group chat or a team meeting, it must still be
noted in the GitHub team repository. Individuals should do their best to not stray from the standard communication
channels unless extenuating circumstances apply or an additional standard communication channel has been established
by the group throughout the project progression. This will help ensure there is consistency in communication amongst
the entire group.\\

At a minimum, the team will provide the supervisor, Dr. Onaizah, with updates at least bi-weekly. Prior to providing
an update for Dr. Onaizah, the team will establish whether a meeting with her is necessary within the next week.
Meetings with Dr. Onaizah will be organized as needed. Key stakeholders will be handled in a similar manner. This
includes the grad student, Kaitlyn Clancy, that created the 3D printer and works directly under Dr. Onaizah. If relevant, Kaitlyn 
may be cc’d onto email updates to Dr. Onaizah or separate meetings may be organized.

\section{Team Member Roles}

\iffalse
\wss{You should identify the types of roles you anticipate, like notetaker,
leader, meeting chair, reviewer.  Assigning specific people to those roles is
not necessary at this stage.  In a student team the role of the individuals will
likely change throughout the year.}
\fi

\textit{Team Manager} - The individual who takes on this role will take the lead on keeping the team on track. They
will closely monitor the deadlines and ensure that the team is on track with meeting the course deadlines as well as
the group deadlines that have been agreed upon ahead of time.\\

\textit{Notetaker} - During meetings, this individual will be responsible for keeping track of what was discussed
and who attended the meetings. They will also be required to create a report, in the form of a GitHub issue, that is accessible to each group member
at the end of a meeting. This report will include a summary of what was discussed, a meeting agenda for what has to
be discussed, a review of what has changed since the previous meeting, a list of key tasks to complete prior to the
next meeting and an outline of deadlines that are relevant at the current stage of the project.\\

\textit{Meeting Chair} - This individual will be responsible for organizing meetings and ensuring that the team is
following the Team Meeting Plan as specified above. They will ensure that the team is meeting at a sufficient
frequency to facilitate effective collaboration both within the group and with Dr. Onaizah. They will also take
the lead on team discussions regarding future meetings and do their best to reasonably accommodate the schedule of
all team members when determining a date and time.\\

\textit{Document Reviewer} - This individual will take on the responsibility of monitoring the review of all
documents. They will ensure that team deadlines provide reasonable buffer for review prior to the official course
due date. They will organize who reviews each section of the relevant documents being submitted as well as complete
an overall final review prior to submission. This final review should include verifying the consistency,
formatting, and grammar of the document. They will also work closely with the code reviewer to ensure the code and
documentation properly align.\\

\textit{Code Reviewer} - This individual will have similar responsibilities to the document reviewer but will focus
on reviewing the code. This includes monitoring the integration of individual code contributions during each
iteration in the development stage. They will organize milestone code check-ins to ensure that after major code
contributions the functionality of previous code has not been compromised in any way and still runs seamlessly. They
will also work closely with the document reviewer to ensure the code and documentation properly align.\\

\textit{Communication Officer} - The individual who takes on this role will be responsible for handling the majority
of the communication with our supervisor and key stakeholders. Specific communication responsibilities include
sending emails when relevant, closely monitoring their email for replies, keeping all relevant parties in the loop
regarding the progress of the project, and conveying important information from these emails to the rest of the team
members.

\section{Workflow Plan}

\begin{itemize}
	\item How will you be using git, including branches, pull request, etc.?
	\item How will you be managing issues, including template issues, issue
	classification, etc.?
  \item Use of CI/CD
\end{itemize}

Git will be utilized as the primary, centralized collaboration space for both project
source code and its related documentation. Members will create new branches for each
feature to be implemented, as well as per each member (eg. branches will not be shared).
When a member is complete with their work on a given branch, they will create a pull request
with appropriate labels and assignees. For instance, an acceptable assignee could be someone
working on a related feature. Labels should give an estimate of how long it might take to
review. Each pull request should also be linked to a given issue.

A milestone will be created for each deliverable (both documentation and revisions) and issgit ues
created will be associated with it. Issue templates will be used in the following ways:

\begin{itemize}
  \item lecture: to track attendance as well as general notes/questions
  \item meeting(s): to record meeting minutes, attendance, and catch up members that miss the meeting
  \item peer review: to request another team review a given section
\end{itemize}

Additionally, for general issues, new labels will be created:

\begin{itemize}
  \item documentation
  \item code
  \item question
  \item easy / hard
  \item bug-fix
  \item need help
\end{itemize}

All issues should also contain a reasonable description describing the bug / feature / 
section of documentation.

To incorporate continuous integration, upon each push/pull request, all tests in the test suite
will be run automatically via GitHub Actions.

\section{Project Decomposition and Scheduling}

\begin{itemize}
  \item How will you be using GitHub projects?
  \item Include a link to your GitHub project
\end{itemize}

\wss{How will the project be scheduled?  This is the big picture schedule, not
details. You will need to reproduce information that is in the course outline
for deadlines.}

\section{Proof of Concept Demonstration Plan}

What is the main risk, or risks, for the success of your project?  What will you
demonstrate during your proof of concept demonstration to convince yourself that
you will be able to overcome this risk?

\section{Expected Technology}

\wss{What programming language or languages do you expect to use?  What external
libraries?  What frameworks?  What technologies.  Are there major components of
the implementation that you expect you will implement, despite the existence of
libraries that provide the required functionality.  For projects with machine
learning, will you use pre-trained models, or be training your own model?  }

\wss{The implementation decisions can, and likely will, change over the course
of the project.  The initial documentation should be written in an abstract way;
it should be agnostic of the implementation choices, unless the implementation
choices are project constraints.  However, recording our initial thoughts on
implementation helps understand the challenge level and feasibility of a
project.  It may also help with early identification of areas where project
members will need to augment their training.}

Topics to discuss include the following:

\begin{itemize}
\item Specific programming language
\item Specific libraries
\item Pre-trained models
\item Specific linter tool (if appropriate)
\item Specific unit testing framework
\item Investigation of code coverage measuring tools
\item Specific plans for Continuous Integration (CI), or an explanation that CI
  is not being done
\item Specific performance measuring tools (like Valgrind), if
  appropriate
\item Tools you will likely be using?
\end{itemize}

\wss{git, GitHub and GitHub projects should be part of your technology.}

\section{Coding Standard}

\wss{What coding standard will you adopt?}

\newpage{}

\section*{Appendix --- Reflection}

\wss{Not required for CAS 741}

The purpose of reflection questions is to give you a chance to assess your own
learning and that of your group as a whole, and to find ways to improve in the
future. Reflection is an important part of the learning process.  Reflection is
also an essential component of a successful software development process.  

Reflections are most interesting and useful when they're honest, even if the
stories they tell are imperfect. You will be marked based on your depth of
thought and analysis, and not based on the content of the reflections
themselves. Thus, for full marks we encourage you to answer openly and honestly
and to avoid simply writing ``what you think the evaluator wants to hear.''

Please answer the following questions.  Some questions can be answered on the
team level, but where appropriate, each team member should write their own
response:


\begin{enumerate}
    \item Why is it important to create a development plan prior to starting the
    project?
    \item In your opinion, what are the advantages and disadvantages of using
    CI/CD?
    \item What disagreements did your group have in this deliverable, if any,
    and how did you resolve them?
\end{enumerate}

\newpage{}

\section*{Appendix --- Team Charter}

\wss{borrows from
\href{https://engineering.up.edu/industry_partnerships/files/team-charter.pdf}
{University of Portland Team Charter}}

\subsection*{External Goals}

\wss{What are your team's external goals for this project? These are not the
goals related to the functionality or quality fo the project.  These are the
goals on what the team wishes to achieve with the project.  Potential goals are
to win a prize at the Capstone EXPO, or to have something to talk about in
interviews, or to get an A+, etc.}

\subsection*{Attendance}

\subsubsection*{Expectations}

\wss{What are your team's expectations regarding meeting attendance (being on
time, leaving early, missing meetings, etc.)?}

\subsubsection*{Acceptable Excuse}

\wss{What constitutes an acceptable excuse for missing a meeting or a deadline?
What types of excuses will not be considered acceptable?}

\subsubsection*{In Case of Emergency}

\wss{What process will team members follow if they have an emergency and cannot
attend a team meeting or complete their individual work promised for a team
deliverable?}

\subsection*{Accountability and Teamwork}

\subsubsection*{Quality} 

\wss{What are your team's expectations regarding the quality
of team members' preparation for team meetings and the quality of the
deliverables that members bring to the team?}

\subsubsection*{Attitude}

\wss{What are your team's expectations regarding team members' ideas,
interactions with the team, cooperation, attitudes, and anything else regarding
team member contributions?  Do you want to introduce a code of conduct?  Do you
want a conflict resolution plan?  Can adopt existing codes of conduct.}

\subsubsection*{Stay on Track}

\iffalse
\wss{What methods will be used to keep the team on track? How will your team
ensure that members contribute as expected to the team and that the team
performs as expected? How will your team reward members who do well and manage
members whose performance is below expectations?  What are the consequences for
someone not contributing their fair share?}

\wss{You may wish to use the project management metrics collected for the TA and
instructor for this.}

\wss{You can set target metrics for attendance, commits, etc.  What are the
consequences if someone doesn't hit their targets?  Do they need to bring the
coffee to the next team meeting?  Does the team need to make an appointment with
their TA, or the instructor?  Are there incentives for reaching targets early?}
\fi

To stay on track as a team, we will follow these necessary methods:

\begin{enumerate}
  \item \textbf{Regular Team Meetings:} We will hold meetings at least once a week to discuss progress, address any issues, and plan for the next deliverable. This meeting will be communicated through the team chat \textit{WhatsApp}. This meeting will be attended by all team members, unless otherwise specified.
  \item \textbf{Task Management:} We will use GitHub Projects to assign tasks, track progress, and ensure everyone is contributing their fair share. This is crucial to ensure that all team members contribute their fair share to the project.
  \item \textbf{Regular Updates:} We will update the supervisor and stakeholders regularly to keep them informed about our progress. This will be done through emails to the supervisor and the stakeholders. The supervisor will be updated at least once every two weeks, unless any questions arise.
\end{enumerate}

For team contribution and performance, we will follow these methods:

\begin{enumerate}
  \item \textbf{Attendance:} We will use GitHub Projects to track the attendance of each team member for team meetings, lectures and supervisor meetings. This is important to ensure that all team members are contributing their fair share to the project.
  \item \textbf{Contributions:} We will use GitHub Projects to track the contributions of each team member to the project. Each team member will be responsible for contributing to this capstone project to the best of their ability. Issues will be evaluated based on the quality of the work, the deadline and the effort put into the issue.
  \item \textbf{Code Review:} Each team member will be responsible for reviewing the code of the other team members to ensure that the work is completed at a high quality and on time. Each pull request will be reviewed by at least two other team members before it is merged into the main branch.
  \item \textbf{Performance Metrics:} We will use commits, meetings attended, and issues completed to evaluate the performance of each team member.
\end{enumerate}

For the rewards and consequences, we will follow these methods:

\begin{enumerate}
  \item \textbf{Rewards:} To reward members who do well and to encourage good performance, we will recognize their contributions and celebrate their achievements. They will have the option to decide which task they would like to work on for the next deliverable.
  \item \textbf{Managing underperformers:} We will consider a team member to be underperforming if they are not contributing to the project or are not meeting deadlines. To manage this, we will have a team meeting discussing the issue and offering support and guidance to the team member. If the issue persists, the team member will need to contribute to more tasks to make up for the work of the other team members.
\end{enumerate}


\subsubsection*{Team Building}

\wss{How will you build team cohesion (fun time, group rituals, etc.)? }

\subsubsection*{Decision Making} 

\wss{How will you make decisions in your group? Consensus?  Vote? How will you
handle disagreements? }

\end{document}
