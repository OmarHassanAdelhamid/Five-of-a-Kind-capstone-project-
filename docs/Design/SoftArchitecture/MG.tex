\documentclass[12pt, titlepage]{article}

\usepackage{fullpage}
\usepackage[round]{natbib}
\usepackage{multirow}
\usepackage{booktabs}
\usepackage{tabularx}
\usepackage{graphicx}
\usepackage{float}
\usepackage{amsmath}
\usepackage{amssymb}
\usepackage{hyperref}
\hypersetup{
    colorlinks,
    citecolor=blue,
    filecolor=black,
    linkcolor=red,
    urlcolor=blue
}

\input{../../Comments}
%% Common Parts

\newcommand{\progname}{AutoVox} % PUT YOUR PROGRAM NAME HERE
\newcommand{\authname}{Team \#10, Five of a Kind
\\ Omar Abdelhamid
\\ Daniel Maurer
\\ Andrew Bovbel
\\ Olivia Reich
\\ Khalid Farag
} % AUTHOR NAMES                  

\usepackage{hyperref}
    \hypersetup{colorlinks=true, linkcolor=blue, citecolor=blue, filecolor=blue,
                urlcolor=blue, unicode=false}
    \urlstyle{same}
                                


\newcounter{acnum}
\newcommand{\actheacnum}{AC\theacnum}
\newcommand{\acref}[1]{AC\ref{#1}}

\newcounter{ucnum}
\newcommand{\uctheucnum}{UC\theucnum}
\newcommand{\uref}[1]{UC\ref{#1}}

\newcounter{mnum}
\newcommand{\mthemnum}{M\themnum}
\newcommand{\mref}[1]{M\ref{#1}}

\begin{document}

\title{Module Guide for \progname{}} 
\author{\authname}
\date{\today}

\maketitle

\pagenumbering{roman}

\section{Revision History}

\begin{tabularx}{\textwidth}{p{3cm}p{2cm}X}
\toprule {\bf Date} & {\bf Version} & {\bf Notes}\\
\midrule
Date 1 & 1.0 & Notes\\
Date 2 & 1.1 & Notes\\
\bottomrule
\end{tabularx}

\newpage

\section{Reference Material}

This section records information for easy reference.

\subsection{Abbreviations and Acronyms}

\renewcommand{\arraystretch}{1.2}
\begin{tabular}{l l} 
  \toprule		
  \textbf{symbol} & \textbf{description}\\
  \midrule 
  AC & Anticipated Change\\
  DAG & Directed Acyclic Graph \\
  M & Module \\
  MG & Module Guide \\
  OS & Operating System \\
  R & Requirement\\
  SC & Scientific Computing \\
  SRS & Software Requirements Specification\\
  \progname & Explanation of program name\\
  UC & Unlikely Change \\
  \wss{etc.} & \wss{...}\\
  \bottomrule
\end{tabular}\\

\newpage

\tableofcontents

\listoftables

\listoffigures

\newpage

\pagenumbering{arabic}

\section{Introduction}

Decomposing a system into modules is a commonly accepted approach to developing
software.  A module is a work assignment for a programmer or programming
team~\citep{ParnasEtAl1984}.  We advocate a decomposition
based on the principle of information hiding~\citep{Parnas1972a}.  This
principle supports design for change, because the ``secrets'' that each module
hides represent likely future changes.  Design for change is valuable in SC,
where modifications are frequent, especially during initial development as the
solution space is explored.  

Our design follows the rules layed out by \citet{ParnasEtAl1984}, as follows:
\begin{itemize}
\item System details that are likely to change independently should be the
  secrets of separate modules.
\item Each data structure is implemented in only one module.
\item Any other program that requires information stored in a module's data
  structures must obtain it by calling access programs belonging to that module.
\end{itemize}

After completing the first stage of the design, the Software Requirements
Specification (SRS), the Module Guide (MG) is developed~\citep{ParnasEtAl1984}. The MG
specifies the modular structure of the system and is intended to allow both
designers and maintainers to easily identify the parts of the software.  The
potential readers of this document are as follows:

\begin{itemize}
\item New project members: This document can be a guide for a new project member
  to easily understand the overall structure and quickly find the
  relevant modules they are searching for.
\item Maintainers: The hierarchical structure of the module guide improves the
  maintainers' understanding when they need to make changes to the system. It is
  important for a maintainer to update the relevant sections of the document
  after changes have been made.
\item Designers: Once the module guide has been written, it can be used to
  check for consistency, feasibility, and flexibility. Designers can verify the
  system in various ways, such as consistency among modules, feasibility of the
  decomposition, and flexibility of the design.
\end{itemize}

The rest of the document is organized as follows. Section
\ref{SecChange} lists the anticipated and unlikely changes of the software
requirements. Section \ref{SecMH} summarizes the module decomposition that
was constructed according to the likely changes. Section \ref{SecConnection}
specifies the connections between the software requirements and the
modules. Section \ref{SecMD} gives a detailed description of the
modules. Section \ref{SecTM} includes two traceability matrices. One checks
the completeness of the design against the requirements provided in the SRS. The
other shows the relation between anticipated changes and the modules. Section
\ref{SecUse} describes the use relation between modules.

\section{Anticipated and Unlikely Changes} \label{SecChange}

This section lists possible changes to the system. According to the likeliness
of the change, the possible changes are classified into two
categories. Anticipated changes are listed in Section \ref{SecAchange}, and
unlikely changes are listed in Section \ref{SecUchange}.

\subsection{Anticipated Changes} \label{SecAchange}

Anticipated changes are the source of the information that is to be hidden
inside the modules. Ideally, changing one of the anticipated changes will only
require changing the one module that hides the associated decision. The approach
adapted here is called design for
change.

\begin{description}
\item[\refstepcounter{acnum} \actheacnum \label{acHardware}:] The specific
  hardware on which the software is running.
\item[\refstepcounter{acnum} \actheacnum \label{acInput}:] The format of the
  initial input data.
\item ...
\end{description}

\wss{Anticipated changes relate to changes that would be made in requirements,
design or implementation choices.  They are not related to changes that are made
at run-time, like the values of parameters.}

\subsection{Unlikely Changes} \label{SecUchange}

The module design should be as general as possible. However, a general system is
more complex. Sometimes this complexity is not necessary. Fixing some design
decisions at the system architecture stage can simplify the software design. If
these decision should later need to be changed, then many parts of the design
will potentially need to be modified. Hence, it is not intended that these
decisions will be changed.

\begin{description}
\item[\refstepcounter{ucnum} \uctheucnum \label{ucIO}:] Input/Output devices
  (Input: File and/or Keyboard, Output: File, Memory, and/or Screen).
\item ...
\end{description}

\section{Module Hierarchy} \label{SecMH}

This section provides an overview of the module design. Modules are summarized
in a hierarchy decomposed by secrets in Table \ref{TblMH}. The modules listed
below, which are leaves in the hierarchy tree, are the modules that will
actually be implemented.

\begin{description}
\item [\refstepcounter{mnum} \mthemnum \label{mHH}:] Hardware-Hiding Module
\item ...
\end{description}


\begin{table}[h!]
\centering
\begin{tabular}{p{0.3\textwidth} p{0.6\textwidth}}
\toprule
\textbf{Level 1} & \textbf{Level 2}\\
\midrule

{Hardware-Hiding Module} & ~ \\
\midrule

\multirow{7}{0.3\textwidth}{Behaviour-Hiding Module} & ?\\
& ?\\
& ?\\
& ?\\
& ?\\
& ?\\
& ?\\ 
& ?\\
\midrule

\multirow{3}{0.3\textwidth}{Software Decision Module} & {?}\\
& ?\\
& ?\\
\bottomrule

\end{tabular}
\caption{Module Hierarchy}
\label{TblMH}
\end{table}

\section{Connection Between Requirements and Design} \label{SecConnection}

The design of the system is intended to satisfy the requirements developed in
the SRS. In this stage, the system is decomposed into modules. The connection
between requirements and modules is listed in Table~\ref{TblRT}.

\wss{The intention of this section is to document decisions that are made
  ``between'' the requirements and the design.  To satisfy some requirements,
  design decisions need to be made.  Rather than make these decisions implicit,
  they are explicitly recorded here.  For instance, if a program has security
  requirements, a specific design decision may be made to satisfy those
  requirements with a password.}

\section{Module Decomposition} \label{SecMD}

Modules are decomposed according to the principle of ``information hiding''
proposed by \citet{ParnasEtAl1984}. The \emph{Secrets} field in a module
decomposition is a brief statement of the design decision hidden by the
module. The \emph{Services} field specifies \emph{what} the module will do
without documenting \emph{how} to do it. For each module, a suggestion for the
implementing software is given under the \emph{Implemented By} title. If the
entry is \emph{OS}, this means that the module is provided by the operating
system or by standard programming language libraries.  \emph{\progname{}} means the
module will be implemented by the \progname{} software.

Only the leaf modules in the hierarchy have to be implemented. If a dash
(\emph{--}) is shown, this means that the module is not a leaf and will not have
to be implemented.

\subsection{Hardware Hiding Modules (\mref{mHH})}

\begin{description}
\item[Secrets:]The data structure and algorithm used to implement the virtual
  hardware.
\item[Services:]Serves as a virtual hardware used by the rest of the
  system. This module provides the interface between the hardware and the
  software. So, the system can use it to display outputs or to accept inputs.
\item[Implemented By:] OS
\end{description}

\subsection{Behaviour-Hiding Module}



\subsubsection{Input Interpreter Module (\mref{m1})}

\begin{description}
\item[Secrets:] How data is extracted, parsed, and normalized from any compatible input file into the model’s internal format. This includes how the system identifies file type (STL, JSON, or binary), parses and interprets geometry or voxel-based structures, applies scaling and coordinate normalization, and handles malformed data or import errors. The internal logic for combining geometric mesh data with project metadata, and the reconstruction of previously exported project files, is hidden within this module.
\item[Services:] Provides a unified import interface for all supported input formats. When the user selects a file to load, this module determines the file type and converts it into a consistent intermediate representation. For STL files, it extracts geometric mesh data (facets, vertices, normals) and scales it to the target voxel grid resolution. For past project files, it deserializes voxel data and associated metadata to rebuild the model state. It outputs a structured Intermediate Model object compatible with downstream modules such as Voxel Slicing (M2) and Serialization Manager (M5).
\item[Implemented By:] \teamname{}
\item[Type of Module:] Library: a reusable parsing and normalization component that handles all input data extraction for both new and existing projects.
\end{description}

\paragraph{Input Interpreter Formalization}





\begin{enumerate}
\item \textbf{Definitions}
\begin{itemize}
\item \textbf{Input File:} $F_{input} \in \{F_{STL}, F_{project}\}$ 
\begin{itemize}
\item $F_{STL} = (H, T)$ where $H$ is the header and $T = \{t_1, t_2, ..., t_n\}$ are triangular facets
\item $F_{project} = (V, L, M)$ where $V$ is voxel grid, $L$ layer data, $M$ metadata
\end{itemize}
\item \textbf{Facet:} $t_i = (n_i, v_{i1}, v_{i2}, v_{i3})$, $n_i \in \mathbb{R}^3$, $v_{ij} \in \mathbb{R}^3$
\item \textbf{Scaling Factor:} $s \in \mathbb{R}^+$
\item \textbf{Intermediate Model:} $M_{intermediate} = (V, F, Meta)$
\end{itemize}

\item \textbf{Import Process}
\begin{itemize}
\item \textbf{File Type Detection:}
$$Type(F_{input}) = 
\begin{cases}
STL & \text{if mesh identifier detected}\\
Project & \text{if serialized metadata signature detected}
\end{cases}$$
\item \textbf{Parsing Logic:}
$$M_{raw} =
\begin{cases}
ParseSTL(F_{STL}) & \text{if } Type = STL\\
Deserialize(F_{project}) & \text{if } Type = Project
\end{cases}$$
\item \textbf{Scaling:} $v' = s \cdot v, \forall v \in M_{raw}$
\item \textbf{Normalization:} $v'' = v' - \frac{(x_{min}+x_{max},y_{min}+y_{max},z_{min}+z_{max})}{2}$
\end{itemize}

\item \textbf{Error Handling}
\begin{itemize}
\item Degenerate facets $\rightarrow$ \texttt{InvalidFacetError}
\item Missing metadata $\rightarrow$ \texttt{ImportFormatError}
\item Unsupported extension $\rightarrow$ \texttt{UnsupportedFileTypeError}
\end{itemize}

\item \textbf{Output}
\begin{itemize}
\item $M_{intermediate} = (V', F', Meta)$
\item Supplies structured geometry or voxel data to M2 and M5
\end{itemize}
\end{enumerate}



\subsubsection{Voxel Slicing Module (\mref{m2})}

\begin{description}
\item[Secrets:] How the input geometry or intermediate model is converted into a 3D voxel grid representation. The algorithms for slicing along the Z-axis, detecting layer boundaries, mapping voxels to spatial coordinates, and optimizing voxelization resolution are hidden within this module.
\item[Services:] Converts geometric mesh or input model into voxel representation. Generates discrete voxel layers by slicing through the 3D model at fixed Z intervals. Produces a voxel grid structure compatible with Model Structure (M16) and Model Manager (M8). Maintains layer order and voxel occupancy data for further processing.
\item[Implemented By:] \teamname{}
\item[Type of Module:] Abstract Object: a computational module responsible for transforming geometry into voxelized form.
\end{description}

\paragraph{Voxel Slicing Formalization}

\begin{enumerate}
\item \textbf{Definitions}
\begin{itemize}
\item \textbf{Input Mesh:} $M_{intermediate} = (V, F)$
\item \textbf{Voxel Grid:} $G = \{v_{xyz}\ |\ x,y,z \in \mathbb{N}\}$, discrete points in grid space
\item \textbf{Voxel Resolution:} $(r_x, r_y, r_z) \in \mathbb{R}^3$
\item \textbf{Layer:} $l_k = \{v_{xyz} \mid z = k \cdot r_z\}$
\end{itemize}

\item \textbf{Voxelization Algorithm}
\begin{itemize}
\item For each facet $t_i$ in $F$, determine all grid cells intersected by $t_i$
\item Mark those grid cells as occupied voxels $v_{xyz}$
\item Group occupied voxels into layers by $z$ coordinate
\end{itemize}

\item \textbf{Transformations}
\[
G = Voxelize(M_{intermediate}, r_x, r_y, r_z)
\]
\[
L = \{l_1, l_2, ..., l_n\}
\]

\item \textbf{Output}
\begin{itemize}
\item Returns $(G, L)$ to Model Manager (M8)
\item Provides geometric discretization data for magnetization and material assignment
\end{itemize}
\end{enumerate}



\subsubsection{Serialization Manager Module (\mref{m5})}

\begin{description}
\item[Secrets:] How data structures representing the 3D model are serialized into and deserialized from persistent storage formats (JSON or binary). Includes compression, backward compatibility logic, and error recovery during serialization.
\item[Services:] Converts model data between in-memory representation and serialized form. Encodes voxel grid, layer, and metadata into JSON/binary for saving and reconstructs them when loading. Used by Database Handler (M18) and Export Manager (M13) to ensure consistent data persistence.
\item[Implemented By:] \teamname{}
\item[Type of Module:] Library: a reusable component for encoding and decoding model data.
\end{description}

\paragraph{Serialization Manager Formalization}

\begin{enumerate}
\item \textbf{Definitions}
\begin{itemize}
\item \textbf{Model Structure:} $M = (G, L, Meta)$
\item \textbf{Serialized Form:} $S = Encode(M)$, where $S$ is a JSON/binary string
\item \textbf{Deserialized Structure:} $M' = Decode(S)$
\end{itemize}

\item \textbf{Conversion Process}
\begin{itemize}
\item \textbf{Serialization:} $Encode(M) = JSON.stringify(M)$
\item \textbf{Deserialization:} $Decode(S) = JSON.parse(S)$
\item Maintain consistency: $Decode(Encode(M)) = M$
\end{itemize}

\item \textbf{Error Handling}
\begin{itemize}
\item If corrupted $S$: raise \texttt{DeserializationError}
\item If missing fields: raise \texttt{SchemaMismatchError}
\end{itemize}

\item \textbf{Output}
\begin{itemize}
\item Returns $S$ for saving or $M'$ for reconstruction
\end{itemize}
\end{enumerate}



\subsubsection{Model Manager Module (\mref{m8})}

\begin{description}
\item[Secrets:] How voxel updates, material and magnetization assignments, and auto-save processes are handled internally. How changes propagate through layers, how consistency is maintained across the model, and how synchronization with other modules occurs.
\item[Services:] Provides editing functionality for voxel data. Allows adding, removing, or modifying voxels, and updating material or magnetization values. Coordinates auto-save and triggers History Manager (M9) updates. Interacts with Serialization Manager (M5) for persistent state.
\item[Implemented By:] \teamname{}
\item[Type of Module:] Abstract Object: a controller for maintaining and updating the voxel model state.
\end{description}

\paragraph{Model Manager Formalization}

\begin{enumerate}
\item \textbf{Definitions}
\begin{itemize}
\item \textbf{Model:} $M = (G, L, Meta)$
\item \textbf{Voxel:} $v = (p, m, mag)$
\item \textbf{Modification:} $\Delta M = \{ \Delta v_1, \Delta v_2, ...\}$ set of voxel edits
\end{itemize}

\item \textbf{Update Process}
\[
M' = ApplyChanges(M, \Delta M)
\]
\[
HistoryManager.Push(\Delta M)
\]

\item \textbf{Auto-save:}
\[
if\ elapsed\_time > t_{threshold} \Rightarrow SerializationManager.Save(M')
\]

\item \textbf{Constraints}
\begin{itemize}
\item Layer integrity preserved: $Order(L) = constant$
\item Valid material assignment: $\forall v, m \in M_{types}$
\item Valid magnetization vector: $\forall v, \|mag\| \leq 1$
\end{itemize}
\end{enumerate}



\subsubsection{History Manager Module (\mref{m9})}

\begin{description}
\item[Secrets:] How model changes are tracked, stored, and reversed. Implementation of delta-based versioning, memory-efficient state saving, and undo/redo indexing.
\item[Services:] Maintains chronological history of model changes. Enables undo/redo functionality by restoring previous states. Works closely with Model Manager (M8) to record modifications and Serialization Manager (M5) to store snapshots.
\item[Implemented By:] \teamname{}
\item[Type of Module:] Abstract Object: a component for model state tracking and version control.
\end{description}

\paragraph{History Manager Formalization}

\begin{enumerate}
\item \textbf{Definitions}
\begin{itemize}
\item \textbf{State Sequence:} $\Sigma = [M_0, M_1, ..., M_t]$
\item \textbf{Current State Index:} $i \in [0,t]$
\end{itemize}

\item \textbf{Operations}
\begin{itemize}
\item \textbf{Record Change:} $Push(M_{i+1}) \Rightarrow \Sigma = \Sigma \cup M_{i+1}$
\item \textbf{Undo:} $i' = i - 1$, restore $M_{i'}$
\item \textbf{Redo:} $i' = i + 1$, restore $M_{i'}$
\end{itemize}

\item \textbf{Constraints}
\begin{itemize}
\item History length limited: $|\Sigma| \leq N_{max}$
\item No forward redo after new change: $M_{i+1..t}$ discarded on new edit
\end{itemize}
\end{enumerate}
\subsubsection{Voxel Tracking Module (\mref{m10})}

\begin{description}
\item[Secrets:] How the system determines which voxels satisfy a given property such as material type, magnetization vector, or spatial location. The internal search algorithms, indexing structures, and filtering logic are hidden within this module. Optimizations for performance (e.g., spatial hashing or pre-computed lookup tables) are also concealed.
\item[Services:] Provides query and filtering capabilities over the voxel grid. Given a property condition, it returns the set of voxels that match. Supplies data for Highlight Manager (M11) to visualize these voxels. Supports compound queries such as logical AND/OR between material, magnetization, and position constraints.
\item[Implemented By:] \teamname{}
\item[Type of Module:] Abstract Object: a computational component for property-based voxel identification and filtering.
\end{description}

\paragraph{Voxel Tracking Formalization}

\begin{enumerate}
\item \textbf{Definitions}
\begin{itemize}
\item \textbf{Voxel:} $v = (p, m, mag)$ where $p=(x,y,z)\in\mathbb{Z}^3$, $m\in M_{types}$, $mag\in\mathbb{R}^3$
\item \textbf{Voxel Grid:} $G=\{v_1,v_2,\ldots,v_n\}$
\item \textbf{Property Predicate:} $\phi(v)$ — boolean condition evaluating a voxel’s property
\item \textbf{Result Set:} $R=\{v\in G \mid \phi(v)=true\}$
\end{itemize}

\item \textbf{Tracking Process}
\begin{itemize}
\item Evaluate $\phi(v)$ for all $v\in G$
\item Collect $R=\{v_i\ |\ \phi(v_i)\}$ for highlighting or further analysis
\end{itemize}

\item \textbf{Supported Predicates}
\begin{itemize}
\item Material match: $\phi_m(v)\equiv(v.m=m_{target})$
\item Magnetization threshold: $\phi_{mag}(v)\equiv(\|v.mag - mag_{target}\|<\epsilon)$
\item Spatial region: $\phi_{p}(v)\equiv(v.p\in region)$
\end{itemize}

\item \textbf{Composite Queries}
\[
\phi(v)=
\begin{cases}
\phi_m(v)\land\phi_{mag}(v) & \text{(AND)}\\
\phi_m(v)\lor\phi_{p}(v) & \text{(OR)}
\end{cases}
\]

\item \textbf{Output}
\begin{itemize}
\item Returns set $R$ to Highlight Manager (M11)
\item Guarantees deterministic results for identical predicates
\end{itemize}
\end{enumerate}



\begin{description}
\item[Secrets:]The contents of the required behaviours.
\item[Services:]Includes programs that provide externally visible behaviour of
  the system as specified in the software requirements specification (SRS)
  documents. This module serves as a communication layer between the
  hardware-hiding module and the software decision module. The programs in this
  module will need to change if there are changes in the SRS.
\item[Implemented By:] --
\end{description}

\subsubsection{Input Format Module (\mref{mInput})}

\begin{description}
\item[Secrets:]The format and structure of the input data.
\item[Services:]Converts the input data into the data structure used by the
  input parameters module.
\item[Implemented By:] [Your Program Name Here]
\item[Type of Module:] [Record, Library, Abstract Object, or Abstract Data Type]
  [Information to include for leaf modules in the decomposition by secrets tree.]
\end{description}

\subsubsection{Etc.}


\subsection{Software Decision Module}

\subsubsection{Database Handler Module (\mref{m18})}

\begin{description}
\item[Secrets:] How persistent project and model data are stored, indexed, and accessed. The specific file-system or database technology (e.g., JSON files, SQLite, binary blobs) and its connection management are hidden. Includes caching and concurrency handling mechanisms.
\item[Services:] Provides read and write access between the application and storage layer. Saves serialized models and retrieves them when requested. Ensures data integrity and version consistency using the Serialization Manager (M5). Supports CRUD operations for project entries and manages file metadata such as timestamps and file paths.
\item[Implemented By:] \teamname{}
\item[Type of Module:] Library: a persistence interface between the application logic and the underlying storage mechanism.
\end{description}

\paragraph{Database Handler Formalization}

\begin{enumerate}
\item \textbf{Definitions}
\begin{itemize}
\item \textbf{Serialized Data:} $S=Encode(M)$ from Serialization Manager (M5)
\item \textbf{Storage Mapping:} $\mu: ProjectID \rightarrow S$
\item \textbf{Database:} $D=\{(id_i,S_i)\mid id_i\in ID\_{projects}\}$
\end{itemize}

\item \textbf{Operations}
\begin{itemize}
\item \textbf{Save:} $Store(D,id,S): D\leftarrow D\cup (id,S)$
\item \textbf{Load:} $Retrieve(D,id)\rightarrow S$
\item \textbf{Delete:} $Remove(D,id): D\leftarrow D\setminus (id,S)$
\end{itemize}

\item \textbf{Constraints}
\begin{itemize}
\item Unique project identifier: $\forall id_i,id_j, i\neq j \Rightarrow id_i\neq id_j$
\item Integrity: $Decode(Retrieve(D,id))=M$
\item Atomic write: partial failures do not corrupt $D$
\end{itemize}

\item \textbf{Output}
\begin{itemize}
\item Provides serialized model data to higher-level modules (M5, M4)
\item Ensures persistent consistency across sessions
\end{itemize}
\end{enumerate}



\subsubsection{Export Structure Module (\mref{m19})}

\begin{description}
\item[Secrets:] How exported files are structured internally, including field ordering, data types, and encoding format. The schema definition linking voxel properties, layers, and materials is hidden. Details of compatibility with printer software or external tools are encapsulated.
\item[Services:] Defines the data schema used during export. Specifies how voxel data, layer information, and material/magnetization fields are organized. Provides structural templates used by Export Manager (M13) to generate valid output files. Ensures consistency between internal and external representations.
\item[Implemented By:] \teamname{}
\item[Type of Module:] Abstract Data Type: specifies and maintains the structure of exported model data.
\end{description}

\paragraph{Export Structure Formalization}

\begin{enumerate}
\item \textbf{Definitions}
\begin{itemize}
\item \textbf{Export File:} $F_{export}=(Header,Body)$
\item \textbf{Header:} metadata fields $\{version, date, projectID\}$
\item \textbf{Body:} ordered list of voxel entries $E=\{e_1,e_2,\ldots,e_n\}$
\item Each $e_i=(x_i,y_i,z_i,m_i,mag_i)$ with $m_i\in M_{types}$, $mag_i\in\mathbb{R}^3$
\end{itemize}

\item \textbf{Transformation Rule}
\[
F_{export}=Structure(M) =
\{(Header(M), Body(M))\}
\]
where $Body(M)$ flattens voxel grid $G$ into ordered entries $E$.

\item \textbf{Constraints}
\begin{itemize}
\item Field order preserved: $(x,y,z,m,mag)$
\item All entries valid: $\forall e_i, m_i\neq\emptyset, \|mag_i\|\le1$
\item Backward-compatible schema versioning maintained
\end{itemize}

\item \textbf{Output}
\begin{itemize}
\item Returns structural definition to Export Manager (M13)
\item Guarantees consistency between internal model representation and exported file layout
\end{itemize}
\end{enumerate}
\begin{description}
\item[Secrets:] The design decision based on mathematical theorems, physical
  facts, or programming considerations. The secrets of this module are
  \emph{not} described in the SRS.
\item[Services:] Includes data structure and algorithms used in the system that
  do not provide direct interaction with the user. 
  % Changes in these modules are more likely to be motivated by a desire to
  % improve performance than by externally imposed changes.
\item[Implemented By:] --
\end{description}

\subsubsection{Etc.}

\section{Traceability Matrix} \label{SecTM}

This section shows two traceability matrices: between the modules and the
requirements and between the modules and the anticipated changes.

% the table should use mref, the requirements should be named, use something
% like fref
\begin{table}[H]
\centering
\begin{tabular}{p{0.2\textwidth} p{0.6\textwidth}}
\toprule
\textbf{Req.} & \textbf{Modules}\\
\midrule
R1 & \mref{mHH}, \mref{mInput}, \mref{mParams}, \mref{mControl}\\
R2 & \mref{mInput}, \mref{mParams}\\
R3 & \mref{mVerify}\\
R4 & \mref{mOutput}, \mref{mControl}\\
R5 & \mref{mOutput}, \mref{mODEs}, \mref{mControl}, \mref{mSeqDS}, \mref{mSolver}, \mref{mPlot}\\
R6 & \mref{mOutput}, \mref{mODEs}, \mref{mControl}, \mref{mSeqDS}, \mref{mSolver}, \mref{mPlot}\\
R7 & \mref{mOutput}, \mref{mEnergy}, \mref{mControl}, \mref{mSeqDS}, \mref{mPlot}\\
R8 & \mref{mOutput}, \mref{mEnergy}, \mref{mControl}, \mref{mSeqDS}, \mref{mPlot}\\
R9 & \mref{mVerifyOut}\\
R10 & \mref{mOutput}, \mref{mODEs}, \mref{mControl}\\
R11 & \mref{mOutput}, \mref{mODEs}, \mref{mEnergy}, \mref{mControl}\\
\bottomrule
\end{tabular}
\caption{Trace Between Requirements and Modules}
\label{TblRT}
\end{table}

\begin{table}[H]
\centering
\begin{tabular}{p{0.2\textwidth} p{0.6\textwidth}}
\toprule
\textbf{AC} & \textbf{Modules}\\
\midrule
\acref{acHardware} & \mref{mHH}\\
\acref{acInput} & \mref{mInput}\\
\acref{acParams} & \mref{mParams}\\
\acref{acVerify} & \mref{mVerify}\\
\acref{acOutput} & \mref{mOutput}\\
\acref{acVerifyOut} & \mref{mVerifyOut}\\
\acref{acODEs} & \mref{mODEs}\\
\acref{acEnergy} & \mref{mEnergy}\\
\acref{acControl} & \mref{mControl}\\
\acref{acSeqDS} & \mref{mSeqDS}\\
\acref{acSolver} & \mref{mSolver}\\
\acref{acPlot} & \mref{mPlot}\\
\bottomrule
\end{tabular}
\caption{Trace Between Anticipated Changes and Modules}
\label{TblACT}
\end{table}

\section{Use Hierarchy Between Modules} \label{SecUse}

In this section, the uses hierarchy between modules is
provided. \citet{Parnas1978} said of two programs A and B that A {\em uses} B if
correct execution of B may be necessary for A to complete the task described in
its specification. That is, A {\em uses} B if there exist situations in which
the correct functioning of A depends upon the availability of a correct
implementation of B.  Figure \ref{FigUH} illustrates the use relation between
the modules. It can be seen that the graph is a directed acyclic graph
(DAG). Each level of the hierarchy offers a testable and usable subset of the
system, and modules in the higher level of the hierarchy are essentially simpler
because they use modules from the lower levels.

\wss{The uses relation is not a data flow diagram.  In the code there will often
be an import statement in module A when it directly uses module B.  Module B
provides the services that module A needs.  The code for module A needs to be
able to see these services (hence the import statement).  Since the uses
relation is transitive, there is a use relation without an import, but the
arrows in the diagram typically correspond to the presence of import statement.}

\wss{If module A uses module B, the arrow is directed from A to B.}

\begin{figure}[H]
\centering
%\includegraphics[width=0.7\textwidth]{UsesHierarchy.png}
\caption{Use hierarchy among modules}
\label{FigUH}
\end{figure}

%\section*{References}

\section{User Interfaces}

\wss{Design of user interface for software and hardware.  Attach an appendix if
needed. Drawings, Sketches, Figma}

\section{Design of Communication Protocols}

\wss{If appropriate}

\section{Timeline}

\wss{Schedule of tasks and who is responsible}

\wss{You can point to GitHub if this information is included there}

\bibliographystyle {plainnat}
\bibliography{../../../refs/References}

\newpage{}

\end{document}
