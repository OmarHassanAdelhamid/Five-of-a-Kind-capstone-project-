\documentclass[12pt, titlepage]{article}

\usepackage{fullpage}
\usepackage[round]{natbib}
\usepackage{multirow}
\usepackage{booktabs}
\usepackage{tabularx}
\usepackage{graphicx}
\usepackage{float}
\usepackage{hyperref}
\usepackage{amsmath}
\usepackage{amsfonts}
\hypersetup{
    colorlinks,
    citecolor=blue,
    filecolor=black,
    linkcolor=red,
    urlcolor=blue
}

\input{../../Comments}
%% Common Parts

\newcommand{\progname}{AutoVox} % PUT YOUR PROGRAM NAME HERE
\newcommand{\authname}{Team \#10, Five of a Kind
\\ Omar Abdelhamid
\\ Daniel Maurer
\\ Andrew Bovbel
\\ Olivia Reich
\\ Khalid Farag
} % AUTHOR NAMES                  

\usepackage{hyperref}
    \hypersetup{colorlinks=true, linkcolor=blue, citecolor=blue, filecolor=blue,
                urlcolor=blue, unicode=false}
    \urlstyle{same}
                                


\newcounter{acnum}
\newcommand{\actheacnum}{AC\theacnum}
\newcommand{\acref}[1]{AC\ref{#1}}

\newcounter{ucnum}
\newcommand{\uctheucnum}{UC\theucnum}
\newcommand{\uref}[1]{UC\ref{#1}}

\newcounter{mnum}
\newcommand{\mthemnum}{M\themnum}
\newcommand{\mref}[1]{M\ref{#1}}

\newcommand{\projectname}{AutoVox}

\begin{document}

\title{Module Guide for \progname{}} 
\author{\authname}
\date{\today}

\maketitle

\pagenumbering{roman}

\section{Revision History}

\begin{tabularx}{\textwidth}{p{3cm}p{2cm}X}
\toprule {\bf Date} & {\bf Version} & {\bf Notes}\\
\midrule
November 10, 2025 & 1.0 & Initial draft by All\\
\bottomrule
\end{tabularx}

\newpage

\section{Reference Material}

This section records information for easy reference.

\subsection{Abbreviations and Acronyms}

\renewcommand{\arraystretch}{1.2}
\begin{tabular}{l l} 
  \toprule		
  \textbf{symbol} & \textbf{description}\\
  \midrule 
  AC & Anticipated Change\\
  API & Application Programming Interface \\
  \progname & Explanation of program name\\
  CAD & Computer-Aided Design\\
  CSV & Comma Separated Values\\
  DAG & Directed Acyclic Graph \\
  M & Module \\
  MG & Module Guide \\
  OBJ & Object file format (3D model)\\
  OS & Operating System \\
  R & Requirement\\
  SC & Scientific Computing \\
  SRS & Software Requirements Specification\\
  STL & Stereolithography file format\\
  UI & User Interface\\
  UC & Unlikely Change \\
  UI & User Interface \\
  XML & Extensible Markup Language\\
  3D & Three Dimensional \\
  \bottomrule
\end{tabular}\\

\newpage

\tableofcontents

\listoftables

\listoffigures

\newpage

\pagenumbering{arabic}

\section{Introduction}

Decomposing a system into modules is a commonly accepted approach to developing
software.  A module is a work assignment for a programmer or programming
team~\citep{ParnasEtAl1984}.  We advocate a decomposition
based on the principle of information hiding~\citep{Parnas1972a}.  This
principle supports design for change, because the ``secrets'' that each module
hides represent likely future changes.  Design for change is valuable in SC,
where modifications are frequent, especially during initial development as the
solution space is explored.  

Our design follows the rules layed out by \citet{ParnasEtAl1984}, as follows:
\begin{itemize}
\item System details that are likely to change independently should be the
  secrets of separate modules.
\item Each data structure is implemented in only one module.
\item Any other program that requires information stored in a module's data
  structures must obtain it by calling access programs belonging to that module.
\end{itemize}

After completing the first stage of the design, the Software Requirements
Specification (SRS), the Module Guide (MG) is developed~\citep{ParnasEtAl1984}. The MG
specifies the modular structure of the system and is intended to allow both
designers and maintainers to easily identify the parts of the software.  The
potential readers of this document are as follows:

\begin{itemize}
\item New project members: This document can be a guide for a new project member
  to easily understand the overall structure and quickly find the
  relevant modules they are searching for.
\item Maintainers: The hierarchical structure of the module guide improves the
  maintainers' understanding when they need to make changes to the system. It is
  important for a maintainer to update the relevant sections of the document
  after changes have been made.
\item Designers: Once the module guide has been written, it can be used to
  check for consistency, feasibility, and flexibility. Designers can verify the
  system in various ways, such as consistency among modules, feasibility of the
  decomposition, and flexibility of the design.
\end{itemize}

The rest of the document is organized as follows. Section
\ref{SecChange} lists the anticipated and unlikely changes of the software
requirements. Section \ref{SecMH} summarizes the module decomposition that
was constructed according to the likely changes. Section \ref{SecConnection}
specifies the connections between the software requirements and the
modules. Section \ref{SecMD} gives a detailed description of the
modules. Section \ref{SecTM} includes two traceability matrices. One checks
the completeness of the design against the requirements provided in the SRS. The
other shows the relation between anticipated changes and the modules. Section
\ref{SecUse} describes the use relation between modules.

\section{Anticipated and Unlikely Changes} \label{SecChange}

This section lists possible changes to the system. According to the likeliness
of the change, the possible changes are classified into two
categories. Anticipated changes are listed in Section \ref{SecAchange}, and
unlikely changes are listed in Section \ref{SecUchange}.

\subsection{Anticipated Changes} \label{SecAchange}

Anticipated changes are the source of the information that is to be hidden
inside the modules. Ideally, changing one of the anticipated changes will only
require changing the one module that hides the associated decision. The approach
adapted here is called design for
change.

\begin{description}
\item[\refstepcounter{acnum} \actheacnum \label{acInput}:] The format of the initial input file (CAD file formats). Initially, the system will support STL files, but anticipated changes could require supporting additional CAD formats used in the industry. The change in input format affects the parsing logic, error handling, and inetgration with visualization manager. Changes will require updates to file importers and related data validation processes.

\item[\refstepcounter{acnum} \actheacnum \label{acInputConstraints}:] The constraints on the input parameters such as the maximum number of voxels and the maximum number of layers that can be displayed. These limits are currently fixed at 13,996,800,000 voxels and 518,400 voxels, but may need to be configurable in the future. Changes to these constraints will require updates to the input validation and parsing logic.

\item[\refstepcounter{acnum} \actheacnum \label{acVoxelSize}:] The dimensions of the voxel grid. The dimensions of the voxel grid are currently fixed at $300 \times 300\,\mu\mathrm{m}$ (XY) and $110\,\mu\mathrm{m}$ (Z), but may need to be configurable in the future as the user may want to use a different voxel size. Changes to these constraints will require updates to the voxelization algorithm and the layer navigation logic.

\item[\refstepcounter{acnum} \actheacnum \label{acVoxelization}:] The algorithm used to convert the mesh CAD file into a voxel grid. Anticipated changes will require updates to the voxelization algorithm to improve the accuracy, performance, or new features. 

\item[\refstepcounter{acnum} \actheacnum \label{acMagnetization}:] The representation and format of magnetization direction data placed on the voxels, either by vector representation or coordinate system. This could involve changing the data structure (e.g., from Cartesian coordinates to Euler angles), supporting both absolute and relative directions, or accommodating multiple coordinate systems. Changes will require updates to the magnetization assignment logic.

\item[\refstepcounter{acnum} \actheacnum \label{acSelection}:] The implementation of selecting multiple voxels. Currently multi-selection is supported by selecting each voxel individually or an entire layer at once. Changes will require updates to the selection logic considering a more flexible selection method such lasso selection or rectangle selection.

\item[\refstepcounter{acnum} \actheacnum \label{acExport}:] The exported file format for the custom printer software is currently CSV. The file must contain the voxel location, layer, and magnetization direction and material ID data. Anticipated changes could require supporting other formats (such as JSON, XML, or other file formats). Changes will require updates to the export logic.

\item[\refstepcounter{acnum} \actheacnum \label{acDataStructure}:] The implementation of the data structures used to store and manage voxel grids and voxel properties. This could involve using different data structures such as a tree structure or a graph structure optimizing the performance metrics of the system, metrics such as memory usage, search speed, or parallel processing. Anticipating such changes allows for future performance and scalability enhancements. Changes will require updates to the data structures.

\item[\refstepcounter{acnum} \actheacnum \label{acUI}:] The layout and interaction methods of the user interface, including menu structure, keyboard shortcuts, and user feedback mechanisms. This anticipated change helps accommodate different user workflows, accessibility requirements, or even support for new input devices.  Changes will require updates to the UI logic.

\end{description}

\iffalse
\wss{Anticipated changes relate to changes that would be made in requirements,
design or implementation choices.  They are not related to changes that are made
at run-time, like the values of parameters.}
\fi

\subsection{Unlikely Changes} \label{SecUchange}

The module design should be as general as possible. However, a general system is
more complex. Sometimes this complexity is not necessary. Fixing some design
decisions at the system architecture stage can simplify the software design. If
these decision should later need to be changed, then many parts of the design
will potentially need to be modified. Hence, it is not intended that these
decisions will be changed.

\begin{description}
\item[\refstepcounter{ucnum} \uctheucnum \label{ucIO}:] Input/Output devices. The system is designed with the expectation that it operates with a limited and specific set of input and output devices. Input is handled through files (such as CAD input files or configuration files) and/or keyboard entry (for command-line or interactive parameter input). Output is provided via files (for export and logging), memory (for internal data storage and communication between modules), and/or a graphical screen (for visualization and user interface feedback).

\item[\refstepcounter{ucnum} \uctheucnum \label{ucArchitecture}:] The overall system architecture. The software follows a rigid four-manager structure that divides system functionality into Import, Visualization, Editing, and Export managers. Each manager is responsible for a core facet of application operation, and this separation is central to the modular design.

\item[\refstepcounter{ucnum} \uctheucnum \label{ucLayerOrder}:] The ordering of layers from bottom to top (Z-axis ordering). Layers within the voxel grid are organized along the Z-axis, proceeding from the lowest layer (the base) to the highest (the top). This convention aligns with common additive manufacturing and 3D modeling practices.

\item[\refstepcounter{ucnum} \uctheucnum \label{ucInteraction}:] The fundamental user interaction paradigm. The system targets a desktop usage scenario, using mouse and keyboard as primary input methods. This supports rapid interaction, direct manipulation of interface elements, and fine control over editing tasks.

\end{description}

\section{Module Hierarchy} \label{SecMH}

This section provides an overview of the module design. Modules are summarized
in a hierarchy decomposed by secrets in Table \ref{TblMH}. The modules listed
below, which are leaves in the hierarchy tree, are the modules that will
actually be implemented.

\begin{description}
% Behaviour Hiding Modules
\item [\refstepcounter{mnum} \mthemnum \label{m1}:] Input Interpreter Module
\item [\refstepcounter{mnum} \mthemnum \label{m2}:] Voxel Slicing Module
\item [\refstepcounter{mnum} \mthemnum \label{m3}:] Display Partitioning Module
\item [\refstepcounter{mnum} \mthemnum \label{m4}:] Project Manager Module
\item [\refstepcounter{mnum} \mthemnum \label{m5}:] Serialization Manager Module
\item [\refstepcounter{mnum} \mthemnum \label{m6}:] Backend Communication Manager Module
\item [\refstepcounter{mnum} \mthemnum \label{m7}:] Interaction Controller Module
\item [\refstepcounter{mnum} \mthemnum \label{m8}:] Visualization State Manager Module
\item [\refstepcounter{mnum} \mthemnum \label{m9}:] Model Manager Module
\item [\refstepcounter{mnum} \mthemnum \label{m10}:] History Manager Module
\item [\refstepcounter{mnum} \mthemnum \label{m11}:] Autosave Manager Module
\item [\refstepcounter{mnum} \mthemnum \label{m12}:] Voxel Tracking Module
\item [\refstepcounter{mnum} \mthemnum \label{m13}:] Highlight Manager Module
\item [\refstepcounter{mnum} \mthemnum \label{m14}:] Export Validation Module
\item [\refstepcounter{mnum} \mthemnum \label{m15}:] Export Manager Module
\item [\refstepcounter{mnum} \mthemnum \label{m16}:] Error Diagnostic Handler Module

% Software Decision Modules
\item [\refstepcounter{mnum} \mthemnum \label{m17}:] Model Structure Module
\item [\refstepcounter{mnum} \mthemnum \label{m18}:] Graphics Adapter Module
\item [\refstepcounter{mnum} \mthemnum \label{m19}:] Database Handler Module
\item [\refstepcounter{mnum} \mthemnum \label{m20}:] Export Structure Module
\end{description}




\section{Connection Between Requirements and Design} \label{SecConnection}

The design of the system is intended to satisfy the requirements developed in
the SRS. In this stage, the system is decomposed into modules. The connection
between requirements and modules is listed in Table~\ref{TblMH}.

\begin{table}[h!]
\centering
\begin{tabular}{p{0.3\textwidth} p{0.6\textwidth}}
\toprule
\textbf{Level 1} & \textbf{Level 2}\\
\midrule

{Hardware-Hiding Module} & None \\
\midrule

\multirow{7}{0.3\textwidth}{Behaviour-Hiding Module} & Input Interpreter Module\\
& Voxel Slicing Module\\
& Display Partitioning Module\\
& Project Manager Module\\
& Serialization Manager Module\\
& Backend Communication Manager Module\\
& Interaction Controller Module\\
& Visualization State Manager Module\\
& Model Manager Module\\
& History Manager Module\\
& Autosave Manager Module\\
& Voxel Tracking Module\\
& Highlight Manager Module\\
& Export Validation Module\\
& Export Manager Module\\
& Error Diagnostic Handler Module\\
\midrule

\multirow{3}{0.3\textwidth}{Software Decision Module} & Model Structure Module\\
& Graphics Adapter Module\\
& Database Handler Module\\
& Export Structure Module\\
\bottomrule

\end{tabular}
\caption{Module Hierarchy}
\label{TblMH}
\end{table}

\iffalse
\wss{The intention of this section is to document decisions that are made
  ``between'' the requirements and the design.  To satisfy some requirements,
  design decisions need to be made.  Rather than make these decisions implicit,
  they are explicitly recorded here.  For instance, if a program has security
  requirements, a specific design decision may be made to satisfy those
  requirements with a password.}
\fi

\section{Module Decomposition} \label{SecMD}

Modules are decomposed according to the principle of ``information hiding''
proposed by \citet{ParnasEtAl1984}. The \emph{Secrets} field in a module
decomposition is a brief statement of the design decision hidden by the
module. The \emph{Services} field specifies \emph{what} the module will do
without documenting \emph{how} to do it. For each module, a suggestion for the
implementing software is given under the \emph{Implemented By} title. If the
entry is \emph{OS}, this means that the module is provided by the operating
system or by standard programming language libraries.  \emph{\progname{}} means the
module will be implemented by the \progname{} software.

Only the leaf modules in the hierarchy have to be implemented. If a dash
(\emph{--}) is shown, this means that the module is not a leaf and will not have
to be implemented.

\subsection{Hardware Hiding Modules (None)}

This system has no hardware components.

\subsection{Behaviour-Hiding Module}

\subsubsection{Input Interpreter Module (\mref{m1})}

\begin{description}
  \item[Secrets:]
  \item[Services:]
  \item[Implemented By:] \projectname{}
  \item[Type of Module:] [Record, Library, Abstract Object, or Abstract Data Type]
  [Information to include for leaf modules in the decomposition by secrets tree.]
\end{description}

\subsubsection{Voxel Slicing Module (\mref{m2})}

\begin{description}
  \item[Secrets:]
  \item[Services:]
  \item[Implemented By:] \projectname{}
  \item[Type of Module:] [Record, Library, Abstract Object, or Abstract Data Type]
  [Information to include for leaf modules in the decomposition by secrets tree.]
\end{description}

\subsubsection{Display Partitioning Module (\mref{m3})}

\begin{description}
  \item[Secrets:] How to determine the boundaries that define model partitions. How to define the relationship
  between display segments. How to represent partition boundaries. How to minimize the number of partitions created. How to ensure that display partitions maintain balanced spatial proportions. How to maintain partition stability during model modification. How to validate that a partition of voxels is viable to display within the UI.
  \item[Services:] Provide functionality for partitioning the model into distinct display segments defined by specific voxel groupings. Manages the relationships and alignment between display segments.
  \item[Implemented By:] \projectname{}
  \item[Type of Module:] Abstract Object
\end{description}

\subsubsection{Project Manager Module (\mref{m4})}

\begin{description}
\item[Secrets:] How to create and initialize a new project workspace. How to establish the intitial project structure, metadata, and default configuration settings. How to allocate and manage project resources and state.
\item[Services:] Creates new projects with appropriate initialization. Manages project setup cycle starting from the creation of a new project workspace to the an interactive user interface.
\item[Implemented By:] \projectname{}
\item[Type of Module:] Abstract Object: A component responsible for project initialization and workspace management.
\end{description} 

\subsubsection{Serialization Manager Module (\mref{m5})}

\begin{description}
  \item[Secrets:]
  \item[Services:]
  \item[Implemented By:] \projectname{}
  \item[Type of Module:] [Record, Library, Abstract Object, or Abstract Data Type]
  [Information to include for leaf modules in the decomposition by secrets tree.]
\end{description}

\subsubsection{Backend Communication Manager Module (\mref{m6})}

\begin{description}
  \item[Secrets:] How the connection between the backend and frontend is established. How data is transformed, validated and transferred between the backend and frontend. How synchronization errors are detected during communication. How failed communication attempts are detected and handled.
  \item[Services:] Handles communication between the backend server and the frontend UI. Ensures synchronization of model data throughout the system.
  \item[Implemented By:] \projectname{}
  \item[Type of Module:] Abstract Object
\end{description}

\subsubsection{Interaction Controller Module (\mref{m7})}

\begin{description}
  \item[Secrets:] How events are internally represented. How to differentiate between similar events. How to define the rules that associate high-level action with a raw event. How to handle raw events where user intent is inconclusive. How to determine when multiple events should be interpreted as a single user action. How to apply context from UI when interpreting events.
  \item[Services:] Maps raw events from interaction with UI to high-level system actions that reflect user intent.
  \item[Implemented By:] \projectname{}
  \item[Type of Module:] Abstract Object
\end{description}

\subsubsection{Visualization State Manager Module (\mref{m8})}

\begin{description}
  \item[Secrets:] How changes are grouped and aggregated. How the unified state-change representation is structured. How a state change is validated as complete. How event propagation from multiple sources is tracked concurrently. How conditions that trigger a UI update are determined. How the order of UI updates is established. How conflicting state changes are resolved. How invalid or irrelevant data-change notifications are handled.
  \item[Services:] Aggregates data-change notifications from event propagation into a unified state-change representation. Triggers UI update used the derived state differences.
  \item[Implemented By:] \projectname{}
  \item[Type of Module:] Abstract Object
\end{description}

\subsubsection{Model Manager Module (\mref{m9})}

\begin{description}
  \item[Secrets:]
  \item[Services:]
  \item[Implemented By:] \projectname{}
  \item[Type of Module:] [Record, Library, Abstract Object, or Abstract Data Type]
  [Information to include for leaf modules in the decomposition by secrets tree.]
\end{description}

\subsubsection{History Manager Module (\mref{m10})}

\begin{description}
  \item[Secrets:]
  \item[Services:]
  \item[Implemented By:] \projectname{}
  \item[Type of Module:] [Record, Library, Abstract Object, or Abstract Data Type]
  [Information to include for leaf modules in the decomposition by secrets tree.]
\end{description}

\subsubsection{Autosave Manager Module (\mref{m11})}

\begin{description}
  \item[Secrets:] How to collect and aggregate metrics to represent the current status of update operations. How to handle errors in metric collection. How multiple metrics are tracked concurrently. How the state of update execution is defined from collected metric data. How autosave execution is triggered. How autosave failures or interruptions are detected and handled.
  \item[Services:] Maintains real-time autosave status. Tracks metrics that are used to determine the current state of update execution.
  \item[Implemented By:] \projectname{}
  \item[Type of Module:] Abstract Object
\end{description}

\subsubsection{Voxel Tracking Module (\mref{m12})}

\begin{description}
  \item[Secrets:]
  \item[Services:]
  \item[Implemented By:] \projectname{}
  \item[Type of Module:] [Record, Library, Abstract Object, or Abstract Data Type]
  [Information to include for leaf modules in the decomposition by secrets tree.]
\end{description}

\subsubsection{Highlight Manager Module (\mref{m13})}

\begin{description}
  \item[Secrets:] How to define the rules that associate voxel semantics with a highlight colour. How highlight colours are internally represented. How conflicts are resolved when a voxel qualifies for multiple highlight colours simultaneously. How colour palette, which encapsulates all highlight colours, is chosen. How accessibility requirements influence the selection of highlight colours.
  \item[Services:] Maps relationship between highlight colour and semantic meaning of highlighted voxels.
  \item[Implemented By:] \projectname{}
  \item[Type of Module:] Abstract Object
\end{description}

\subsubsection{Export Validation Module (\mref{m14})}

\begin{description}
\item[Secrets:] How to validate whether a project metadata file is suitable for export. How to check export file format requirements, verify that all voxels have required material and magnetization properties, and ensure data integrity constraints are satisfied. How to determine export readiness based on validation rules.
\item[Services:] Validates export readiness by checking export file format requirements, verifying completeness of voxel properties (material and magnetization), and ensuring all export constraints are met. Provides validation results and error reporting for export issues.
\item[Implemented By:] \projectname{}
\item[Type of Module:] Library: A reusable component providing validation functionality for export operations.
\end{description} 

\paragraph{Export Validation Formalization}
\begin{enumerate}
\item \textbf{Definitions}
\begin{itemize}
\item \textbf{Project Model:} A structure $P = (V, L, M)$ where:
\begin{itemize}
\item $V = \{v_1, v_2, \ldots, v_n\}$ is a set of voxels
\item $L = \{l_1, l_2, \ldots, l_k\}$ is a set of layers
\item $M$ is metadata containing project information
\end{itemize}
\item \textbf{Voxel:} An element $v_i = (p, m, mag)$ where:
\begin{itemize}
\item $p = (x, y, z)$ represents spatial position
\item $m \in M_{types}$ represents material type (may be $\emptyset$)
\item $mag \in \mathbb{R}^3$ represents magnetization vector (may be $\emptyset$)
\end{itemize}
\item \textbf{Printer Compatibility:} A constraint function $C_{printer}(P) \rightarrow \{valid, invalid\}$
\item \textbf{Property Completeness:} A predicate $Complete(v) \equiv (m \neq \emptyset) \land (mag \neq \emptyset)$
\item \textbf{Export Readiness:} A validation result $R = (status, errors)$ where $status \in \{ready, not\_ready\}$ and $errors$ is a set of validation error messages
\end{itemize}

\item \textbf{Validation Process}
\begin{itemize}
\item \textbf{Printer Compatibility Check:}
$$C_{printer}(P) = \begin{cases}
valid & \text{if } P \text{ satisfies printer specifications}\\
invalid & \text{otherwise}
\end{cases}$$
\item \textbf{Property Completeness Check:}
$$V_{complete} = \{v_i \in V \mid Complete(v_i)\}$$
$$V_{incomplete} = V \setminus V_{complete}$$
\item \textbf{Export Readiness Determination:}
$$R.status = \begin{cases}
ready & \text{if } C_{printer}(P) = valid \land V_{incomplete} = \emptyset\\
not\_ready & \text{otherwise}
\end{cases}$$
$$R.errors = \begin{cases}
\{C_{printer}(P).errors\} & \text{if } C_{printer}(P) = invalid\\
\{\text{``Voxel } v_i \text{ missing property''} \mid v_i \in V_{incomplete}\} & \text{otherwise}
\end{cases}$$
\end{itemize}
\item \textbf{Validation Constraints}
\begin{itemize}
\item All voxels must have assigned material: $\forall v_i \in V, m_i \neq \emptyset$
\item All voxels must have assigned magnetization: $\forall v_i \in V, mag_i \neq \emptyset$
\item Model must satisfy printer compatibility constraints: $C_{printer}(P) = valid$
\end{itemize}
\end{enumerate}

\subsubsection{Export Manager Module (\mref{m15})}

\begin{description}
\item[Secrets:] How to export a project with the defined data structure format. How to transform internal model data into the export file format. How to serialize project data including voxel grids, metadata, material properties, and magnetization information according to export specifications.
\item[Services:] Exports projects to external file formats with the defined data structure. Handles the conversion of internal model representations to export-compatible formats, ensuring data integrity and completeness during the export process.
\item[Implemented By:] \projectname{}
\item[Type of Module:] Abstract Object: A component responsible for coordinating the export process and data transformation.
\end{description}

\paragraph{Export Manager Formalization}
\begin{enumerate}
\item \textbf{Definitions}
\begin{itemize}
\item \textbf{Internal Model:} A structure $M_{internal} = (V, L, Meta)$ where:
\begin{itemize}
\item $V = \{v_1, v_2, \ldots, v_n\}$ is the set of voxels with positions $p_i = (x_i, y_i, z_i)$
\item $L = \{l_1, l_2, \ldots, l_k\}$ is the ordered set of layers
\item $Meta$ contains project metadata
\end{itemize}
\item \textbf{Export Format:} A structure $F_{export} = (structure)$ where:
\begin{itemize}
\item $structure$ defines the export data organization
\end{itemize}
\item \textbf{Transformation Function:} A mapping $T : M_{internal} \rightarrow F_{export}$
\item \textbf{Export File:} A serialized representation $E = Serialize(F_{export})$
\end{itemize}
\item \textbf{Export Process}
\begin{itemize}
\item \textbf{Data Transformation:}
$$F_{export} = T(M_{internal})$$
where $T$ performs:
\begin{itemize}
\item Voxel aggregation: $V \rightarrow V_{export}$ (grouping by layer/region)
\item Metadata mapping: $Meta \rightarrow Meta_{export}$
\item Property serialization: $(m_i, mag_i) \rightarrow ExportFormat(m_i, mag_i)$
\end{itemize}
\item \textbf{Serialization:}
$$E = CSVEncode(F_{export})$$
\item \textbf{Integrity Preservation:}
$$\forall v_i \in V, \exists v'_i \in V_{export} : Properties(v_i) = Properties(v'_i)$$
where $Properties(v) = (position, material, magnetization)$
\end{itemize}
\item \textbf{Export Constraints}
\begin{itemize}
\item All voxels from $M_{internal}$ must be represented in $F_{export}$
\item Layer ordering must be preserved: $Order(L) = Order(L_{export})$
\item Material and magnetization properties must be preserved: $\forall v_i, Properties(v_i) = Properties(T(v_i))$
\item Export file must conform to defined export structure specification
\end{itemize}
\end{enumerate}

\subsubsection{Error Diagnostic Handler Module (\mref{m16})}

\begin{description}
\item[Secrets:] How to detect and communicate errors related to unresponsiveness of the model, graphics update failures, and model file issues. How to classify error types and determine error sources. How to handle error recovery and logging mechanisms.
\item[Services:] Detects and diagnoses errors occurring during model updates, graphics rendering, or file operations. Provides error classification and logging.
\item[Implemented By:] \projectname{}
\item[Type of Module:] Library: A reusable component providing error detection, diagnosis, and handling functionality.
\end{description}



\subsection{Software Decision Module}

\subsubsection{Model Structure Module (\mref{m17})}

\begin{description}
  \item[Secrets:]
  \item[Services:]
  \item[Implemented By:] \projectname{}
  \item[Type of Module:] [Record, Library, Abstract Object, or Abstract Data Type]
  [Information to include for leaf modules in the decomposition by secrets tree.]
\end{description}

\subsubsection{Graphics Adapter Module (\mref{m18})}

\begin{description}
  \item[Secrets:] How the graphics API is initialized and configured for rendering. How input data and formats are configured to remain compliant with API requirements. How rendering parameters are derived from visualization state. How resources are allocated and managed during rendering. 
  \item[Services:] Facilitates the use of the graphics API to render the  3D model and general UI in accordance with the state provided by the Visualization State Manager.  
  \item[Implemented By:] \projectname{}
  \item[Type of Module:] Library
\end{description}

\subsubsection{Database Handler Module (\mref{m19})}

\begin{description}
  \item[Secrets:]
  \item[Services:]
  \item[Implemented By:] \projectname{}
  \item[Type of Module:] [Record, Library, Abstract Object, or Abstract Data Type]
  [Information to include for leaf modules in the decomposition by secrets tree.]
\end{description}

\subsubsection{Export Structure Module (\mref{m20})}

\begin{description}
  \item[Secrets:]
  \item[Services:]
  \item[Implemented By:] \projectname{}
  \item[Type of Module:] [Record, Library, Abstract Object, or Abstract Data Type]
  [Information to include for leaf modules in the decomposition by secrets tree.]
\end{description}

\section{Traceability Matrix} \label{SecTM}

This section shows two traceability matrices: between the modules and the
requirements and between the modules and the anticipated changes.

% the table should use mref, the requirements should be named, use something
% like fref

\newcommand{\obox}{\fbox{\rule{0pt}{1.5ex}\hspace{1.5ex}}}
\newcommand{\xbox}{\colorbox{black}{\rule{0pt}{1.5ex}\hspace{1.5ex}}}


\begin{table}[H]
\centering
\small
\setlength{\tabcolsep}{3pt}
\resizebox{\textwidth}{!}{
\begin{tabular}{l|*{20}{c}}
\toprule
\textbf{Req.} 
& \mref{m1} & \mref{m2} & \mref{m3} & \mref{m4} & \mref{m5}
& \mref{m6} & \mref{m7} & \mref{m8} & \mref{m9} & \mref{m10}
& \mref{m11} & \mref{m12} & \mref{m13} & \mref{m14} & \mref{m15}
& \mref{m16} & \mref{m17} & \mref{m18} & \mref{m19} & \mref{m20} \\
\midrule

F211 &
\xbox & \obox & \obox & \xbox & \obox & \obox & \obox & \obox & \obox & \obox &
\obox & \obox & \obox & \obox & \obox & \obox & \obox & \obox & \obox & \obox \\

F212 &
\xbox & \obox & \obox & \xbox & \xbox & \obox & \obox & \obox & \obox & \obox &
\obox & \obox & \obox & \obox & \obox & \obox & \obox & \obox & \obox & \obox \\

F213 &
\xbox & \xbox & \obox & \obox & \obox & \xbox & \xbox & \obox & \obox & \obox &
\obox & \obox & \obox & \obox & \obox & \obox & \obox & \obox & \obox & \obox \\

F214 &
\xbox & \obox & \obox & \obox & \obox & \xbox & \xbox & \obox & \obox & \obox &
\obox & \obox & \obox & \obox & \obox & \obox & \obox & \obox & \obox & \obox \\

F215 &
\xbox & \xbox & \obox & \obox & \obox & \obox & \obox & \obox & \obox & \obox &
\obox & \obox & \obox & \obox & \obox & \obox & \obox & \obox & \obox & \obox \\

F216 &
\obox & \obox & \xbox & \obox & \obox & \obox & \obox & \obox & \obox & \obox &
\obox & \obox & \obox & \obox & \obox & \obox & \obox & \obox & \obox & \obox \\

% --------------------------------------------------------------------------------

F221 &
\obox & \obox & \obox & \obox & \obox & \obox & \obox & \xbox & \obox & \obox &
\obox & \obox & \obox & \obox & \obox & \obox & \xbox & \xbox & \obox & \obox \\

F222 &
\obox & \obox & \xbox & \obox & \obox & \obox & \xbox & \xbox & \obox & \obox &
\obox & \obox & \obox & \obox & \obox & \obox & \xbox & \obox & \obox & \obox \\

F223 &
\obox & \obox & \obox & \obox & \obox & \obox & \xbox & \xbox & \obox & \obox &
\obox & \obox & \obox & \obox & \obox & \obox & \xbox & \xbox & \obox & \obox \\

F224 &
\obox & \obox & \obox & \obox & \obox & \obox & \xbox & \xbox & \obox & \obox &
\obox & \obox & \obox & \obox & \obox & \obox & \xbox & \xbox & \obox & \obox \\

F225 &
\obox & \obox & \obox & \obox & \obox & \obox & \xbox & \xbox & \obox & \obox &
\obox & \xbox & \xbox & \obox & \obox & \obox & \xbox & \xbox & \obox & \obox \\

F226 &
\obox & \obox & \obox & \obox & \obox & \obox & \obox & \xbox & \obox & \obox &
\obox & \xbox & \xbox & \obox & \obox & \obox & \xbox & \xbox & \obox & \obox \\

F227 &
\obox & \obox & \obox & \obox & \obox & \obox & \obox & \xbox & \obox & \obox &
\obox & \xbox & \xbox & \obox & \obox & \obox & \xbox & \xbox & \obox & \obox \\

% --------------------------------------------------------------------------------

F231 &
\obox & \obox & \obox & \obox & \obox & \xbox & \xbox & \obox & \xbox & \obox &
\obox & \obox & \obox & \obox & \obox & \obox & \xbox & \obox & \xbox & \obox \\

F232 &
\obox & \obox & \obox & \obox & \obox & \xbox & \xbox & \obox & \xbox & \obox &
\obox & \obox & \xbox & \obox & \obox & \obox & \obox & \obox & \obox & \obox \\

F233 &
\obox & \obox & \obox & \obox & \obox & \xbox & \xbox & \obox & \xbox & \obox &
\obox & \obox & \obox & \obox & \obox & \obox & \xbox & \obox & \xbox & \obox \\

F234 &
\obox & \obox & \obox & \obox & \obox & \xbox & \xbox & \obox & \xbox & \obox &
\obox & \obox & \xbox & \obox & \obox & \obox & \obox & \obox & \obox & \obox \\

F235 &
\obox & \obox & \obox & \obox & \obox & \xbox & \xbox & \obox & \xbox & \obox &
\obox & \xbox & \obox & \obox & \obox & \obox & \xbox & \obox & \xbox & \obox \\

F236 &
\obox & \obox & \obox & \obox & \xbox & \obox & \obox & \obox & \obox & \xbox &
\xbox & \obox & \obox & \obox & \obox & \xbox & \obox & \obox & \obox & \obox \\

F237 &
\obox & \obox & \obox & \obox & \obox & \xbox & \xbox & \obox & \xbox & \xbox &
\obox & \obox & \obox & \obox & \obox & \obox & \obox & \obox & \xbox & \obox \\

F238 &
\obox & \obox & \obox & \obox & \obox & \xbox & \xbox & \obox & \xbox & \obox &
\obox & \obox & \xbox & \obox & \obox & \obox & \xbox & \obox & \xbox & \obox \\

F239 &
\obox & \obox & \obox & \obox & \obox & \xbox & \xbox & \obox & \xbox & \obox &
\obox & \obox & \obox & \obox & \obox & \obox & \xbox & \obox & \xbox & \obox \\

F2310 &
\obox & \obox & \obox & \obox & \obox & \xbox & \xbox & \obox & \xbox & \obox &
\obox & \obox & \obox & \obox & \obox & \obox & \xbox & \obox & \xbox & \obox \\

% --------------------------------------------------------------------------------

F241 &
\obox & \obox & \obox & \obox & \obox & \obox & \obox & \obox & \obox & \obox &
\obox & \xbox & \obox & \xbox & \obox & \obox & \xbox & \obox & \xbox & \obox \\

F242 &
\obox & \obox & \obox & \obox & \obox & \obox & \obox & \obox & \obox & \obox &
\obox & \obox & \obox & \obox & \xbox & \obox & \xbox & \obox & \xbox & \xbox \\

F243 &
\obox & \obox & \obox & \obox & \obox & \obox & \obox & \obox & \obox & \obox &
\obox & \obox & \obox & \obox & \xbox & \obox & \xbox & \obox & \xbox & \obox \\

F244 &
\obox & \obox & \obox & \obox & \obox & \obox & \obox & \obox & \obox & \obox &
\obox & \obox & \obox & \obox & \xbox & \xbox & \obox & \obox & \obox & \xbox \\

F245 &
\obox & \obox & \obox & \obox & \obox & \obox & \obox & \obox & \obox & \obox &
\obox & \obox & \obox & \obox & \xbox & \obox & \xbox & \obox & \xbox & \xbox \\

\bottomrule
\end{tabular}
}
\caption{Trace Between Functional Requirements and Modules}
\label{TblRT}
\end{table}

\begin{table}[H]
\centering
\small
\setlength{\tabcolsep}{3pt}
\resizebox{\textwidth}{!}{
\begin{tabular}{l|*{20}{c}}
\toprule
\textbf{Req.} 
& \mref{m1} & \mref{m2} & \mref{m3} & \mref{m4} & \mref{m5}
& \mref{m6} & \mref{m7} & \mref{m8} & \mref{m9} & \mref{m10}
& \mref{m11} & \mref{m12} & \mref{m13} & \mref{m14} & \mref{m15}
& \mref{m16} & \mref{m17} & \mref{m18} & \mref{m19} & \mref{m20} \\
\midrule

NF211 &
\xbox & \xbox & \obox & \xbox & \obox & \obox & \obox & \obox & \obox & \obox &
\obox & \obox & \obox & \obox & \obox & \xbox & \obox & \obox & \obox & \obox \\

NF212 &
\xbox & \xbox & \obox & \obox & \xbox & \obox & \obox & \obox & \obox & \obox &
\obox & \obox & \obox & \obox & \obox & \xbox & \xbox & \obox & \xbox & \obox \\

% --------------------------------------------------------------------------------

NF221 &
\obox & \obox & \obox & \obox & \obox & \obox & \obox & \xbox & \xbox & \obox &
\obox & \obox & \obox & \obox & \obox & \xbox & \obox & \xbox & \obox & \obox \\

NF222 &
\obox & \obox & \xbox & \obox & \obox & \obox & \obox & \obox & \obox & \obox &
\obox & \obox & \obox & \obox & \obox & \xbox & \obox & \xbox & \obox & \obox \\

NF223 &
\obox & \obox & \obox & \obox & \obox & \obox & \obox & \obox & \obox & \obox &
\obox & \obox & \xbox & \obox & \obox & \obox & \obox & \obox & \obox & \obox \\

% --------------------------------------------------------------------------------

NF231 &

\obox & \obox & \obox & \obox & \obox & \obox & \obox & \xbox & \obox & \obox &
\obox & \obox & \obox & \obox & \obox & \obox & \obox & \obox & \xbox & \obox \\

NF232 &
\obox & \obox & \obox & \obox & \obox & \xbox & \obox & \obox & \xbox & \obox &
\obox & \obox & \obox & \obox & \obox & \xbox & \xbox & \obox & \xbox & \obox \\

% --------------------------------------------------------------------------------

NF241 &
\obox & \obox & \obox & \obox & \obox & \obox & \obox & \obox & \obox & \obox &
\obox & \obox & \obox & \obox & \xbox & \xbox & \obox & \obox & \obox & \obox \\

NF242 &
\obox & \obox & \obox & \obox & \obox & \obox & \obox & \obox & \obox & \obox &
\obox & \obox & \obox & \obox & \xbox & \xbox & \xbox & \obox & \xbox & \xbox \\

\bottomrule
\end{tabular}
}
\caption{Trace Between Non-Functional Requirements and Modules}
\label{TblRT}
\end{table} 




\begin{table}[H]
\centering
\small
\setlength{\tabcolsep}{3pt}
\resizebox{\textwidth}{!}{
\begin{tabular}{l|*{20}{c}}
\toprule
\textbf{Req.} 
& \mref{m1} & \mref{m2} & \mref{m3} & \mref{m4} & \mref{m5}
& \mref{m6} & \mref{m7} & \mref{m8} & \mref{m9} & \mref{m10}
& \mref{m11} & \mref{m12} & \mref{m13} & \mref{m14} & \mref{m15}
& \mref{m16} & \mref{m17} & \mref{m18} & \mref{m19} & \mref{m20} \\
\midrule

\acref{acInput} &
\xbox & \obox & \obox & \obox & \obox & \obox & \obox & \obox & \obox & \obox &
\obox & \obox & \obox & \obox & \obox & \obox & \xbox & \obox & \obox & \obox \\

\acref{acInputConstraints} &
\xbox & \obox & \xbox & \obox & \obox & \obox & \obox & \obox & \obox & \obox &
\obox & \obox & \obox & \obox & \obox & \obox & \xbox & \obox & \obox & \obox \\

\acref{acVoxelSize} &
\obox & \xbox & \xbox & \obox & \obox & \obox & \obox & \obox & \obox & \obox &
\obox & \obox & \obox & \obox & \obox & \obox & \xbox & \obox & \obox & \obox \\

\acref{acVoxelization} &
\xbox & \xbox & \obox & \obox & \obox & \obox & \obox & \obox & \obox & \obox &
\obox & \obox & \obox & \obox & \obox & \obox & \obox & \obox & \obox & \obox \\
  
\mref{acMagnetization} &
\obox & \obox & \obox & \obox & \obox & \obox & \xbox & \obox & \xbox & \obox &
\obox & \obox & \obox & \obox & \obox & \obox & \xbox & \obox & \xbox & \obox \\
  
\acref{acSelection} &
\obox & \obox & \obox & \obox & \obox & \obox & \xbox & \obox & \obox & \obox &
\obox & \xbox & \obox & \obox & \obox & \obox & \obox & \xbox & \obox & \obox \\

\acref{acExport} &
\obox & \obox & \obox & \obox & \obox & \obox & \obox & \obox & \obox & \obox &
\obox & \obox & \obox & \obox & \xbox & \obox & \obox & \obox & \obox & \xbox \\

\acref{acDataStructure} &
\obox & \obox & \obox & \obox & \xbox & \obox & \obox & \obox & \obox & \obox &
\obox & \obox & \obox & \obox & \obox & \obox & \obox & \obox & \xbox & \obox \\

\acref{acUI} &
\obox & \obox & \obox & \obox & \obox & \obox & \xbox & \xbox & \obox & \obox &
\obox & \obox & \obox & \obox & \obox & \obox & \obox & \xbox & \obox & \obox \\

\bottomrule
\end{tabular}
}
\caption{Trace Between Anticipated Changes and Modules}
\label{TblACT}
\end{table}

\section{Use Hierarchy Between Modules} \label{SecUse}

In this section, the uses hierarchy between modules is
provided. \citet{Parnas1978} said of two programs A and B that A {\em uses} B if
correct execution of B may be necessary for A to complete the task described in
its specification. That is, A {\em uses} B if there exist situations in which
the correct functioning of A depends upon the availability of a correct
implementation of B.  Figure \ref{FigUH} illustrates the use relation between
the modules. It can be seen that the graph is a directed acyclic graph
(DAG). Each level of the hierarchy offers a testable and usable subset of the
system, and modules in the higher level of the hierarchy are essentially simpler
because they use modules from the lower levels.

\wss{The uses relation is not a data flow diagram.  In the code there will often
be an import statement in module A when it directly uses module B.  Module B
provides the services that module A needs.  The code for module A needs to be
able to see these services (hence the import statement).  Since the uses
relation is transitive, there is a use relation without an import, but the
arrows in the diagram typically correspond to the presence of import statement.}

\wss{If module A uses module B, the arrow is directed from A to B.}

\begin{figure}[H]
\centering
%\includegraphics[width=0.7\textwidth]{UsesHierarchy.png}
\caption{Use hierarchy among modules}
\label{FigUH}
\end{figure}

%\section*{References}

\bibliographystyle {plainnat}
\bibliography{../../../refs/References}

\newpage{}

\end{document}