\documentclass[12pt, titlepage]{article}

\usepackage{fullpage}
\usepackage[round]{natbib}
\usepackage{multirow}
\usepackage{booktabs}
\usepackage{tabularx}
\usepackage{graphicx}
\usepackage{float}
\usepackage{hyperref}
\hypersetup{
    colorlinks,
    citecolor=blue,
    filecolor=black,
    linkcolor=red,
    urlcolor=blue
}

\input{../../Comments}
%% Common Parts

\newcommand{\progname}{AutoVox} % PUT YOUR PROGRAM NAME HERE
\newcommand{\authname}{Team \#10, Five of a Kind
\\ Omar Abdelhamid
\\ Daniel Maurer
\\ Andrew Bovbel
\\ Olivia Reich
\\ Khalid Farag
} % AUTHOR NAMES                  

\usepackage{hyperref}
    \hypersetup{colorlinks=true, linkcolor=blue, citecolor=blue, filecolor=blue,
                urlcolor=blue, unicode=false}
    \urlstyle{same}
                                


\newcounter{acnum}
\newcommand{\actheacnum}{AC\theacnum}
\newcommand{\acref}[1]{AC\ref{#1}}

\newcounter{ucnum}
\newcommand{\uctheucnum}{UC\theucnum}
\newcommand{\uref}[1]{UC\ref{#1}}

\newcounter{mnum}
\newcommand{\mthemnum}{M\themnum}
\newcommand{\mref}[1]{M\ref{#1}}

\newcommand{\projectname}{AutoVox}

\begin{document}

\title{Module Guide for \progname{}} 
\author{\authname}
\date{\today}

\maketitle

\pagenumbering{roman}

\section{Revision History}

\begin{tabularx}{\textwidth}{p{3cm}p{2cm}X}
\toprule {\bf Date} & {\bf Version} & {\bf Notes}\\
\midrule
Date 1 & 1.0 & Notes\\
Date 2 & 1.1 & Notes\\
\bottomrule
\end{tabularx}

\newpage

\section{Reference Material}

This section records information for easy reference.

\subsection{Abbreviations and Acronyms}

\renewcommand{\arraystretch}{1.2}
\begin{tabular}{l l} 
  \toprule		
  \textbf{symbol} & \textbf{description}\\
  \midrule 
  AC & Anticipated Change\\
  DAG & Directed Acyclic Graph \\
  M & Module \\
  MG & Module Guide \\
  OS & Operating System \\
  R & Requirement\\
  SC & Scientific Computing \\
  SRS & Software Requirements Specification\\
  \progname & Explanation of program name\\
  UC & Unlikely Change \\
  UI & User Interface \\
  3D & Three Dimensional \\
  API & Application Programming Interface \\
  \bottomrule
\end{tabular}\\

\newpage

\tableofcontents

\listoftables

\listoffigures

\newpage

\pagenumbering{arabic}

\section{Introduction}

Decomposing a system into modules is a commonly accepted approach to developing
software.  A module is a work assignment for a programmer or programming
team~\citep{ParnasEtAl1984}.  We advocate a decomposition
based on the principle of information hiding~\citep{Parnas1972a}.  This
principle supports design for change, because the ``secrets'' that each module
hides represent likely future changes.  Design for change is valuable in SC,
where modifications are frequent, especially during initial development as the
solution space is explored.  

Our design follows the rules layed out by \citet{ParnasEtAl1984}, as follows:
\begin{itemize}
\item System details that are likely to change independently should be the
  secrets of separate modules.
\item Each data structure is implemented in only one module.
\item Any other program that requires information stored in a module's data
  structures must obtain it by calling access programs belonging to that module.
\end{itemize}

After completing the first stage of the design, the Software Requirements
Specification (SRS), the Module Guide (MG) is developed~\citep{ParnasEtAl1984}. The MG
specifies the modular structure of the system and is intended to allow both
designers and maintainers to easily identify the parts of the software.  The
potential readers of this document are as follows:

\begin{itemize}
\item New project members: This document can be a guide for a new project member
  to easily understand the overall structure and quickly find the
  relevant modules they are searching for.
\item Maintainers: The hierarchical structure of the module guide improves the
  maintainers' understanding when they need to make changes to the system. It is
  important for a maintainer to update the relevant sections of the document
  after changes have been made.
\item Designers: Once the module guide has been written, it can be used to
  check for consistency, feasibility, and flexibility. Designers can verify the
  system in various ways, such as consistency among modules, feasibility of the
  decomposition, and flexibility of the design.
\end{itemize}

The rest of the document is organized as follows. Section
\ref{SecChange} lists the anticipated and unlikely changes of the software
requirements. Section \ref{SecMH} summarizes the module decomposition that
was constructed according to the likely changes. Section \ref{SecConnection}
specifies the connections between the software requirements and the
modules. Section \ref{SecMD} gives a detailed description of the
modules. Section \ref{SecTM} includes two traceability matrices. One checks
the completeness of the design against the requirements provided in the SRS. The
other shows the relation between anticipated changes and the modules. Section
\ref{SecUse} describes the use relation between modules.

\section{Anticipated and Unlikely Changes} \label{SecChange}

This section lists possible changes to the system. According to the likeliness
of the change, the possible changes are classified into two
categories. Anticipated changes are listed in Section \ref{SecAchange}, and
unlikely changes are listed in Section \ref{SecUchange}.

\subsection{Anticipated Changes} \label{SecAchange}

Anticipated changes are the source of the information that is to be hidden
inside the modules. Ideally, changing one of the anticipated changes will only
require changing the one module that hides the associated decision. The approach
adapted here is called design for
change.

\begin{description}
\item[\refstepcounter{acnum} \actheacnum \label{acHardware}:] The specific
  hardware on which the software is running.
\item[\refstepcounter{acnum} \actheacnum \label{acInput}:] The format of the
  initial input data.
\item ...
\end{description}

\wss{Anticipated changes relate to changes that would be made in requirements,
design or implementation choices.  They are not related to changes that are made
at run-time, like the values of parameters.}

\subsection{Unlikely Changes} \label{SecUchange}

The module design should be as general as possible. However, a general system is
more complex. Sometimes this complexity is not necessary. Fixing some design
decisions at the system architecture stage can simplify the software design. If
these decision should later need to be changed, then many parts of the design
will potentially need to be modified. Hence, it is not intended that these
decisions will be changed.

\begin{description}
\item[\refstepcounter{ucnum} \uctheucnum \label{ucIO}:] Input/Output devices
  (Input: File and/or Keyboard, Output: File, Memory, and/or Screen).
\item ...
\end{description}

\section{Module Hierarchy} \label{SecMH}

This section provides an overview of the module design. Modules are summarized
in a hierarchy decomposed by secrets in Table \ref{TblMH}. The modules listed
below, which are leaves in the hierarchy tree, are the modules that will
actually be implemented.

\begin{description}
% Behaviour Hiding Modules
\item [\refstepcounter{mnum} \mthemnum \label{m1}:] Input Interpreter Module
\item [\refstepcounter{mnum} \mthemnum \label{m2}:] Voxel Slicing Module
\item [\refstepcounter{mnum} \mthemnum \label{m3}:] Display Partitioning Module
\item [\refstepcounter{mnum} \mthemnum \label{m4}:] Project Manager Module
\item [\refstepcounter{mnum} \mthemnum \label{m5}:] Serialization Manager Module
\item [\refstepcounter{mnum} \mthemnum \label{m6}:] Backend Communication Manager Module
\item [\refstepcounter{mnum} \mthemnum \label{m7}:] Interaction Controller Module
\item [\refstepcounter{mnum} \mthemnum \label{m8}:] Visualization State Manager Module
\item [\refstepcounter{mnum} \mthemnum \label{m9}:] Model Manager Module
\item [\refstepcounter{mnum} \mthemnum \label{m10}:] History Manager Module
\item [\refstepcounter{mnum} \mthemnum \label{m11}:] Autosave Manager Module
\item [\refstepcounter{mnum} \mthemnum \label{m12}:] Voxel Tracking Module
\item [\refstepcounter{mnum} \mthemnum \label{m13}:] Highlight Manager Module
\item [\refstepcounter{mnum} \mthemnum \label{m14}:] Export Validation Module
\item [\refstepcounter{mnum} \mthemnum \label{m15}:] Export Manager Module
\item [\refstepcounter{mnum} \mthemnum \label{m16}:] Error Diagnostic Handler Module

% Software Decision Modules
\item [\refstepcounter{mnum} \mthemnum \label{m17}:] Model Structure Module
\item [\refstepcounter{mnum} \mthemnum \label{m18}:] Graphics Adapter Module
\item [\refstepcounter{mnum} \mthemnum \label{m19}:] Database Handler Module
\item [\refstepcounter{mnum} \mthemnum \label{m20}:] Export Structure Module
\end{description}


\begin{table}[h!]
\centering
\begin{tabular}{p{0.3\textwidth} p{0.6\textwidth}}
\toprule
\textbf{Level 1} & \textbf{Level 2}\\
\midrule

{Hardware-Hiding Module} & None \\
\midrule

\multirow{7}{0.3\textwidth}{Behaviour-Hiding Module} & Input Interpreter Module\\
& Voxel Slicing Module\\
& Display Partitioning Module\\
& Project Manager Module\\
& Serialization Manager Module\\
& Backend Communication Manager Module\\
& Interaction Controller Module\\
& Visualization State Manager Module\\
& Model Manager Module\\
& History Manager Module\\
& Autosave Manager Module\\
& Voxel Tracking Module\\
& Highlight Manager Module\\
& Export Validation Module\\
& Export Manager Module\\
& Error Diagnostic Handler Module\\
\midrule

\multirow{3}{0.3\textwidth}{Software Decision Module} & Model Structure Module\\
& Graphics Adapter Module\\
& Database Handler Module\\
& Export Structure Module\\
\bottomrule

\end{tabular}
\caption{Module Hierarchy}
\label{TblMH}
\end{table}

\section{Connection Between Requirements and Design} \label{SecConnection}

The design of the system is intended to satisfy the requirements developed in
the SRS. In this stage, the system is decomposed into modules. The connection
between requirements and modules is listed in Table~\ref{TblRT}.

\wss{The intention of this section is to document decisions that are made
  ``between'' the requirements and the design.  To satisfy some requirements,
  design decisions need to be made.  Rather than make these decisions implicit,
  they are explicitly recorded here.  For instance, if a program has security
  requirements, a specific design decision may be made to satisfy those
  requirements with a password.}

\section{Module Decomposition} \label{SecMD}

Modules are decomposed according to the principle of ``information hiding''
proposed by \citet{ParnasEtAl1984}. The \emph{Secrets} field in a module
decomposition is a brief statement of the design decision hidden by the
module. The \emph{Services} field specifies \emph{what} the module will do
without documenting \emph{how} to do it. For each module, a suggestion for the
implementing software is given under the \emph{Implemented By} title. If the
entry is \emph{OS}, this means that the module is provided by the operating
system or by standard programming language libraries.  \emph{\progname{}} means the
module will be implemented by the \progname{} software.

Only the leaf modules in the hierarchy have to be implemented. If a dash
(\emph{--}) is shown, this means that the module is not a leaf and will not have
to be implemented.

\subsection{Hardware Hiding Modules (\mref{mHH})}

This system has no hardware components.

\subsection{Behaviour-Hiding Module}

\subsubsection{Input Interpreter Module (\mref{m1})}

\begin{description}
  \item[Secrets:]
  \item[Services:]
  \item[Implemented By:] \projectname{}
  \item[Type of Module:] [Record, Library, Abstract Object, or Abstract Data Type]
  [Information to include for leaf modules in the decomposition by secrets tree.]
\end{description}

\subsubsection{Voxel Slicing Module (\mref{m2})}

\begin{description}
  \item[Secrets:]
  \item[Services:]
  \item[Implemented By:] \projectname{}
  \item[Type of Module:] [Record, Library, Abstract Object, or Abstract Data Type]
  [Information to include for leaf modules in the decomposition by secrets tree.]
\end{description}

\subsubsection{Display Partitioning Module (\mref{m3})}

\begin{description}
  \item[Secrets:] How to determine the boundaries that define model partitions. How to define the relationship
  between display segments. How to represent partition boundaries. How to minimize the number of partitions created. How to ensure that display partitions maintain balanced spatial proportions. How to maintain partition stability during model modification. How to validate that a partition of voxels is viable to display within the UI.
  \item[Services:] Provide functionality for partitioning the model into distinct display segments defined by specific voxel groupings. Manages the relationships and alignment between display segments.
  \item[Implemented By:] \projectname{}
  \item[Type of Module:] Abstract Object
\end{description}

\subsubsection{Project Manager Module (\mref{m4})}

\begin{description}
  \item[Secrets:]
  \item[Services:]
  \item[Implemented By:] \projectname{}
  \item[Type of Module:] [Record, Library, Abstract Object, or Abstract Data Type]
  [Information to include for leaf modules in the decomposition by secrets tree.]
\end{description}

\subsubsection{Serialization Manager Module (\mref{m5})}

\begin{description}
  \item[Secrets:]
  \item[Services:]
  \item[Implemented By:] \projectname{}
  \item[Type of Module:] [Record, Library, Abstract Object, or Abstract Data Type]
  [Information to include for leaf modules in the decomposition by secrets tree.]
\end{description}

\subsubsection{Backend Communication Manager Module (\mref{m6})}

\begin{description}
  \item[Secrets:] How the connection between the backend and frontend is established. How data is transformed, validated and transferred between the backend and frontend. How synchronization errors are detected during communication. How failed communication attempts are detected and handled.
  \item[Services:] Handles communication between the backend server and the frontend UI. Ensures synchronization of model data throughout the system.
  \item[Implemented By:] \projectname{}
  \item[Type of Module:] Abstract Object
\end{description}

\subsubsection{Interaction Controller Module (\mref{m7})}

\begin{description}
  \item[Secrets:] How events are internally represented. How to differentiate between similar events. How to define the rules that associate high-level action with a raw event. How to handle raw events where user intent is inconclusive. How to determine when multiple events should be interpreted as a single user action. How to apply context from UI when interpreting events.
  \item[Services:] Maps raw events from interaction with UI to high-level system actions that reflect user intent.
  \item[Implemented By:] \projectname{}
  \item[Type of Module:] Abstract Object
\end{description}

\subsubsection{Visualization State Manager Module (\mref{m8})}

\begin{description}
  \item[Secrets:] How changes are grouped and aggregated. How the unified state-change representation is structured. How a state change is validated as complete. How event propagation from multiple sources is tracked concurrently. How conditions that trigger a UI update are determined. How the order of UI updates is established. How conflicting state changes are resolved. How invalid or irrelevant data-change notifications are handled.
  \item[Services:] Aggregates data-change notifications from event propagation into a unified state-change representation. Triggers UI update used the derived state differences.
  \item[Implemented By:] \projectname{}
  \item[Type of Module:] Abstract Object
\end{description}

\subsubsection{Model Manager Module (\mref{m9})}

\begin{description}
  \item[Secrets:]
  \item[Services:]
  \item[Implemented By:] \projectname{}
  \item[Type of Module:] [Record, Library, Abstract Object, or Abstract Data Type]
  [Information to include for leaf modules in the decomposition by secrets tree.]
\end{description}

\subsubsection{History Manager Module (\mref{m10})}

\begin{description}
  \item[Secrets:]
  \item[Services:]
  \item[Implemented By:] \projectname{}
  \item[Type of Module:] [Record, Library, Abstract Object, or Abstract Data Type]
  [Information to include for leaf modules in the decomposition by secrets tree.]
\end{description}

\subsubsection{Autosave Manager Module (\mref{m11})}

\begin{description}
  \item[Secrets:] How to collect and aggregate metrics to represent the current status of update operations. How to handle errors in metric collection. How multiple metrics are tracked concurrently. How the state of update execution is defined from collected metric data. How autosave execution is triggered. How autosave failures or interruptions are detected and handled.
  \item[Services:] Maintains real-time autosave status. Tracks metrics that are used to determine the current state of update execution.
  \item[Implemented By:] \projectname{}
  \item[Type of Module:] Abstract Object
\end{description}

\subsubsection{Voxel Tracking Module (\mref{m12})}

\begin{description}
  \item[Secrets:]
  \item[Services:]
  \item[Implemented By:] \projectname{}
  \item[Type of Module:] [Record, Library, Abstract Object, or Abstract Data Type]
  [Information to include for leaf modules in the decomposition by secrets tree.]
\end{description}

\subsubsection{Highlight Manager Module (\mref{m13})}

\begin{description}
  \item[Secrets:] How to define the rules that associate voxel semantics with a highlight colour. How highlight colours are internally represented. How conflicts are resolved when a voxel qualifies for multiple highlight colours simultaneously. How colour palette, which encapsulates all highlight colours, is chosen. How accessibility requirements influence the selection of highlight colours.
  \item[Services:] Maps relationship between highlight colour and semantic meaning of highlighted voxels.
  \item[Implemented By:] \projectname{}
  \item[Type of Module:] Abstract Object
\end{description}

\subsubsection{Export Validation Module (\mref{m14})}

\begin{description}
  \item[Secrets:]
  \item[Services:]
  \item[Implemented By:] \projectname{}
  \item[Type of Module:] [Record, Library, Abstract Object, or Abstract Data Type]
  [Information to include for leaf modules in the decomposition by secrets tree.]
\end{description}

\subsubsection{Export Manager Module (\mref{m15})}

\begin{description}
  \item[Secrets:]
  \item[Services:]
  \item[Implemented By:] \projectname{}
  \item[Type of Module:] [Record, Library, Abstract Object, or Abstract Data Type]
  [Information to include for leaf modules in the decomposition by secrets tree.]
\end{description}

\subsubsection{Error Diagnostic Handler Module (\mref{m16})}

\begin{description}
  \item[Secrets:]
  \item[Services:]
  \item[Implemented By:] \projectname{}
  \item[Type of Module:] [Record, Library, Abstract Object, or Abstract Data Type]
  [Information to include for leaf modules in the decomposition by secrets tree.]
\end{description}



\subsection{Software Decision Module}

\subsubsection{Model Structure Module (\mref{m17})}

\begin{description}
  \item[Secrets:]
  \item[Services:]
  \item[Implemented By:] \projectname{}
  \item[Type of Module:] [Record, Library, Abstract Object, or Abstract Data Type]
  [Information to include for leaf modules in the decomposition by secrets tree.]
\end{description}

\subsubsection{Graphics Adapter Module (\mref{m18})}

\begin{description}
  \item[Secrets:] How the graphics API is initialized and configured for rendering. How input data and formats are configured to remain compliant with API requirements. How rendering parameters are derived from visualization state. How resources are allocated and managed during rendering. 
  \item[Services:] Facilitates the use of the graphics API to render the  3D model and general UI in accordance with the state provided by the Visualization State Manager.  
  \item[Implemented By:] \projectname{}
  \item[Type of Module:] Library
\end{description}

\subsubsection{Database Handler Module (\mref{m19})}

\begin{description}
  \item[Secrets:]
  \item[Services:]
  \item[Implemented By:] \projectname{}
  \item[Type of Module:] [Record, Library, Abstract Object, or Abstract Data Type]
  [Information to include for leaf modules in the decomposition by secrets tree.]
\end{description}

\subsubsection{Export Structure Module (\mref{m20})}

\begin{description}
  \item[Secrets:]
  \item[Services:]
  \item[Implemented By:] \projectname{}
  \item[Type of Module:] [Record, Library, Abstract Object, or Abstract Data Type]
  [Information to include for leaf modules in the decomposition by secrets tree.]
\end{description}

\section{Traceability Matrix} \label{SecTM}

This section shows two traceability matrices: between the modules and the
requirements and between the modules and the anticipated changes.

% the table should use mref, the requirements should be named, use something
% like fref

\newcommand{\obox}{\fbox{\rule{0pt}{1.5ex}\hspace{1.5ex}}}
\newcommand{\xbox}{\colorbox{black}{\rule{0pt}{1.5ex}\hspace{1.5ex}}}


\begin{table}[H]
\centering
\small
\setlength{\tabcolsep}{3pt}
\resizebox{\textwidth}{!}{
\begin{tabular}{l|*{20}{c}}
\toprule
\textbf{Req.} 
& \mref{m1} & \mref{m2} & \mref{m3} & \mref{m4} & \mref{m5}
& \mref{m6} & \mref{m7} & \mref{m8} & \mref{m9} & \mref{m10}
& \mref{m11} & \mref{m12} & \mref{m13} & \mref{m14} & \mref{m15}
& \mref{m16} & \mref{m17} & \mref{m18} & \mref{m19} & \mref{m20} \\
\midrule

F211 &
\xbox & \obox & \obox & \xbox & \obox & \obox & \obox & \obox & \obox & \obox &
\obox & \obox & \obox & \obox & \obox & \obox & \obox & \obox & \obox & \obox \\

F212 &
\xbox & \obox & \obox & \xbox & \xbox & \obox & \obox & \obox & \obox & \obox &
\obox & \obox & \obox & \obox & \obox & \obox & \obox & \obox & \obox & \obox \\

F213 &
\xbox & \xbox & \obox & \obox & \obox & \obox & \xbox & \obox & \xbox & \obox &
\obox & \obox & \obox & \obox & \obox & \obox & \obox & \obox & \obox & \obox \\

F214 &
\xbox & \obox & \obox & \obox & \obox & \obox & \xbox & \obox & \xbox & \obox &
\obox & \obox & \obox & \obox & \obox & \obox & \obox & \obox & \obox & \obox \\

F215 &
\xbox & \xbox & \obox & \obox & \obox & \obox & \obox & \obox & \obox & \obox &
\obox & \obox & \obox & \obox & \obox & \obox & \obox & \obox & \obox & \obox \\

F216 &
\obox & \obox & \xbox & \obox & \obox & \obox & \obox & \obox & \obox & \obox &
\obox & \obox & \obox & \obox & \obox & \obox & \obox & \obox & \obox & \obox \\

% --------------------------------------------------------------------------------

F221 &
\obox & \obox & \obox & \obox & \obox & \obox & \obox & \xbox & \obox & \obox &
\obox & \obox & \obox & \obox & \obox & \obox & \xbox & \xbox & \obox & \obox \\

F222 &
\obox & \obox & \xbox & \obox & \obox & \obox & \xbox & \xbox & \obox & \obox &
\obox & \obox & \obox & \obox & \obox & \obox & \xbox & \obox & \obox & \obox \\

F223 &
\obox & \obox & \obox & \obox & \obox & \obox & \xbox & \xbox & \obox & \obox &
\obox & \obox & \obox & \obox & \obox & \obox & \xbox & \xbox & \obox & \obox \\

F224 &
\obox & \obox & \obox & \obox & \obox & \obox & \xbox & \xbox & \obox & \obox &
\obox & \obox & \obox & \obox & \obox & \obox & \xbox & \xbox & \obox & \obox \\

F225 &
\obox & \obox & \obox & \obox & \obox & \obox & \xbox & \xbox & \obox & \obox &
\obox & \xbox & \xbox & \obox & \obox & \obox & \xbox & \xbox & \obox & \obox \\

F226 &
\obox & \obox & \obox & \obox & \obox & \obox & \obox & \xbox & \obox & \obox &
\obox & \xbox & \xbox & \obox & \obox & \obox & \xbox & \xbox & \obox & \obox \\

F227 &
\obox & \obox & \obox & \obox & \obox & \obox & \obox & \xbox & \obox & \obox &
\obox & \xbox & \xbox & \obox & \obox & \obox & \xbox & \xbox & \obox & \obox \\

% --------------------------------------------------------------------------------

F231 &
\obox & \obox & \obox & \obox & \obox & \xbox & \xbox & \obox & \xbox & \obox &
\obox & \obox & \obox & \obox & \obox & \obox & \xbox & \obox & \xbox & \obox \\

F232 &
\obox & \obox & \obox & \obox & \obox & \xbox & \xbox & \obox & \xbox & \obox &
\obox & \obox & \xbox & \obox & \obox & \obox & \obox & \obox & \obox & \obox \\

F233 &
\obox & \obox & \obox & \obox & \obox & \xbox & \xbox & \obox & \xbox & \obox &
\obox & \obox & \obox & \obox & \obox & \obox & \xbox & \obox & \xbox & \obox \\

F234 &
\obox & \obox & \obox & \obox & \obox & \xbox & \xbox & \obox & \xbox & \obox &
\obox & \obox & \xbox & \obox & \obox & \obox & \obox & \obox & \obox & \obox \\

F235 &
\obox & \obox & \obox & \obox & \obox & \xbox & \xbox & \obox & \xbox & \xbox &
\obox & \xbox & \obox & \obox & \obox & \obox & \xbox & \obox & \xbox & \obox \\

F236 &
\obox & \obox & \obox & \obox & \xbox & \obox & \obox & \obox & \obox & \obox &
\xbox & \obox & \obox & \obox & \obox & \xbox & \obox & \obox & \obox & \obox \\

F237 &
\obox & \obox & \obox & \obox & \obox & \xbox & \xbox & \obox & \xbox & \xbox &
\obox & \obox & \obox & \obox & \obox & \obox & \obox & \obox & \xbox & \obox \\

F238 &
\obox & \obox & \obox & \obox & \obox & \xbox & \xbox & \obox & \xbox & \obox &
\obox & \obox & \xbox & \obox & \obox & \obox & \xbox & \obox & \xbox & \obox \\

F239 &
\obox & \obox & \obox & \obox & \obox & \xbox & \xbox & \obox & \xbox & \obox &
\obox & \obox & \obox & \obox & \obox & \obox & \xbox & \obox & \xbox & \obox \\

F2310 &
\obox & \obox & \obox & \obox & \obox & \xbox & \xbox & \obox & \xbox & \obox &
\obox & \obox & \obox & \obox & \obox & \obox & \xbox & \obox & \xbox & \obox \\

% --------------------------------------------------------------------------------

F241 &
\obox & \obox & \obox & \obox & \obox & \obox & \obox & \obox & \obox & \obox &
\obox & \xbox & \obox & \xbox & \obox & \obox & \xbox & \obox & \xbox & \obox \\

F242 &
\obox & \obox & \obox & \obox & \obox & \obox & \obox & \obox & \obox & \obox &
\obox & \obox & \obox & \obox & \xbox & \obox & \xbox & \obox & \xbox & \xbox \\

F243 &
\obox & \obox & \obox & \obox & \obox & \obox & \obox & \obox & \obox & \obox &
\obox & \obox & \obox & \obox & \xbox & \obox & \xbox & \obox & \xbox & \obox \\

F244 &
\obox & \obox & \obox & \obox & \obox & \obox & \obox & \obox & \obox & \obox &
\obox & \obox & \obox & \obox & \xbox & \xbox & \obox & \obox & \obox & \xbox \\

F245 &
\obox & \obox & \obox & \obox & \obox & \obox & \obox & \obox & \obox & \obox &
\obox & \obox & \obox & \obox & \xbox & \obox & \xbox & \obox & \xbox & \xbox \\

\bottomrule
\end{tabular}
}
\caption{Trace Between Functional Requirements and Modules}
\label{TblRT}
\end{table}

\begin{table}[H]
\centering
\small
\setlength{\tabcolsep}{3pt}
\resizebox{\textwidth}{!}{
\begin{tabular}{l|*{20}{c}}
\toprule
\textbf{Req.} 
& \mref{m1} & \mref{m2} & \mref{m3} & \mref{m4} & \mref{m5}
& \mref{m6} & \mref{m7} & \mref{m8} & \mref{m9} & \mref{m10}
& \mref{m11} & \mref{m12} & \mref{m13} & \mref{m14} & \mref{m15}
& \mref{m16} & \mref{m17} & \mref{m18} & \mref{m19} & \mref{m20} \\
\midrule

NF211 &
\xbox & \xbox & \obox & \xbox & \obox & \obox & \obox & \obox & \obox & \obox &
\obox & \obox & \obox & \obox & \obox & \xbox & \obox & \obox & \obox & \obox \\

NF212 &
\xbox & \xbox & \obox & \obox & \xbox & \obox & \obox & \obox & \obox & \obox &
\obox & \obox & \obox & \obox & \obox & \xbox & \xbox & \obox & \xbox & \obox \\

% --------------------------------------------------------------------------------

NF221 &
\obox & \obox & \obox & \obox & \obox & \obox & \obox & \xbox & \xbox & \obox &
\obox & \obox & \obox & \obox & \obox & \xbox & \obox & \xbox & \obox & \obox \\

NF222 &
\obox & \obox & \xbox & \obox & \obox & \obox & \obox & \obox & \obox & \obox &
\obox & \obox & \obox & \obox & \obox & \xbox & \obox & \xbox & \obox & \obox \\

NF223 &
\obox & \obox & \obox & \obox & \obox & \obox & \obox & \obox & \obox & \obox &
\obox & \obox & \xbox & \obox & \obox & \obox & \obox & \obox & \obox & \obox \\

% --------------------------------------------------------------------------------

NF231 &

\obox & \obox & \obox & \obox & \obox & \obox & \obox & \xbox & \obox & \obox &
\obox & \obox & \obox & \obox & \obox & \obox & \obox & \obox & \xbox & \obox \\

NF232 &
\obox & \obox & \obox & \obox & \obox & \xbox & \obox & \obox & \xbox & \obox &
\obox & \obox & \obox & \obox & \obox & \xbox & \xbox & \obox & \xbox & \obox \\

% --------------------------------------------------------------------------------

NF241 &
\obox & \obox & \obox & \obox & \obox & \obox & \obox & \obox & \obox & \obox &
\obox & \obox & \obox & \obox & \xbox & \xbox & \obox & \obox & \obox & \obox \\

NF242 &
\obox & \obox & \obox & \obox & \obox & \obox & \obox & \obox & \obox & \obox &
\obox & \obox & \obox & \obox & \xbox & \xbox & \xbox & \obox & \xbox & \xbox \\

\bottomrule
\end{tabular}
}
\caption{Trace Between Non-Functional Requirements and Modules}
\label{TblRT}
\end{table} 




\begin{table}[H]
\centering
\small
\setlength{\tabcolsep}{3pt}
\resizebox{\textwidth}{!}{
\begin{tabular}{l|*{20}{c}}
\toprule
\textbf{Req.} 
& \mref{m1} & \mref{m2} & \mref{m3} & \mref{m4} & \mref{m5}
& \mref{m6} & \mref{m7} & \mref{m8} & \mref{m9} & \mref{m10}
& \mref{m11} & \mref{m12} & \mref{m13} & \mref{m14} & \mref{m15}
& \mref{m16} & \mref{m17} & \mref{m18} & \mref{m19} & \mref{m20} \\
\midrule

\mref{acInput} &
\obox & \obox & \obox & \obox & \obox & \obox & \obox & \obox & \obox & \obox &
\obox & \obox & \obox & \obox & \obox & \obox & \obox & \obox & \obox & \obox \\

\mref{acInputConstraints} &
\obox & \obox & \obox & \obox & \obox & \obox & \obox & \obox & \obox & \obox &
\obox & \obox & \obox & \obox & \obox & \obox & \obox & \obox & \obox & \obox \\
  
\mref{acVoxelSize} &
\obox & \obox & \obox & \obox & \obox & \obox & \obox & \obox & \obox & \obox &
\obox & \obox & \obox & \obox & \obox & \obox & \obox & \obox & \obox & \obox \\

\mref{acVoxelization} &
\obox & \obox & \obox & \obox & \obox & \obox & \obox & \obox & \obox & \obox &
\obox & \obox & \obox & \obox & \obox & \obox & \obox & \obox & \obox & \obox \\
  
\mref{acMagnetization} &
\obox & \obox & \obox & \obox & \obox & \obox & \obox & \obox & \obox & \obox &
\obox & \obox & \obox & \obox & \obox & \obox & \obox & \obox & \obox & \obox \\

\mref{acSelection} &
\obox & \obox & \obox & \obox & \obox & \obox & \obox & \obox & \obox & \obox &
\obox & \obox & \obox & \obox & \obox & \obox & \obox & \obox & \obox & \obox \\
  
\mref{acExport} &
\obox & \obox & \obox & \obox & \obox & \obox & \obox & \obox & \obox & \obox &
\obox & \obox & \obox & \obox & \obox & \obox & \obox & \obox & \obox & \obox \\

\mref{acDataStructure} &
\obox & \obox & \obox & \obox & \obox & \obox & \obox & \obox & \obox & \obox &
\obox & \obox & \obox & \obox & \obox & \obox & \obox & \obox & \obox & \obox \\

\mref{acUI} &
\obox & \obox & \obox & \obox & \obox & \obox & \obox & \obox & \obox & \obox &
\obox & \obox & \obox & \obox & \obox & \obox & \obox & \obox & \obox & \obox \\

\bottomrule
\end{tabular}
}
\caption{Trace Between Anticipated Changes and Modules}
\label{TblACT}
\end{table}

\section{Use Hierarchy Between Modules} \label{SecUse}

In this section, the uses hierarchy between modules is
provided. \citet{Parnas1978} said of two programs A and B that A {\em uses} B if
correct execution of B may be necessary for A to complete the task described in
its specification. That is, A {\em uses} B if there exist situations in which
the correct functioning of A depends upon the availability of a correct
implementation of B.  Figure \ref{FigUH} illustrates the use relation between
the modules. It can be seen that the graph is a directed acyclic graph
(DAG). Each level of the hierarchy offers a testable and usable subset of the
system, and modules in the higher level of the hierarchy are essentially simpler
because they use modules from the lower levels.

\wss{The uses relation is not a data flow diagram.  In the code there will often
be an import statement in module A when it directly uses module B.  Module B
provides the services that module A needs.  The code for module A needs to be
able to see these services (hence the import statement).  Since the uses
relation is transitive, there is a use relation without an import, but the
arrows in the diagram typically correspond to the presence of import statement.}

\wss{If module A uses module B, the arrow is directed from A to B.}

\begin{figure}[H]
\centering
%\includegraphics[width=0.7\textwidth]{UsesHierarchy.png}
\caption{Use hierarchy among modules}
\label{FigUH}
\end{figure}

%\section*{References}

\bibliographystyle {plainnat}
\bibliography{../../../refs/References}

\newpage{}

\end{document}