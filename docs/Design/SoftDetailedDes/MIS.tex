\documentclass[12pt, titlepage]{article}

\usepackage{amsmath, mathtools}

\usepackage[round]{natbib}
\usepackage{amsfonts}
\usepackage{amssymb}
\usepackage{graphicx}
\usepackage{colortbl}
\usepackage{xr}
\usepackage{hyperref}
\usepackage{longtable}
\usepackage{xfrac}
\usepackage{tabularx}
\usepackage{float}
\usepackage{siunitx}
\usepackage{booktabs}
\usepackage{multirow}
\usepackage[section]{placeins}
\usepackage{caption}
\usepackage{fullpage}

\hypersetup{
bookmarks=true,     % show bookmarks bar?
colorlinks=true,       % false: boxed links; true: colored links
linkcolor=red,          % color of internal links (change box color with linkbordercolor)
citecolor=blue,      % color of links to bibliography
filecolor=magenta,  % color of file links
urlcolor=cyan          % color of external links
}

\usepackage{array}

\externaldocument{../../SRS/SRS}

\input{../../Comments}
%% Common Parts

\newcommand{\progname}{AutoVox} % PUT YOUR PROGRAM NAME HERE
\newcommand{\authname}{Team \#10, Five of a Kind
\\ Omar Abdelhamid
\\ Daniel Maurer
\\ Andrew Bovbel
\\ Olivia Reich
\\ Khalid Farag
} % AUTHOR NAMES                  

\usepackage{hyperref}
    \hypersetup{colorlinks=true, linkcolor=blue, citecolor=blue, filecolor=blue,
                urlcolor=blue, unicode=false}
    \urlstyle{same}
                                


\begin{document}

\title{Module Interface Specification for \progname{}}

\author{\authname}

\date{\today}

\maketitle

\pagenumbering{roman}

\section{Revision History}

\begin{tabularx}{\textwidth}{p{3cm}p{2cm}X}
\toprule {\bf Date} & {\bf Version} & {\bf Notes}\\
\midrule
Date 1 & 1.0 & Notes\\
Date 2 & 1.1 & Notes\\
\bottomrule
\end{tabularx}

~\newpage

\section{Symbols, Abbreviations and Acronyms}

See SRS Documentation at \wss{give url}

\wss{Also add any additional symbols, abbreviations or acronyms}

\newpage

\tableofcontents

\newpage

\pagenumbering{arabic}

\section{Introduction}

The following document details the Module Interface Specifications for
\wss{Fill in your project name and description}

Complementary documents include the System Requirement Specifications
and Module Guide.  The full documentation and implementation can be
found at \url{...}.  \wss{provide the url for your repo}

\section{Notation}

\wss{You should describe your notation.  You can use what is below as
  a starting point.}

The structure of the MIS for modules comes from \citet{HoffmanAndStrooper1995},
with the addition that template modules have been adapted from
\cite{GhezziEtAl2003}.  The mathematical notation comes from Chapter 3 of
\citet{HoffmanAndStrooper1995}.  For instance, the symbol := is used for a
multiple assignment statement and conditional rules follow the form $(c_1
\Rightarrow r_1 | c_2 \Rightarrow r_2 | ... | c_n \Rightarrow r_n )$.

The following table summarizes the primitive data types used by \progname. 

\begin{center}
\renewcommand{\arraystretch}{1.2}
\noindent 
\begin{tabular}{l l p{7.5cm}} 
\toprule 
\textbf{Data Type} & \textbf{Notation} & \textbf{Description}\\ 
\midrule
character & char & a single symbol or digit\\
integer & $\mathbb{Z}$ & a number without a fractional component in (-$\infty$, $\infty$) \\
natural number & $\mathbb{N}$ & a number without a fractional component in [1, $\infty$) \\
real & $\mathbb{R}$ & any number in (-$\infty$, $\infty$)\\
\bottomrule
\end{tabular} 
\end{center}

\noindent
The specification of \progname \ uses some derived data types: sequences, strings, and
tuples. Sequences are lists filled with elements of the same data type. Strings
are sequences of characters. Tuples contain a list of values, potentially of
different types. In addition, \progname \ uses functions, which
are defined by the data types of their inputs and outputs. Local functions are
described by giving their type signature followed by their specification.

\section{Module Decomposition}

The following table is taken directly from the Module Guide document for this project.

\begin{table}[h!]
  \centering
  \begin{tabular}{p{0.3\textwidth} p{0.6\textwidth}}
  \toprule
  \textbf{Level 1} & \textbf{Level 2}\\
  \midrule
  
  {Hardware-Hiding Module} & None \\
  \midrule
  
  \multirow{7}{0.3\textwidth}{Behaviour-Hiding Module} & Input Interpreter Module\\
  & Voxel Slicing Module\\
  & Display Partitioning Module\\
  & Project Manager Module\\
  & Serialization Manager Module\\
  & Backend Communication Manager Module\\
  & Interaction Controller Module\\
  & Visualization State Manager Module\\
  & Model Manager Module\\
  & History Manager Module\\
  & Autosave Manager Module\\
  & Voxel Tracking Module\\
  & Highlight Manager Module\\
  & Export Validation Module\\
  & Export Manager Module\\
  & Error Diagnostic Handler Module\\
  \midrule
  
  \multirow{3}{0.3\textwidth}{Software Decision Module} & Model Structure Module\\
  & Graphics Adapter Module\\
  & Database Handler Module\\
  & Export Structure Module\\
  \bottomrule
  
  \end{tabular}
  \caption{Module Hierarchy}
  \label{TblMH}
  \end{table}
\newpage
~\newpage

\section{MIS of STL Interpreter} \label{Module1} \wss{Use labels for
  cross-referencing}

\wss{You can reference SRS labels, such as R\ref{R_Inputs}.}

\wss{It is also possible to use \LaTeX for hypperlinks to external documents.}

\subsection{Module}

\wss{Short name for the module}

\subsection{Uses}


\subsection{Syntax}

\subsubsection{Exported Constants}

\subsubsection{Exported Access Programs}

\begin{center}
\begin{tabular}{p{2cm} p{4cm} p{4cm} p{2cm}}
\hline
\textbf{Name} & \textbf{In} & \textbf{Out} & \textbf{Exceptions} \\
\hline
\wss{accessProg} & - & - & - \\
\hline
\end{tabular}
\end{center}

\subsection{Semantics}

\subsubsection{State Variables}

\wss{Not all modules will have state variables.  State variables give the module
  a memory.}

\subsubsection{Environment Variables}

\wss{This section is not necessary for all modules.  Its purpose is to capture
  when the module has external interaction with the environment, such as for a
  device driver, screen interface, keyboard, file, etc.}

\subsubsection{Assumptions}

\wss{Try to minimize assumptions and anticipate programmer errors via
  exceptions, but for practical purposes assumptions are sometimes appropriate.}

\newpage

\section{MIS of Voxel Slicing} \label{Module2} \wss{Use labels for
  cross-referencing}

\wss{You can reference SRS labels, such as R\ref{R_Inputs}.}

\wss{It is also possible to use \LaTeX for hypperlinks to external documents.}

\subsection{Module}

\wss{Short name for the module}

\subsection{Uses}


\subsection{Syntax}

\subsubsection{Exported Constants}

\subsubsection{Exported Access Programs}

\begin{center}
\begin{tabular}{p{2cm} p{4cm} p{4cm} p{2cm}}
\hline
\textbf{Name} & \textbf{In} & \textbf{Out} & \textbf{Exceptions} \\
\hline
\wss{accessProg} & - & - & - \\
\hline
\end{tabular}
\end{center}

\subsection{Semantics}

\subsubsection{State Variables}

\wss{Not all modules will have state variables.  State variables give the module
  a memory.}

\subsubsection{Environment Variables}

\wss{This section is not necessary for all modules.  Its purpose is to capture
  when the module has external interaction with the environment, such as for a
  device driver, screen interface, keyboard, file, etc.}

\subsubsection{Assumptions}

\wss{Try to minimize assumptions and anticipate programmer errors via
  exceptions, but for practical purposes assumptions are sometimes appropriate.}

\newpage

\section{MIS of Display Partitioning} \label{Module3} 

\subsection{Module}
Provides core functionality for partitioning the model into distinct display segments.

\subsection{Uses}

\texttt{ModelStructure} to obtain the sliced model’s data representation.

\subsection{Syntax}

\noindent\textbf{Exported Constants}

\begin{itemize}
  \item \texttt{MAX\_PARTITION}: \textit{int} — maximum number of partitions that can be created.
  \item \texttt{MAX\_VOXEL\_PARTITION}: \textit{int} — maximum number of voxels per partition.
  \item \texttt{AVAILABLE\_PARTITIONS}: \textit{List[string]} — configuration of valid partition IDs used by the display layout.
\end{itemize}

\noindent\textbf{Exported Access Programs}:
\begin{center}
  \renewcommand{\arraystretch}{1.55}
  \begin{tabular}{p{4cm} p{5cm} p{3cm} p{3cm}}
  \hline
  \textbf{Name} & \textbf{In} & \textbf{Out} & \textbf{Exceptions} \\
  \hline
  
  \texttt{getPartitions} &
  None &
  \textit{dict[string]} &
  None \\
  
  \texttt{setCurrentPartition} &
  id: \textit{string} \newline
  partition: \textit{PartitionItem} &
  void &
  None \\
  
  \texttt{resizePartitions} &
  id: \textit{string} \newline
  width: \textit{int} \newline
  height: \textit{int} \newline
  depth: \textit{int} &
  void &
  None \\
  
  \end{tabular}
  \end{center}
\subsection{Semantics}

\subsubsection{State Variables}
\begin{itemize}
  \item \texttt{PartitionItem}: \textit{record} — configuration for data that describes a partition.
  \item \texttt{PartitionDict}: \textit{dict[string] = PartitionItem} — tracks all partitions.
\end{itemize}

\subsubsection{Environment Variables}
None

  \subsubsection{Assumptions}
  \begin{itemize}
    \item A valid and complete \texttt{ModelStructure} exists.
    \item Display partitions adhere only to positive dimensions.
    \item \texttt{AVAILABLE\_PARTITIONS} provides the display layout with a preset mapping to UI grid positions.
  \end{itemize}
\newpage

\section{MIS of Serialization Manager} \label{Module5} \wss{Use labels for
  cross-referencing}

\wss{You can reference SRS labels, such as R\ref{R_Inputs}.}

\wss{It is also possible to use \LaTeX for hypperlinks to external documents.}

\subsection{Module}

\wss{Short name for the module}

\subsection{Uses}


\subsection{Syntax}

\subsubsection{Exported Constants}

\subsubsection{Exported Access Programs}

\begin{center}
\begin{tabular}{p{2cm} p{4cm} p{4cm} p{2cm}}
\hline
\textbf{Name} & \textbf{In} & \textbf{Out} & \textbf{Exceptions} \\
\hline
\wss{accessProg} & - & - & - \\
\hline
\end{tabular}
\end{center}

\subsection{Semantics}

\subsubsection{State Variables}

\wss{Not all modules will have state variables.  State variables give the module
  a memory.}

\subsubsection{Environment Variables}

\wss{This section is not necessary for all modules.  Its purpose is to capture
  when the module has external interaction with the environment, such as for a
  device driver, screen interface, keyboard, file, etc.}

\subsubsection{Assumptions}

\wss{Try to minimize assumptions and anticipate programmer errors via
  exceptions, but for practical purposes assumptions are sometimes appropriate.}

\newpage

\section{MIS of Backend Communication Manager} \label{Module6} 
  
\subsection{Module}
BackendCommunicationManager manages the transfer and synchronization of data between the backend server and frontend UI.

\subsection{Uses}
\begin{itemize}
  \item ModelManager for interpretation into backend-compatible commands.
  \item SerializationManager for necessary JSON and binary transforms.
  \item InteractionController for processing raw UI information.
\end{itemize}

\subsection{Syntax}

\noindent\textbf{Exported Constants}

\begin{itemize}
  \item None
\end{itemize}

\noindent\textbf{Exported Access Programs}

\begin{center}
\renewcommand{\arraystretch}{1.55}
\begin{tabular}{p{5cm} p{3cm} p{3cm} p{3cm}}
\hline
\textbf{Name} & \textbf{In} & \textbf{Out} & \textbf{Exceptions} \\
\hline

\texttt{checkServerStatus} &
None &
Promise\textless void\textgreater&
Network Error \\

\texttt{initDataLogs} &
None &
Promise\textless boolean\textgreater &
Network Error \\

\texttt{getDataLogs} &
None &
string &
None \\

\texttt{getInteraction} &
None &
UIdata: string &
None \\

\end{tabular}
\end{center}


\subsection{Semantics}

\subsubsection{State Variables}

\begin{itemize}
  \item \texttt{serverStatus}: \textit{ServerStatusType} — tracks the current connection status of the backend server.
  \item \texttt{data\_log}: \textit{string} — all data gathered from the frontend that requires interpretation on the backend.
\end{itemize}

\subsubsection{Environment Variables}

\begin{itemize}
  \item \texttt{SERVER\_URL}: \textit{string} — backend server URL from environment configuration.
\end{itemize}

\subsubsection{Assumptions}

\begin{itemize}
  \item Backend service is reachable and operational.
  \item Connection is available for data synchronization.
  \item Valid configurations exist for necessary data transfer operations.
\end{itemize}


\newpage

\section{Interaction Controller} \label{Module7} 

\subsection{Module}
InteractionController manages the process of raw events from interaction with UI to associated actions within internal code.

\subsection{Uses}
\begin{itemize}
  \item \texttt{VisualizationManager} to establish current active view.
  \item \texttt{BackendCommunicator} for taking the collected raw UI data from the frontend and sending it for interpretation.
  \item \texttt{ModelManager} for handling user intent related to model modification.
\end{itemize}

\subsection{Syntax}

\noindent\textbf{Exported Constants}

\begin{itemize}
  \item \texttt{UI\_EVENTS}: \textit{list[string]} — identifies all supported UI events that can be detected from user interaction.
  \item \texttt{UI\_ACTIONS}: \textit{list[string]} — identifies all supported interactions that can be derived from UI events.
\end{itemize}

\noindent\textbf{Exported Access Programs}

\begin{center}
\renewcommand{\arraystretch}{1.55}
\begin{tabular}{p{4cm} p{4cm} p{3cm} p{3cm}}
\hline
\textbf{Name} & \textbf{In} & \textbf{Out} & \textbf{Exceptions} \\
\hline

\texttt{interpretEvent} &
UIdata: \textit{string} &
UIEvent &
None \\

\end{tabular}
\end{center}


\subsection{Semantics}

\subsubsection{State Variables}

\begin{itemize}
  \item \texttt{currentEvent}: \textit{UIEvent} — configuration for a UI event.
  \item \texttt{currentView}: \textit{string} — identifies what view is currently displayed during interaction.
  \item \texttt{pointerPosition}: \textit{[int, int]} — pointer coordinates captured from the most recent relevant event.
  \item \texttt{allEvents}: \textit{list[string]} — identifies all supported UI events that can be detected from user interaction.
\end{itemize}

\subsubsection{Environment Variables}
None


\subsubsection{Assumptions}
\begin{itemize}
  \item User intent does not exceed the scope of supported actions that can be derived.
  \item There can only be one current active view at a given moment.
  \item \texttt{pointerPosition} is contained within the screen resolution.
\end{itemize}

\newpage

\section{MIS of Visualization State Manager} \label{Module8} 

\subsection{Module}
VisualizationManager oversees the creation of UI views while managing subsequent updates to the display state of the specified current UI views on a backend level.

\subsection{Uses}
\begin{itemize}
  \item Display Partitioning for tracking different interface states across partitions.
  \item Graphics Adapter for rendering updates visible to user.
  \item Model Manager for managing model and metadata updates that affect state visualization.
  \item Highlight Manager for managing change in voxels being highlighted.
  \item ModelStructure for managing the specification of update bundles within the model.
\end{itemize}

\subsection{Syntax}

\noindent\textbf{Exported Constants}

\begin{itemize}
  \item \texttt{DEFAULT\_VIEWS}: \textit{list[string]} — list of identifiers corresponding to the default views generated upon project initialization.
\end{itemize}

\noindent\textbf{Exported Access Programs}

\begin{center}
\renewcommand{\arraystretch}{1.55}
\begin{tabular}{p{4cm} p{6cm} p{3cm} p{2.5cm}}
\hline
\textbf{Name} & \textbf{In} & \textbf{Out} & \textbf{Exceptions} \\
\hline

\texttt{getCurrentView} &
None &
ViewState &
None \\

\texttt{setView} &
state: \textit{ViewState} &
void &
None \\

\texttt{getHighlightState} &
None &
HighlightState &
None \\

\texttt{updateModel} &
voxelCoords: \textit{List[VoxelCoord]} \newline
updateType: \textit{ModelUpdateType} &
void &
None \\

\texttt{updateHighlight} &
semanticKey: \textit{string} &
void &
None \\

\end{tabular}
\end{center}


\subsection{Semantics}

\subsubsection{State Variables}

\begin{itemize}
  \item \texttt{currentView}: \textit{string} — identifies which UI view is currently shown to the user.
  \item \texttt{viewStatus}: \textit{dict[string] = ViewItem} — stores all data encapsulated within a UI view.
  \item \texttt{updateBundle}: \textit{dict[string] = UpdateItem} — stores all data needed to update a UI view upon a render request.
  \item \texttt{pendingUpdates}: \textit{list[UpdateKey]} — updates waiting to be processed in rendering.
  \item \texttt{renderStatus}: \textit{boolean} — records the success status of the rendering process.
  \item \texttt{supportedViews}: \textit{list[string]} — identifiers corresponding to all available project views.
\end{itemize}

\subsubsection{Environment Variables}
None

\subsubsection{Assumptions}
\begin{itemize}
  \item Partitions are initialized.
  \item ViewItem is properly formatted in accordance with ViewItem specification.
  \item UpdateItem is properly formatted in accordance with UpdateItem specification.
  \item \texttt{currentView}, \texttt{ViewState}, and \texttt{DEFAULT\_VIEWS} are elements of \texttt{supportedViews}.
\end{itemize}

\newpage


\newpage

\section{MIS of Model Manager} \label{Module9} \wss{Use labels for
  cross-referencing}

\wss{You can reference SRS labels, such as R\ref{R_Inputs}.}

\wss{It is also possible to use \LaTeX for hypperlinks to external documents.}

\subsection{Module}

\wss{Short name for the module}

\subsection{Uses}


\subsection{Syntax}

\subsubsection{Exported Constants}

\subsubsection{Exported Access Programs}

\begin{center}
\begin{tabular}{p{2cm} p{4cm} p{4cm} p{2cm}}
\hline
\textbf{Name} & \textbf{In} & \textbf{Out} & \textbf{Exceptions} \\
\hline
\wss{accessProg} & - & - & - \\
\hline
\end{tabular}
\end{center}

\subsection{Semantics}

\subsubsection{State Variables}

\wss{Not all modules will have state variables.  State variables give the module
  a memory.}

\subsubsection{Environment Variables}

\wss{This section is not necessary for all modules.  Its purpose is to capture
  when the module has external interaction with the environment, such as for a
  device driver, screen interface, keyboard, file, etc.}

\subsubsection{Assumptions}

\wss{Try to minimize assumptions and anticipate programmer errors via
  exceptions, but for practical purposes assumptions are sometimes appropriate.}

\newpage

\section{MIS of History Manager} \label{Module10} \wss{Use labels for
  cross-referencing}

\wss{You can reference SRS labels, such as R\ref{R_Inputs}.}

\wss{It is also possible to use \LaTeX for hypperlinks to external documents.}

\subsection{Module}

\wss{Short name for the module}

\subsection{Uses}


\subsection{Syntax}

\subsubsection{Exported Constants}

\subsubsection{Exported Access Programs}

\begin{center}
\begin{tabular}{p{2cm} p{4cm} p{4cm} p{2cm}}
\hline
\textbf{Name} & \textbf{In} & \textbf{Out} & \textbf{Exceptions} \\
\hline
\wss{accessProg} & - & - & - \\
\hline
\end{tabular}
\end{center}

\subsection{Semantics}

\subsubsection{State Variables}

\wss{Not all modules will have state variables.  State variables give the module
  a memory.}

\subsubsection{Environment Variables}

\wss{This section is not necessary for all modules.  Its purpose is to capture
  when the module has external interaction with the environment, such as for a
  device driver, screen interface, keyboard, file, etc.}

\subsubsection{Assumptions}

\wss{Try to minimize assumptions and anticipate programmer errors via
  exceptions, but for practical purposes assumptions are sometimes appropriate.}

\newpage

\section{MIS of Autosave Manager} \label{Module11} 

\subsection{Module}
AutosaveManager prepares data to be saved, monitors when updates require saving, and invokes periodic scheduled transfers to file-based storage.

\subsection{Uses}
\begin{itemize}
  \item ModelManager for tracking changes that require savings.
  \item SerializationManager for data preparation.
  \item HistoryManager for logging recent changes.
\end{itemize}

\subsection{Syntax}

\noindent\textbf{Exported Constants}

\begin{itemize}
  \item \texttt{AUTOSAVE\_INTERVAL\_MS}: \textit{int} — minimum amount of time (in milliseconds) between autosave operations.
  \item \texttt{FILE\_ID}: \textit{string} — file to which autosave operations are written.
\end{itemize}

\noindent\textbf{Exported Access Programs}

\begin{center}
\renewcommand{\arraystretch}{1.55}
\begin{tabular}{p{4cm} p{3cm} p{3cm} p{3cm}}
\hline
\textbf{Name} & \textbf{In} & \textbf{Out} & \textbf{Exceptions} \\
\hline

\texttt{forceSave} &
None &
None &
IO Error \\

\texttt{enableAutosave} &
None &
None &
None \\

\texttt{disableAutosave} &
None &
None &
None \\

\texttt{lastAutosaveTime} &
None &
DateTime &
None \\

\end{tabular}
\end{center}


\subsection{Semantics}

\subsubsection{State Variables}

\begin{itemize}
  \item \texttt{autosavePermission}: \textit{bool} — indicates whether autosave is enabled or disabled.
  \item \texttt{lastAutosave}: \textit{DateTime} — timestamp of last transfer to file-based storage.
  \item \texttt{autosaveHistory}: \textit{dict[DateTime] = Update} — complete log of transfers to file-based storage.
\end{itemize}


\subsubsection{Environment Variables}

\begin{itemize}
  \item \texttt{SYSTEM\_TIME}: \textit{DateTime} — global time supplied by the current system.
\end{itemize}


\subsubsection{Assumptions}
\begin{itemize}
  \item \texttt{SYSTEM\_TIME} always increases monotonically.
  \item Data preparation method is compatible with file-based storage.
  \item Connection is available to file-based storage.
  \item Valid configuration exists for operations to file-based storage.
\end{itemize}

\newpage

\section{MIS of Voxel Tracking} \label{Module12} \wss{Use labels for
  cross-referencing}

\wss{You can reference SRS labels, such as R\ref{R_Inputs}.}

\wss{It is also possible to use \LaTeX for hypperlinks to external documents.}

\subsection{Module}

\wss{Short name for the module}

\subsection{Uses}


\subsection{Syntax}

\subsubsection{Exported Constants}

\subsubsection{Exported Access Programs}

\begin{center}
\begin{tabular}{p{2cm} p{4cm} p{4cm} p{2cm}}
\hline
\textbf{Name} & \textbf{In} & \textbf{Out} & \textbf{Exceptions} \\
\hline
\wss{accessProg} & - & - & - \\
\hline
\end{tabular}
\end{center}

\subsection{Semantics}

\subsubsection{State Variables}

\wss{Not all modules will have state variables.  State variables give the module
  a memory.}

\subsubsection{Environment Variables}

\wss{This section is not necessary for all modules.  Its purpose is to capture
  when the module has external interaction with the environment, such as for a
  device driver, screen interface, keyboard, file, etc.}

\subsubsection{Assumptions}

\wss{Try to minimize assumptions and anticipate programmer errors via
  exceptions, but for practical purposes assumptions are sometimes appropriate.}

\newpage

\section{MIS of Highlight Manager} \label{Module13} 

\subsection{Module}
HighlightManager manages the voxel highlights in accordance with visual semantic meaning.

\subsection{Uses}
\begin{itemize}
  \item \texttt{Visualization State Manager} for managing visual updates related to voxel highlights.
\end{itemize}

\subsection{Syntax}

\noindent\textbf{Exported Constants}

\begin{itemize}
  \item \texttt{DEFAULT\_HIGHLIGHT\_MAP}: \textit{dict[string, string]} — mapping from semantic keys to their default highlight colours.
\end{itemize}

\noindent\textbf{Exported Access Programs}

\begin{center}
\renewcommand{\arraystretch}{1.55}
\begin{tabular}{p{4cm} p{5cm} p{3cm} p{3cm}}
\hline
\textbf{Name} & \textbf{In} & \textbf{Out} & \textbf{Exceptions} \\
\hline

\texttt{getHighlight} &
semanticKey: \textit{string} &
ColourValue &
None \\

\texttt{editPalette} &
semanticKey: \textit{string} \newline
newColour: \textit{ColourValue} &
void &
None \\

\texttt{setHighlight} &
semanticKey: \textit{string} &
void &
None \\

\texttt{resetPalette} &
None &
void &
None \\

\end{tabular}
\end{center}


\subsection{Semantics}

\subsubsection{State Variables}

\begin{itemize}
  \item \texttt{highlightColourMap}: \textit{dict[string, string]} — current mapping from semantic keys to active highlight colours.
\end{itemize}

\subsubsection{Environment Variables}
None


\subsubsection{Assumptions}
\begin{itemize}
  \item Semantic keys must correspond to valid entries maintained by the TrackingManager.
  \item Colour values are valid colour representations.
\end{itemize}

\newpage

\section{MIS of Export Validation} \label{Module14} \wss{Use labels for
  cross-referencing}

\wss{You can reference SRS labels, such as R\ref{R_Inputs}.}

\wss{It is also possible to use \LaTeX for hypperlinks to external documents.}

\subsection{Module}

\wss{Short name for the module}

\subsection{Uses}


\subsection{Syntax}

\subsubsection{Exported Constants}

\subsubsection{Exported Access Programs}

\begin{center}
\begin{tabular}{p{2cm} p{4cm} p{4cm} p{2cm}}
\hline
\textbf{Name} & \textbf{In} & \textbf{Out} & \textbf{Exceptions} \\
\hline
\wss{accessProg} & - & - & - \\
\hline
\end{tabular}
\end{center}

\subsection{Semantics}

\subsubsection{State Variables}

\wss{Not all modules will have state variables.  State variables give the module
  a memory.}

\subsubsection{Environment Variables}

\wss{This section is not necessary for all modules.  Its purpose is to capture
  when the module has external interaction with the environment, such as for a
  device driver, screen interface, keyboard, file, etc.}

\subsubsection{Assumptions}

\wss{Try to minimize assumptions and anticipate programmer errors via
  exceptions, but for practical purposes assumptions are sometimes appropriate.}

\newpage

\section{MIS of Export Manager} \label{Module15} \wss{Use labels for
  cross-referencing}

\wss{You can reference SRS labels, such as R\ref{R_Inputs}.}

\wss{It is also possible to use \LaTeX for hypperlinks to external documents.}

\subsection{Module}

\wss{Short name for the module}

\subsection{Uses}


\subsection{Syntax}

\subsubsection{Exported Constants}

\subsubsection{Exported Access Programs}

\begin{center}
\begin{tabular}{p{2cm} p{4cm} p{4cm} p{2cm}}
\hline
\textbf{Name} & \textbf{In} & \textbf{Out} & \textbf{Exceptions} \\
\hline
\wss{accessProg} & - & - & - \\
\hline
\end{tabular}
\end{center}

\subsection{Semantics}

\subsubsection{State Variables}

\wss{Not all modules will have state variables.  State variables give the module
  a memory.}

\subsubsection{Environment Variables}

\wss{This section is not necessary for all modules.  Its purpose is to capture
  when the module has external interaction with the environment, such as for a
  device driver, screen interface, keyboard, file, etc.}

\subsubsection{Assumptions}

\wss{Try to minimize assumptions and anticipate programmer errors via
  exceptions, but for practical purposes assumptions are sometimes appropriate.}

\newpage

\section{MIS of Error Diagnostic Handler } \label{Module16} \wss{Use labels for
  cross-referencing}

\wss{You can reference SRS labels, such as R\ref{R_Inputs}.}

\wss{It is also possible to use \LaTeX for hypperlinks to external documents.}

\subsection{Module}

\wss{Short name for the module}

\subsection{Uses}


\subsection{Syntax}

\subsubsection{Exported Constants}

\subsubsection{Exported Access Programs}

\begin{center}
\begin{tabular}{p{2cm} p{4cm} p{4cm} p{2cm}}
\hline
\textbf{Name} & \textbf{In} & \textbf{Out} & \textbf{Exceptions} \\
\hline
\wss{accessProg} & - & - & - \\
\hline
\end{tabular}
\end{center}

\subsection{Semantics}

\subsubsection{State Variables}

\wss{Not all modules will have state variables.  State variables give the module
  a memory.}

\subsubsection{Environment Variables}

\wss{This section is not necessary for all modules.  Its purpose is to capture
  when the module has external interaction with the environment, such as for a
  device driver, screen interface, keyboard, file, etc.}

\subsubsection{Assumptions}

\wss{Try to minimize assumptions and anticipate programmer errors via
  exceptions, but for practical purposes assumptions are sometimes appropriate.}

\newpage

\section{MIS of Model Structure} \label{Module17} \wss{Use labels for
  cross-referencing}

\wss{You can reference SRS labels, such as R\ref{R_Inputs}.}

\wss{It is also possible to use \LaTeX for hypperlinks to external documents.}

\subsection{Module}

\wss{Short name for the module}

\subsection{Uses}


\subsection{Syntax}

\subsubsection{Exported Constants}

\subsubsection{Exported Access Programs}

\begin{center}
\begin{tabular}{p{2cm} p{4cm} p{4cm} p{2cm}}
\hline
\textbf{Name} & \textbf{In} & \textbf{Out} & \textbf{Exceptions} \\
\hline
\wss{accessProg} & - & - & - \\
\hline
\end{tabular}
\end{center}

\subsection{Semantics}

\subsubsection{State Variables}

\wss{Not all modules will have state variables.  State variables give the module
  a memory.}

\subsubsection{Environment Variables}

\wss{This section is not necessary for all modules.  Its purpose is to capture
  when the module has external interaction with the environment, such as for a
  device driver, screen interface, keyboard, file, etc.}

\subsubsection{Assumptions}

\wss{Try to minimize assumptions and anticipate programmer errors via
  exceptions, but for practical purposes assumptions are sometimes appropriate.}

\newpage

\section{MIS of Graphics Adapter} \label{Module18} 

\subsection{Module}
GraphicsAdapter handles all communication with the graphics API to enable visual rendering and generate a model on the UI.

\subsection{Uses}
\begin{itemize}
  \item VisualizationManager aggregates necessary backend data and relays updates to the rendered view.
\end{itemize}

\subsection{Syntax}

\noindent\textbf{Exported Constants}

\begin{itemize}
  \item None
\end{itemize}

\noindent\textbf{Exported Access Programs}

\begin{center}
\renewcommand{\arraystretch}{1.55}
\begin{tabular}{p{4cm} p{5cm} p{3cm} p{3cm}}
\hline
\textbf{Name} & \textbf{In} & \textbf{Out} & \textbf{Exceptions} \\
\hline

\texttt{requestRender} &
str: \textit{UpdateKey} \newline
update: \textit{UpdateBundle} &
void &
Network Error \\

\texttt{getUpdateStatus} &
str: \textit{UpdateKey} &
boolean &
None \\

\texttt{checkServerStatus} &
None &
Promise\textless void\textgreater &
Network Error \\

\end{tabular}
\end{center}

\subsection{Semantics}

\subsubsection{State Variables}

\begin{itemize}
  \item \texttt{updateStatus}: \textit{boolean} — tracks status of rendering completion.
\end{itemize}

\subsubsection{Environment Variables}

\begin{itemize}
  \item \texttt{API\_BASE\_URL}: \textit{string} — external base URL for establishing environment configurations and enabling backend API requests.
\end{itemize}

\subsubsection{Assumptions}

\begin{itemize}
  \item Backend service is reachable and operational.
  \item Connection is available for API.
  \item Valid configurations exist for necessary API operations.
\end{itemize}

\newpage

\section{MIS of Database Handler} \label{Module19} \wss{Use labels for
  cross-referencing}

\wss{You can reference SRS labels, such as R\ref{R_Inputs}.}

\wss{It is also possible to use \LaTeX for hypperlinks to external documents.}

\subsection{Module}

\wss{Short name for the module}

\subsection{Uses}


\subsection{Syntax}

\subsubsection{Exported Constants}

\subsubsection{Exported Access Programs}

\begin{center}
\begin{tabular}{p{2cm} p{4cm} p{4cm} p{2cm}}
\hline
\textbf{Name} & \textbf{In} & \textbf{Out} & \textbf{Exceptions} \\
\hline
\wss{accessProg} & - & - & - \\
\hline
\end{tabular}
\end{center}

\subsection{Semantics}

\subsubsection{State Variables}

\wss{Not all modules will have state variables.  State variables give the module
  a memory.}

\subsubsection{Environment Variables}

\wss{This section is not necessary for all modules.  Its purpose is to capture
  when the module has external interaction with the environment, such as for a
  device driver, screen interface, keyboard, file, etc.}

\subsubsection{Assumptions}

\wss{Try to minimize assumptions and anticipate programmer errors via
  exceptions, but for practical purposes assumptions are sometimes appropriate.}

\newpage

\section{MIS of Export Structure} \label{Module20} \wss{Use labels for
  cross-referencing}

\wss{You can reference SRS labels, such as R\ref{R_Inputs}.}

\wss{It is also possible to use \LaTeX for hypperlinks to external documents.}

\subsection{Module}

\wss{Short name for the module}

\subsection{Uses}


\subsection{Syntax}

\subsubsection{Exported Constants}

\subsubsection{Exported Access Programs}

\begin{center}
\begin{tabular}{p{2cm} p{4cm} p{4cm} p{2cm}}
\hline
\textbf{Name} & \textbf{In} & \textbf{Out} & \textbf{Exceptions} \\
\hline
\wss{accessProg} & - & - & - \\
\hline
\end{tabular}
\end{center}

\subsection{Semantics}

\subsubsection{State Variables}

\wss{Not all modules will have state variables.  State variables give the module
  a memory.}

\subsubsection{Environment Variables}

\wss{This section is not necessary for all modules.  Its purpose is to capture
  when the module has external interaction with the environment, such as for a
  device driver, screen interface, keyboard, file, etc.}

\subsubsection{Assumptions}

\wss{Try to minimize assumptions and anticipate programmer errors via
  exceptions, but for practical purposes assumptions are sometimes appropriate.}

\newpage

\bibliographystyle {plainnat}
\bibliography {../../../refs/References}

\newpage

\section{Appendix} \label{Appendix}

\wss{Extra information if required}

\newpage{}

\section*{Appendix --- Reflection}

\wss{Not required for CAS 741 projects}

The information in this section will be used to evaluate the team members on the
graduate attribute of Problem Analysis and Design.

\input{../../Reflection.tex}

\begin{enumerate}
  \item What went well while writing this deliverable? 
  \item What pain points did you experience during this deliverable, and how
    did you resolve them?
  \item Which of your design decisions stemmed from speaking to your client(s)
  or a proxy (e.g. your peers, stakeholders, potential users)? For those that
  were not, why, and where did they come from?
  \item While creating the design doc, what parts of your other documents (e.g.
  requirements, hazard analysis, etc), it any, needed to be changed, and why?
  \item What are the limitations of your solution?  Put another way, given
  unlimited resources, what could you do to make the project better? (LO\_ProbSolutions)
  \item Give a brief overview of other design solutions you considered.  What
  are the benefits and tradeoffs of those other designs compared with the chosen
  design?  From all the potential options, why did you select the documented design?
  (LO\_Explores)
\end{enumerate}


\end{document}